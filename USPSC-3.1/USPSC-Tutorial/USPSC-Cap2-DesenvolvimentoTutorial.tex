%% USPSC-Cap2-DesenvolvimentoTutorial.tex 

% ---
% Este capítulo, utilizado por diferentes exemplos do abnTeX2, ilustra o uso de
% comandos do abnTeX2 e de LaTeX.
% ---

\chapter{Desenvolvimento}\label{cap_exemplos}
Este capítulo é parte principal do trabalho acadêmico e deve conter a exposição ordenada e detalhada do assunto. Divide-se em seções e subseções, em conformidade com a abordagem do tema e do método, abrangendo: revisão bibliográfica, materiais e métodos, técnicas utilizadas, resultados obtidos e discussão.

O conteúdo deste documento visa apresentar um tutorial para utilização do Pacote USPSC, composto da Classe USPSC, tutorial e modelos, utilizando a estrutura de trabalhos acadêmicos, mas por questões didáticas adotou-se capítulo, seções e subseções diferentes das usualmente utilizadas.


\section{Pacote USPSC: Classe USPSC e modelos de trabalhos acadêmicos}\label{Pacote}
A versão 3.1 do Pacote USPSC traz os modelos simplificados de trabalhos acadêmicos \textbf{USPSC-modelo.tex} e \textbf{USPSC-TCC-modelo.tex} e o \textbf{Tutorial do Pacote USPSC para modelos de trabalhos acad\^emicos em LaTeX - vers\~ao 3.1}, contendo as instruções precisas e detalhadas para melhor utilização dos recursos do Pacote USPSC. Para tanto, foram acrescidos diversos arquivos, para atender as especificidades do tutorial que possui os elementos pré-textuais distintos para teses, dissertações, TCCs e outros trabalhos acadêmicos, conforme descrito em  \textbf{\ref{Pacote} Pacote USPSC: Classe USPSC e modelos de trabalhos acadêmicos}. A estrutura deste tutorial é igual à estrutura de trabalhos acadêmicos estabelecida pela ABNT NBR 14724, conforme a \autoref{fig_EstruturaTrabAcad}.

Todas as alterações e novas implementações foram relacionadas em \textbf{\ref{Introdução} INTRODUÇÃO} e neste capítulo serão descritas detalhadamente, quando necessário. 

A classe USPSC é uma derivada da classe \textbf{\abnTeX\ v1.9.5} para as Unidades de ensino e pesquisa do Campus USP de São Carlos:
Escola de Engenharia de São Carlos (EESC), Instituto de Arquitetura e Urbanismo (IAU), Instituto de Ciências Matemáticas e de Computação (ICMC), Instituto de Física de São Carlos (IFSC) e Instituto de Química de São Carlos (IQSC).

O objetivo do projeto é disponibilizar modelos em \LaTeX\  para a elaboração de trabalhos acadêmicos (tese, dissertação, trabalho de conclusão de curso (TCC), dentre outros) em conformidade com a \textbf{ABNT NBR 14724}: informação e documentação: trabalhos acadêmicos: apresentação \cite{nbr14724}, \textbf{Diretrizes para apresentação de dissertações e teses da USP}: documento eletrônico e impresso - Parte I (ABNT) \cite{aguia2020} e normas e padrões estabelecidos pelas Unidades.

Este documento e seu código fonte são exemplos de uso da \textbf{Classe USPSC} e do pacote \textbf{abntex2cite}.
Para complementar as instruções contidas neste documento, utilize os manuais \cite{abnetxclasse,abnetxcite,abnetxcitealf} e da classe \textsf{memoir}\cite{memoir2010}. 


Os referidos modelos seguem a estrutura de trabalhos acadêmicos estabelecida pela ABNT NBR 14724, conforme a \autoref{fig_EstruturaTrabAcad}. 

\begin{figure}[htb]
	\caption{\label{fig_EstruturaTrabAcad}Estrutura do trabalho acadêmico}
	\begin{center}
		\includegraphics[scale=0.5]{USPSC-img/USPSC-EstruturaTrabAcad.jpg}
	\end{center}
	\legend{Fonte: \citeonline{nbr14724}}
\end{figure}

A versão v.3.1 do Pacote USPSC, por \textit{default}, é instalado em uma pasta denominada USPSC-v3.1 com a seguinte distribuição dos arquivos utilizados para gerar o documento em formato PDF, mediante a compilação utilizando um dos editores \LaTeX\ :


\begin{alineas}	
	\item \verb+ ...\USPSC-v3.1\ + contendo:	
		\begin{alineas}
			\item arquivo principal do tutorial: USPSC-tutorial.tex;
			\item arquivos principais do modelo para teses e dissertações:
				\begin{subalineas}
					\item USPSC-modelo.tex;
					\item USPSC-modelo-EESC.tex;
					\item USPSC-modelo-IAU.tex;
					\item USPSC-modelo-ICMCe.tex (idioma do texto em inglês);
					\item USPSC-modelo-ICMCp.tex (idioma do texto em português);
					\item USPSC-modelo-IFSCe.tex (idioma do texto em inglês);
					\item USPSC-modelo-IFSCp.tex (idioma do texto em português);
					\item USPSC-modelo-IQSC.tex;
				\end{subalineas}
			\item arquivos principais do modelo para TCC: 
			\begin{subalineas}
				\item USPSC-TCC-modelo.tex;
				\item USPSC-TCC-modelo-EESC.tex;
				\item USPSC-TCC-modelo-ICMCe.tex (idioma do texto em inglês);
				\item USPSC-TCC-modelo-ICMCp.tex (idioma do texto em português);
				\item USPSC-TCC-modelo-IQSC.tex;
			\end{subalineas}
			\item arquivo que relaciona os arquivos pré-textuais dos programas de Pós-Graduação e TCCs das Unidades: USPSC-unidades.tex;
			\item arquivos pré-textuais: USPSC-pre-textual-UUUU.tex e USPSC-TCC-pre-textual-UUUU.tex;
			\item pastas especificadas nos itens abaixo;
	   	\end{alineas}
	   	
	\item \verb+ ...\USPSC-v3.1\USPSC-bib\ + com o arquivo de dados das citações e referências utilizadas: 
		\begin{alineas}	
			\item USPSC-modelo-references.bib;
		\end{alineas} 
		
	\item \verb+ ...\USPSC-v3.1\USPSC-classe\ + com os arquivos da classe USPSC, incluindo os referentes à compatibilização com a ABNT NBR 6023:2018:
		\begin{alineas}
			\item USPSC.cls;
			\item USPSC1.cls; 
			\item arquivo ABNT6023-2018.sty, para tirar <> da URL, chamado mediante o comando  \verb+ \usepackage{USPSC-classe/ABNT6023-2018}+;
			\item as demais compatibilizações estão nos arquivos abntex2-alf-USPSC.bst, abntex2-alfeng-USPSC.bst, abntex2-num-USPSC.bst e abntex2-numeng-USPSC.bst, chamados através do comando \newline \verb+\bibliographystyle{USPSC-classe/abntex2-alf-USPSC} + ou \newline
			\verb+\bibliographystyle{USPSC-classe/abntex2-alfeng-USPSC} + ou \newline
			\verb+\bibliographystyle{USPSC-classe/abntex2-num-USPSC}+ ou \newline \verb+\bibliographystyle{USPSC-classe/abntex2-numeng-USPSC}+, dependendo se o Sistemas de chamada for autor-data ou numérico; 
		\end{alineas}	
	    
	\item \verb+ ...\USPSC-v3.1\USPSC-Tutorial\ + contendo:
		\begin{subalineas}
			\item USPSC-fichacatalograficaTutorial.tex;
			\item USPSC-ErrataTutorial.tex;
			\item USPSC-DedicatoriaTutorial.tex;
			\item USPSC-AgradecimentosTutorial.tex;
			\item USPSC-EpigrafeTutorial.tex;
			\item USPSC-ResumoTutorial.tex;
			\item USPSC-AbstractTutorial.tex;
			\item USPSC-AbreviaturasSiglasTutorial.tex;
			\item USPSC-SimbolosTutorial.tex;
			\item USPSC-Cap1-IntroducaoTutorial.tex;
			\item USPSC-Cap2-DesenvolvimentoTutorial;
			\item USPSC-Cap3-CitacoesTutorial.tex;
			\item USPSC-Cap4-ReferenciasTutorial.tex;
			\item USPSC-Cap5-ConclusaoTutorial.tex;
			\item USPSC-ApendicesTutorial.tex;
			\item USPSC-AnexosTutorial.tex;
			\item USPSC-IndicesRemissivosTutorial.tex.
		\end{subalineas}
	
	\item \verb+ ...\USPSC-v3.1\USPSC-TA-PreTextual\ + com os arquivos relativos aos elementos pré-textuais de trabalhos acadêmicos:
		\begin{alineas}
				\item USPSC-CapaICMC.tex;
				\item USPSC-fichacatalografica.tex;
				\item USPSC-fichacatalografica.pdf;
				\item USPSC-folhadeaprovacao.tex;
				\item USPSC-folhadeaprovacao.pdf;
				\item USPSC-Errata.tex;
				\item USPSC-Dedicatoria.tex;
				\item USPSC-Agradecimentos.tex;
				\item USPSC-Epigrafe.tex;
				\item USPSC-Resumo.tex;
				\item USPSC-Abstract.tex;
				\item USPSC-AbreviaturasSiglas.tex
				\item USPSC-Simbolos.tex 
				\item USPSC-PaginaEmBranco.pdf
			\end{alineas}
			
	\item \verb+ ...\USPSC-v3.1\USPSC-TA-Textual\ + contendo os arquivos relativos aos elementos textuais de trabalhos acadêmicos:
		\begin{alineas}
			\item USPSC-Cap1-Introducao.tex;
			\item USPSC-Cap2-Desenvolvimento.tex;
			\item USPSC-Cap3-Conclusao.tex;
		\end{alineas}
		
	\item \verb+ ...\USPSC-v3.1\USPSC-TA-PosTextual\ + para os arquivos relativos aos elementos pós-textuais de trabalhos acadêmicos:	
		\begin{alineas}
			\item USPSC-Apendices.tex;
			\item USPSC-Anexos.tex;
			\item USPSC-IndicesRemissivos.tex;
		\end{alineas}
	
	\item \verb+ ...\USPSC-v3.1\USPSC-img\ + contendo os arquivos de imagens e os PDFs relacionados no texto dos modelos: 
		\begin{alineas}	
			\item USPSC-AcentuacaoLaTeX.png;
			\item CapaICMC.jpg;
			\item USPSC-EstruturaTrabAcad.jpg;
			\item USPSC-LetrasGregas.png;
			\item USPSC-modelo-img-grafico.pdf;
			\item USPSC-modelo-img-marca.pdf;
			\item USPSC-SimbolosUteis.png;
		\end{alineas}	
	
	\item \verb+ ...\USPSC-v3.1\USPSC-Siglas\ + traz os arquivos que relacionam as siglas das Unidades e as definidas para os cursos de graduação e programas de pós-graduação, que não necessariamente são as oficiais utilizadas pela Universidade: 
		\begin{alineas}	
			\item USPSC-Siglas estabelecidas para os Programas de Pós-Graduação por Unidade.xlsx;
			\item USPSC-TCC-Siglas estabelecidas para as Graduações por Unidade.xlsx;
		\end{alineas}
		
	\item \verb+ ...\USPSC-v3.1\USPSC-Sobre\ + contém arquivos sobre cada versão do Pacote USPSC.

\end{alineas}	 

		
Para tese ou dissertação deverá ser utilizado o arquivo USPSC-modelo.tex, onde o autor deverá indicar a sigla da Unidade e a sigla do programa de pós-graduação que está vinculado, a exemplo dos comandos abaixo:
		
			\begin{verbatim}
				\siglaunidade{IQSC}
				\programa{MQOB}
			\end{verbatim}
			
Para o \textbf{Modelo para teses e dissertações em \LaTeX\ utilizando o Pacote USPSC} estão definidos os seguintes arquivos pré-textuais:
			
			\begin{alineas}	 
				\item USPSC-pre-textual-EESC.tex;
				\item USPSC-pre-textual-IAU.tex;
				\item USPSC-pre-textual-ICMC.tex;
				\item USPSC-pre-textual-IFSC.tex;
				\item USPSC-pre-textual-IQSC.tex.
			\end{alineas}
			
Para TCC deverá ser utilizado o arquivo USPSC-TCC-modelo.tex, onde o autor deverá indicar a \textbf{'sigla da Unidade'} + \textbf{'-TCC'} (Exemplo: EESC-TCC) e a sigla do curso de graduação que está vinculado, a exemplo dos comandos abaixo:
			
			\begin{verbatim}
			\siglaunidade{EESC-TCC}
			\programa{EAMB}
			\end{verbatim}
			
Atualmente estão disponíveis os dados pré-textuais para a EESC, ICMC e IQSC:
			
			\begin{alineas}	 
				\item USPSC-TCC-pre-textual-EESC.tex;
				\item USPSC-TCC-pre-textual-ICMC.tex;
				\item USPSC-TCC-pre-textual-IQSC.tex.
			\end{alineas}
			
Serão incluídos os demais arquivos quando as demais Unidades do Campus USP de São Carlos estabelecerem seus padrões. 
			
Para utilizar corretamente os dados pré-textuais, é necessário consultar as siglas estabelecidas para os cursos de graduação e para os programas de pós-graduação da Unidade de vínculo nos quadros dos \textbf{APÊNDICES B-I} ou nas planilhas \textbf{USPSC-TCC-Siglas estabelecidas para as Graduações por Unidade.xlsx} e \textbf{USPSC-Siglas estabelecidas para os programas de pós-graduação por Unidade.xlsx}. 

Os arquivos com dados pre-textuais estão nominados como USPSC-pre-textual-UUUU.tex ou USPSC-TCC-pre-textual-UUUU.tex, onde UUUU é a sigla da Unidade. Inicialmente estão disponibilizados apenas os pré-textuais das Unidades do Campus USP de São Carlos.
			
Foram definidos os arquivos USPSC-pre-textual-OUTRO.tex e USPSC-TCC-pre-textual-OUTRO.tex que serão executados quando uma das siglas for diferente das explicitadas para as Unidades e/ou para os cursos de graduação e/ou para os programas de Pós-Graduação. O preâmbulo será incompleto e apresentando "..." no final, evidenciando que o autor deverá rever as siglas utilizadas.

Através do comando \verb+ %% USPSC-unidades.tex
% Camando para definição do programa de Pós-Graduação, Especialidade do Título e Instituição
\newcommand{\siglaunidade}[1]{


% EESC-TCC ==========================================================================
    \ifthenelse{\equal{#1}{EESC-TCC}}{
     			%% USPSC-TCC-pre-textual-EESC.tex
%% Camandos para defini��o do tipo de documento (tese ou disserta��o), �rea de concentra��o, op��o, pre�mbulo, titula��o 
%% referentes aos Programas de P�s-Gradua��o
\instituicao{Escola de Engenharia de S\~ao Carlos, Universidade de S\~ao Paulo}
\unidade{ESCOLA DE ENGENHARIA DE S\~AO CARLOS}
\unidademin{Escola de Engenharia de S\~ao Carlos}
\universidademin{Universidade de S\~ao Paulo}
% A EESC n�o inclui a nota "Vers�o original", portanto o comando abaixo n�o tem a mensagem entre {}
\notafolharosto{ }
%Para a vers�o corrigida tire a % do comando/declara��o abaixo e inclua uma % antes do comando acima
%\notafolharosto{VERS\~AO CORRIGIDA}
% ---
% dados complementares para CAPA e FOLHA DE ROSTO
% ---
\universidade{UNIVERSIDADE DE S\~AO PAULO}
\titulo{Modelo para TCC em \LaTeX\ utilizando o Pacote USPSC para a EESC}
\titleabstract{Model for TCC in \LaTeX\ using the USPSC Package to the EESC}
\tituloresumo{Modelo para TCC em \LaTeX\ utilizando o Pacote USPSC para a EESC}
\autor{Jos\'e da Silva}
\autorficha{Silva, Jos\'e da}
\autorabr{SILVA, J.}

\cutter{S856m}
% Para gerar a ficha catalogr�fica sem o C�digo Cutter, basta 
% incluir uma % na linha acima e tirar a % da linha abaixo
%\cutter{ }

\local{S\~ao Carlos}
\data{2021}
% Quando for Orientador, basta incluir uma % antes do comando abaixo
\renewcommand{\orientadorname}{Orientadora:}
% Quando for Coorientadora, basta tirar a % do comando abaixo
%\newcommand{\coorientadorname}{Coorientador:}
\orientador{Profa. Dra. Elisa Gon\c{c}alves Rodrigues}
%\orientadorcorpoficha{orientadora Elisa Gon\c{c}alves Rodrigues}
%\orientadorficha{Rodrigues, Elisa Gon\c{c}alves, orient}
%Se houver co-orientador, inclua % antes das duas linhas (antes dos comandos \orientadorcorpoficha e \orientadorficha) 
%          e tire a % antes dos 3 comandos abaixo
\coorientador{Prof. Dr. Jo\~ao Alves Serqueira}
\orientadorcorpoficha{orientadora Elisa Gon\c{c}alves Rodrigues ;  co-orientador Jo\~ao Alves Serqueira}
\orientadorficha{Rodrigues, Elisa Gon\c{c}alves, orient. II. Serqueira, Jo\~ao Alves, co-orient}

\notaautorizacao{AUTORIZO A REPRODU\c{C}\~AO E DIVULGA\c{C}\~AO TOTAL OU PARCIAL DESTE TRABALHO, POR QUALQUER MEIO CONVENCIONAL OU ELETR\^ONICO PARA FINS DE ESTUDO E PESQUISA, DESDE QUE CITADA A FONTE.}
\notabib{~  ~}

\newcommand{\programa}[1]{


% EAMB ==========================================================================
\ifthenelse{\equal{#1}{EAMB}}{
    \tipotrabalho{Monografia (Trabalho de Conclus\~ao de Curso)}
    \tipotrabalhoabs{Monograph (Conclusion Course Paper)}
    %\area{Nome da �rea}
	%\opcao{Nome da Op��o}
    % O preambulo deve conter o tipo do trabalho, o objetivo, 
	% o nome da institui��o, a �rea de concentra��o e op��o quando houver
	\preambulo{Monografia apresentada ao Curso de Engenharia Ambiental, da Escola de Engenharia de S\~ao Carlos da Universidade de S\~ao Paulo, como parte dos requisitos para obten\c{c}\~ao do t\'itulo de Engenheiro Ambiental.}
	\notaficha{Monografia (Gradua\c{c}\~ao em Engenharia Ambiental)}
    }{
% EAER ===========================================================================
\ifthenelse{\equal{#1}{EAER}}{
	\tipotrabalho{Monografia (Trabalho de Conclus\~ao de Curso)}
	\tipotrabalhoabs{Monograph (Conclusion Course Paper)}
	%\area{Nome da �rea}
	%\opcao{Nome da Op��o}
	% O preambulo deve conter o tipo do trabalho, o objetivo, 
	% o nome da institui��o, a �rea de concentra��o e op��o quando houver
	\preambulo{Monografia apresentada ao Curso de Engenharia Aeron\'autica, da Escola de Engenharia de S\~ao Carlos da Universidade de S\~ao Paulo, como parte dos requisitos para obten\c{c}\~ao do t\'itulo de Engenheiro Aeron\'autico.}
	\notaficha{Monografia (Gradua\c{c}\~ao em Engenharia Aeron\'autica)}
    }{
% ECIV =======================================================================
\ifthenelse{\equal{#1}{ECIV}}{
    \tipotrabalho{Monografia (Trabalho de Conclus\~ao de Curso)}
    \tipotrabalhoabs{Monograph (Conclusion Course Paper)}
    %\area{Nome da �rea}
    %\opcao{Nome da Op��o}
    % O preambulo deve conter o tipo do trabalho, o objetivo, 
	% o nome da institui��o, a �rea de concentra��o e op��o quando houver
	\preambulo{Monografia apresentada ao Curso de Engenharia Civil, da Escola de Engenharia de S\~ao Carlos da Universidade de S\~ao Paulo, como parte dos requisitos para obten\c{c}\~ao do t\'itulo de Engenheiro Civil.}
	\notaficha{Monografia (Gradua\c{c}\~ao em Engenharia Civil)}
    }{
% ECOM ===========================================================================
\ifthenelse{\equal{#1}{ECOM}}{
	\tipotrabalho{Monografia (Trabalho de Conclus\~ao de Curso)}
	\tipotrabalhoabs{Monograph (Conclusion Course Paper)}
	%\area{Nome da �rea}
	%\opcao{Nome da Op��o}
	% O preambulo deve conter o tipo do trabalho, o objetivo, 
	% o nome da institui��o, a �rea de concentra��o e op��o quando houver
	\preambulo{Monografia apresentada ao Curso de Engenharia de Computa\c{c}\~ao, da Escola de Engenharia de S\~ao Carlos e Instituto de Ci\^encias Matem\'aticas e de Computa\c{c}\~ao da Universidade de S\~ao Paulo, como parte dos requisitos para obten\c{c}\~ao do t\'itulo de Engenheiro de Computa\c{c}\~ao.}
	\notaficha{Monografia (Gradua\c{c}\~ao em Engenharia de Computa\c{c}\~ao)}
    }{
% EELT ==========================================================================
\ifthenelse{\equal{#1}{EELT}}{
    \tipotrabalho{Monografia (Trabalho de Conclus\~ao de Curso)}
    \tipotrabalhoabs{Monograph (Conclusion Course Paper)}
	%\area{Nome da �rea}
    %\opcao{Nome da Op��o}
    % O preambulo deve conter o tipo do trabalho, o objetivo, 
    % o nome da institui��o, a �rea de concentra��o e op��o quando houver
    \preambulo{Monografia apresentada ao Curso de Engenharia El\'etrica com \^Enfase em Eletr\^onica, da Escola de Engenharia de S\~ao Carlos da Universidade de S\~ao Paulo, como parte dos requisitos para obten\c{c}\~ao do t\'itulo de Engenheiro Eletricista.}
    \notaficha{Monografia (Gradua\c{c}\~ao em Engenharia El\'etrica com \^Enfase em Eletr\^onica)}
    }{
% EELS ===========================================================================
\ifthenelse{\equal{#1}{EELS}}{
	\tipotrabalho{Monografia (Trabalho de Conclus\~ao de Curso)}
	\tipotrabalhoabs{Monograph (Conclusion Course Paper)}
	%\area{Nome da �rea}
	%\opcao{Nome da Op��o}
	% O preambulo deve conter o tipo do trabalho, o objetivo, 
	% o nome da institui��o, a �rea de concentra��o e op��o quando houver
	\preambulo{Monografia apresentada ao Curso de Curso de Engenharia El\'etrica com \^Enfase em Sistemas de Energia e Automa\c{c}\~ao, da Escola de Engenharia de S\~ao Carlos da Universidade de S\~ao Paulo, como parte dos requisitos para obten\c{c}\~ao do t\'itulo de Engenheiro Eletricista.}
	\notaficha{Monografia (Gradua\c{c}\~ao em Engenharia El\'etrica com \^Enfase Sistemas de Energia e Automa\c{c}\~ao)}
    }{			
% EMAT ==========================================================================
\ifthenelse{\equal{#1}{EMAT}}{
    \tipotrabalho{Monografia (Trabalho de Conclus\~ao de Curso)}
    \tipotrabalhoabs{Monograph (Conclusion Course Paper)}
    %\area{Nome da �rea}
    %\opcao{Nome da Op��o}
    % O preambulo deve conter o tipo do trabalho, o objetivo, 
    % o nome da institui��o, a �rea de concentra��o e op��o quando houver
    \preambulo{Monografia apresentada ao Curso de Engenharia de Materiais e Manufatura, da Escola de Engenharia de S\~ao Carlos da Universidade de S\~ao Paulo, como parte dos requisitos para obten\c{c}\~ao do t\'itulo de Engenheiro de Materiais e de Manufatura.}
    \notaficha{Monografia (Gradua\c{c}\~ao em Engenharia de Materiais e Manufatura)}
    }{
% EMEC ===========================================================================
\ifthenelse{\equal{#1}{EMEC}}{
	\tipotrabalho{Monografia (Trabalho de Conclus\~ao de Curso)}
	\tipotrabalhoabs{Monograph (Conclusion Course Paper)}
	%\area{Nome da �rea}
	%\opcao{Nome da Op��o}
	% O preambulo deve conter o tipo do trabalho, o objetivo, 
	% o nome da institui��o, a �rea de concentra��o e op��o quando houver
	\preambulo{Monografia apresentada ao Curso de Engenharia Mec\^anica, da Escola de Engenharia de S\~ao Carlos da Universidade de S\~ao Paulo, como parte dos requisitos para obten\c{c}\~ao do t\'itulo de Engenheiro Mec\^anico.}
	\notaficha{Monografia (Gradua\c{c}\~ao em Engenharia Mec\^anica)}
    }{			
% EMET ===========================================================================
\ifthenelse{\equal{#1}{EMET}}{
    \tipotrabalho{Monografia (Trabalho de Conclus\~ao de Curso)}
    \tipotrabalhoabs{Monograph (Conclusion Course Paper)}
    %\area{Nome da �rea}
    %\opcao{Nome da Op��o}
    % O preambulo deve conter o tipo do trabalho, o objetivo, 
    % o nome da institui��o, a �rea de concentra��o e op��o quando houver
    \preambulo{Monografia apresentada ao Curso de Engenharia Mecatr\^onica, da Escola de Engenharia de S\~ao Carlos da Universidade de S\~ao Paulo, como parte dos requisitos para obten\c{c}\~ao do t\'itulo de Engenheiro Mecatr\^onico.}
    \notaficha{Monografia (Gradua\c{c}\~ao em Engenharia Mecatr\^onica)}
    }{		
% EPRO ===========================================================================
\ifthenelse{\equal{#1}{EPRO}}{
	\tipotrabalho{Monografia (Trabalho de Conclus\~ao de Curso)}
	\tipotrabalhoabs{Monograph (Conclusion Course Paper)}
	%\area{Nome da �rea}
	%\opcao{Nome da Op��o}
	% O preambulo deve conter o tipo do trabalho, o objetivo, 
	% o nome da institui��o, a �rea de concentra��o e op��o quando houver
	\preambulo{Monografia apresentada ao Curso de Engenharia de Produ\c{c}\~ao, da Escola de Engenharia de S\~ao Carlos da Universidade de S\~ao Paulo, como parte dos requisitos para obten\c{c}\~ao do t\'itulo de Engenheiro de Produ\c{c}\~ao.}
	\notaficha{Monografia (Gradua\c{c}\~ao em Engenharia de Produ\c{c}\~a)}
}{		         	
% Outros
	\tipotrabalho{Monografia (Trabalho de Conclus\~ao de Curso)}
	\tipotrabalhoabs{Monograph (Conclusion Course Paper)}
	%\area{Nome da �rea}
	%\opcao{Nome da Op��o}
	% O preambulo deve conter o tipo do trabalho, o objetivo, 
	% o nome da institui��o, a �rea de concentra��o e op��o quando houver
	\preambulo{Monografia apresentada ao Curso de Engenharia (?), da Escola de Engenharia de S\~ao Carlos da Universidade de S\~ao Paulo, como parte dos requisitos para obten\c{c}\~ao do t\'itulo de Engenheiro (?).}
	\notaficha{Monografia (Gradua\c{c}\~ao em Engenharia (?))}		
        }}}}}}}}}}}
        				
% ---
    }{
% EESC ==========================================================================
	\ifthenelse{\equal{#1}{EESC}}{
	%% USPSC-pre-textual-EESC.tex
%% Camandos para defini��o do tipo de documento (tese ou disserta��o), �rea de concentra��o, op��o, pre�mbulo, titula��o 
%% referentes aos Programas de P�s-Gradua��o
\instituicao{Escola de Engenharia de S\~ao Carlos, Universidade de S\~ao Paulo}
\unidade{ESCOLA DE ENGENHARIA DE S\~AO CARLOS}
\unidademin{Escola de Engenharia de S\~ao Carlos}
\universidademin{Universidade de S\~ao Paulo}

% A EESC n�o inclui a nota "Vers�o original", portanto o comando abaixo n�o tem a mensagem entre {}
\notafolharosto{ }
%Para a vers�o corrigida tire a % do comando/declara��o abaixo e inclua uma % antes do comando acima
%\notafolharosto{VERS\~AO CORRIGIDA}
% ---
% dados complementares para CAPA e FOLHA DE ROSTO
% ---
\universidade{UNIVERSIDADE DE S\~AO PAULO}
\titulo{Modelo para teses e disserta\c{c}\~oes em \LaTeX\ utilizando o Pacote USPSC para a EESC}
\titleabstract{Model for thesis and dissertations in \LaTeX\ using the USPSC Package to the EESC}
\tituloresumo{Modelo para teses e disserta\c{c}\~oes em \LaTeX\ utilizando o Pacote USPSC para a EESC}
\autor{Jos\'e da Silva}
\autorficha{Silva, Jos\'e da}
\autorabr{SILVA, J.}

\cutter{S856m}
% Para gerar a ficha catalogr�fica sem o C�digo Cutter, basta 
% incluir uma % na linha acima e tirar a % da linha abaixo
%\cutter{ }

\local{S\~ao Carlos}
\data{2021}
% Quando for Orientador, basta incluir uma % antes do comando abaixo
\renewcommand{\orientadorname}{Orientadora:}
% Quando for Coorientadora, basta tirar a % utilizar o comando abaixo
%\newcommand{\coorientadorname}{Coorientador:}
\orientador{Profa. Dra. Elisa Gon\c{c}alves Rodrigues}
\orientadorcorpoficha{orientadora Elisa Gon\c{c}alves Rodrigues}
\orientadorficha{Rodrigues, Elisa Gon\c{c}alves, orient}
%Se houver co-orientador, inclua % antes das duas linhas (antes dos comandos \orientadorcorpoficha e \orientadorficha) 
%          e tire a % antes dos 3 comandos abaixo
%\coorientador{Prof. Dr. Jo\~ao Alves Serqueira}
%\orientadorcorpoficha{orientadora Elisa Gon\c{c}alves Rodrigues ;  co-orientador Jo\~ao Alves Serqueira}
%\orientadorficha{Rodrigues, Elisa Gon\c{c}alves, orient. II. Serqueira, Jo\~ao Alves, co-orient}

\notaautorizacao{AUTORIZO A REPRODU\c{C}\~AO E DIVULGA\c{C}\~AO TOTAL OU PARCIAL DESTE TRABALHO, POR QUALQUER MEIO CONVENCIONAL OU ELETR\^ONICO PARA FINS DE ESTUDO E PESQUISA, DESDE QUE CITADA A FONTE.}
\notabib{~  ~}

\newcommand{\programa}[1]{

% DCEA ==========================================================================
\ifthenelse{\equal{#1}{DCEA}}{
    \tipotrabalho{Tese (Doutorado)}
    \tipotrabalhoabs{Thesis (Doctor)}
    \area{Ci\^encias da Engenharia Ambiental}
	%\opcao{Nome da Op��o}
    % O preambulo deve conter o tipo do trabalho, o objetivo, 
	% o nome da institui��o, a �rea de concentra��o e op��o quando houver
	\preambulo{Tese apresentada \`a Escola de Engenharia de S\~ao Carlos da Universidade de S\~ao Paulo, para obten\c{c}\~ao do t\'itulo de Doutor em Ci\^encias - Programa de P\'os-Gradua\c{c}\~ao em Ci\^encias da Engenharia Ambiental.}
	\notaficha{Tese (Doutorado) - Programa de P\'os-Gradua\c{c}\~ao e \'Area de Concentra\c{c}\~ao em Ci\^encias da Engenharia Ambiental}
    }{
% MCEA ===========================================================================
\ifthenelse{\equal{#1}{MCEA}}{
	\tipotrabalho{Disserta\c{c}\~ao (Mestrado)}
	\tipotrabalhoabs{Dissertation (Master)}
	\area{Ci\^encias da Engenharia Ambiental}
	%\opcao{Nome da Op��o}
	% O preambulo deve conter o tipo do trabalho, o objetivo, 
	% o nome da institui��o, a �rea de concentra��o e op��o quando houver
	\preambulo{Disserta\c{c}\~ao apresentada \`a Escola de Engenharia de S\~ao Carlos da Universidade de S\~ao Paulo, para obten\c{c}\~ao do t\'itulo de Mestre em Ci\^encias - Programa de P\'os-Gradua\c{c}\~ao em Ci\^encias da Engenharia Ambiental.}
	\notaficha{Disserta\c{c}\~ao (Mestrado) - Programa de P\'os-Gradua\c{c}\~ao e \'Area de Concentra\c{c}\~ao em Ci\^encias da Engenharia Ambiental}
        }{
% DEE =======================================================================
\ifthenelse{\equal{#1}{DEE}}{
    \tipotrabalho{Tese (Doutorado)}
    \tipotrabalhoabs{Thesis (Doctor)}
    \area{Estruturas}
	%\opcao{Nome da Op��o}
    % O preambulo deve conter o tipo do trabalho, o objetivo, 
	% o nome da institui��o, a �rea de concentra��o e op��o quando houver
	\preambulo{Tese apresentada \`a Escola de Engenharia de S\~ao Carlos da Universidade de S\~ao Paulo, para obten\c{c}\~ao do t\'itulo de Doutor em Ci\^encias - Programa de P\'os-Gradua\c{c}\~ao em Engenharia Civil (Engenharia de Estruturas).}
	\notaficha{Tese (Doutorado) - Programa de P\'os-Gradua\c{c}\~ao em Engenharia Civil (Engenharia de Estruturas) e \'Area de Concentra\c{c}\~ao em Estruturas}
    }{
% MEE ===========================================================================
\ifthenelse{\equal{#1}{MEE}}{
	\tipotrabalho{Disserta\c{c}\~ao (Mestrado)}
	\tipotrabalhoabs{Dissertation (Master)}
	\area{Estruturas}
	%\opcao{Nome da Op��o}
	% O preambulo deve conter o tipo do trabalho, o objetivo, 
	% o nome da institui��o, a �rea de concentra��o e op��o quando houver
	\preambulo{Disserta\c{c}\~ao apresentada \`a Escola de Engenharia de S\~ao Carlos da Universidade de S\~ao Paulo, para obten\c{c}\~ao do t\'itulo de Mestre em Ci\^encias - Programa de P\'os-Gradua\c{c}\~ao em Engenharia Civil (Engenharia de Estruturas).}
	\notaficha{Disserta\c{c}\~ao (Mestrado) - Programa de P\'os-Gradua\c{c}\~ao em Engenharia Civil (Engenharia de Estruturas) e \'Area de Concentra\c{c}\~ao em Estruturas}
    }{
% DEPP ==========================================================================
\ifthenelse{\equal{#1}{DEPP}}{
    \tipotrabalho{Tese (Doutorado)}
    \tipotrabalhoabs{Thesis (Doctor)}
    \area{Processos e Gest\~ao de Opera\c{c}\~oes}
	%\opcao{Nome da Op��o}
    % O preambulo deve conter o tipo do trabalho, o objetivo, 
	% o nome da institui��o, a �rea de concentra��o e op��o quando houver
	\preambulo{Tese apresentada \`a Escola de Engenharia de S\~ao Carlos da Universidade de S\~ao Paulo, para obten\c{c}\~ao do t\'itulo de Doutor em Ci\^encias - Programa de P\'os-Gradua\c{c}\~ao em Engenharia de Produ\c{c}\~ao.}
	\notaficha{Tese (Doutorado) - Programa de P\'os-Gradua\c{c}\~ao em Engenharia de Produ\c{c}\~ao e \'Area de Concentra\c{c}\~ao em Processos e Gest\~ao de Opera\c{c}\~oes}
    }{
% MEPP ===========================================================================
\ifthenelse{\equal{#1}{MEPP}}{
	\tipotrabalho{Disserta\c{c}\~ao (Mestrado)}
	\tipotrabalhoabs{Dissertation (Master)}
	\area{Processos e Gest\~ao de Opera\c{c}\~oes}
	%\opcao{Nome da Op��o}
	% O preambulo deve conter o tipo do trabalho, o objetivo, 
	% o nome da institui��o, a �rea de concentra��o e op��o quando houver
	\preambulo{Disserta\c{c}\~ao apresentada \`a Escola de Engenharia de S\~ao Carlos da Universidade de S\~ao Paulo, para obten\c{c}\~ao do t\'itulo de Mestre em Ci\^encias - Programa de P\'os-Gradua\c{c}\~ao em Engenharia de Produ\c{c}\~ao.}
	\notaficha{Disserta\c{c}\~ao (Mestrado) - Programa de P\'os-Gradua\c{c}\~ao em Engenharia de Produ\c{c}\~ao e \'Area de Concentra\c{c}\~ao em Processos e Gest\~ao de Opera\c{c}\~oes}
    }{			
% DEPE ==========================================================================
\ifthenelse{\equal{#1}{DEPE}}{
    \tipotrabalho{Tese (Doutorado)}
    \tipotrabalhoabs{Thesis (Doctor)}
    \area{Economia, Organiza\c{c}\~oes e Gest\~ao do Conhecimento}
	%\opcao{Nome da Op��o}
    % O preambulo deve conter o tipo do trabalho, o objetivo, 
	% o nome da institui��o, a �rea de concentra��o e op��o quando houver
	\preambulo{Tese apresentada \`a Escola de Engenharia de S\~ao Carlos da Universidade de S\~ao Paulo, para obten\c{c}\~ao do t\'itulo de Doutor em Ci\^encias - Programa de P\'os-Gradua\c{c}\~ao em Engenharia de Produ\c{c}\~ao.}
	\notaficha{Tese (Doutorado) - Programa de P\'os-Gradua\c{c}\~ao em Engenharia de Produ\c{c}\~ao e \'Area de Concentra\c{c}\~ao em Economia, Organiza\c{c}\~oes e Gest\~ao do Conhecimento}
    }{
% MEPE ===========================================================================
\ifthenelse{\equal{#1}{MEPE}}{
	\tipotrabalho{Disserta\c{c}\~ao (Mestrado)}
	\tipotrabalhoabs{Dissertation (Master)}
	\area{Economia, Organiza\c{c}\~oes e Gest\~ao do Conhecimento}
	%\opcao{Nome da Op��o}
	% O preambulo deve conter o tipo do trabalho, o objetivo, 
	% o nome da institui��o, a �rea de concentra��o e op��o quando houver
	\preambulo{Disserta\c{c}\~ao apresentada \`a Escola de Engenharia de S\~ao Carlos da Universidade de S\~ao Paulo, para obten\c{c}\~ao do t\'itulo de Mestre em Ci\^encias - Programa de P\'os-Gradua\c{c}\~ao em Engenharia de Produ\c{c}\~ao.}
	\notaficha{Disserta\c{c}\~ao (Mestrado) - Programa de P\'os-Gradua\c{c}\~ao em Engenharia de Produ\c{c}\~ao e \'Area de Concentra\c{c}\~ao em Economia, Organiza\c{c}\~oes e Gest\~ao do Conhecimento}
    }{			
% DETI ==========================================================================
\ifthenelse{\equal{#1}{DETI}}{
    \tipotrabalho{Tese (Doutorado)}
    \tipotrabalhoabs{Thesis (Doctor)}
    \area{Infraestrutura de Transportes}
	%\opcao{Nome da Op��o}
    % O preambulo deve conter o tipo do trabalho, o objetivo, 
	% o nome da institui��o, a �rea de concentra��o e op��o quando houver
	\preambulo{Tese apresentada \`a Escola de Engenharia de S\~ao Carlos da Universidade de S\~ao Paulo, para obten\c{c}\~ao do t\'itulo de Doutor em Ci\^encias - Programa de P\'os-Gradua\c{c}\~ao em Engenharia de Transportes.}
	\notaficha{Tese (Doutorado) - Programa de P\'os-Gradua\c{c}\~ao em Engenharia de Transportes e \'Area de Concentra\c{c}\~ao em Infraestrutura de Transportes}
    }{
% METI ===========================================================================
\ifthenelse{\equal{#1}{METI}}{
	\tipotrabalho{Disserta\c{c}\~ao (Mestrado)}
	\tipotrabalhoabs{Dissertation (Master)}
	\area{Infraestrutura de Transportes}
	%\opcao{Nome da Op��o}
	% O preambulo deve conter o tipo do trabalho, o objetivo, 
	% o nome da institui��o, a �rea de concentra��o e op��o quando houver
	\preambulo{Disserta\c{c}\~ao apresentada \`a Escola de Engenharia de S\~ao Carlos da Universidade de S\~ao Paulo, para obten\c{c}\~ao do t\'itulo de Mestre em Ci\^encias - Programa de P\'os-Gradua\c{c}\~ao em Engenharia de Transportes.}
	\notaficha{Disserta\c{c}\~ao (Mestrado) - Programa de P\'os-Gradua\c{c}\~ao em Engenharia de Transportes e \'Area de Concentra\c{c}\~ao em Infraestrutura de Transportes}
    }{	   			
% DETT ==========================================================================
\ifthenelse{\equal{#1}{DETT}}{
    \tipotrabalho{Tese (Doutorado)}
    \tipotrabalhoabs{Thesis (Doctor)}
    \area{Transportes}
	%\opcao{Nome da Op��o}
    % O preambulo deve conter o tipo do trabalho, o objetivo, 
	% o nome da institui��o, a �rea de concentra��o e op��o quando houver
	\preambulo{Tese apresentada \`a Escola de Engenharia de S\~ao Carlos da Universidade de S\~ao Paulo, para obten\c{c}\~ao do t\'itulo de Doutor em Ci\^encias - Programa de P\'os-Gradua\c{c}\~ao em Engenharia de Transportes.}
	\notaficha{Tese (Doutorado) - Programa de P\'os-Gradua\c{c}\~ao em Engenharia de Transportes e \'Area de Concentra\c{c}\~ao em Transportes}
    }{
% METT ===========================================================================
\ifthenelse{\equal{#1}{METT}}{
	\tipotrabalho{Disserta\c{c}\~ao (Mestrado)}
	\tipotrabalhoabs{Dissertation (Master)}
	\area{Transportes}
	%\opcao{Nome da Op��o}
	% O preambulo deve conter o tipo do trabalho, o objetivo, 
	% o nome da institui��o, a �rea de concentra��o e op��o quando houver
	\preambulo{Disserta\c{c}\~ao apresentada \`a Escola de Engenharia de S\~ao Carlos da Universidade de S\~ao Paulo, para obten\c{c}\~ao do t\'itulo de Mestre em Ci\^encias - Programa de P\'os-Gradua\c{c}\~ao em Engenharia de Transportes.}
	\notaficha{Disserta\c{c}\~ao (Mestrado) - Programa de P\'os-Gradua\c{c}\~ao em Engenharia de Transportes e \'Area de Concentra\c{c}\~ao em Transportes}
    }{	  				
% DETP ==========================================================================
\ifthenelse{\equal{#1}{DETP}}{
    \tipotrabalho{Tese (Doutorado)}
    \tipotrabalhoabs{Thesis (Doctor)}
    \area{Planejamento e Opera\c{c}\~ao de Sistemas de Transporte}
	%\opcao{Nome da Op��o}
    % O preambulo deve conter o tipo do trabalho, o objetivo, 
	% o nome da institui��o, a �rea de concentra��o e op��o quando houver
	\preambulo{Tese apresentada \`a Escola de Engenharia de S\~ao Carlos da Universidade de S\~ao Paulo, para obten\c{c}\~ao do t\'itulo de Doutor em Ci\^encias - Programa de P\'os-Gradua\c{c}\~ao em Engenharia de Transportes.}
	\notaficha{Tese (Doutorado) - Programa de P\'os-Gradua\c{c}\~ao em Engenharia de Transportes e \'Area de Concentra\c{c}\~ao em Planejamento e Opera\c{c}\~ao de Sistemas de Transporte}
    }{
% METP ===========================================================================
\ifthenelse{\equal{#1}{METP}}{
	\tipotrabalho{Disserta\c{c}\~ao (Mestrado)}
	\tipotrabalhoabs{Dissertation (Master)}
	\area{Planejamento e Opera\c{c}\~ao de Sistemas de Transporte}
	%\opcao{Nome da Op��o}
	% O preambulo deve conter o tipo do trabalho, o objetivo, 
	% o nome da institui��o, a �rea de concentra��o e op��o quando houver
	\preambulo{Disserta\c{c}\~ao apresentada \`a Escola de Engenharia de S\~ao Carlos da Universidade de S\~ao Paulo, para obten\c{c}\~ao do t\'itulo de Mestre em Ci\^encias - Programa de P\'os-Gradua\c{c}\~ao em Engenharia de Transportes.}
	\notaficha{Disserta\c{c}\~ao (Mestrado) - Programa de P\'os-Gradua\c{c}\~ao em Engenharia de Transportes e \'Area de Concentra\c{c}\~ao em Planejamento e Opera\c{c}\~ao de Sistemas de Transporte}
    }{	    
% DEEP ==========================================================================
\ifthenelse{\equal{#1}{DEEP}}{
    \tipotrabalho{Tese (Doutorado)}
    \tipotrabalhoabs{Thesis (Doctor)}
    \area{Processamento de Sinais e Instrumenta\c{c}\~ao}
	%\opcao{Nome da Op��o}
	% O preambulo deve conter o tipo do trabalho, o objetivo, 
	% o nome da institui��o, a �rea de concentra��o e op��o quando houver
	\preambulo{Tese apresentada \`a Escola de Engenharia de S\~ao Carlos da Universidade de S\~ao Paulo, para obten\c{c}\~ao do t\'itulo de Doutor em Ci\^encias - Programa de P\'os-Gradua\c{c}\~ao em Engenharia El\'etrica.}
	\notaficha{Tese (Doutorado) - Programa de P\'os-Gradua\c{c}\~ao em Engenharia El\'etrica e \'Area de Concentra\c{c}\~ao em Processamento de Sinais e Instrumenta\c{c}\~ao}
    }{
% MEEP ===========================================================================
\ifthenelse{\equal{#1}{MEEP}}{
	\tipotrabalho{Disserta\c{c}\~ao (Mestrado)}
	\tipotrabalhoabs{Dissertation (Master)}
	\area{Processamento de Sinais e Instrumenta\c{c}\~ao}
	%\opcao{Nome da Op��o}
	% O preambulo deve conter o tipo do trabalho, o objetivo, 
	% o nome da institui��o, a �rea de concentra��o e op��o quando houver
	\preambulo{Disserta\c{c}\~ao apresentada \`a Escola de Engenharia de S\~ao Carlos da Universidade de S\~ao Paulo, para obten\c{c}\~ao do t\'itulo de Mestre em Ci\^encias - Programa de P\'os-Gradua\c{c}\~ao em Engenharia El\'etrica.}
	\notaficha{Disserta\c{c}\~ao (Mestrado) - Programa de P\'os-Gradua\c{c}\~ao em Engenharia El\'etrica e \'Area de Concentra\c{c}\~ao em Processamento de Sinais e Instrumenta\c{c}\~ao}
    }{	  
% DEED ==========================================================================
\ifthenelse{\equal{#1}{DEED}}{
    \tipotrabalho{Tese (Doutorado)}
    \tipotrabalhoabs{Thesis (Doctor)}
    \area{Sistemas Din\^amicos}
	%\opcao{Nome da Op��o}
    % O preambulo deve conter o tipo do trabalho, o objetivo, 
	% o nome da institui��o, a �rea de concentra��o e op��o quando houver
	\preambulo{Tese apresentada \`a Escola de Engenharia de S\~ao Carlos da Universidade de S\~ao Paulo, para obten\c{c}\~ao do t\'itulo de Doutor em Ci\^encias - Programa de P\'os-Gradua\c{c}\~ao em Engenharia El\'etrica.}
	\notaficha{Tese (Doutorado) - Programa de P\'os-Gradua\c{c}\~ao em Engenharia El\'etrica e \'Area de Concentra\c{c}\~ao em Sistemas Din\^amicos}
    }{
% MEED ===========================================================================
\ifthenelse{\equal{#1}{MEED}}{
	\tipotrabalho{Disserta\c{c}\~ao (Mestrado)}
	\tipotrabalhoabs{Dissertation (Master)}
	\area{Sistemas Din\^amicos}
	%\opcao{Nome da Op��o}
	% O preambulo deve conter o tipo do trabalho, o objetivo, 
	% o nome da institui��o, a �rea de concentra��o e op��o quando houver
	\preambulo{Disserta\c{c}\~ao apresentada \`a Escola de Engenharia de S\~ao Carlos da Universidade de S\~ao Paulo, para obten\c{c}\~ao do t\'itulo de Mestre em Ci\^encias - Programa de P\'os-Gradua\c{c}\~ao em Engenharia El\'etrica.}
	\notaficha{Disserta\c{c}\~ao (Mestrado) - Programa de P\'os-Gradua\c{c}\~ao em Engenharia El\'etrica e \'Area de Concentra\c{c}\~ao em Sistemas Din\^amicos}
    }{	  
% DEEE ==========================================================================
\ifthenelse{\equal{#1}{DEEE}}{
    \tipotrabalho{Tese (Doutorado)}
    \tipotrabalhoabs{Thesis (Doctor)}
    \area{Sistemas El\'etricos de Pot\^encia}
	%\opcao{Nome da Op��o}
    % O preambulo deve conter o tipo do trabalho, o objetivo, 
	% o nome da institui��o, a �rea de concentra��o e op��o quando houver
	\preambulo{Tese apresentada \`a Escola de Engenharia de S\~ao Carlos da Universidade de S\~ao Paulo, para obten\c{c}\~ao do t\'itulo de Doutor em Ci\^encias - Programa de P\'os-Gradua\c{c}\~ao em Engenharia El\'etrica.}
	\notaficha{Tese (Doutorado) - Programa de P\'os-Gradua\c{c}\~ao em Engenharia El\'etrica e \'Area de Concentra\c{c}\~ao em Sistemas El\'etricos de Pot\^encia}
    }{
% MEEE ===========================================================================
\ifthenelse{\equal{#1}{MEEE}}{
    \tipotrabalho{Disserta\c{c}\~ao (Mestrado)}
    \tipotrabalhoabs{Dissertation (Master)}
    \area{Sistemas El\'etricos de Pot\^encia}
	%\opcao{Nome da Op��o}
    % O preambulo deve conter o tipo do trabalho, o objetivo, 
	% o nome da institui��o, a �rea de concentra��o e op��o quando houver
	\preambulo{Disserta\c{c}\~ao apresentada \`a Escola de Engenharia de S\~ao Carlos da Universidade de S\~ao Paulo, para obten\c{c}\~ao do t\'itulo de Mestre em Ci\^encias - Programa de P\'os-Gradua\c{c}\~ao em Engenharia El\'etrica.}
	\notaficha{Disserta\c{c}\~ao (Mestrado) - Programa de P\'os-Gradua\c{c}\~ao em Engenharia El\'etrica e \'Area de Concentra\c{c}\~ao em Sistemas El\'etricos de Pot\^encia}
    }{	  
% DEET ==========================================================================
\ifthenelse{\equal{#1}{DEET}}{
    \tipotrabalho{Tese (Doutorado)}
    \tipotrabalhoabs{Thesis (Doctor)}
    \area{Telecomunica\c{c}\~oes}
	%\opcao{Nome da Op��o}
    % O preambulo deve conter o tipo do trabalho, o objetivo, 
	% o nome da institui��o, a �rea de concentra��o e op��o quando houver
	\preambulo{Tese apresentada \`a Escola de Engenharia de S\~ao Carlos da Universidade de S\~ao Paulo, para obten\c{c}\~ao do t\'itulo de Doutor em Ci\^encias - Programa de P\'os-Gradua\c{c}\~ao em Engenharia El\'etrica.}
	\notaficha{Tese (Doutorado) - Programa de P\'os-Gradua\c{c}\~ao em Engenharia El\'etrica e \'Area de Concentra\c{c}\~ao em Telecomunica\c{c}\~oes}
    }{
% MEET ===========================================================================
\ifthenelse{\equal{#1}{MEET}}{
	\tipotrabalho{Disserta\c{c}\~ao (Mestrado)}
	\area{Telecomunica\c{c}\~oes}
	%\opcao{Nome da Op��o}
	% O preambulo deve conter o tipo do trabalho, o objetivo, 
	% o nome da institui��o, a �rea de concentra��o e op��o quando houver
	\preambulo{Disserta\c{c}\~ao apresentada \`a Escola de Engenharia de S\~ao Carlos da Universidade de S\~ao Paulo, para obten\c{c}\~ao do t\'itulo de Mestre em Ci\^encias - Programa de P\'os-Gradua\c{c}\~ao em Engenharia El\'etrica.}
	\notaficha{Disserta\c{c}\~ao (Mestrado) - Programa de P\'os-Gradua\c{c}\~ao em Engenharia El\'etrica e \'Area de Concentra\c{c}\~ao em Telecomunica\c{c}\~oes}
    }{	
% DEHS ==========================================================================
\ifthenelse{\equal{#1}{DEHS}}{
    \tipotrabalho{Tese (Doutorado)}
    \tipotrabalhoabs{Thesis (Doctor)}
    \area{Hidr\'aulica e Saneamento}
	%\opcao{Nome da Op��o}
    % O preambulo deve conter o tipo do trabalho, o objetivo, 
	% o nome da institui��o, a �rea de concentra��o e op��o quando houver
	\preambulo{Tese apresentada \`a Escola de Engenharia de S\~ao Carlos da Universidade de S\~ao Paulo, para obten\c{c}\~ao do t\'itulo de Doutor em Ci\^encias - Programa de P\'os-Gradua\c{c}\~ao em Engenharia Hidr\'aulica e Saneamento.}
	\notaficha{Tese (Doutorado) - Programa de P\'os-Gradua\c{c}\~ao e \'Area de Concentra\c{c}\~ao em Engenharia Hidr\'aulica e Saneamento}
    }{
% MEHS ===========================================================================
\ifthenelse{\equal{#1}{MEHS}}{
	\tipotrabalho{Disserta\c{c}\~ao (Mestrado)}
	\tipotrabalhoabs{Dissertation (Master)}
	\area{Hidr\'aulica e Saneamento}
	%\opcao{Nome da Op��o}
	% O preambulo deve conter o tipo do trabalho, o objetivo, 
	% o nome da institui��o, a �rea de concentra��o e op��o quando houver
	\preambulo{Disserta\c{c}\~ao apresentada \`a Escola de Engenharia de S\~ao Carlos da Universidade de S\~ao Paulo, para obten\c{c}\~ao do t\'itulo de Mestre em Ci\^encias - Programa de P\'os-Gradua\c{c}\~ao em Hidr\'aulica e Saneamento.}
	\notaficha{Disserta\c{c}\~ao (Mestrado) - Programa de P\'os-Gradua\c{c}\~ao e \'Area de Concentra\c{c}\~ao em Engenharia Hidr\'aulica e Saneamento} 
    }{	
% DEMA ==========================================================================
\ifthenelse{\equal{#1}{DEMA}}{
    \tipotrabalho{Tese (Doutorado)}
    \tipotrabalhoabs{Thesis (Doctor)}
    \area{Aerona\'utica}
	%\opcao{Nome da Op��o}
    % O preambulo deve conter o tipo do trabalho, o objetivo, 
	% o nome da institui��o, a �rea de concentra��o e op��o quando houver
	\preambulo{Tese apresentada \`a Escola de Engenharia de S\~ao Carlos da Universidade de S\~ao Paulo, para obten\c{c}\~ao do t\'itulo de Doutor em Ci\^encias - Programa de P\'os-Gradua\c{c}\~ao em Engenharia Mec\^anica.}
	\notaficha{Tese (Doutorado) - Programa de P\'os-Gradua\c{c}\~ao em Engenharia Mec\^anica e \'Area de Concentra\c{c}\~ao em Aerona\'utica}
    }{
% MEMA ===========================================================================
\ifthenelse{\equal{#1}{MEMA}}{
	\tipotrabalho{Disserta\c{c}\~ao (Mestrado)}
	\tipotrabalhoabs{Dissertation (Master)}
	\area{Aerona\'utica}
	%\opcao{Nome da Op��o}
	% O preambulo deve conter o tipo do trabalho, o objetivo, 
	% o nome da institui��o, a �rea de concentra��o e op��o quando houver
	\preambulo{Disserta\c{c}\~ao apresentada \`a Escola de Engenharia de S\~ao Carlos da Universidade de S\~ao Paulo, para obten\c{c}\~ao do t\'itulo de Mestre em Ci\^encias - Programa de P\'os-Gradua\c{c}\~ao em Engenharia Mec\^anica.}
	\notaficha{Disserta\c{c}\~ao (Mestrado) - Programa de P\'os-Gradua\c{c}\~ao em Engenharia Mec\^anica e \'Area de Concentra\c{c}\~ao em Aerona\'utica}
    }{	
% DEMD ==========================================================================
\ifthenelse{\equal{#1}{DEMD}}{
    \tipotrabalho{Tese (Doutorado)}
    \tipotrabalhoabs{Thesis (Doctor)}
    \area{Din\^amica e Mecatr\^onica}
	%\opcao{Nome da Op��o}
    % O preambulo deve conter o tipo do trabalho, o objetivo, 
	% o nome da institui��o, a �rea de concentra��o e op��o quando houver
	\preambulo{Tese apresentada \`a Escola de Engenharia de S\~ao Carlos da Universidade de S\~ao Paulo, para obten\c{c}\~ao do t\'itulo de Doutor em Ci\^encias - Programa de P\'os-Gradua\c{c}\~ao em Engenharia Mec\^anica.}
	\notaficha{Tese (Doutorado) - Programa de P\'os-Gradua\c{c}\~ao em Engenharia Mec\^anica e \'Area de Concentra\c{c}\~ao em Din\^amica e Mecatr\^onica}
    }{
% MEMD ===========================================================================
\ifthenelse{\equal{#1}{MEMD}}{
	\tipotrabalho{Disserta\c{c}\~ao (Mestrado)}
	\tipotrabalhoabs{Dissertation (Master)}
	\area{Din\^amica e Mecatr\^onica}
	%\opcao{Nome da Op��o}
	% O preambulo deve conter o tipo do trabalho, o objetivo, 
	% o nome da institui��o, a �rea de concentra��o e op��o quando houver
	\preambulo{Disserta\c{c}\~ao apresentada \`a Escola de Engenharia de S\~ao Carlos da Universidade de S\~ao Paulo, para obten\c{c}\~ao do t\'itulo de Mestre em Ci\^encias - Programa de P\'os-Gradua\c{c}\~ao em Engenharia Mec\^anica.}
	\notaficha{Disserta\c{c}\~ao (Mestrado) - Programa de P\'os-Gradua\c{c}\~ao em Engenharia Mec\^anica e \'Area de Concentra\c{c}\~ao em Din\^amica e Mecatr\^onica}
    }{	
% DEMF ==========================================================================
\ifthenelse{\equal{#1}{DEMF}}{
    \tipotrabalho{Tese (Doutorado)}
    \tipotrabalhoabs{Thesis (Doctor)}
    \area{Projeto, Materiais e Manufatura}
	%\opcao{Nome da Op��o}
    % O preambulo deve conter o tipo do trabalho, o objetivo, 
	% o nome da institui��o, a �rea de concentra��o e op��o quando houver
	\preambulo{Tese apresentada \`a Escola de Engenharia de S\~ao Carlos da Universidade de S\~ao Paulo, para obten\c{c}\~ao do t\'itulo de Doutor em Ci\^encias - Programa de P\'os-Gradua\c{c}\~ao em Engenharia Mec\^anica.}
	\notaficha{Tese (Doutorado) - Programa de P\'os-Gradua\c{c}\~ao em Engenharia Mec\^anica e \'Area de Concentra\c{c}\~ao em Projeto, Materiais e Manufatura}
    }{
% MEMF ===========================================================================
\ifthenelse{\equal{#1}{MEMF}}{
	\tipotrabalho{Disserta\c{c}\~ao (Mestrado)}
	\tipotrabalhoabs{Dissertation (Master)}
	\area{Projeto, Materiais e Manufatura}
	%\opcao{Nome da Op��o}
	% O preambulo deve conter o tipo do trabalho, o objetivo, 
	% o nome da institui��o, a �rea de concentra��o e op��o quando houver
	\preambulo{Disserta\c{c}\~ao apresentada \`a Escola de Engenharia de S\~ao Carlos da Universidade de S\~ao Paulo, para obten\c{c}\~ao do t\'itulo de Mestre em Ci\^encias - Programa de P\'os-Gradua\c{c}\~ao em Engenharia Mec\^anica.}
	\notaficha{Disserta\c{c}\~ao (Mestrado) - Programa de P\'os-Gradua\c{c}\~ao em Engenharia Mec\^anica e \'Area de Concentra\c{c}\~ao em Projeto, Materiais e Manufatura}
    }{	
% DEMT ==========================================================================
\ifthenelse{\equal{#1}{DEMT}}{
    \tipotrabalho{Tese (Doutorado)}
    \tipotrabalhoabs{Thesis (Doctor)}
    \area{Termoci\^encias e Mec\^anica de Fluidos}
	%\opcao{Nome da Op��o}
    % O preambulo deve conter o tipo do trabalho, o objetivo, 
	% o nome da institui��o, a �rea de concentra��o e op��o quando houver
	\preambulo{Tese apresentada \`a Escola de Engenharia de S\~ao Carlos da Universidade de S\~ao Paulo, para obten\c{c}\~ao do t\'itulo de Doutor em Ci\^encias - Programa de P\'os-Gradua\c{c}\~ao em Engenharia Mec\^anica.}
	\notaficha{Tese (Doutorado) - Programa de P\'os-Gradua\c{c}\~ao em Engenharia Mec\^anica e \'Area de Concentra\c{c}\~ao em Termoci\^encias e Mec\^anica de Fluidos}
    }{
% MEMT ===========================================================================
\ifthenelse{\equal{#1}{MEMT}}{
	\tipotrabalho{Disserta\c{c}\~ao (Mestrado)}
	\tipotrabalhoabs{Dissertation (Master)}
	\area{Termoci\^encias e Mec\^anica de Fluidos}
	%\opcao{Nome da Op��o}
	% O preambulo deve conter o tipo do trabalho, o objetivo, 
	% o nome da institui��o, a �rea de concentra��o e op��o quando houver
	\preambulo{Disserta\c{c}\~ao apresentada \`a Escola de Engenharia de S\~ao Carlos da Universidade de S\~ao Paulo, para obten\c{c}\~ao do t\'itulo de Mestre em Ci\^encias - Programa de P\'os-Gradua\c{c}\~ao em Engenharia Mec\^anica.}
	\notaficha{Disserta\c{c}\~ao (Mestrado) - Programa de P\'os-Gradua\c{c}\~ao em Engenharia Mec\^anica e \'Area de Concentra\c{c}\~ao em Termoci\^encias e Mec\^anica de Fluidos}
    }{	
% DCEM ==========================================================================
\ifthenelse{\equal{#1}{DCEM}}{
    \tipotrabalho{Tese (Doutorado)}
    \tipotrabalhoabs{Thesis (Doctor)}
    \area{Caracteriza\c{c}\~ao, Desenvolvimento e Aplica\c{c}\~ao de Materiais}
	%\opcao{Nome da Op��o}
    % O preambulo deve conter o tipo do trabalho, o objetivo, 
	% o nome da institui��o, a �rea de concentra��o e op��o quando houver
	\preambulo{Tese apresentada \`a Escola de Engenharia de S\~ao Carlos da Universidade de S\~ao Paulo, para obten\c{c}\~ao do t\'itulo de Doutor em Ci\^encias - Programa de P\'os-Gradua\c{c}\~ao em Ci\^encia e Engenharia de Materiais.}
	\notaficha{Tese (Doutorado) - Programa de P\'os-Gradua\c{c}\~ao em Ci\^encias e Engenharia de Materiais e \'Area de Concentra\c{c}\~ao em Desenvolvimento, Caracteriza\c{c}\~ao e Aplica\c{c}\~ao de Materiais}
    }{
% MCEM ===========================================================================
\ifthenelse{\equal{#1}{MCEM}}{
	\tipotrabalho{Disserta\c{c}\~ao (Mestrado)}
	\tipotrabalhoabs{Dissertation (Master)}
	\area{Caracteriza\c{c}\~ao, Desenvolvimento e Aplica\c{c}\~ao de Materiais}
	%\opcao{Nome da Op��o}
	% O preambulo deve conter o tipo do trabalho, o objetivo, 
	% o nome da institui��o, a �rea de concentra��o e op��o quando houver
	\preambulo{Disserta\c{c}\~ao apresentada \`a Escola de Engenharia de S\~ao Carlos da Universidade de S\~ao Paulo, para obten\c{c}\~ao do t\'itulo de Mestre em Ci\^encias - Programa de P\'os-Gradua\c{c}\~ao em Ci\^encia e Engenharia de Materiais.}
	\notaficha{Disserta\c{c}\~ao (Mestrado) - Programa de P\'os-Gradua\c{c}\~ao em Ci\^encias e Engenharia de Materiais e \'Area de Concentra\c{c}\~ao em Desenvolvimento, Caracteriza\c{c}\~ao e Aplica\c{c}\~ao de Materiais}
    }{	
% DGEO ==========================================================================
\ifthenelse{\equal{#1}{DGEO}}{
    \tipotrabalho{Tese (Doutorado)}
    \tipotrabalhoabs{Thesis (Doctor)}
    \area{Geotecnia}
	%\opcao{Nome da Op��o}
    % O preambulo deve conter o tipo do trabalho, o objetivo, 
	% o nome da institui��o, a �rea de concentra��o e op��o quando houver
	\preambulo{Tese apresentada \`a Escola de Engenharia de S\~ao Carlos da Universidade de S\~ao Paulo, para obten\c{c}\~ao do t\'itulo de Doutor em Ci\^encias - Programa de P\'os-Gradua\c{c}\~ao em Geotecnia.}
	\notaficha{Tese (Doutorado) - Programa de P\'os-Gradua\c{c}\~ao e \'Area de Concentra\c{c}\~ao em Geotecnia}
    }{
% MGEO ===========================================================================
\ifthenelse{\equal{#1}{MGEO}}{
	\tipotrabalho{Disserta\c{c}\~ao (Mestrado)}
	\tipotrabalhoabs{Dissertation (Master)}
	\area{Geotecnia}
	%\opcao{Nome da Op��o}
	% O preambulo deve conter o tipo do trabalho, o objetivo, 
	% o nome da institui��o, a �rea de concentra��o e op��o quando houver
	\preambulo{Disserta\c{c}\~ao apresentada \`a Escola de Engenharia de S\~ao Carlos da Universidade de S\~ao Paulo, para obten\c{c}\~ao do t\'itulo de Mestre em Ci\^encias - Programa de P\'os-Gradua\c{c}\~ao em Geotecnia.}
	\notaficha{Disserta\c{c}\~ao (Mestrado) - Programa de P\'os-Gradua\c{c}\~ao e \'Area de Concentra\c{c}\~ao em Geotecnia}
    }{	
% DIUB ==========================================================================
\ifthenelse{\equal{#1}{DIUB}}{
    \tipotrabalho{Tese (Doutorado)}
    \tipotrabalhoabs{Thesis (Doctor)}
    \area{Bioengenharia}
	%\opcao{Nome da Op��o}
    % O preambulo deve conter o tipo do trabalho, o objetivo, 
	% o nome da institui��o, a �rea de concentra��o e op��o quando houver
	\preambulo{Tese apresentada \`a Escola de Engenharia de S\~ao Carlos da Universidade de S\~ao Paulo, para obten\c{c}\~ao do t\'itulo de Doutor em Ci\^encias - Programa de P\'os-Gradua\c{c}\~ao Interunidades em Bioengenharia.}
	\notaficha{Tese (Doutorado) - Programa de P\'os-Gradua\c{c}\~ao Interunidades em Bioengenharia e \'Area de Concentra\c{c}\~ao em Bioengenharia}
    }{
% MIUB ===========================================================================
\ifthenelse{\equal{#1}{MIUB}}{
	\tipotrabalho{Disserta\c{c}\~ao (Mestrado)}
	\tipotrabalhoabs{Dissertation (Master)}
	\area{Bioengenharia}
	%\opcao{Nome da Op��o}
	% O preambulo deve conter o tipo do trabalho, o objetivo, 
	% o nome da institui��o, a �rea de concentra��o e op��o quando houver
	\preambulo{Disserta\c{c}\~ao apresentada \`a Escola de Engenharia de S\~ao Carlos da Universidade de S\~ao Paulo, para obten\c{c}\~ao do t\'itulo de Mestre em Ci\^encias - Programa de P\'os-Gradua\c{c}\~ao Interunidades em Bioengenharia.}
	\notaficha{Disserta\c{c}\~ao (Mestrado) - Programa de P\'os-Gradua\c{c}\~ao Interunidades em Bioengenharia e \'Area de Concentra\c{c}\~ao em Bioengenharia}
    }{	
% MRNECA ===========================================================================
\ifthenelse{\equal{#1}{MRNECA}}{
	\tipotrabalho{Disserta\c{c}\~ao (Mestrado)}
	\tipotrabalhoabs{Dissertation (Master)}
	\area{Ensino de Ci\^encias Ambientais}
	%\opcao{Nome da Op��o}
	% O preambulo deve conter o tipo do trabalho, o objetivo, 
	% o nome da institui��o, a �rea de concentra��o e op��o quando houver
	\preambulo{Disserta\c{c}\~ao apresentada \`a Escola de Engenharia de S\~ao Carlos da Universidade de S\~ao Paulo, para obten\c{c}\~ao do t\'itulo de Mestre em Ci\^encias - Programa de P\'os-Gradua\c{c}\~ao em Rede Nacional para Ensino das Ci\^encias Ambientais.}
	\notaficha{Disserta\c{c}\~ao (Mestrado) - Programa de P\'os-Gradua\c{c}\~ao em Rede Nacional para Ensino das Ci\^encias Ambientais e \'Area de Concentra\c{c}\~ao em Ensino de Ci\^encias Ambientais}
    }{	         
% Outros
	\tipotrabalho{Disserta\c{c}\~ao/Tese (Mestrado/Doutorado)}
	\tipotrabalhoabs{Dissertation/Thesis (Master/Doctor)}
    \area{Nome da \'Area}
    \opcao{Nome da Op\c{c}\~ao}
    % O preambulo deve conter o tipo do trabalho, o objetivo, 
	% o nome da institui��o, a �rea de concentra��o e op��o quando houver
	\preambulo{Disserta\c{c}\~ao/Tese apresentada \`a Escola de Engenharia de S\~ao Carlos da Universidade de S\~ao Paulo, para obten\c{c}\~ao do t\'itulo de Mestre/Doutor em Ci\^encias - Programa de P\'os-Gradua\c{c}\~ao em Engenharia.}
	\notaficha{Disserta\c{c}\~ao/Tese (Mestrado/Doutorado) - Programa de P\'os-Gradua\c{c}\~ao e \'Area de Concentra\c{c}\~ao em Engenharia}		
    }}}}}}}}}}}}}}}}}}}}}
    }}}}}}}}}}}}}}}}}}}
	% ---
}{    
% IAU ===========================================================================
        \ifthenelse{\equal{#1}{IAU}}{
        %% USPSC-pre-textual-IAU.tex
%% Camandos para defini��o do tipo de documento (tese ou disserta��o), �rea de concentra��o, op��o, pre�mbulo, titula��o 
%% referentes ao Programa de P�s-Gradua��o o IQSC
\instituicao{Instituto de Arquitetura e Urbanismo, Universidade de S\~ao Paulo}
\unidade{INSTITUTO DE ARQUITETURA E URBANISMO}
\unidademin{Instituto de Arquitetura e Urbanismo}
\universidademin{Universidade de S\~ao Paulo}

\notafolharosto{Vers\~ao original}
%Para vers�o original em ingl�s, comente do comando/declara��o 
%     acima(inclua % antes do comando acima) e tire a % do 
%     comando/declara��o abaixo no idioma do texto
%\notafolharosto{Original version} 
%Para vers�o corrigida, comente do comando/declara��o da 
%     vers�o original acima (inclua % antes do comando acima) 
%     e tire a % do comando/declara��o de um dos comandos 
%     abaixo em conformidade com o idioma do texto
%\notafolharosto{Vers\~ao corrigida \\(Vers\~ao original dispon\'ivel na Unidade que aloja o Programa)}
%\notafolharosto{Corrected version \\(Original version available on the Program Unit)}

% ---
% dados complementares para CAPA e FOLHA DE ROSTO
% ---
\universidade{UNIVERSIDADE DE S\~AO PAULO}
\titulo{Modelo para teses e disserta\c{c}\~oes em \LaTeX\ utilizando o Pacote USPSC para o IAU}
\titleabstract{Model for thesis and dissertations in \LaTeX\ using the USPSC Package to the IAU}
\tituloresumo{Modelo para teses e disserta\c{c}\~oes em \LaTeX\ utilizando o Pacote USPSC para o IAU}
\autor{Jos\'e da Silva}
\autorficha{Silva, Jos\'e da}
\autorabr{SILVA, J.}

\cutter{S856m}
% Para gerar a ficha catalogr�fica sem o C�digo Cutter, basta 
% incluir uma % na linha acima e tirar a % da linha abaixo
%\cutter{ }

\local{S\~ao Carlos}
\data{2021}
% Quando for Orientador, basta incluir uma % antes do comando abaixo
\renewcommand{\orientadorname}{Orientadora:}
% Quando for Coorientadora, basta tirar a % utilizar o comando abaixo
%\newcommand{\coorientadorname}{Coorientador:}
\orientador{Profa. Dra. Elisa Gon\c{c}alves Rodrigues}
\orientadorcorpoficha{orientadora Elisa Gon\c{c}alves Rodrigues}
\orientadorficha{Rodrigues, Elisa Gon\c{c}alves, orient}
%Se houver co-orientador, inclua % antes das duas linhas (antes dos comandos \orientadorcorpoficha e \orientadorficha) 
%          e tire a % antes dos 3 comandos abaixo
%\coorientador{Prof. Dr. Jo\~ao Alves Serqueira}
%\orientadorcorpoficha{orientadora Elisa Gon\c{c}alves Rodrigues ;  co-orientador Jo\~ao Alves Serqueira}
%\orientadorficha{Rodrigues, Elisa Gon\c{c}alves, orient. II. Serqueira, Jo\~ao Alves, co-orient}

\notaautorizacao{AUTORIZO A REPRODU\c{C}\~AO E DIVULGA\c{C}\~AO TOTAL OU PARCIAL DESTE TRABALHO, POR QUALQUER MEIO CONVENCIONAL OU ELETR\^ONICO PARA FINS DE ESTUDO E PESQUISA, DESDE QUE CITADA A FONTE.}
\notabib{Ficha catalogr\'afica elaborada pela Biblioteca do Instituto de Arquitetura e Urbanismo - IAU/USP, com os dados fornecidos pelo(a) autor(a)}

\newcommand{\programa}[1]{

% DAUT ==========================================================================
\ifthenelse{\equal{#1}{DAUT}}{
    \area{Arquitetura, Urbanismo e Tecnologia}
	\tipotrabalho{Tese (Doutorado)}
	\tipotrabalhoabs{Thesis (Doctor)}
	%\opcao{Nome da Op��o}
    % O preambulo deve conter o tipo do trabalho, o objetivo, 
	% o nome da institui��o e a �rea de concentra��o 
	\preambulo{Tese apresentada ao Programa de P\'os-Gradua\c{c}\~ao em Arquitetura e Urbanismo do Instituto de Arquitetura e Urbanismo, Universidade de S\~ao Paulo, como parte dos requisitos para a obten\c{c}\~ao do t\'itulo de Doutor em Arquitetura e Urbanismo.}
	\notaficha{Tese (Doutorado - Programa de P\'os-Gradua\c{c}\~ao em Arquitetura e Urbanismo e \'Area de Concentra\c{c}\~ao em ~\imprimirarea)}
    }{
% MAUT ===========================================================================
\ifthenelse{\equal{#1}{MAUT}}{
    \area{Arquitetura, Urbanismo e Tecnologia}
	\tipotrabalho{Disserta\c{c}\~ao (Mestrado)}
	\tipotrabalhoabs{Dissertation (Master)}
	%\opcao{Nome da Op��o}
    % O preambulo deve conter o tipo do trabalho, o objetivo, 
	% o nome da institui��o e a �rea de concentra��o 
	\preambulo{Disserta\c{c}\~ao apresentada ao Programa de P\'os-Gradua\c{c}\~ao em Arquitetura e Urbanismo do Instituto de Arquitetura e Urbanismo, Universidade de S\~ao Paulo, como parte dos requisitos para a obten\c{c}\~ao do t\'itulo de Mestre em Arquitetura e Urbanismo.}
	\notaficha{Disserta\c{c}\~ao (Mestrado - Programa de P\'os-Gradua\c{c}\~ao em Arquitetura e Urbanismo e \'Area de Concentra\c{c}\~ao em ~\imprimirarea)}
    }{
% DAUH ==========================================================================
\ifthenelse{\equal{#1}{DAUH}}{
    \area{Teoria e Hist\'oria da Arquitetura e do Urbanismo}
	\tipotrabalho{Tese (Doutorado)}
	\tipotrabalhoabs{Thesis (Doctor)}
	%\opcao{Nome da Op��o}
    % O preambulo deve conter o tipo do trabalho, o objetivo, 
	% o nome da institui��o e a �rea de concentra��o 
	\preambulo{Tese apresentada ao Programa de P\'os-Gradua\c{c}\~ao em Arquitetura e Urbanismo do Instituto de Arquitetura e Urbanismo, Universidade de S\~ao Paulo, como parte dos requisitos para a obten\c{c}\~ao do t\'itulo de Doutor em Arquitetura e Urbanismo.}
	\notaficha{Tese (Doutorado - Programa de P\'os-Gradua\c{c}\~ao em Arquitetura e Urbanismo e \'Area de Concentra\c{c}\~ao em ~\imprimirarea)}
    }{
% MAUH ===========================================================================
\ifthenelse{\equal{#1}{MAUH}}{
    \area{Teoria e Hist\'oria da Arquitetura e do Urbanismo}
	\tipotrabalho{Disserta\c{c}\~ao (Mestrado)}
	\tipotrabalhoabs{Dissertation (Master)}
	%\opcao{Nome da Op��o}
    % O preambulo deve conter o tipo do trabalho, o objetivo, 
	% o nome da institui��o e a �rea de concentra��o 
	\preambulo{Disserta\c{c}\~ao apresentada ao Programa de P\'os-Gradua\c{c}\~ao em Arquitetura e Urbanismo do Instituto de Arquitetura e Urbanismo, Universidade de S\~ao Paulo, como parte dos requisitos para a obten\c{c}\~ao do t\'itulo de Mestre em Arquitetura e Urbanismo.}
	\notaficha{Disserta\c{c}\~ao (Mestrado - Programa de P\'os-Gradua\c{c}\~ao em Arquitetura e Urbanismo e \'Area de Concentra\c{c}\~ao em ~\imprimirarea)}
    }{
% Outros
	\tipotrabalho{Disserta\c{c}\~ao/Tese (Mestrado/Doutorado)}
	\tipotrabalhoabs{Dissertation/Thesis (Master/Doctor)}
	\area{Nome da \'Area}
	\opcao{Nome da Op\c{c}\~ao}
	% O preambulo deve conter o tipo do trabalho, o objetivo, 
	% o nome da institui��o e a �rea de concentra��o 
	\preambulo{Disserta\c{c}\~ao/Tese apresentada ao Programa de P\'os-Gradua\c{c}\~ao em Arquitetura e Urbanismo do Instituto de Arquitetura e Urbanismo, Universidade de S\~ao Paulo, como parte dos requisitos para a obten\c{c}\~ao do t\'itulo de Mestre/Doutor em Arquitetura e Urbanismo.}
	\notaficha{Disserta\c{c}\~ao/Tese (Mestrado/Doutorado - Programa de P\'os-Gradua\c{c}\~ao em Arquitetura e Urbanismo e \'Area de Concentra\c{c}\~ao em ~\imprimirarea)}
    }}}}}
				    
        }{
% ICMC ===========================================================================
        \ifthenelse{\equal{#1}{ICMC}}{
        %% USPSC-pre-textual-ICMC.tex
%% Camandos para defini��o do tipo de documento (tese ou disserta��o ou monografia), �rea de concentra��o, op��o, pre�mbulo, titula��o 
%% referentes ao Programa de P�s-Gradua��o o ICMC
\instituicao{Instituto de Ci\^encias Matem\'aticas e de Computa\c{c}\~ao, Universidade de S\~ao Paulo}
\unidade{INSTITUTO DE CI\^ENCIAS MATEM\'ATICAS E DE COMPUTA\c{C}\~AO}
\unidademin{Instituto de Ci\^encias Matem\'aticas e de Computa\c{c}\~ao}
\universidademin{Universidade de S\~ao Paulo}
\setorpos{SERVI\c{C}O DE P\'OS-GRADUA\c{C}\~AO DO ICMC-USP}

\notafolharosto{Vers\~ao original}
\notafolharostoadic{Original version}
%Para vers�o original em ingl�s, comente os comandos/declara��es acima (inclua % antes do comando acima) 
% e tire a % dos comandos/declara��es abaixo no idioma do texto
%\notafolharosto{Original version}
%\notafolharostoadic{Vers\~ao original}
 
%Para vers�o revisada, comente os comandos/declara��es acima (inclua % antes do comando acima) 
% e tire a % dos comandos/declara��es abaixo, em conformidade com o idioma do texto
% Se o Idioma do texto for portugu�s: 
%\notafolharosto{Vers\~ao revisada}
%\notafolharosto{Final version}
% Se o Idioma do texto for Ingl�s: 
%\notafolharosto{Final version}
%\notafolharosto{Vers\~ao revisada}
% ---
% dados complementares para CAPA e FOLHA DE ROSTO
% ---
\universidade{UNIVERSIDADE DE S\~AO PAULO}

% Idioma do texto em PORTUGU�S
\titulo{Modelo para teses e disserta\c{c}\~oes em \LaTeX\ utilizando o Pacote USPSC para o ICMC} 
\titleabstract{Model for thesis and dissertations in \LaTeX\ using the USPSC Package to the ICMC}
\tituloadic{Model for thesis and dissertations in \LaTeX\ using the USPSC Package to the ICMC}
\tituloresumo{Modelo para teses e disserta\c{c}\~oes em LaTeX utilizando o Pacote USPSC para o ICMC}

% Idioma do texto em INGL�S
% 22/02/2017 - T�tulo para p�gina de rosto adicional
% para a vers�o em ingl�s, utilize os comandos abaixo, e inclua % no in�cio dos 4 comandos logo acima,  cada comando acima que s�o referentes ao texto do Trabalho Acad�mico em portugu�s
%\titulo{Model for thesis and dissertations in LaTeX using the USPSC Package to the ICMC}
%\titleabstract{Model for thesis and dissertations in LaTeX using the USPSC Package to the ICMC}
%\tituloadic{Modelo para teses e disserta\c{c}\~oes em LaTeX utilizando o Pacote USPSC para o ICMC}
%\tituloresumo{Modelo para teses e disserta\c{c}\~oes em LaTeX utilizando o Pacote USPSC para o ICMC}

\autor{Jos\'e da Silva}
\autorficha{Silva, Jos\'e da}
\autorabr{SILVA, J.}

\cutter{S856m}
% Para gerar a ficha catalogr�fica sem o C�digo Cutter, basta 
% incluir uma % na linha acima e tirar a % da linha abaixo
%\cutter{ }

\local{S\~ao Carlos}
\data{2021}

% Para o idioma portugu�s:
\renewcommand{\orientadorname}{Orientadora:}
\orientador{Profa. Dra. Elisa Gon\c{c}alves Rodrigues}
\orientadoradic{Advisor: Profa. Dra. Elisa Gon\c{c}alves Rodrigues}
\orientadorcorpoficha{orientadora Elisa Gon\c{c}alves Rodrigues}
\orientadorficha{Rodrigues, Elisa Gon\c{c}alves, orient}
%Para incluir o nome do(a) coorientados(a), inclua % nos 2 comandos acima e retire a % dos 2 comandos abaixo
%\orientadorcorpoficha{orientadora Elisa Gon\c{c}alves Rodrigues ;  co-orientador Jo\~ao Alves Serqueira}
%\orientadorficha{Rodrigues, Elisa Gon\c{c}alves, orient. II. Serqueira, Jo\~ao Alves, co-orient}


%Se o idoma for o ingl�s, inclua % nos comandos acima e exclua dos comandos abaixo
%\renewcommand{\orientadorname}{Advisor:}
%\orientador{Profa. Dra. Elisa Gon\c{c}alves Rodrigues}
%\orientadoradic{Orientadora: Profa. Dra. Elisa Gon\c{c}alves Rodrigues}
%\orientadorcorpoficha{orientadora Elisa Gon\c{c}alves Rodrigues}
%\orientadorficha{Rodrigues, Elisa Gon\c{c}alves, orient}
%Para incluir o nome do(a) coorientados(a), inclua % nos 2 comandos acima e retire a % dos 2 comandos abaixo
%\orientadorcorpoficha{orientadora Elisa Gon\c{c}alves Rodrigues ;  co-orientador Jo\~ao Alves Serqueira}
%\orientadorficha{Rodrigues, Elisa Gon\c{c}alves, orient. II. Serqueira, Jo\~ao Alves, co-orient}

% Quando houver Coorientador(a): 
% Para o idioma portugu�s:
% basta retirar  % antes de um dos comandos abaixo
%\newcommand{\coorientadorname}{Coorientador:}
%\newcommand{\coorientadorname}{Coorientadora:}
% Para o idoma ingl�s:
% basta retirar  % antes do comando abaixo
%\newcommand{\coorientadorname}{Coorientador:}

% Quando houver Coorientador(a), basta tirar a % utilizar o comando abaixo
%\newcommand{\coorientadorname}{Coadvisor:}
%Se houver co-orientador, inclua % antes das duas linhas (antes dos comandos \orientadorcorpoficha e \orientadorficha) 
%          e tire a % antes dos 3 comandos abaixo
%\coorientador{Prof. Dr. Jo\~ao Alves Serqueira}
%\coorientadoradic{ Co-orientador: Prof. Dr. Jo\~ao Alves Serqueira}
%\orientadorcorpoficha{orientadora Elisa Gon\c{c}alves Rodrigues ;  co-orientador Jo\~ao Alves Serqueira}
%\orientadorficha{Rodrigues, Elisa Gon\c{c}alves, orient. II. Serqueira, Jo\~ao Alves, co-orient}

%Para o idioma Ingl�s, retire a % antes da linha abaixo
%\renewcommand{\areaname}{Concentration area: }

		
\notaautorizacao{AUTORIZO A REPRODU\c{C}\~AO E DIVULGA\c{C}\~AO TOTAL OU PARCIAL DESTE TRABALHO, POR QUALQUER MEIO CONVENCIONAL OU ELETR\^ONICO PARA FINS DE ESTUDO E PESQUISA, DESDE QUE CITADA A FONTE.}
% Se o idioma for o ingl�s, inclua a % antes do campo \notaautorizacao acima e retire a % da linha abaixo
%\notaautorizacao{I AUTORIZE THE REPRODUCTION AND DISSEMINATION OF TOTAL OR PARTIAL COPIES OF THIS DOCUMENT, BY CONVENCIONAL OR ELECTRONIC MEDIA FOR STUDY OR RESEARCH PURPOSE, SINCE IT IS REFERENCED.}

\notabib{Ficha catalogr\'afica elaborada pela Biblioteca Prof. Achille Bassi, ICMC/USP, com os dados fornecidos pelo(a) autor(a)}

\newcommand{\programa}[1]{
% MPMp ==========================================================================
\ifthenelse{\equal{#1}{MPMp}}{
	\tipotrabalho{Disserta\c{c}\~ao (Mestrado em Ci\^encias)}
	\tipotrabalhoabs{Dissertation (Master in Science)}
	\area{Matem\'atica em Rede Nacional}
	\areaadic{Concentration area: Mathematics in National Network}
	%\opcao{Nome da Op��o em portugu�s}
	%\opcaoadic{Nome da Op��o em ingl�s}
	% O preambulo deve conter o tipo do trabalho, o objetivo, 
	% o nome da institui��o, a �rea de concentra��o e op��o quando houver
	\preambulo{Disserta\c{c}\~ao apresentada ao Instituto de Ci\^encias Matem\'aticas e de Computa\c{c}\~ao, Universidade de S\~ao Paulo - ICMC/USP, como parte dos requisitos para obten\c{c}\~ao do t\'itulo de Mestre em Ci\^encias - Mestrado Profissional em Matem\'atica em Rede Nacional.}
	\preambuloadic{Dissertation submitted to the Instituto de Ci\^encias Matem\'aticas e de Computa\c{c}\~ao, Universidade de S\~ao Paulo - ICMC/USP, in partial fulfillment of the requirements for the degree of the Master in Science - Professional Master in Mathematics in National Network.}
	\notaficha{Disserta\c{c}\~ao (Mestrado - Programa de Mestrado Profissional em Matem\'atica em Rede Nacional)}
	\notacapaicmc{Disserta\c{c}\~ao de Mestrado do Programa de Mestrado Profissional em \\Matem\'atica em Rede Nacional (PROFMAT)}
    }{
% MPMe ==========================================================================
\ifthenelse{\equal{#1}{MPMe}}{
	\renewcommand{\areaname}{Concentration area:}
	\tipotrabalho{Disserta\c{c}\~ao (Mestrado em Ci\^encias)}
	\tipotrabalhoabs{Dissertation (Master in Science)}
	\area{Mathematics in National Network}
	\areaadic{\'Area de concentra\c{c}\~ao: Matem\'atica em Rede Nacional}
	%\opcao{Nome da Op��o em ingl�s}
	%\opcaoadic{Nome da Op��o em portugu�s}
	% O preambulo deve conter o tipo do trabalho, o objetivo, 
	% o nome da institui��o, a �rea de concentra��o e op��o quando houver
	\preambulo{Dissertation submitted to the Instituto de Ci\^encias Matem\'aticas e de Computa\c{c}\~ao, Universidade de S\~ao Paulo - ICMC/USP, in partial fulfillment of the requirements for the degree of the Master in Science - Professional Master in Mathematics in National Network.}		
	\preambuloadic{Disserta\c{c}\~ao apresentada ao Instituto de Ci\^encias Matem\'aticas e de Computa\c{c}\~ao, Universidade de S\~ao Paulo - ICMC/USP, como parte dos requisitos para obten\c{c}\~ao do t\'itulo de Mestre em Ci\^encias - Mestrado Profissional em Matem\'atica em Rede Nacional.}
	\notaficha{Dissertation (Master - Professional Master\'{}s Program in Mathematics on the National Network)}
	\notacapaicmc{Master\'{}s Dissertation of the Professional Master\'{}s Program in \\Mathematics on the National Network (PROFMAT)}
    }{   
% MPMECAIp ==========================================================================
\ifthenelse{\equal{#1}{MPMECAIp}}{
	\tipotrabalho{Disserta\c{c}\~ao (Mestrado em Ci\^encias)}
	\tipotrabalhoabs{Dissertation (Master in Science)}
	\area{Matem\'atica, Estat\'istica e Computa\c{c}\~ao}
	\areaadic{Concentration area: Mathematics, Statistics and Computing}
	%\opcao{Nome da Op��o em portugu�s}
	%\opcaoadic{Nome da Op��o em ingl�s}
	% O preambulo deve conter o tipo do trabalho, o objetivo, 
	% o nome da institui��o, a �rea de concentra��o e op��o quando houver
	\preambulo{Disserta\c{c}\~ao apresentada ao Instituto de Ci\^encias Matem\'aticas e de Computa\c{c}\~ao, Universidade de S\~ao Paulo - ICMC/USP, como parte dos requisitos para obten\c{c}\~ao do t\'itulo de Mestre em Ci\^encias - Mestrado Profissional em Matem\'atica, Estat\'istica e Computa\c{c}\~ao Aplicadas \`a Ind\'ustria.}
	\preambuloadic{Dissertation submitted to the Instituto de Ci\^encias Matem\'aticas e de Computa\c{c}\~ao, Universidade de S\~ao Paulo - ICMC/USP, in partial fulfillment of the requirements for the degree of the Master in Science - Professional Masters in Mathematics, Statistics and Computing Applied to Industry.}
	\notaficha{Disserta\c{c}\~ao (Mestrado - Programa de Mestrado Profissional em Matem\'atica, Estat\'istica e Computa\c{c}\~ao Aplicadas \`a Ind\'ustria)}
	\notacapaicmc{Disserta\c{c}\~ao de Mestrado do Programa de Mestrado Profissional em \\Matem\'atica, Estat\'istica e Computa\c{c}\~ao Aplicadas \`a Ind\'ustria (MECAI)} 
}{
% MPMECAIe ==========================================================================
\ifthenelse{\equal{#1}{MPMECAIe}}{
	\renewcommand{\areaname}{Concentration area:}
	\tipotrabalho{Disserta\c{c}\~ao (Mestrado em Ci\^encias)}
	\tipotrabalhoabs{Dissertation (Master in Science)}
	\area{Mathematics, Statistics and Computing}
	\areaadic{\'Area de concentra\c{c}\~ao: Matem\'atica, Estat\'istica e Computa\c{c}\~ao}
	%\opcao{Nome da Op��o em ingl�s}
	%\opcaoadic{Nome da Op��o em portugu�s}
	% O preambulo deve conter o tipo do trabalho, o objetivo, 
	% o nome da institui��o, a �rea de concentra��o e op��o quando houver
	\preambulo{Dissertation submitted to the Instituto de Ci\^encias Matem\'aticas e de Computa\c{c}\~ao, Universidade de S\~ao Paulo - ICMC/USP, in partial fulfillment of the requirements for the degree of the Master in Science - Professional Masters in Mathematics, Statistics and Computing Applied to Industry.}		
	\preambuloadic{Disserta\c{c}\~ao apresentada ao Instituto de Ci\^encias Matem\'aticas e de Computa\c{c}\~ao, Universidade de S\~ao Paulo - ICMC/USP, como parte dos requisitos para obten\c{c}\~ao do t\'itulo de Mestre em Ci\^encias - Mestrado Profissional em Matem\'atica, Estat\'istica e Computa\c{c}\~ao Aplicadas \`a Ind\'ustria.}
	\notaficha{Dissertation (Master - Professional Master's Program in Mathematics, Statistics and Computing Applied to Industry)}
	\notacapaicmc{Master's Dissertation of the Professional Master's Program in Mathematics, \\Statistics and Computing Applied to Industry (MECAI)}
}{    
% DMAp ==========================================================================
\ifthenelse{\equal{#1}{DMAp}}{
    \tipotrabalho{Tese (Doutorado em Ci\^encias)}
    \tipotrabalhoabs{Thesis (Doctorate in Science)}
    \area{Matem\'atica}
    \areaadic{Concentration area: Mathematics}
	%\opcao{Nome da Op��o em portugu�s}
	%\opcaoadic{Nome da Op��o em ingl�s}
    % O preambulo deve conter o tipo do trabalho, o objetivo, 
	% o nome da institui��o, a �rea de concentra��o e op��o quando houver
	\preambulo{Tese apresentada ao Instituto de Ci\^encias Matem\'aticas e de Computa\c{c}\~ao, Universidade de S\~ao Paulo - ICMC/USP, como parte dos requisitos para obten\c{c}\~ao do t\'itulo de Doutor em Ci\^encias - Matem\'atica.}	
	\preambuloadic{Thesis submitted to the Instituto de Ci\^encias Matem\'aticas e de Computa\c{c}\~ao, Universidade de S\~ao Paulo - ICMC/USP, in partial fulfillment of the requirements for the degree of the Doctor in Science - Mathematics.}
	\notaficha{Tese (Doutorado - Programa de P\'os-Gradua\c{c}\~ao em Matem\'atica)}
	\notacapaicmc{Tese de Doutorado do Programa de P\'os-Gradua\c{c}\~ao em \\Matem\'atica (PPG-Mat)}
    }{
% DMAe ==========================================================================
\ifthenelse{\equal{#1}{DMAe}}{
	\tipotrabalho{Tese (Doutorado em Ci\^encias)}
    \tipotrabalhoabs{Thesis (Doctorate in Science)}
	\renewcommand{\areaname}{Concentration area:}
    \area{Mathematics}
    \areaadic{\'Area de concentra\c{c}\~ao: Matem\'atica}
	%\opcao{Nome da Op��o em ingl�s}
	%\opcaoadic{Nome da Op��o em portugu�s}
    % O preambulo deve conter o tipo do trabalho, o objetivo, 
	% o nome da institui��o, a �rea de concentra��o e op��o quando houver
	\preambulo{Thesis submitted to the Instituto de Ci\^encias Matem\'aticas e de Computa\c{c}\~ao, Universidade de S\~ao Paulo - ICMC/USP, in partial fulfillment of the requirements for the degree of the Doctor in Science - Mathematics.}
	\preambuloadic{Tese apresentada ao Instituto de Ci\^encias Matem\'aticas e de Computa\c{c}\~ao, Universidade de S\~ao Paulo - ICMC/USP, como parte dos requisitos para obten\c{c}\~ao do t\'itulo de Doutor em Ci\^encias - Matem\'atica.}
	\notaficha{Thesis (Doctorate - Program in Mathematics)}
	\notacapaicmc{Doctoral Thesis of the Postgraduate Program in Mathematics (PPG-Mat)}
    }{
% MMAp ==========================================================================
\ifthenelse{\equal{#1}{MMAp}}{
    \tipotrabalho{Disserta\c{c}\~ao (Mestrado em Ci\^encias)}
    \tipotrabalhoabs{Dissertation (Master in Science)}
    \area{Matem\'atica}
    \areaadic{Concentration area: Mathematics}
	%\opcao{Nome da Op��o em portugu�s}
	%\opcaoadic{Nome da Op��o em ingl�s}
    % O preambulo deve conter o tipo do trabalho, o objetivo, 
	% o nome da institui��o, a �rea de concentra��o e op��o quando houver
	\preambulo{Disserta\c{c}\~ao apresentada ao Instituto de Ci\^encias Matem\'aticas e de Computa\c{c}\~ao, Universidade de S\~ao Paulo - ICMC/USP, como parte dos requisitos para obten\c{c}\~ao do t\'itulo de Mestre em Ci\^encias - Matem\'atica.}	
	\preambuloadic{Dissertation submitted to the Instituto de Ci\^encias Matem\'aticas e de Computa\c{c}\~ao, Universidade de S\~ao Paulo - ICMC/USP, in partial fulfillment of the requirements for the degree of the Master in Science - Mathematics.}
	\notaficha{Disserta\c{c}\~ao (Mestrado - Programa de P\'os-Gradua\c{c}\~ao em Matem\'atica)}
	\notacapaicmc{Disserta\c{c}\~ao de Mestrado do Programa de P\'os-Gradua\c{c}\~ao em \\Matem\'atica (PPG-Mat)}
    }{
% MMAe ==========================================================================
\ifthenelse{\equal{#1}{MMAe}}{
	\tipotrabalho{Disserta\c{c}\~ao (Mestrado em Ci\^encias)}
    \tipotrabalhoabs{Dissertation (Master in Science)}
	\renewcommand{\areaname}{Concentration area:}
    \area{Mathematics}
    \areaadic{\'Area de concentra\c{c}\~ao: Matem\'atica}
	%\opcao{Nome da Op��o em ingl�s}
	%\opcaoadic{Nome da Op��o em portugu�s}
    % O preambulo deve conter o tipo do trabalho, o objetivo, 
	% o nome da institui��o, a �rea de concentra��o e op��o quando houver
	\preambulo{Dissertation submitted to the Instituto de Ci\^encias Matem\'aticas e de Computa\c{c}\~ao, Universidade de S\~ao Paulo - ICMC/USP, in partial fulfillment of the requirements for the degree of the Master in Science - Mathematics.}
	\preambuloadic{Disserta\c{c}\~ao apresentada ao Instituto de Ci\^encias Matem\'aticas e de Computa\c{c}\~ao, Universidade de S\~ao Paulo - ICMC/USP, como parte dos requisitos para obten\c{c}\~ao do t\'itulo de Mestre em Ci\^encias - Matem\'atica.}
	\notaficha{Dissertation (Master - Program in Mathematics)}
	\notacapaicmc{Master\'{}s Dissertation of the Postgraduate Program in Mathematics (PPG-Mat)}
    }{
% DESp ==========================================================================
\ifthenelse{\equal{#1}{DESp}}{
    \tipotrabalho{Tese (Doutorado em Estat\'istica)}
    \tipotrabalhoabs{Thesis (Doctorate in Statistics)}
    \area{Estat\'istica}
    \areaadic{Concentration area: Statistics}
    \instituicao{Instituto de Ci\^encias Matem\'aticas e de Computa\c{c}\~ao, Universidade de S\~ao Paulo; Departamento de Estat\'istica, Universidade Federal de S\~ao Carlos}
	%\opcao{Nome da Op��o em portugu�s}
	%\opcaoadic{Nome da Op��o em ingl�s}
    % O preambulo deve conter o tipo do trabalho, o objetivo, 
	% o nome da institui��o, a �rea de concentra��o e op��o quando houver
	\preambulo{Tese apresentada ao Instituto de Ci\^encias Matem\'aticas e de Computa\c{c}\~ao, Universidade de S\~ao Paulo - ICMC/USP e ao Departamento de Estat\'istica, Universidade Federal de S\~ao Carlos - DEs/UFSCar, como parte dos requisitos para obten\c{c}\~ao do t\'itulo de Doutor em Estat\'istica - Interinstitucional de P\'os-Gradua\c{c}\~ao em Estat\'istica.}
	\preambuloadic{Thesis submitted to the Instituto de Ci\^encias Matem\'aticas e de Computa\c{c}\~ao, Universidade de S\~ao Paulo - ICMC/USP and to the Departamento de Estat\'istica, Universidade Federal de S\~ao Carlos - DEs/UFSCar, in partial fulfillment of the requirements for the degree of the Doctor in Statistics - Interagency Program Graduate in Statistics.}
	\notaficha{Tese (Doutorado - Interinstitucional de P\'os-Gradua\c{c}\~ao em Estat\'istica)}
	\notacapaicmc{Tese de Doutorado do Programa Interinstitucional de P\'os-Gradua\c{c}\~ao em \\Estat\'istica (PIPGEs)}
    }{
% DESe ==========================================================================
\ifthenelse{\equal{#1}{DESe}}{
	\tipotrabalho{Tese (Doutorado em Estat\'istica)}
    \tipotrabalhoabs{Thesis (Doctorate in Statistics)}
	\renewcommand{\areaname}{Concentration area:}
    \area{Statistics}
    \areaadic{\'Area de concentra\c{c}\~ao: Estat\'istica}
    \instituicao{Instituto de Ci\^encias Matem\'aticas e de Computa\c{c}\~ao, Universidade de S\~ao Paulo; Departamento de Estat\'istica, Universidade Federal de S\~ao Carlos}
	%\opcao{Nome da Op��o em ingl�s}
	%\opcaoadic{Nome da Op��o em portugu�s}
    % O preambulo deve conter o tipo do trabalho, o objetivo, 
	% o nome da institui��o, a �rea de concentra��o e op��o quando houver
	\preambulo{Thesis submitted to the Instituto de Ci\^encias Matem\'aticas e de Computa\c{c}\~ao, Universidade de S\~ao Paulo - ICMC/USP and to the Departamento
	de Estat\'istica, Universidade Federal de S\~ao Carlos - DEs/UFSCar, in partial fulfillment of the requirements for the degree of the Doctor in Statistics - Interagency Program Graduate in Statistics.}
	\preambuloadic{Tese apresentada ao Instituto de Ci\^encias Matem\'aticas e de Computa\c{c}\~ao, Universidade de S\~ao Paulo - ICMC/USP e ao Departamento de Estat\'istica, Universidade Federal de S\~ao Carlos - DEs/UFSCar, como parte dos requisitos para obten\c{c}\~ao do t\'itulo de Doutor em Estat\'istica - Interinstitucional de P\'os-Gradua\c{c}\~ao em Estat\'istica.}
	\notaficha{Thesis (Doctorate - Joint Graduate Program in Statistics)}
	\notacapaicmc{Doctoral Thesis of the Interagency Postgraduate Program in Statistics (PIPGEs)}
    }{     
% MESp ==========================================================================
\ifthenelse{\equal{#1}{MESp}}{
    \tipotrabalho{Disserta\c{c}\~ao (Mestrado em Estat\'istica)}
    \tipotrabalhoabs{Dissertation (Master in Statistics)}
    \renewcommand{\areaname}{Concentration area:}
    \area{Estat\'istica}
    \areaadic{Concentration area: Statistics}
    \instituicao{Instituto de Ci\^encias Matem\'aticas e de Computa\c{c}\~ao, Universidade de S\~ao Paulo; Departamento de Estat\'istica, Universidade Federal de S\~ao Carlos}
	%\opcao{Nome da Op��o em portugu�s}
	%\opcaoadic{Nome da Op��o em ingl�s}
    % O preambulo deve conter o tipo do trabalho, o objetivo, 
	% o nome da institui��o, a �rea de concentra��o e op��o quando houver
	\preambulo{Disserta\c{c}\~ao apresentada ao Instituto de Ci\^encias Matem\'aticas e de Computa\c{c}\~ao, Universidade de S\~ao Paulo - ICMC/USP e ao Departamento de Estat\'istica, Universidade Federal de S\~ao Carlos - DEs/UFSCar, como parte dos requisitos para obten\c{c}\~ao do t\'itulo de Mestre em Estat\'istica - Interinstitucional de P\'os-Gradua\c{c}\~ao em Estat\'istica.}
	\preambuloadic{Dissertation submitted to the Instituto de Ci\^encias Matem\'aticas e de Computa\c{c}\~ao, Universidade de S\~ao Paulo - ICMC/USP and to the Departamento de Estat\'istica- DEs, Universidade Federal de S\~ao Carlos - DEs/UFSCar, in partial fulfillment of the requirements for the degree of the Master in Statistics - Joint Graduate Program in Statistics.}
	\notaficha{Disserta\c{c}\~ao (Mestrado - Interinstitucional de P\'os-Gradua\c{c}\~ao em Estat\'istica)}
	\notacapaicmc{Disserta\c{c}\~ao de Mestrado do Programa Interinstitucional de \\P\'os-Gradua\c{c}\~ao em Estat\'istica (PIPGEs)}
    }{
% MESe ==========================================================================
\ifthenelse{\equal{#1}{MESe}}{
	\tipotrabalho{Disserta\c{c}\~ao (Mestrado em Estat\'istica)}
    \tipotrabalhoabs{Dissertation (Master in Statistics)}
    \renewcommand{\areaname}{Concentration area:}
	\area{Statistics}
    \areaadic{\'Area de concentra\c{c}\~ao: Estat\'istica}
    \instituicao{Instituto de Ci\^encias Matem\'aticas e de Computa\c{c}\~ao, Universidade de S\~ao Paulo; Departamento de Estat\'istica, Universidade Federal de S\~ao Carlos}
	%\opcao{Nome da Op��o em ingl�s}
	%\opcaoadic{Nome da Op��o em portugu�s}
    % O preambulo deve conter o tipo do trabalho, o objetivo, 
	% o nome da institui��o, a �rea de concentra��o e op��o quando houver
	\preambulo{Dissertation submitted to the Instituto de Ci\^encias Matem\'aticas e de Computa\c{c}\~ao, Universidade de S\~ao Paulo - ICMC/USP and to the Departamento de Estat\'istica - DEs, Universidade Federal de S\~ao Carlos - DEs/UFSCar, in partial fulfillment of the requirements for the degree of the Master in Statistics - Interagency Program Graduate in Statistics.} 
	\preambuloadic{Disserta\c{c}\~ao apresentada ao Instituto de Ci\^encias Matem\'aticas e de Computa\c{c}\~ao, Universidade de S\~ao Paulo - ICMC/USP e ao Departamento de Estat\'istica, Universidade Federal de S\~ao Carlos - DEs/UFSCar, como parte dos requisitos para obten\c{c}\~ao do t\'itulo de Mestre em Estat\'istica - Interinstitucional de P\'os-Gradua\c{c}\~ao em Estat\'istica.}
	\notaficha{Dissertation (Master - Joint Graduate Program in Statistics)}
	\notacapaicmc{Master\'{}s Dissertation of the Interagency Postgraduate Program in\\ Statistics (PIPGEs)}
	}{  
% DCCp ==========================================================================
\ifthenelse{\equal{#1}{DCCp}}{
    \tipotrabalho{Tese (Doutorado em Ci\^encias)}
    \tipotrabalhoabs{Thesis (Doctorate in Science)}
    \area{Ci\^encias de Computa\c{c}\~ao e Matem\'atica Computacional}
    \areaadic{Concentration area: Computer Science and Computational Mathematics} 
	%\opcao{Nome da Op��o em portugu�s}
	%\opcaoadic{Nome da Op��o em ingl�s}
    % O preambulo deve conter o tipo do trabalho, o objetivo, 
	% o nome da institui��o, a �rea de concentra��o e op��o quando houver
	\preambulo{Tese apresentada ao Instituto de Ci\^encias Matem\'aticas e de Computa\c{c}\~ao, Universidade de S\~ao Paulo - ICMC/USP, como parte dos requisitos para obten\c{c}\~ao do t\'itulo de Doutor em Ci\^encias - Ci\^encias de Computa\c{c}\~ao e Matem\'atica Computacional.}
	\preambuloadic{Thesis submitted to the Instituto de Ci\^encias Matem\'aticas e de Computa\c{c}\~ao, Universidade de S\~ao Paulo - ICMC/USP, in partial fulfillment of the requirements for the degree of the Doctor in Science - Program in Computer Science and Computational Mathematics.}
	\notaficha{Tese (Doutorado - Programa de P\'os-Gradua\c{c}\~ao em Ci\^encias de Computa\c{c}\~ao e Matem\'atica Computacional)}	 
	\notacapaicmc{Tese de Doutorado do Programa de P\'os-Gradua\c{c}\~ao em Ci\^encias de\\ Computa\c{c}\~ao e Matem\'atica Computacional (PPG-CCMC)}		
    }{
% DCCe ==========================================================================
\ifthenelse{\equal{#1}{DCCe}}{
    \tipotrabalho{Tese (Doutorado em Ci\^encias)}
    \tipotrabalhoabs{Thesis (Doctorate in Science)}
	\renewcommand{\areaname}{Concentration area:}
    \area{Computer Science and Computational Mathematics}
    \areaadic{\'Area de concentra\c{c}\~ao: Ci\^encias de Computa\c{c}\~ao e Matem\'atica Computacional}
	%\opcao{Nome da Op��o em ingl�s}
	%\opcaoadic{Nome da Op��o em portugu�s}
    % O preambulo deve conter o tipo do trabalho, o objetivo, 
	% o nome da institui��o, a �rea de concentra��o e op��o quando houver
	\preambulo{Thesis submitted to the Instituto de Ci\^encias Matem\'aticas e de Computa\c{c}\~ao, Universidade de S\~ao Paulo - ICMC/USP, in partial fulfillment of the requirements for the degree of the Doctor in Science - Program in Computer Science and Computational Mathematics.}
	\preambuloadic{Tese apresentada ao Instituto de Ci\^encias Matem\'aticas e de Computa\c{c}\~ao, Universidade de S\~ao Paulo - ICMC/USP, como parte dos requisitos para obten\c{c}\~ao do t\'itulo de Doutor em Ci\^encias - Ci\^encias de Computa\c{c}\~ao e Matem\'atica Computacional.}
	\notaficha{Thesis (Doctorate - Program in Computer Science and Computational Mathematics)}
	\notacapaicmc{Doctoral Thesis of the Postgraduate Program in Computer Science and\\ Computational Mathematics (PPG-CCMC)}	
    }{			
% MCCp ==========================================================================
\ifthenelse{\equal{#1}{MCCp}}{
    \tipotrabalho{Disserta\c{c}\~ao (Mestrado em Ci\^encias)}
    \tipotrabalhoabs{Dissertation (Master in Science)}
    \area{Ci\^encias de Computa\c{c}\~ao e Matem\'atica Computacional}
    \areaadic{Concentration area: Computer Science and Computational Mathematics}         
	%\opcao{Nome da Op��o em portugu�s}
	%\opcaoadic{Nome da Op��o em ingl�s}
    % O preambulo deve conter o tipo do trabalho, o objetivo, 
	% o nome da institui��o, a �rea de concentra��o e op��o quando houver
	\preambulo{Disserta\c{c}\~ao apresentada ao Instituto de Ci\^encias Matem\'aticas e de Computa\c{c}\~ao, Universidade de S\~ao Paulo - ICMC/USP, como parte dos requisitos para obten\c{c}\~ao do t\'itulo de Mestre em Ci\^encias - Ci\^encias de Computa\c{c}\~ao e Matem\'atica Computacional.}
	\preambuloadic{Dissertation submitted to the Instituto de Ci\^encias Matem\'aticas e de Computa\c{c}\~ao, Universidade de S\~ao Paulo - ICMC/USP, in partial fulfillment of the requirements for the degree of the Master in Science - Program in Computer Science and Computational Mathematics.}
	\notaficha{Disserta\c{c}\~ao (Mestrado - Programa de P\'os-Gradua\c{c}\~ao em Ci\^encias de Computa\c{c}\~ao e Matem\'atica Computacional)}
	\notacapaicmc{Disserta\c{c}\~ao de Mestrado do Programa de P\'os-Gradua\c{c}\~ao em Ci\^encias de\\ Computa\c{c}\~ao e Matem\'atica Computacional (PPG-CCMC)}	
    }{
% MCCe ==========================================================================
\ifthenelse{\equal{#1}{MCCe}}{
    \tipotrabalho{Disserta\c{c}\~ao (Mestrado em Ci\^encias)}
    \tipotrabalhoabs{Dissertation (Master in Science)}
	\renewcommand{\areaname}{Concentration area:}
    \area{Computer Science and Computational Mathematics}
    \areaadic{\'Area de concentra\c{c}\~ao: Ci\^encias de Computa\c{c}\~ao e Matem\'atica Computacional}
	%\opcao{Nome da Op��o em ingl�s}
	%\opcaoadic{Nome da Op��o em portugu�s}
    % O preambulo deve conter o tipo do trabalho, o objetivo, 
	% o nome da institui��o, a �rea de concentra��o e op��o quando houver
	\preambulo{Dissertation submitted to the Instituto de Ci\^encias Matem\'aticas e de Computa\c{c}\~ao, Universidade de S\~ao Paulo - ICMC/USP, in partial fulfillment of the requirements for the degree of the Master in Science - Program in Computer Science and Computational Mathematics.}
	\preambuloadic{Disserta\c{c}\~ao apresentada ao Instituto de Ci\^encias Matem\'aticas e de Computa\c{c}\~ao, Universidade de S\~ao Paulo - ICMC/USP, como parte dos requisitos para obten\c{c}\~ao do t\'itulo de Mestre em Ci\^encias - Ci\^encias de Computa\c{c}\~ao e Matem\'atica Computacional.}
	\notaficha{Dissertation (Master - Program in Computer Science and Computational Mathematics)}
	\notacapaicmc{Master's Dissertation of the Postgraduate Program in Computer Science and\\ Computational Mathematics (PPG-CCMC)}
    }{		
% MBACDp ==========================================================================
\ifthenelse{\equal{#1}{MBACDp}}{
	\tipotrabalho{Monografia (MBA em Ci\^encias de Dados)}
	\tipotrabalhoabs{Monograph (MBA in Data Sciences)}
	\area{Ci\^encias de Dados}
	\areaadic{Concentration area: Data Science} 
	%\opcao{Nome da Op��o em portugu�s}
	%\opcaoadic{Nome da Op��o em ingl�s}
	% O preambulo deve conter o tipo do trabalho, o objetivo, 
	% o nome da institui��o, a �rea de concentra��o e op��o quando houver
	\preambulo{Monografia apresentada ao Centro de Ci\^encias Matem\'aticas Aplicadas \`a Ind\'ustria do Instituto de Ci\^encias Matem\'aticas e de Computa\c{c}\~ao, Universidade de S\~ao Paulo - ICMC/USP, como parte dos requisitos para obten\c{c}\~ao do t\'itulo de Especialista em Ci\^encias de Dados.}
	\preambuloadic{Monograph presented to the Centro de Ci\^encias Matem\'aticas Aplicadas \`a Ind\'ustria do Instituto de Ci\^encias Matem\'aticas e de Computa\c{c}\~ao, Universidade de S\~ao Paulo - ICMC/USP, as part of the requirements for obtaining the title of Specialist in Data Science.}
	\instituicao{Centro de Ci\^encias Matem\'aticas Aplicadas \`a Ind\'ustria, Instituto de Ci\^encias Matem\'aticas e de Computa\c{c}\~ao, Universidade de S\~ao Paulo}
	\notaficha{Monografia (MBA em Ci\^encias de Dados)}
	\notacapaicmc{Monografia - MBA em Ci\^encia de Dados (CEMEAI)}
    }{    		
% MBACDe ==========================================================================
\ifthenelse{\equal{#1}{MBACDe}}{
	\tipotrabalho{Monografia (MBA em Ci\^encias de Dados)}
	\tipotrabalhoabs{Monograph (MBA in Data Sciences)}
	\renewcommand{\areaname}{Concentration area:}
	\area{Data Sciences}
	\areaadic{\'Area de concentra\c{c}\~ao: Ci\^encias de Dados}
 	%\opcao{Nome da Op��o em ingl�s}
 	%\opcaoadic{Nome da Op��o em portugu�s}
	% O preambulo deve conter o tipo do trabalho, o objetivo, 
	% o nome da institui��o, a �rea de concentra��o e op��o quando houver
	\preambulo{Monograph presented to the Centro de Ci\^encias Matem\'aticas Aplicadas \`a Ind\'ustria do Instituto de Ci\^encias Matem\'aticas e de Computa\c{c}\~ao, Universidade de S\~ao Paulo - ICMC/USP, as part of the requirements for obtaining the title of Specialist in Data Science.}
	\preambuloadic{Monografia apresentada ao Centro de Ci\^encias Matem\'aticas Aplicadas \`a Ind\'ustria do Instituto de Ci\^encias Matem\'aticas e de Computa\c{c}\~ao, Universidade de S\~ao Paulo - ICMC/USP, como parte dos requisitos para obten\c{c}\~ao do t\'itulo de Especialista em Ci\^encias de Dados.}
	\instituicao{Centro de Ci\^encias Matem\'aticas Aplicadas \`a Ind\'ustria, Instituto de Ci\^encias Matem\'aticas e de Computa\c{c}\~ao, Universidade de S\~ao Paulo}
	\notaficha{Monograph (MBA in Data Sciences)}
	\notacapaicmc{Monograph - MBA in Data Science (CEMEAI)}
    }{   
% MBAIAp ==========================================================================
\ifthenelse{\equal{#1}{MBAIAp}}{
	\tipotrabalho{Monografia (MBA em Intelig\^encia Artificial e Big Data)}
	\tipotrabalhoabs{Monograph (MBA in Artificial Intelligence and Big Data)}
	\area{Intelig\^encia Artificial}
	\areaadic{Concentration area: Artificial Intelligence} 
	%\opcao{Nome da Op��o em portugu�s}
	%\opcaoadic{Nome da Op��o em ingl�s}
	% O preambulo deve conter o tipo do trabalho, o objetivo, 
	% o nome da institui��o, a �rea de concentra��o e op��o quando houver
	\preambulo{Monografia apresentada ao Departamento de Ci\^encias de Computa\c{c}\~ao do Instituto de Ci\^encias Matem\'aticas e de Computa\c{c}\~ao, Universidade de S\~ao Paulo - ICMC/USP, como parte dos requisitos para obten\c{c}\~ao do t\'itulo de Especialista em Intelig\^encia Artificial e Big Data.}
	\preambuloadic{Monograph presented to the Departamento de Ci\^encias de Computa\c{c}\~ao do Instituto de Ci\^encias Matem\'aticas e de Computa\c{c}\~ao, Universidade de S\~ao Paulo - ICMC/USP, as part of the requirements for obtaining the title of Specialist in Artificial Intelligence and Big Data.}
	\notaficha{Monografia (MBA em Intelig\^encia Artificial e Big Data)}
	\notacapaicmc{Monografia - MBA em Intelig\^encia Artificial e Big Data}
    }{    		
% MBAIAe ==========================================================================
\ifthenelse{\equal{#1}{MBAIAe}}{
	\tipotrabalho{Monografia (MBA em Intelig\^encia Artificial e Big Data)}
	\tipotrabalhoabs{Monograph (MBA in Artificial Intelligence and Big Data)}
	\renewcommand{\areaname}{Concentration area:}
	\area{Artificial Intelligence and Big Data}
	\areaadic{\'Area de concentra\c{c}\~ao: Intelig\^encia Artificial e Big Data}
	%\opcao{Nome da Op��o em ingl�s}
	%\opcaoadic{Nome da Op��o em portugu�s}
	% O preambulo deve conter o tipo do trabalho, o objetivo, 
	% o nome da institui��o, a �rea de concentra��o e op��o quando houver
	\preambulo{Monograph presented to the Departamento de Ci\^encias de Computa\c{c}\~ao do Instituto de Ci\^encias Matem\'aticas e de Computa\c{c}\~ao, Universidade de S\~ao Paulo - ICMC/USP, as part of the requirements for obtaining the title of Specialist in Artificial Intelligence and Big Data.}
	\preambuloadic{Monografia apresentada ao Departamento de Ci\^encias de Computa\c{c}\~ao do Instituto de Ci\^encias Matem\'aticas e de Computa\c{c}\~ao, Universidade de S\~ao Paulo - ICMC/USP, como parte dos requisitos para obten\c{c}\~ao do t\'itulo de Especialista em Intelig\^encia Artificial e Big Data.}
	\notaficha{Monograph (MBA in Artificial Intelligence and Big Data)}
	\notacapaicmc{Monograph - MBA in Artificial Intelligence and Big Data}
    }{  
% MBASDp ==========================================================================
\ifthenelse{\equal{#1}{MBASDp}}{
	\tipotrabalho{Monografia (MBA em Seguran\c{c}a de Dados)}
	\tipotrabalhoabs{Monograph (MBA in Data Security)}
	\area{Seguran\c{c}a de Dados}
	\areaadic{Concentration area: Data Security} 
	%\opcao{Nome da Op��o em portugu�s}
	%\opcaoadic{Nome da Op��o em ingl�s}
	% O preambulo deve conter o tipo do trabalho, o objetivo, 
	% o nome da institui��o, a �rea de concentra��o e op��o quando houver
	\preambulo{Monografia apresentada ao Centro de Ci\^encias Matem\'aticas Aplicadas \`a Ind\'ustria do Instituto de Ci\^encias Matem\'aticas e de Computa\c{c}\~ao, Universidade de S\~ao Paulo - ICMC/USP, como parte dos requisitos para obten\c{c}\~ao do t\'itulo de Especialista em Seguran\c{c}a de Dados.}
	\preambuloadic{Monograph presented to the Centro de Ci\^encias Matem\'aticas Aplicadas \`a Ind\'ustria do Instituto de Ci\^encias Matem\'aticas e de Computa\c{c}\~ao, Universidade de S\~ao Paulo - ICMC/USP, as part of the requirements for obtaining the title of Specialist in Artificial Intelligence and Big Data.}
	\notaficha{Monografia (MBA em Seguran\c{c}a de Dados)}
	\notacapaicmc{Monografia - MBA em Seguran\c{c}a de Dados (CEMEAI)}
    }{    		
% MBASDe ==========================================================================
\ifthenelse{\equal{#1}{MBASDe}}{
	\tipotrabalho{Monografia (MBA em Seguran\c{c}a de Dados)}
	\tipotrabalhoabs{Monograph (MBA in Data Security)}
	\renewcommand{\areaname}{Concentration area:}
	\area{Data Security}
	\areaadic{\'Area de concentra\c{c}\~ao: Seguran\c{c}a de Dados}
	%\opcao{Nome da Op��o em ingl�s}
	%\opcaoadic{Nome da Op��o em portugu�s}
	% O preambulo deve conter o tipo do trabalho, o objetivo, 
	% o nome da institui��o, a �rea de concentra��o e op��o quando houver
	\preambulo{Monograph presented to the Centro de Ci\^encias Matem\'aticas Aplicadas \`a Ind\'ustria do Instituto de Ci\^encias Matem\'aticas e de Computa\c{c}\~ao, Universidade de S\~ao Paulo - ICMC/USP, as part of the requirements for obtaining the title of Specialist in Artificial Intelligence and Big Data.}
	\preambuloadic{Monografia apresentada ao Centro de Ci\^encias Matem\'aticas Aplicadas \`a Ind\'ustria do Instituto de Ci\^encias Matem\'aticas e de Computa\c{c}\~ao, Universidade de S\~ao Paulo - ICMC/USP, como parte dos requisitos para obten\c{c}\~ao do t\'itulo de Especialista em Seguran\c{c}a de Dados.}
	\notaficha{Monograph (MBA in Data Security)}
	\notacapaicmc{Monograph - MBA in Data Security (CEMEAI)}
    }{  
% Outros
	\tipotrabalho{Disserta\c{c}\~ao/Tese (Mestrado/Doutorado)}
	\tipotrabalhoabs{Dissertation/Thesis (Master/Doctor)}
	\area{Nome da \'Area}
	\opcao{Nome da Op\c{c}\~ao}
	\areaadic{Additional area}
	%\opcao{Nome da Op��o em portugu�s}
	%\opcaoadic{Nome da Op��o em ingl�s}
    % O preambulo deve conter o tipo do trabalho, o objetivo, 
	% o nome da institui��o, a �rea de concentra��o e op��o quando houver				
	\preambulo{Disserta\c{c}\~ao/Tese apresentada ao Instituto de Ci\^encias Matem\'aticas e de Computa\c{c}\~ao, Universidade de S\~ao Paulo - ICMC/USP, como parte dos requisitos para obten\c{c}\~ao do t\'itulo de Mestre/Doutor em Ci\^encias - Programa.}
	\preambuloadic{Dissertation/Thesis submitted to the Instituto de Ci\^encias Matem\'aticas e de Computa\c{c}\~ao, Universidade de S\~ao Paulo - ICMC/USP, in partial fulfillment of the requirements for the degree of the Master/Doctor in Science - Program.}
	\notaficha{Disserta\c{c}\~ao/Tese (Mestrado/Doutorado - Programa)}
	\notacapaicmc{Master's Dissertation/Doctoral Thesis of the Postgraduate Program in ...}
        }}}}}}}}}}}}}}}}}}}}}}}		







    
        }{
% ICMC-TCC ===========================================================================
        \ifthenelse{\equal{#1}{ICMC-TCC}}{
        	%% USPSC-TCC-pre-textual-ICMC.tex
%% Camandos para defini��o do tipo de documento (tese ou disserta��o), �rea de concentra��o, op��o, pre�mbulo, titula��o 
%% referentes ao Programa de P�s-Gradua��o o IFSC
\instituicao{Instituto de Ci\^encias Matem\'aticas e de Computa\c{c}\~ao, Universidade de S\~ao Paulo}
\unidade{INSTITUTO DE CI\^ENCIAS MATEM\'ATICAS E DE COMPUTA\c{C}\~AO}
\unidademin{Instituto de Ci\^encias Matem\'aticas e de Computa\c{c}\~ao}
\universidademin{Universidade de S\~ao Paulo}
\setorpos{SERVI\c{C}O DE GRADUA\c{C}\~AO DO ICMC-USP}

\notafolharosto{Vers\~ao original}
\notafolharostoadic{Original version}
%Para vers�o original em ingl�s, comente os comandos/declara��es acima (inclua % antes do comando acima) 
% e tire a % dos comandos/declara��es abaixo no idioma do texto
%\notafolharosto{Original version} 
%\notafolharostoadic{Vers\~ao original}

%Para vers�o revisada, comente os comandos/declara��es acima (inclua % antes do comando acima) 
% e tire a % dos comandos/declara��es abaixo, em conformidade com o idioma do texto
% Se o Idioma do texto for portugu�s: 
%\notafolharosto{Vers\~ao revisada}
%\notafolharosto{Final version}
% Se o Idioma do texto for Ingl�s: 
%\notafolharosto{Final version}
%\notafolharosto{Vers\~ao revisada}
% ---
% dados complementares para CAPA e FOLHA DE ROSTO
% ---
\universidade{UNIVERSIDADE DE S\~AO PAULO}

% Idioma do texto em PORTUGU�S
\titulo{Modelo para TCC em \LaTeX\ utilizando o Pacote USPSC para o ICMC} 
\titleabstract{Model for TCC in \LaTeX\ using the USPSC Package to the ICMC}
\tituloadic{Model for TCC in \LaTeX\ using the USPSC Package to the ICMC}
\tituloresumo{Modelo para TCC em \LaTeX\ utilizando o Pacote USPSC para o ICMC}

% Idioma do texto em INGL�S
% 22/02/2017 - T�tulo para p�gina de rosto adicional
% para a vers�o em ingl�s, utilize os comandos abaixo
%\titulo{Model for TCC in \LaTeX\ using the USPSC Package to the ICMC}
%\titleabstract{Model for TCC in \LaTeX\ using the USPSC Package to the ICMC}
%\tituloadic{Modelo para TCC em \LaTeX\ utilizando o Pacote USPSC para o ICMC}
%\tituloresumo{Modelo para TCC em \LaTeX\ utilizando o Pacote USPSC para o ICMC}

\autor{Jos\'e da Silva}
\autorficha{Silva, Jos\'e da}
\autorabr{SILVA, J.}

\cutter{S856m}
% Para gerar a ficha catalogr�fica sem o C�digo Cutter, basta 
% incluir uma % na linha acima e tirar a % da linha abaixo
%\cutter{ }

\local{S\~ao Carlos}
\data{2021}
% Para o idioma portugu�s:
\renewcommand{\orientadorname}{Orientadora:}
\orientador{Profa. Dra. Elisa Gon\c{c}alves Rodrigues}
\orientadoradic{Advisor: Profa. Dra. Elisa Gon\c{c}alves Rodrigues}
\orientadorcorpoficha{orientadora Elisa Gon\c{c}alves Rodrigues}
\orientadorficha{Rodrigues, Elisa Gon\c{c}alves, orient}
%Para incluir o nome do(a) coorientados(a), inclua % nos 2 comandos acima e retire a % dos 2 comandos abaixo
%\orientadorcorpoficha{orientadora Elisa Gon\c{c}alves Rodrigues ;  co-orientador Jo\~ao Alves Serqueira}
%\orientadorficha{Rodrigues, Elisa Gon\c{c}alves, orient. II. Serqueira, Jo\~ao Alves, co-orient}


%Se o idioma for o ingl�s, inclua % nos comandos acima e exclua dos comandos abaixo
%\renewcommand{\orientadorname}{Advisor:}
%\orientador{Profa. Dra. Elisa Gon\c{c}alves Rodrigues}
%\orientadoradic{Orientadora: Profa. Dra. Elisa Gon\c{c}alves Rodrigues}
%\orientadorcorpoficha{orientadora Elisa Gon\c{c}alves Rodrigues}
%\orientadorficha{Rodrigues, Elisa Gon\c{c}alves, orient}
%Para incluir o nome do(a) coorientados(a), inclua % nos 2 comandos acima e retire a % dos 2 comandos abaixo
%\orientadorcorpoficha{orientadora Elisa Gon\c{c}alves Rodrigues ;  co-orientador Jo\~ao Alves Serqueira}
%\orientadorficha{Rodrigues, Elisa Gon\c{c}alves, orient. II. Serqueira, Jo\~ao Alves, co-orient}

% Quando houver Coorientador(a): 
% Para o idioma portugu�s:
% basta retirar  % antes de um dos comandos abaixo
%\newcommand{\coorientadorname}{Coorientador:}
%\newcommand{\coorientadorname}{Coorientadora:}
% Para o idoma ingl�s:
% basta retirar  % antes do comando abaixo
%\newcommand{\coorientadorname}{Coorientador:}

% Quando houver Coorientador(a), basta tirar a % utilizar o comando abaixo
%\newcommand{\coorientadorname}{Coadvisor:}
%Se houver co-orientador, inclua % antes das duas linhas (antes dos comandos \orientadorcorpoficha e \orientadorficha) 
%          e tire a % antes dos 3 comandos abaixo
%\coorientador{Prof. Dr. Jo\~ao Alves Serqueira}
%\coorientadoradic{ Co-orientador: Prof. Dr. Jo\~ao Alves Serqueira}
%\orientadorcorpoficha{orientadora Elisa Gon\c{c}alves Rodrigues ;  co-orientador Jo\~ao Alves Serqueira}
%\orientadorficha{Rodrigues, Elisa Gon\c{c}alves, orient. II. Serqueira, Jo\~ao Alves, co-orient}

%Para o idioma Ingl�s, retire a % antes da linha abaixo
%\renewcommand{\areaname}{Concentration area: }

\notaautorizacao{AUTORIZO A REPRODU\c{C}\~AO E DIVULGA\c{C}\~AO TOTAL OU PARCIAL DESTE TRABALHO, POR QUALQUER MEIO CONVENCIONAL OU ELETR\^ONICO PARA FINS DE ESTUDO E PESQUISA, DESDE QUE CITADA A FONTE.}
% Se o idioma for o ingl�s, inclua a % antes do campo \notaautorizacao acima e retire a % da linha abaixo
%\notaautorizacao{I AUTORIZE THE REPRODUCTION AND DISSEMINATION OF TOTAL OR PARTIAL COPIES OF THIS DOCUMENT, BY CONVENCIONAL OR ELECTRONIC MEDIA FOR STUDY OR RESEARCH PURPOSE, SINCE IT IS REFERENCED.}

\notabib{Ficha catalogr\'afica elaborada pela Biblioteca Prof. Achille Bassi, ICMC/USP, com os dados fornecidos pelo(a) autor(a)}

\newcommand{\programa}[1]{


% BCCe ==========================================================================
\ifthenelse{\equal{#1}{BCCe}}{
	\renewcommand{\areaname}{Concentration area:}
    \tipotrabalho{Monografia (Trabalho de Conclus\~ao de Curso)}
    \tipotrabalhoabs{Monograph (Conclusion Course Paper)}
    \renewcommand{\orientadorname}{Advisor:}
    %Para Orientadora, inclua % antes do comando acima e retire a % antes do comando abaixo
    %\renewcommand{\orientadorname}{Orientadora:}
    % Quando houver Coorientador, basta tirar a % utilizar o comando abaixo
   	%\newcommand{\coorientadorname}{Coorientador:}
    \renewcommand{\areaname}{Concentration area: }
    \area{Computer Science and Computational Mathematics}
    \areaadic{\'Area de concentra\c{c}\~ao: Ci\^encias de Computa\c{c}\~ao e Matem\'atica Computacional}
    %\opcao{Nome da Op��o em ingl�s}
    %\opcaoadic{Nome da Op��o em portugu�s}
    % O preambulo deve conter o tipo do trabalho, o objetivo, 
    % o nome da institui��o, a �rea de concentra��o e op��o quando houver
    % O preambulo deve conter o tipo do trabalho, o objetivo, 
	% o nome da institui��o, a �rea de concentra��o e op��o quando houver
	\preambulo{Conclusion course paper submitted to the Undergraduate Program of the Instituto de Ci\^encias Matem\'aticas e de Computa\c{c}\~ao, Universidade de S\~ao Paulo - ICMC/USP, in partial fulfillment of the  requirements for the degree of the Bachelor in Computer Science.}
	\preambuloadic{Trabalho de conclus\~ao de curso apresentado ao Programa de Gradua\c{c}\~ao, do Instituto de Ci\^encias Matem\'aticas e de Computa\c{c}\~ao, Universidade de S\~ao Paulo - ICMC/USP, como parte dos requisitos para obten\c{c}\~ao do t\'itulo de Bacharel em Ci\^encias de Computa\c{c}\~ao.}
	\notaficha{Monograph (Undergraduate in Computer Science)}
	\notacapaicmc{Conclusion Course Paper to the Undergraduate Program Bachelor's in\\ Computer Sciences}
    }{
 % BCCp ==========================================================================
 \ifthenelse{\equal{#1}{BCCp}}{
    \tipotrabalho{Monografia (Trabalho de Conclus\~ao de Curso)}
    \tipotrabalhoabs{Monograph (Conclusion Course Paper)}
    % Quando for Orientador, basta incluir uma % antes do comando abaixo
    \renewcommand{\orientadorname}{Orientadora:}
    % Quando for Coorientadora, basta tirar a % utilizar o comando abaixo
    %\newcommand{\coorientadorname}{Coorientador:}
    \area{Ci\^encias de Computa\c{c}\~ao e Matem\'atica Computacional}
    \areaadic{Concentration area: Computer Science and Computational Mathematics}
    %\opcao{Nome da Op��o em portugu�s}
    %\opcaoadic{Nome da Op��o em ingl�s}
    % O preambulo deve conter o tipo do trabalho, o objetivo, 
    % o nome da institui��o, a �rea de concentra��o e op��o quando houver
    % O preambulo deve conter o tipo do trabalho, o objetivo, 
    % o nome da institui��o, a �rea de concentra��o e op��o quando houver
    \preambulo{Trabalho de conclus\~ao de curso apresentado ao Programa de Gradua\c{c}\~ao, do Instituto de Ci\^encias Matem\'aticas e de Computa\c{c}\~ao, Universidade de S\~ao Paulo - ICMC/USP, como parte dos requisitos para obten\c{c}\~ao do t\'itulo de Bacharel em Ci\^encias de Computa\c{c}\~ao.}
    \preambuloadic{Conclusion course paper submitted to the Undergraduate Program of the Instituto de Ci\^encias Matem\'aticas e de Computa\c{c}\~ao, Universidade de S\~ao Paulo - ICMC/USP, in partial fulfillment of the  requirements for the degree of the Bachelor in Computer Science.}
    \notaficha{Monografia (Gradua\c{c}\~ao em Ci\^encias de Computa\c{c}\~ao)}
    \notacapaicmc{Trabalho de Conclus\~ao de Curso do Programa de Gradua\c{c}\~ao Bacharelado em\\ Ci\^encias de Computa\c{c}\~ao}
    }{
% BMe ==========================================================================
\ifthenelse{\equal{#1}{BMe}}{
	\renewcommand{\areaname}{Concentration area:}
	\tipotrabalho{Monografia (Trabalho de Conclus\~ao de Curso)}
	\tipotrabalhoabs{Monograph (Conclusion Course Paper)}
	%\renewcommand{\orientadorname}{Advisor:}
	%Para Orientadora, inclua % antes do comando acima e retire a % antes do comando abaixo
	%\renewcommand{\orientadorname}{Orientadora:}
	% Quando houver Coorientador, basta tirar a % utilizar o comando abaixo
	%\newcommand{\coorientadorname}{Coorientador:}
	\renewcommand{\areaname}{Concentration area: }
	\area{Mathematics}
	\areaadic{\'Area de concentra\c{c}\~ao: Matem\'atica}
	%\opcao{Nome da Op��o em ingl�s}
	%\opcaoadic{Nome da Op��o em portugu�s}
	% O preambulo deve conter o tipo do trabalho, o objetivo, 
	% o nome da institui��o, a �rea de concentra��o e op��o quando houver
	% O preambulo deve conter o tipo do trabalho, o objetivo, 
	% o nome da institui��o, a �rea de concentra��o e op��o quando houver
	\preambulo{Conclusion course  paper presented to the Undergraduate Program of the Instituto de Ci\^encias Matem\'aticas e de Computa\c{c}\~ao, Universidade de S\~ao Paulo - ICMC/USP, in partial fulfillment of the  requirements for the degree of the Bachelor in Mathematics.}
	\preambuloadic{Trabalho de conclus\~ao de curso apresentado ao Programa de Gradua\c{c}\~ao do Instituto de Ci\^encias Matem\'aticas e de Computa\c{c}\~ao, Universidade de S\~ao Paulo - ICMC/USP, como parte dos requisitos para obten\c{c}\~ao do t\'itulo de Bacharel em Matem\'atica.}	
	\notaficha{Monograph (Undergraduate in Mathematics)}
	\notacapaicmc{Conclusion Course Paper to the Undergraduate Program Bachelor's in\\ Mathematics}
    }{
% BMp ==========================================================================
\ifthenelse{\equal{#1}{BMp}}{
    \tipotrabalho{Monografia (Trabalho de Conclus\~ao de Curso)}
    \tipotrabalhoabs{Monograph (Conclusion Course Paper)}
    \area{Matem\'atica}
	\areaadic{Concentration area: Mathematics}
	%\opcao{Nome da Op��o em portugu�s}
	%\opcaoadic{Nome da Op��o em ingl�s}
	% O preambulo deve conter o tipo do trabalho, o objetivo, 
	% o nome da institui��o, a �rea de concentra��o e op��o quando houver
	% O preambulo deve conter o tipo do trabalho, o objetivo, 
	% o nome da institui��o, a �rea de concentra��o e op��o quando houver
	\preambulo{Trabalho de conclus\~ao de curso apresentado ao Programa de Gradua\c{c}\~ao do Instituto de Ci\^encias Matem\'aticas e de Computa\c{c}\~ao, Universidade de S\~ao Paulo - ICMC/USP, como parte dos requisitos para obten\c{c}\~ao do t\'itulo de Bacharel em Matem\'atica.}	
	\preambuloadic{Conclusion course  paper presented to the Undergraduate Program of the Instituto de Ci\^encias Matem\'aticas e de Computa\c{c}\~ao, Universidade de S\~ao Paulo - ICMC/USP, in partial fulfillment of the  requirements for the degree of the Bachelor in Mathematics.}
	\notaficha{Monografia (Gradua\c{c}\~ao em Matem\'atica)}
	\notacapaicmc{Trabalho de Conclus\~ao de Curso do Programa de Gradua\c{c}\~ao Bacharelado em\\ Matem\'atica}
    }{
% BMAe ==========================================================================
\ifthenelse{\equal{#1}{BMAe}}{
	\renewcommand{\areaname}{Concentration area:}
	\tipotrabalho{Monografia (Trabalho de Conclus\~ao de Curso)}
	\tipotrabalhoabs{Monograph (Conclusion Course Paper)}
	\area{Applied Mathematics and Scientific Computing}
	\areaadic{\'Area de concentra\c{c}\~ao: Matem\'atica Aplicada e Computa\c{c}\~ao Cient\'ifica}
	%\opcao{Nome da Op��o em ingl�s}
	%\opcaoadic{Nome da Op��o em portugu�s}
	% O preambulo deve conter o tipo do trabalho, o objetivo, 
	% o nome da institui��o, a �rea de concentra��o e op��o quando houver
	% O preambulo deve conter o tipo do trabalho, o objetivo, 
	% o nome da institui��o, a �rea de concentra��o e op��o quando houver
	\preambulo{Conclusion course paper presented to the Undergraduate Program of the Instituto de Ci\^encias Matem\'aticas e de Computa\c{c}\~ao, Universidade de S\~ao Paulo - ICMC/USP, in partial fulfillment of the requirements for the degree of the Bachelor in Applied Mathematics and Scientific Computing.}
	\preambuloadic{Trabalho de conclus\~ao de curso apresentado ao Programa de Gradua\c{c}\~ao, do Instituto de Ci\^encias Matem\'aticas e de Computa\c{c}\~ao, Universidade de S\~ao Paulo - ICMC/USP, como parte dos requisitos para obten\c{c}\~ao do t\'itulo de Bacharel em Matem\'atica Aplicada e Computa\c{c}\~ao Cient\'ifica.}	
	\notaficha{Monograph (Undergraduate in Applied Mathematics and Scientific Computing)}
	\notacapaicmc{Conclusion Course Paper to the Undergraduate Program Bachelor's in\\ Applied Mathematics and Scientific Computing}
    }{
% BMAp ==========================================================================
\ifthenelse{\equal{#1}{BMAp}}{
	\tipotrabalho{Monografia (Trabalho de Conclus\~ao de Curso)}
	\tipotrabalhoabs{Monograph (Conclusion Course Paper)}
	\area{Matem\'atica Aplicada e Computa\c{c}\~ao Cient\'ifica}
	\areaadic{Concentration area: Applied Mathematics and Scientific Computing}
	%\opcao{Nome da Op��o em portugu�s}
	%\opcaoadic{Nome da Op��o em ingl�s}
	% O preambulo deve conter o tipo do trabalho, o objetivo, 
	% o nome da institui��o, a �rea de concentra��o e op��o quando houver
	% O preambulo deve conter o tipo do trabalho, o objetivo, 
	% o nome da institui��o, a �rea de concentra��o e op��o quando houver
	\preambulo{Trabalho de conclus\~ao de curso apresentado ao Programa de Gradua\c{c}\~ao, do Instituto de Ci\^encias Matem\'aticas e de Computa\c{c}\~ao, Universidade de S\~ao Paulo - ICMC/USP, como parte dos requisitos para obten\c{c}\~ao do t\'itulo de Bacharel em Matem\'atica Aplicada e Computa\c{c}\~ao Cient\'ifica.}	
	\preambuloadic{Conclusion course paper presented to the Undergraduate Program of the Instituto de Ci\^encias Matem\'aticas e de Computa\c{c}\~ao, Universidade de S\~ao Paulo - ICMC/USP, in partial fulfillment of the  requirements for the degree of the Bachelor in Applied Mathematics and Scientific Computing.}
	\notaficha{Monografia (Gradua\c{c}\~ao em Matem\'atica Aplicada e Computa\c{c}\~ao Cient\'ifica)}
	\notacapaicmc{Trabalho de Conclus\~ao de Curso do Programa de Gradua\c{c}\~ao Bacharelado em\\ Matem\'atica Aplicada e Computa\c{c}\~ao Cient\'ifica}
    }{
% LMe ==========================================================================
\ifthenelse{\equal{#1}{LMe}}{
	\renewcommand{\areaname}{Concentration area:}
	\tipotrabalho{Monografia (Trabalho de Conclus\~ao de Curso)}
	\tipotrabalhoabs{Monograph (Conclusion Course Paper)}
	\area{Mathematics}
	\areaadic{\'Area de concentra\c{c}\~ao: Matem\'atica}
	%\opcao{Nome da Op��o em ingl�s}
	%\opcaoadic{Nome da Op��o em portugu�s}
	% O preambulo deve conter o tipo do trabalho, o objetivo, 
	% o nome da institui��o, a �rea de concentra��o e op��o quando houver
	% O preambulo deve conter o tipo do trabalho, o objetivo, 
	% o nome da institui��o, a �rea de concentra��o e op��o quando houver
	\preambulo{Conclusion course paper presented to the Undergraduate Program of the Instituto de Ci\^encias Matem\'aticas e de Computa\c{c}\~ao, Universidade de S\~ao Paulo - ICMC/USP, in partial fulfillment of the  requirements for the degree of the Licentiate in Mathematics.}
	\preambuloadic{Trabalho de conclus\~ao de curso apresentado ao Programa de Gradua\c{c}\~ao, do Instituto de Ci\^encias Matem\'aticas e de Computa\c{c}\~ao, Universidade de S\~ao Paulo - ICMC/USP, como parte dos requisitos para obten\c{c}\~ao do t\'itulo de Licenciado em Matem\'atica.}	
	\notaficha{Monograph (Degree in Mathematics)}
	\notacapaicmc{Conclusion Course Paper to the Undergraduate Program Licenciate in\\ Mathematics}
    }{
% LMp ==========================================================================
\ifthenelse{\equal{#1}{LMp}}{
	\tipotrabalho{Monografia (Trabalho de Conclus\~ao de Curso)}
	\tipotrabalhoabs{Monograph (Conclusion Course Paper)}
	\area{Matem\'atica}
	\areaadic{Concentration area: Mathematics}
	%\opcao{Nome da Op��o em portugu�s}
	%\opcaoadic{Nome da Op��o em ingl�s}
	% O preambulo deve conter o tipo do trabalho, o objetivo, 
	% o nome da institui��o, a �rea de concentra��o e op��o quando houver
	% O preambulo deve conter o tipo do trabalho, o objetivo, 
	% o nome da institui��o, a �rea de concentra��o e op��o quando houver
	\preambulo{Trabalho de conclus\~ao de curso apresentado ao Programa de Gradua\c{c}\~ao, do Instituto de Ci\^encias Matem\'aticas e de Computa\c{c}\~ao, Universidade de S\~ao Paulo - ICMC/USP, como parte dos requisitos para obten\c{c}\~ao do t\'itulo de Licenciado em Matem\'atica.}	
	\preambuloadic{Conclusion course paper presented to the Undergraduate Program of the Instituto de Ci\^encias Matem\'aticas e de Computa\c{c}\~ao, Universidade de S\~ao Paulo - ICMC/USP, in partial fulfillment of the  requirements for the degree of the Licentiate in Mathematics.}
	\notaficha{Monografia (Licenciatura em Matem\'atica)}
	\notacapaicmc{Trabalho de Conclus\~ao de Curso do Programa de Gradua\c{c}\~ao Licenciatura em\\ Matem\'atica}
    }{
% BCDe ==========================================================================
\ifthenelse{\equal{#1}{BCDe}}{
	\renewcommand{\areaname}{Concentration area:}
	\tipotrabalho{Monografia (Trabalho de Conclus\~ao de Curso)}
	\tipotrabalhoabs{Monograph (Conclusion Course Paper)}
	\area{Data Science}
	\areaadic{\'Area de concentra\c{c}\~ao: Ci\^encia de Dados}
	%\opcao{Nome da Op��o em ingl�s}
	%\opcaoadic{Nome da Op��o em portugu�s}
	% O preambulo deve conter o tipo do trabalho, o objetivo, 
	% o nome da institui��o, a �rea de concentra��o e op��o quando houver
	% O preambulo deve conter o tipo do trabalho, o objetivo, 
	% o nome da institui��o, a �rea de concentra��o e op��o quando houver
	\preambulo{Conclusion course paper presented to the Undergraduate Program of the Instituto de Ci\^encias Matem\'aticas e de Computa\c{c}\~ao, Universidade de S\~ao Paulo - ICMC/USP, in partial fulfillment of the  requirements for the degree of the Bachelor in Data Science.}
	\preambuloadic{Trabalho de conclus\~ao de curso apresentado ao Programa de Gradua\c{c}\~ao do Instituto de Ci\^encias Matem\'aticas e de Computa\c{c}\~ao, Universidade de S\~ao Paulo - ICMC/USP, como parte dos requisitos para obten\c{c}\~ao do t\'itulo de Bacharel em Ci\^encia de Dados.}	
	\notaficha{Monograph (Undergraduate in Statistics and Data Science)}
	\notacapaicmc{Conclusion Course Paper to the Undergraduate Program Bachelor's in\\ Data Science}
    }{
% BCDp ==========================================================================
\ifthenelse{\equal{#1}{BCDp}}{
    \tipotrabalho{Monografia (Trabalho de Conclus\~ao de Curso)}
    \tipotrabalhoabs{Monograph (Conclusion Course Paper)}
	\area{Ci\^encia de Dados}
	\areaadic{Concentration area: Data Science}
	%\opcao{Nome da Op��o em portugu�s}
	%\opcaoadic{Nome da Op��o em ingl�s}
	% O preambulo deve conter o tipo do trabalho, o objetivo, 
	% o nome da institui��o, a �rea de concentra��o e op��o quando houver
	% O preambulo deve conter o tipo do trabalho, o objetivo, 
	% o nome da institui��o, a �rea de concentra��o e op��o quando houver
	\preambulo{Trabalho de conclus\~ao de curso apresentado ao Programa de Gradua\c{c}\~ao do Instituto de Ci\^encias Matem\'aticas e de Computa\c{c}\~ao, Universidade de S\~ao Paulo - ICMC/USP, como parte dos requisitos para obten\c{c}\~ao do t\'itulo de Bacharel em Ci\^encia de Dados.}	
	\preambuloadic{Conclusion course paper presented to the Undergraduate Program of the Instituto de Ci\^encias Matem\'aticas e de Computa\c{c}\~ao, Universidade de S\~ao Paulo - ICMC/USP, in partial fulfillment of the requirements for the degree of the Bachelor in Data Science.}
	\notaficha{Monografia (Gradua\c{c}\~ao em Ci\^encia de Dados)}
	\notacapaicmc{Trabalho de Conclus\~ao de Curso do Programa de Gradua\c{c}\~ao Bacharelado em\\ Ci\^encia de Dados}
    }{
% EBECDe ==========================================================================
\ifthenelse{\equal{#1}{EBECDe}}{
	\renewcommand{\areaname}{Concentration area:}
	\tipotrabalho{Projeto de gradua\c{c}\~ao (Trabalho de Conclus\~ao de Curso)}
	\tipotrabalhoabs{Graduation project (Conclusion Course Paper)}
	\area{Statistics and Data Science}
	\areaadic{\'Area de concentra\c{c}\~ao: Estat\'istica e Ci\^encia de Dados}
	%\opcao{Nome da Op��o em ingl�s}
	%\opcaoadic{Nome da Op��o em portugu�s}
	% O preambulo deve conter o tipo do trabalho, o objetivo, 
	% o nome da institui��o, a �rea de concentra��o e op��o quando houver
	% O preambulo deve conter o tipo do trabalho, o objetivo, 
	% o nome da institui��o, a �rea de concentra��o e op��o quando houver
	\preambulo{Graduation project presented to the Undergraduate Program of the Instituto de Ci\^encias Matem\'aticas e de Computa\c{c}\~ao, Universidade de S\~ao Paulo - ICMC/USP, in partial fulfillment of the requirements for the degree of the Bachelor in Statistics and Data Science.}
	\preambuloadic{Projeto de gradua\c{c}\~ao apresentado ao Programa de Gradua\c{c}\~ao do Instituto de Ci\^encias Matem\'aticas e de Computa\c{c}\~ao, Universidade de S\~ao Paulo - ICMC/USP, como parte dos requisitos para obten\c{c}\~ao do t\'itulo de Bacharel em Estat\'istica e Ci\^encia de Dados.}
	\notaficha{Graduation Project (Undergraduate in Statistics and Data Science)}
	\notacapaicmc{Graduation Project to the Undergraduate Program Bachelor's in\\ Statistics and Data Science}	
    }{
% EBECDp  ==========================================================================
\ifthenelse{\equal{#1}{EBECDp}}{
	\tipotrabalho{Projeto de gradua\c{c}\~ao (Trabalho de Conclus\~ao de Curso)}
	\tipotrabalhoabs{Graduation project (Conclusion Course Paper)}
	\area{Estat\'istica e Ci\^encia de Dados}
	\areaadic{Concentration area: Statistics and Data Science}
	%\opcao{Nome da Op��o em portugu�s}
	%\opcaoadic{Nome da Op��o em ingl�s}
	% O preambulo deve conter o tipo do trabalho, o objetivo, 
	% o nome da institui��o, a �rea de concentra��o e op��o quando houver
	% O preambulo deve conter o tipo do trabalho, o objetivo, 
	% o nome da institui��o, a �rea de concentra��o e op��o quando houver
	\preambulo{Projeto de gradua\c{c}\~ao apresentado ao Programa de Gradua\c{c}\~ao do Instituto de Ci\^encias Matem\'aticas e de Computa\c{c}\~ao, Universidade de S\~ao Paulo - ICMC/USP, como parte dos requisitos para obten\c{c}\~ao do t\'itulo de Bacharel em Estat\'istica e Ci\^encia de Dados.}	
	\preambuloadic{Graduation project presented to the Undergraduate Program of the Instituto de Ci\^encias Matem\'aticas e de Computa\c{c}\~ao, Universidade de S\~ao Paulo - ICMC/USP, in partial fulfillment of the requirements for the degree of the Bachelor in Statistics and Data Science.}
	\notaficha{Projeto de gradua\c{c}\~ao (Gradua\c{c}\~ao em Estat\'istica e Ci\^encia de Dados)}
	\notacapaicmc{Projeto de gradua\c{c}\~ao do Programa de Gradua\c{c}\~ao Bacharelado em\\ Estat\'istica e Ci\^encia de Dados}
    }{
% BECDe ==========================================================================
\ifthenelse{\equal{#1}{BECDe}}{
	\renewcommand{\areaname}{Concentration area:}
	\tipotrabalho{Monografia (Trabalho de Conclus\~ao de Curso)}
	\tipotrabalhoabs{Monograph (Conclusion Course Paper)}
	\area{Statistics and Data Science}
	\areaadic{\'Area de concentra\c{c}\~ao: Estat\'istica  e Ci\^encia de Dados}
	%\opcao{Nome da Op��o em ingl�s}
	%\opcaoadic{Nome da Op��o em portugu�s}
	% O preambulo deve conter o tipo do trabalho, o objetivo, 
	% o nome da institui��o, a �rea de concentra��o e op��o quando houver
	% O preambulo deve conter o tipo do trabalho, o objetivo, 
	% o nome da institui��o, a �rea de concentra��o e op��o quando houver
	\preambulo{Conclusion course paper presented to the Undergraduate Program of the Instituto de Ci\^encias Matem\'aticas e de Computa\c{c}\~ao, Universidade de S\~ao Paulo - ICMC/USP, in partial fulfillment of the requirements for the degree of the Bachelor in Statistics and Data Science.}
	\preambuloadic{Trabalho de conclus\~ao de curso apresentado ao Programa de Gradua\c{c}\~ao do Instituto de Ci\^encias Matem\'aticas e de Computa\c{c}\~ao, Universidade de S\~ao Paulo - ICMC/USP, como parte dos requisitos para obten\c{c}\~ao do t\'itulo de Bacharel em  Estat\'istica  e Ci\^encia de Dados.}	
	\notaficha{Monograph (Undergraduate in Statistics and Data Science)}
	\notacapaicmc{Conclusion Course Paper to the Undergraduate Program Bachelor's in\\ Statistics and Data Science}
    }{
% BECDp ==========================================================================
\ifthenelse{\equal{#1}{BECDp}}{
	\tipotrabalho{Monografia (Trabalho de Conclus\~ao de Curso)}
	\tipotrabalhoabs{Monograph (Conclusion Course Paper)}
	\area{Estat\'istica  e Ci\^encia de Dados}
	\areaadic{Concentration area: Statistics and Data Science}
	%\opcao{Nome da Op��o em portugu�s}
	%\opcaoadic{Nome da Op��o em ingl�s}
	% O preambulo deve conter o tipo do trabalho, o objetivo, 
	% o nome da institui��o, a �rea de concentra��o e op��o quando houver
	% O preambulo deve conter o tipo do trabalho, o objetivo, 
	% o nome da institui��o, a �rea de concentra��o e op��o quando houver
	\preambulo{Trabalho de conclus\~ao de curso apresentado ao Programa de Gradua\c{c}\~ao do Instituto de Ci\^encias Matem\'aticas e de Computa\c{c}\~ao, Universidade de S\~ao Paulo - ICMC/USP, como parte dos requisitos para obten\c{c}\~ao do t\'itulo de Bacharel em  Estat\'istica  e Ci\^encia de Dados.}	
	\preambuloadic{Conclusion course paper presented to the Undergraduate Program of the Instituto de Ci\^encias Matem\'aticas e de Computa\c{c}\~ao, Universidade de S\~ao Paulo - ICMC/USP, in partial fulfillment of the requirements for the degree of the Bachelor in Statistics and Data Science.}
	\notaficha{Monografia (Gradua\c{c}\~ao em Estat\'istica e Ci\^encia de Dados)}
	\notacapaicmc{Trabalho de Conclus\~ao de Curso do Programa de Gradua\c{c}\~ao Bacharelado em\\ Estat\'istica e Ci\^encia de Dados}
    }{
% BSIe ==========================================================================
\ifthenelse{\equal{#1}{BSIe}}{
	\renewcommand{\areaname}{Concentration area:}
	\tipotrabalho{Monografia (Trabalho de Conclus\~ao de Curso)}
	\tipotrabalhoabs{Monograph (Conclusion Course Paper)}
	\area{Information Systems}
	\areaadic{\'Area de concentra\c{c}\~ao: Sistemas de Informa\c{c}\~ao}
	%\opcao{Nome da Op��o em ingl�s}
	%\opcaoadic{Nome da Op��o em portugu�s}
	% O preambulo deve conter o tipo do trabalho, o objetivo, 
	% o nome da institui��o, a �rea de concentra��o e op��o quando houver
	% O preambulo deve conter o tipo do trabalho, o objetivo, 
	% o nome da institui��o, a �rea de concentra��o e op��o quando houver
	\preambulo{Conclusion course paper presented to the Undergraduate Program of the Instituto de Ci\^encias Matem\'aticas e de Computa\c{c}\~ao, Universidade de S\~ao Paulo - ICMC/USP, in partial fulfillment of the requirements for the degree of the Bachelor in Information Systems.}
	\preambuloadic{Trabalho de conclus\~ao de curso apresentado ao Programa de Gradua\c{c}\~ao do Instituto de Ci\^encias Matem\'aticas e de Computa\c{c}\~ao, Universidade de S\~ao Paulo - ICMC/USP, como parte dos requisitos para obten\c{c}\~ao do t\'itulo de Bacharel em Sistemas de Informa\c{c}\~ao.}	
	\notaficha{Monograph (Degree in Information Systems)}
	\notacapaicmc{Conclusion Course Paper to the Undergraduate Program Bachelor's in\\ Information Systems}	
    }{
% BSIp ==========================================================================
\ifthenelse{\equal{#1}{BSIp}}{
	\tipotrabalho{Monografia (Trabalho de Conclus\~ao de Curso)}
	\tipotrabalhoabs{Monograph (Conclusion Course Paper)}
	\area{Sistemas de Informa\c{c}\~ao}
	\areaadic{Concentration area: Information Systems}
	%\opcao{Nome da Op��o em portugu�s}
	%\opcaoadic{Nome da Op��o em ingl�s}
	% O preambulo deve conter o tipo do trabalho, o objetivo, 
	% o nome da institui��o, a �rea de concentra��o e op��o quando houver
	% O preambulo deve conter o tipo do trabalho, o objetivo, 
	% o nome da institui��o, a �rea de concentra��o e op��o quando houver
	\preambulo{Trabalho de conclus\~ao de curso apresentado ao Programa de Gradua\c{c}\~ao do Instituto de Ci\^encias Matem\'aticas e de Computa\c{c}\~ao, Universidade de S\~ao Paulo - ICMC/USP, como parte dos requisitos para obten\c{c}\~ao do t\'itulo de Bacharel em Sistemas de Informa\c{c}\~ao.}	
	\preambuloadic{Conclusion course paper presented to the Undergraduate Program of the Instituto de Ci\^encias Matem\'aticas e de Computa\c{c}\~ao, Universidade de S\~ao Paulo - ICMC/USP, in partial fulfillment of the requirements for the degree of the Bachelor in Information Systems.}
	\notaficha{Monografia (Gradua\c{c}\~ao em Sistemas de Informa\c{c}\~ao}
	\notacapaicmc{Trabalho de Conclus\~ao de Curso do Programa de Gradua\c{c}\~ao Bacharelado em\\ Sistemas de Informa\c{c}\~ao}
    }{
% ECe ==========================================================================
\ifthenelse{\equal{#1}{ECe}}{
	\renewcommand{\areaname}{Concentration area:}
	\tipotrabalho{Monografia (Trabalho de Conclus\~ao de Curso)}
	\tipotrabalhoabs{Monograph (Conclusion Course Paper)}
	\area{Computer Engineering}
	\areaadic{\'Area de concentra\c{c}\~ao: Engenharia de Computa\c{c}\~ao}
	\instituicao{Instituto de Ci\^encias Matem\'aticas e de Computa\c{c}\~ao, Universidade de S\~ao Paulo; Escola de Engenharia de S\~ao Carlos, Universidade de S\~ao Paulo}	
	%\opcao{Nome da Op��o em ingl�s}
	%\opcaoadic{Nome da Op��o em portugu�s}
	% O preambulo deve conter o tipo do trabalho, o objetivo, 
	% o nome da institui��o, a �rea de concentra��o e op��o quando houver
	% O preambulo deve conter o tipo do trabalho, o objetivo, 
	% o nome da institui��o, a �rea de concentra��o e op��o quando houver
	\preambulo{Conclusion course paper presented to the Undergraduate Program of the Instituto de Ci\^encias Matem\'aticas e de Computa\c{c}\~ao, and the Escola de Engenharia de S\~ao Carlos, Universidade de S\~ao Paulo, in partial fulfillment of the requirements for the degree of the Computer Engineer.}
	\preambuloadic{Trabalho de conclus\~ao de curso apresentado ao Programa de Gradua\c{c}\~ao do Instituto de Ci\^encias Matem\'aticas e de Computa\c{c}\~ao e Escola de Engenharia de S\~ao Carlos, Universidade de S\~ao Paulo - ICMC - EESC/USP, como parte dos requisitos para obten\c{c}\~ao do t\'itulo de Engenheiro de Computa\c{c}\~ao.}
	\notaficha{Monograph (Degree in Computer Engineering)}
	\notacapaicmc{Conclusion Course Paper to the Undergraduate Program Bachelor's in\\ Computer Engineering}	
    }{
% ECp ==========================================================================
\ifthenelse{\equal{#1}{ECp}}{
	\tipotrabalho{Monografia (Trabalho de Conclus\~ao de Curso)}
	\tipotrabalhoabs{Monograph (Conclusion Course Paper)}
	\area{Engenharia de Computa\c{c}\~ao}
	\areaadic{Concentration area: Computer Engineering}
	\instituicao{Instituto de Ci\^encias Matem\'aticas e de Computa\c{c}\~ao, Universidade de S\~ao Paulo; Escola de Engenharia de S\~ao Carlos, Universidade de S\~ao Paulo}
	%\opcao{Nome da Op��o em portugu�s}
	%\opcaoadic{Nome da Op��o em ingl�s}
	% O preambulo deve conter o tipo do trabalho, o objetivo, 
	% o nome da institui��o, a �rea de concentra��o e op��o quando houver
	% O preambulo deve conter o tipo do trabalho, o objetivo, 
	% o nome da institui��o, a �rea de concentra��o e op��o quando houver
	\preambulo{Trabalho de conclus\~ao de curso apresentado ao Programa de Gradua\c{c}\~ao do Instituto de Ci\^encias Matem\'aticas e de Computa\c{c}\~ao e Escola de Engenharia de S\~ao Carlos, Universidade de S\~ao Paulo - ICMC - EESC/USP, como parte dos requisitos para obten\c{c}\~ao do t\'itulo de Engenheiro de Computa\c{c}\~ao.}
	\preambuloadic{Conclusion course paper presented to the Undergraduate Program of the Instituto de Ci\^encias Matem\'aticas e de Computa\c{c}\~ao, and the Escola de Engenharia de S\~ao Carlos, Universidade de S\~ao Paulo, in partial fulfillment of the requirements for the degree of the Computer Engineer.}
	\notaficha{Monografia (Gradua\c{c}\~ao em Engenharia de Computa\c{c}\~ao)}
	\notacapaicmc{Trabalho de Conclus\~ao de Curso do Programa de Gradua\c{c}\~ao Bacharelado em\\ Engenharia de Computa\c{c}\~ao}
	}{
% Outros
	\tipotrabalho{Monografia (Trabalho de Conclus\~ao de Curso)}
	\tipotrabalhoabs{Monograph (Conclusion Course Paper)}
	\area{Nome da \'Area}
	\opcao{Nome da Op\c{c}\~ao}
    % O preambulo deve conter o tipo do trabalho, o objetivo, 
	% o nome da institui��o, a �rea de concentra��o e op��o quando houver				
	\preambulo{Trabalho de conclus\~ao de curso apresentado ao Instituto de Ci\^encias Matem\'aticas e de Computa\c{c}\~ao, Universidade de S\~ao Paulo - ICMC/USP, como parte dos requisitos para obten\c{c}\~ao do t\'itulo de ...}
	\preambuloadic{Term paper submitted to the Instituto de Ci\^encias Matem\'aticas e de Computa\c{c}\~ao, Universidade de S\~ao Paulo - ICMC/USP, in partial fulfillment of the requirements for the degree of the ...}
	\notaficha{Monografia (Gradua\c{c}\~ao em ...)}
	\notacapaicmc{Trabalho de Conclus\~ao de Curso do Programa de Gradua\c{c}\~ao em ...}
    }}}}}}}}}}}}}}}}}}}			







    
        }{
% IFSC ===========================================================================
        \ifthenelse{\equal{#1}{IFSC}}{
        %% USPSC-pre-textual-IFSC.tex
%% Camandos para defini��o do tipo de documento (tese ou disserta��o), �rea de concentra��o, op��o, pre�mbulo, titula��o 
%% referentes ao Programa de P�s-Gradua��o o IFSC
\instituicao{Instituto de F\'isica de S\~ao Carlos, Universidade de S\~ao Paulo}
\unidade{INSTITUTO DE F\'ISICA DE S\~AO CARLOS}
\unidademin{Instituto de F\'isica de S\~ao Carlos}
\universidademin{Universidade de S\~ao Paulo}

\notafolharosto{Vers\~ao original}
%Para vers�o original em ingl�s, comente do comando/declara��o 
%     acima(inclua % antes do comando acima) e tire a % do 
%     comando/declara��o abaixo no idioma do texto
%\notafolharosto{Original version}
 
%Para vers�o corrigida, comente do comando/declara��o da 
%     vers�o original acima (inclua % antes do comando acima) 
%     e tire a % do comando/declara��o de um dos comandos 
%     abaixo em conformidade com o idioma do texto
%\notafolharosto{Vers\~ao corrigida \\(Vers\~ao original dispon\'ivel na Unidade que aloja o Programa)}
%\notafolharosto{Corrected version \\(Original version available on the Program Unit)}

% Para utilizar Sistema Num�rico diferente da ABNT
% Numera��o entre colchetes:
%\citebrackets[] 
% Numera��o entre parenteses:

% ---
% dados complementares para CAPA e FOLHA DE ROSTO
% ---
\universidade{UNIVERSIDADE DE S\~AO PAULO}
\titulo{Modelo para teses e disserta\c{c}\~oes em \LaTeX\ utilizando o Pacote USPSC para o IFSC}
% Se o idioma do texto for o ingl�s, inclua % antes da linha acima e retire a % da linha abaixo
%\titulo{Model for thesis and dissertations in \LaTeX\ using the USPSC Package to the IFSC}

\titleabstract{Model for thesis and dissertations in \LaTeX\ using the USPSC Package to the IFSC}
\tituloresumo{Modelo para teses e disserta\c{c}\~oes em \LaTeX\ utilizando o Pacote USPSC para o IFSC}
\autor{Jos\'e da Silva}
\autorficha{Silva, Jos\'e da}
\autorabr{SILVA, J.}

\cutter{S856m}
% Para gerar a ficha catalogr�fica sem o C�digo Cutter, basta 
% incluir uma % na linha acima e tirar a % da linha abaixo
%\cutter{ }

\local{S\~ao Carlos}
\data{2021}
% Quando for Orientador, basta incluir uma % antes do comando abaixo
\renewcommand{\orientadorname}{Orientadora:}
% Quando for Coorientadora, basta tirar a % utilizar o comando abaixo
%\newcommand{\coorientadorname}{Coorientador:}
\orientador{Profa. Dra. Elisa Gon\c{c}alves Rodrigues}
\orientadorcorpoficha{orientadora Elisa Gon\c{c}alves Rodrigues}
\orientadorficha{Rodrigues, Elisa Gon\c{c}alves, orient}
%Se houver co-orientador, inclua % antes das duas linhas (antes dos comandos \orientadorcorpoficha e \orientadorficha) 
%          e tire a % antes dos 3 comandos abaixo
%\coorientador{Prof. Dr. Jo\~ao Alves Serqueira}
%\orientadorcorpoficha{orientadora Elisa Gon\c{c}alves Rodrigues ;  co-orientador Jo\~ao Alves Serqueira}
%\orientadorficha{Rodrigues, Elisa Gon\c{c}alves, orient. II. Serqueira, Jo\~ao Alves, co-orient}

\notaautorizacao{AUTORIZO A REPRODU\c{C}\~AO E DIVULGA\c{C}\~AO TOTAL OU PARCIAL DESTE TRABALHO, POR QUALQUER MEIO CONVENCIONAL OU ELETR\^ONICO PARA FINS DE ESTUDO E PESQUISA, DESDE QUE CITADA A FONTE.}
% Se o idioma for o ingl�s, inclua a % antes do campo \notaautorizacao acima e retire a % da linha abaixo
%\notaautorizacao{I AUTORIZE THE REPRODUCTION AND DISSEMINATION OF TOTAL OR PARTIAL COPIES OF THIS DOCUMENT, BY CONVENCIONAL OR ELECTRONIC MEDIA FOR STUDY OR RESEARCH PURPOSE, SINCE IT IS REFERENCED.}
\notabib{Ficha catalogr\'afica revisada pelo Servi\c{c}o de Biblioteca e Informa\c{c}\~ao Prof. Bernhard Gross, com os dados fornecidos pelo(a) autor(a)}

\newcommand{\programa}[1]{

% DFAp ==========================================================================
\ifthenelse{\equal{#1}{DFAp}}{
    \tipotrabalho{Tese (Doutorado em Ci\^encias)}
    \tipotrabalhoabs{Thesis (Doctor in Science)}
    \area{F\'isica Aplicada}
	%\opcao{Nome da Op��o}
    % O preambulo deve conter o tipo do trabalho, o objetivo, 
	% o nome da institui��o, a �rea de concentra��o e op��o quando houver
	\preambulo{Tese apresentada ao Programa de P\'os-Gradua\c{c}\~ao em F\'isica do Instituto de F\'isica de S\~ao Carlos da Universidade de S\~ao Paulo, para obten\c{c}\~ao do t\'itulo de Doutor em Ci\^encias.}
	\notaficha{Tese (Doutorado - Programa de P\'os-Gradua\c{c}\~ao em F\'isica Aplicada)}
    }{
% DFAe ==========================================================================
\ifthenelse{\equal{#1}{DFAe}}{
	\renewcommand{\areaname}{Concentration area:}
	\renewcommand{\opcaoname}{Option:}
	\renewcommand{\orientadorname}{Advisor:}
	\tipotrabalho{Tese (Doutorado em Ci\^encias)}
	\tipotrabalhoabs{Thesis (Doctor in Science)}
	\area{Applied Physics}
	%\opcao{Nome da Op��o}
   	% O preambulo deve conter o tipo do trabalho, o objetivo, 
   	% o nome da institui��o, a �rea de concentra��o e op��o quando houver
   	\preambulo{Thesis presented to the Graduate Program in Physics at the Instituto de F\'isica de S\~ao Carlos da Universidade de S\~ao Paulo, to obtain the degree of Doctor in Science.}
   	\notaficha{Thesis (Doctorate - Graduate Program in Applied Physics)}
    }{    
% MFAp ===========================================================================
\ifthenelse{\equal{#1}{MFAp}}{
	\tipotrabalho{Disserta\c{c}\~ao (Mestrado em Ci\^encias)}
	\tipotrabalhoabs{Dissertation (Master in Science)}
	\area{F\'isica Aplicada}
	%\opcao{Nome da Op��o}
	% O preambulo deve conter o tipo do trabalho, o objetivo, 
	% o nome da institui��o, a �rea de concentra��o e op��o quando houver
	\preambulo{Disserta\c{c}\~ao apresentada ao Programa de P\'os-Gradua\c{c}\~ao em F\'isica do Instituto de F\'isica de S\~ao Carlos da Universidade de S\~ao Paulo, para obten\c{c}\~ao do t\'itulo de Mestre em Ci\^encias.}
	\notaficha{Disserta\c{c}\~ao (Mestrado - Programa de P\'os-Gradua\c{c}\~ao em F\'isica Aplicada)}
    }{
% MFAe ===========================================================================
\ifthenelse{\equal{#1}{MFAe}}{
	\renewcommand{\areaname}{Concentration area:}
	\renewcommand{\opcaoname}{Option:}
	\renewcommand{\orientadorname}{Advisor:}
	\tipotrabalho{Disserta\c{c}\~ao (Mestrado em Ci\^encias)}
	\tipotrabalhoabs{Dissertation (Master in Science)}
	\area{Applied Physics}
	%\opcao{Nome da Op��o}
	% O preambulo deve conter o tipo do trabalho, o objetivo, 
	% o nome da institui��o, a �rea de concentra��o e op��o quando houver
	\preambulo{Dissertation presented to the Graduate Program in Physics at the Instituto de F\'isica de S\~ao Carlos da Universidade de S\~ao Paulo, to obtain the degree of  Master in Science.}
	\notaficha{Dissertation (Master - Graduate Program in Applied Physics)}	
    }{        
% DFAFCp ==========================================================================
\ifthenelse{\equal{#1}{DFAFCp}}{
    \tipotrabalho{Tese (Doutorado em Ci\^encias)}
    \tipotrabalhoabs{Thesis (Doctor in Science)}
    \area{F\'isica Aplicada}
    \opcao{F\'isica Computacional}
    % O preambulo deve conter o tipo do trabalho, o objetivo, 
	% o nome da institui��o, a �rea de concentra��o e op��o quando houver
	\preambulo{Tese apresentada ao Programa de P\'os-Gradua\c{c}\~ao em F\'isica do Instituto de F\'isica de S\~ao Carlos da Universidade de S\~ao Paulo, para obten\c{c}\~ao do t\'itulo de Doutor em Ci\^encias.}
	\notaficha{Tese (Doutorado - Programa de P\'os-Gradua\c{c}\~ao em F\'isica Aplicada)}
    }{
% DFAFCe ==========================================================================
\ifthenelse{\equal{#1}{DFAFCe}}{
    \renewcommand{\areaname}{Concentration area:}
    \renewcommand{\opcaoname}{Option:}
    \renewcommand{\orientadorname}{Advisor:}
    \tipotrabalho{Tese (Doutorado em Ci\^encias)}
    \tipotrabalhoabs{Thesis (Doctor in Science)}
   	\area{Applied Physics}
    \opcao{Computational Physics}
    % O preambulo deve conter o tipo do trabalho, o objetivo, 
    % o nome da institui��o, a �rea de concentra��o e op��o quando houver
    \preambulo{Thesis presented to the Graduate Program in Physics at the Instituto de F\'isica de S\~ao Carlos da Universidade de S\~ao Paulo, to obtain the degree of Doctor in Science.}
    \notaficha{Thesis (Doctorate - Graduate Program in Applied Physics)}
    }{
% MFAFCp ===========================================================================
\ifthenelse{\equal{#1}{MFAFCp}}{
	\tipotrabalho{Disserta\c{c}\~ao (Mestrado em Ci\^encias)}
	\tipotrabalhoabs{Dissertation (Master in Science)}
	\area{F\'isica Aplicada}
	\opcao{F\'isica Computacional}
	% O preambulo deve conter o tipo do trabalho, o objetivo, 
	% o nome da institui��o, a �rea de concentra��o e op��o quando houver
	\preambulo{Disserta\c{c}\~ao apresentada ao Programa de P\'os-Gradua\c{c}\~ao em F\'isica do Instituto de F\'isica de S\~ao Carlos da Universidade de S\~ao Paulo, para obten\c{c}\~ao do t\'itulo de Mestre em Ci\^encias.}
	\notaficha{Disserta\c{c}\~ao (Mestrado - Programa de P\'os-Gradua\c{c}\~ao em F\'isica Aplicada)}
    }{
% MFAFCe ===========================================================================
\ifthenelse{\equal{#1}{MFAFCe}}{
	\renewcommand{\areaname}{Concentration area:}
	\renewcommand{\opcaoname}{Option:}
	\renewcommand{\orientadorname}{Advisor:}
	\tipotrabalho{Disserta\c{c}\~ao (Mestrado em Ci\^encias)}
	\tipotrabalhoabs{Dissertation (Master in Science)}
	\area{Applied Physics}
	\opcao{Computational Physics}
	% O preambulo deve conter o tipo do trabalho, o objetivo, 
	% o nome da institui��o, a �rea de concentra��o e op��o quando houver
	\preambulo{Dissertation presented to the Graduate Program in Physics at the Instituto de F\'isica de S\~ao Carlos da Universidade de S\~ao Paulo, to obtain the degree of  Master in Science.}
	\notaficha{Dissertation (Master - Graduate Program in Applied Physics)}
    }{
% DFAFBp ===========================================================================
\ifthenelse{\equal{#1}{DFAFBp}}{
	\tipotrabalho{Tese (Doutorado em Ci\^encias)}
	\tipotrabalhoabs{Thesis (Doctor in Science)}
	\area{F\'isica Aplicada}
	\opcao{F\'isica Biomolecular}
	% O preambulo deve conter o tipo do trabalho, o objetivo, 
	% o nome da institui��o, a �rea de concentra��o e op��o quando houver
	\preambulo{Tese apresentada ao Programa de P\'os-Gradua\c{c}\~ao em F\'isica do Instituto de F\'isica de S\~ao Carlos da Universidade de S\~ao Paulo, para obten\c{c}\~ao do t\'itulo de Doutor em Ci\^encias.}
	\notaficha{Tese (Doutorado - Programa de P\'os-Gradua\c{c}\~ao em F\'isica Aplicada)}
    }{				
% DFAFBe ===========================================================================
\ifthenelse{\equal{#1}{DFAFBe}}{
	\renewcommand{\areaname}{Concentration area:}
	\renewcommand{\opcaoname}{Option:}
	\renewcommand{\orientadorname}{Advisor:}
	\tipotrabalho{Tese (Doutorado em Ci\^encias)}
	\tipotrabalhoabs{Thesis (Doctor in Science)}
	\area{Applied Physics}
	\opcao{Biomolecular Physics}
	% O preambulo deve conter o tipo do trabalho, o objetivo, 
	% o nome da institui��o, a �rea de concentra��o e op��o quando houver
	\preambulo{Thesis presented to the Graduate Program in Physics at the Instituto de F\'isica de S\~ao Carlos, Universidade de S\~ao Paulo, to obtain the degree of Doctor in Science.}
	\notaficha{Thesis (Doctorate - Graduate Program in Applied Physics)}
    }{				
% MFAFBp ===========================================================================
\ifthenelse{\equal{#1}{MFAFBp}}{
	\tipotrabalho{Disserta\c{c}\~ao (Mestrado em Ci\^encias)}
	\tipotrabalhoabs{Dissertation (Master in Science)}
	\area{F\'isica Aplicada}
	\opcao{F\'isica Biomolecular}
	% O preambulo deve conter o tipo do trabalho, o objetivo, 
	% o nome da institui��o, a �rea de concentra��o e op��o quando houver
	\preambulo{Disserta\c{c}\~ao apresentada ao Programa de P\'os-Gradua\c{c}\~ao em F\'isica do Instituto de F\'isica de S\~ao Carlos da Universidade de S\~ao Paulo, para obten\c{c}\~ao do t\'itulo de Mestre em Ci\^encias.}
	\notaficha{Disserta\c{c}\~ao (Mestrado - Programa de P\'os-Gradua\c{c}\~ao em F\'isica Aplicada)}
    }{
% MFAFBe ===========================================================================
\ifthenelse{\equal{#1}{MFAFBe}}{
	\renewcommand{\areaname}{Concentration area:}
	\renewcommand{\opcaoname}{Option:}
	\renewcommand{\orientadorname}{Advisor:}
	\tipotrabalho{Disserta\c{c}\~ao (Mestrado em Ci\^encias)}
	\tipotrabalhoabs{Dissertation (Master in Science)}
	\area{Applied Physics}
	\opcao{Biomolecular Physics}
	% O preambulo deve conter o tipo do trabalho, o objetivo, 
	% o nome da institui��o, a �rea de concentra��o e op��o quando houver
	\preambulo{Dissertation presented to the Graduate Program in Physics at the Instituto de F\'isica de S\~ao Carlos da Universidade de S\~ao Paulo, to obtain the degree of Master in Science.}
	\notaficha{Dissertation (Master - Graduate Program in Applied Physics}
    }{				
% DFBp ==========================================================================
\ifthenelse{\equal{#1}{DFBp}}{
    \tipotrabalho{Tese (Doutorado em Ci\^encias)}
    \tipotrabalhoabs{Thesis (Doctor in Science)}
    \area{F\'isica B\'asica}
	%\opcao{Nome da Op��o}
    % O preambulo deve conter o tipo do trabalho, o objetivo, 
	% o nome da institui��o, a �rea de concentra��o e op��o quando houver				
	\preambulo{Tese apresentada ao Programa de P\'os-Gradua\c{c}\~ao em F\'isica do Instituto de F\'isica de S\~ao Carlos da Universidade de S\~ao Paulo, para obten\c{c}\~ao do t\'itulo de Doutor em Ci\^encias.}
	\notaficha{Tese (Doutorado - Programa de P\'os-Gradua\c{c}\~ao em F\'isica B\'asica)}
    }{
% DFBe ==========================================================================
\ifthenelse{\equal{#1}{DFBe}}{
    \renewcommand{\areaname}{Concentration area:}
    \renewcommand{\opcaoname}{Option:}
    \renewcommand{\orientadorname}{Advisor:}
    \tipotrabalho{Tese (Doutorado em Ci\^encias)}
    \tipotrabalhoabs{Thesis (Doctor in Science)}
    \area{Basic Physics}
    %\opcao{Nome da Op��o}
    % O preambulo deve conter o tipo do trabalho, o objetivo, 
    % o nome da institui��o, a �rea de concentra��o e op��o quando houver				
    \preambulo{Thesis presented to the Graduate Program in Physics at the Instituto de F\'isica de S\~ao Carlos da Universidade de S\~ao Paulo, to obtain the degree of Doctor in Science.}
    \notaficha{Thesis (Doctorate - Graduate Program in Basic Physics)}
    }{
% MFBp ===========================================================================
\ifthenelse{\equal{#1}{MFBp}}{
	\tipotrabalho{Disserta\c{c}\~ao (Mestrado em Ci\^encias)}
	\tipotrabalhoabs{Dissertation (Master in Science)}
	\area{F\'isica B\'asica}
	%\opcao{Nome da Op��o}
	% O preambulo deve conter o tipo do trabalho, o objetivo, 
	% o nome da institui��o, a �rea de concentra��o e op��o quando houver				
	\preambulo{Disserta\c{c}\~ao apresentada ao Programa de P\'os-Gradua\c{c}\~ao em F\'isica do Instituto de F\'isica de S\~ao Carlos da Universidade de S\~ao Paulo, para obten\c{c}\~ao do t\'itulo de Mestre em Ci\^encias.}
	\notaficha{Disserta\c{c}\~ao (Mestrado - Programa de P\'os-Gradua\c{c}\~ao em F\'isica B\'asica)}
    }{                
% MFBe ===========================================================================
\ifthenelse{\equal{#1}{MFBe}}{
    \renewcommand{\areaname}{Concentration area:}
    \renewcommand{\opcaoname}{Option:}
    \renewcommand{\orientadorname}{Advisor:}
    \tipotrabalho{Disserta\c{c}\~ao (Mestrado em Ci\^encias)}
    \tipotrabalhoabs{Dissertation (Master in Science)}
    \area{Basic Physics}
    %\opcao{Nome da Op��o}
    % O preambulo deve conter o tipo do trabalho, o objetivo, 
    % o nome da institui��o, a �rea de concentra��o e op��o quando houver				
    \preambulo{Dissertation presented to the Graduate Program in Physics at the Instituto de F\'isica de S\~ao Carlos da Universidade de S\~ao Paulo, to obtain the degree of Master in Science.}
    \notaficha{Dissertation (Master - Graduate Program in Basic Physics)}
    }{   
% DFBMp ===========================================================================
\ifthenelse{\equal{#1}{DFBMp}}{
	\tipotrabalho{Tese (Doutorado em Ci\^encias)}
	\tipotrabalhoabs{Thesis (Doctor in Science)}
	\area{F\'isica Biomolecular}
	%\opcao{Nome da Op��o}
	% O preambulo deve conter o tipo do trabalho, o objetivo, 
	% o nome da institui��o, a �rea de concentra��o e op��o quando houver				
	\preambulo{Tese apresentada ao Programa de P\'os-Gradua\c{c}\~ao em F\'isica do Instituto de F\'isica de S\~ao Carlos da Universidade de S\~ao Paulo, para obten\c{c}\~ao do t\'itulo de Doutor em Ci\^encias.}
	\notaficha{Tese (Doutorado - Programa de P\'os-Gradua\c{c}\~ao em F\'isica Biomolecular)}
    }{   
% DFBMe ===========================================================================
\ifthenelse{\equal{#1}{DFBMe}}{
	\renewcommand{\areaname}{Concentration area:}
	\renewcommand{\opcaoname}{Option:}
	\renewcommand{\orientadorname}{Advisor:}
	\tipotrabalho{Tese (Doutorado em Ci\^encias)}
	\tipotrabalhoabs{Thesis (Doctor in Science)}
	\area{Biomolecular Physics}
	%\opcao{Nome da Op��o}
	% O preambulo deve conter o tipo do trabalho, o objetivo, 
	% o nome da institui��o, a �rea de concentra��o e op��o quando houver				
	\preambulo{Thesis presented to the Graduate Program in Physics at the Instituto de F\'isica de S\~ao Carlos da Universidade de S\~ao Paulo, to obtain the degree of Doctor in Science.}
	\notaficha{Thesis (Doctorate - Graduate Program in Biomolecular Physics)}
    }{    
% MFBMp ===========================================================================
\ifthenelse{\equal{#1}{MFBMp}}{
	\tipotrabalho{Disserta\c{c}\~ao (Mestrado em Ci\^encias)}
	\tipotrabalhoabs{Dissertation (Master in Science)}
	\area{F\'isica Biomolecular}
	%\opcao{Nome da Op��o}
	% O preambulo deve conter o tipo do trabalho, o objetivo, 
	% o nome da institui��o, a �rea de concentra��o e op��o quando houver				
	\preambulo{Disserta\c{c}\~ao apresentada ao Programa de P\'os-Gradua\c{c}\~ao em F\'isica do Instituto de F\'isica de S\~ao Carlos da Universidade de S\~ao Paulo, para obten\c{c}\~ao do t\'itulo de Mestre em Ci\^encias.}
	\notaficha{Disserta\c{c}\~ao (Mestrado - Programa de P\'os-Gradua\c{c}\~ao em F\'isica Biomolecular)}
    }{ 
% MFBMe ===========================================================================
\ifthenelse{\equal{#1}{MFBMe}}{
    \renewcommand{\areaname}{Concentration area:}
    \renewcommand{\opcaoname}{Option:}
    \renewcommand{\orientadorname}{Advisor:}
    \tipotrabalho{Disserta\c{c}\~ao (Mestrado em Ci\^encias)}
    \tipotrabalhoabs{Dissertation (Master in Science)}
    \area{Biomolecular Physics}
    %\opcao{Nome da Op��o}
    % O preambulo deve conter o tipo do trabalho, o objetivo, 
    % o nome da institui��o, a �rea de concentra��o e op��o quando houver				
    \preambulo{Dissertation presented to the Graduate Program in Physics at the Instituto de F\'isica de S\~ao Carlos da Universidade de S\~ao Paulo, to obtain the degree of Master in Science.}
    \notaficha{Dissertation (Master - Graduate Program in Biomolecular Physics)}	
    }{ 
% DFCp ==========================================================================
\ifthenelse{\equal{#1}{DFCp}}{
	\tipotrabalho{Tese (Doutorado em Ci\^encias)}
	\tipotrabalhoabs{Thesis (Doctor in Science)}
	\area{F\'isica Computacional}
	%\opcao{Nome da Op��o}
	% O preambulo deve conter o tipo do trabalho, o objetivo, 
	% o nome da institui��o, a �rea de concentra��o e op��o quando houver				
	\preambulo{Tese apresentada ao Programa de P\'os-Gradua\c{c}\~ao em F\'isica do Instituto de F\'isica de S\~ao Carlos da Universidade de S\~ao Paulo, para obten\c{c}\~ao do t\'itulo de Doutor em Ci\^encias.}
	\notaficha{Tese (Doutorado - Programa de P\'os-Gradua\c{c}\~ao em F\'isica Computacional)}
    }{ 
% DFCe ==========================================================================
\ifthenelse{\equal{#1}{DFCe}}{
	\renewcommand{\areaname}{Concentration area:}
	\renewcommand{\opcaoname}{Option:}
	\renewcommand{\orientadorname}{Advisor:}
	\tipotrabalho{Tese (Doutorado em Ci\^encias)}
	\tipotrabalhoabs{Thesis (Doctor in Science)}
	\area{Computational Physics}
	%\opcao{Nome da Op��o}
	% O preambulo deve conter o tipo do trabalho, o objetivo, 
	% o nome da institui��o, a �rea de concentra��o e op��o quando houver				
	\preambulo{Thesis presented to the Graduate Program in Physics at the Instituto de F\'isica de S\~ao Carlos da Universidade de S\~ao Paulo, to obtain the degree of Doctor in Science.}
	\notaficha{Thesis (Doctorate - Graduate Program in Computational Physics)}
    }{ 
% MFCp ===========================================================================
\ifthenelse{\equal{#1}{MFCp}}{
	\tipotrabalho{Disserta\c{c}\~ao (Mestrado em Ci\^encias)}
	\tipotrabalhoabs{Dissertation (Master in Science)}
	\area{F\'isica Computacional}
	%\opcao{Nome da Op��o}
	% O preambulo deve conter o tipo do trabalho, o objetivo, 
	% o nome da institui��o, a �rea de concentra��o e op��o quando houver				
	\preambulo{Disserta\c{c}\~ao apresentada ao Programa de P\'os-Gradua\c{c}\~ao em F\'isica do Instituto de F\'isica de S\~ao Carlos da Universidade de S\~ao Paulo, para obten\c{c}\~ao do t\'itulo de Mestre em Ci\^encias.}
	\notaficha{Disserta\c{c}\~ao (Mestrado - Programa de P\'os-Gradua\c{c}\~ao em F\'isica Computacional)}
    }{ 
% MFCe ===========================================================================
\ifthenelse{\equal{#1}{MFCe}}{
	\renewcommand{\areaname}{Concentration area:}
	\renewcommand{\opcaoname}{Option:}
	\renewcommand{\orientadorname}{Advisor:}
	\tipotrabalho{Disserta\c{c}\~ao (Mestrado em Ci\^encias)}
	\tipotrabalhoabs{Dissertation (Master in Science)}
	\area{Computational Physics}
	%\opcao{Nome da Op��o}
	% O preambulo deve conter o tipo do trabalho, o objetivo, 
	% o nome da institui��o, a �rea de concentra��o e op��o quando houver				
	\preambulo{Dissertation presented to the Graduate Program in Physics at the Instituto de F\'isica de S\~ao Carlos da Universidade de S\~ao Paulo, to obtain the degree of Master in Science.}
	\notaficha{Dissertation (Master - Graduate Program in Computational Physics)}	
    }{  
% DFTEp ==========================================================================
\ifthenelse{\equal{#1}{DFTEp}}{
	\tipotrabalho{Tese (Doutorado em Ci\^encias)}
	\tipotrabalhoabs{Thesis (Doctor in Science)}
	\area{F\'isica Te\'orica e Experimental}
	%\opcao{Nome da Op��o}
	% O preambulo deve conter o tipo do trabalho, o objetivo, 
	% o nome da institui��o, a �rea de concentra��o e op��o quando houver				
	\preambulo{Tese apresentada ao Programa de P\'os-Gradua\c{c}\~ao em F\'isica do Instituto de F\'isica de S\~ao Carlos da Universidade de S\~ao Paulo, para obten\c{c}\~ao do t\'itulo de Doutor em Ci\^encias.}
	\notaficha{Tese (Doutorado - Programa de P\'os-Gradua\c{c}\~ao em F\'isica Te\'orica e Experimental)}
    }{
% DFTEe ==========================================================================
\ifthenelse{\equal{#1}{DFTEe}}{
	\renewcommand{\areaname}{Concentration area:}
	\renewcommand{\opcaoname}{Option:}
	\renewcommand{\orientadorname}{Advisor:}
	\tipotrabalho{Tese (Doutorado em Ci\^encias)}
	\tipotrabalhoabs{Thesis (Doctor in Science)}
	\area{Theoretical and Experimental Physics}
	%\opcao{Nome da Op��o}
	% O preambulo deve conter o tipo do trabalho, o objetivo, 
	% o nome da institui��o, a �rea de concentra��o e op��o quando houver				
	\preambulo{Thesis presented to the Graduate Program in Physics at the Instituto de F\'isica de S\~ao Carlos da Universidade de S\~ao Paulo, to obtain the degree of Doctor in Science.}
	\notaficha{Thesis (Doctorate - Graduate Program in Theoretical and Experimental Physics)}
    }{
% MFTEp ===========================================================================
\ifthenelse{\equal{#1}{MFTEp}}{
	\tipotrabalho{Disserta\c{c}\~ao (Mestrado em Ci\^encias)}
	\tipotrabalhoabs{Dissertation (Master in Science)}
	\area{F\'isica Te\'orica e Experimental}
	%\opcao{Nome da Op��o}
	% O preambulo deve conter o tipo do trabalho, o objetivo, 
	% o nome da institui��o, a �rea de concentra��o e op��o quando houver				
	\preambulo{Disserta\c{c}\~ao apresentada ao Programa de P\'os-Gradua\c{c}\~ao em F\'isica do Instituto de F\'isica de S\~ao Carlos da Universidade de S\~ao Paulo, para obten\c{c}\~ao do t\'itulo de Mestre em Ci\^encias.}
	\notaficha{Disserta\c{c}\~ao (Mestrado - Programa de P\'os-Gradua\c{c}\~ao em F\'isica Te\'orica e Experimental)}
    }{  
% MFTEe ===========================================================================
\ifthenelse{\equal{#1}{MFTEe}}{
	\renewcommand{\areaname}{Concentration area:}
	\renewcommand{\opcaoname}{Option:}
	\renewcommand{\orientadorname}{Advisor:}
	\tipotrabalho{Disserta\c{c}\~ao (Mestrado em Ci\^encias)}
	\tipotrabalhoabs{Dissertation (Master in Science)}
	\area{Theoretical and Experimental Physics}
	%\opcao{Nome da Op��o}
	% O preambulo deve conter o tipo do trabalho, o objetivo, 
	% o nome da institui��o, a �rea de concentra��o e op��o quando houver				
	\preambulo{Dissertation presented to the Graduate Program in Physics at the Instituto de F\'isica de S\~ao Carlos da Universidade de S\~ao Paulo, to obtain the degree of Master in Science.}
	\notaficha{Dissertation (Master - Graduate Program in Theoretical and Experimental Physics)}		
    }{  
% Outros
	\tipotrabalho{Disserta\c{c}\~ao/Tese (Mestrado/Doutorado)}
	\tipotrabalhoabs{Dissertation/Thesis (Master/Doctor)}
	\area{Nome da \'Area}
	\opcao{Nome da Op\c{c}\~ao}
	% O preambulo deve conter o tipo do trabalho, o objetivo, 
	% o nome da institui��o, a �rea de concentra��o e op��o quando houver
	\preambulo{Disserta\c{c}\~ao/Tese apresentada ao Programa de P\�{\o}s-Gradua\c{c}\~ao em F\�{\i}sica do Instituto de F\�{\i}sica de S\~ao Carlos da Universidade de S\~ao Paulo, para obten�\c{c}\~ao do t\�{\i}tulo de Mestre/Doutor em Ci\^encias.}
	\notaficha{Disserta\c{c}\~ao/Tese (Mestrado/Doutorado - Programa de P\'os-Gradua\c{c}\~ao em Nome da \'Area)}
}}}}}}}}}}}}}}}}}}}}}}}}}}}}}
				
				






    
        }{
% IQSC ===========================================================================
        \ifthenelse{\equal{#1}{IQSC}}{
        %% USPSC-pre-textual-IQSC.tex
%% Camandos para defini��o do tipo de documento (tese ou disserta��o), �rea de concentra��o, op��o, pre�mbulo, titula��o 
%% referentes ao Programa de P�s-Gradua��o o IQSC
\instituicao{Instituto de Qu\'imica de S\~ao Carlos, Universidade de S\~ao Paulo}
\unidade{INSTITUTO DE QU\'IMICA DE S\~AO CARLOS}
\unidademin{Instituto de Qu\'imica de S\~ao Carlos}
\universidademin{Universidade de S\~ao Paulo}
% O IQSC n�o inclui a nota "Vers�o original", portanto o comando abaixo n�o tem a mensagem entre {}
\notafolharosto{ }
%Para a vers�o revisada tire a % do comando/declara��o abaixo e inclua uma % antes do comando acima

%\notafolharosto{Exemplar revisado \\O exemplar original encontra-se em acervo reservado na Biblioteca do IQSC-USP}

% ---
% dados complementares para CAPA e FOLHA DE ROSTO
% ---
\universidade{UNIVERSIDADE DE S\~AO PAULO}
\titulo{Modelo para teses e disserta\c{c}\~oes em \LaTeX\ utilizando o Pacote USPSC para o IQSC}
%\titulo{Modelo para elabora\c{c}\~ao de trabalhos acad\^emicos em LaTex utilizando o Pacote USPSC para o IQSC}
\titleabstract{Model for thesis and dissertations in \LaTeX\ using the USPSC Package to the IQSC}
\tituloresumo{Modelo para teses e disserta\c{c}\~oes em \LaTeX\ utilizando o Pacote USPSC para o IQSC}
\autor{Jos\'e da Silva}
\autorficha{Silva, Jos\'e da}
\autorabr{SILVA, J.}

\cutter{S856m}
% Para gerar a ficha catalogr�fica sem o C�digo Cutter, basta 
% incluir uma % na linha acima e tirar a % da linha abaixo
%\cutter{ }

\local{S\~ao Carlos}
\data{2021}
% Quando for Orientador, basta incluir uma % antes do comando abaixo
\renewcommand{\orientadorname}{Orientadora:}
% Quando for Coorientadora, basta tirar a % utilizar o comando abaixo
%\newcommand{\coorientadorname}{Coorientador:}
\orientador{Profa. Dra. Elisa Gon\c{c}alves Rodrigues}
\orientadorcorpoficha{orientadora Elisa Gon\c{c}alves Rodrigues}
\orientadorficha{Rodrigues, Elisa Gon\c{c}alves, orient}
%Se houver co-orientador, inclua % antes das duas linhas (antes dos comandos \orientadorcorpoficha e \orientadorficha) 
%          e tire a % antes dos 3 comandos abaixo
%\coorientador{Prof. Dr. Jo\~ao Alves Serqueira}
%\orientadorcorpoficha{orientadora Elisa Gon\c{c}alves Rodrigues ;  co-orientador Jo\~ao Alves Serqueira}
%\orientadorficha{Rodrigues, Elisa Gon\c{c}alves, orient. II. Serqueira, Jo\~ao Alves, co-orient}

\notaautorizacao{AUTORIZO A REPRODU\c{C}\~AO E DIVULGA\c{C}\~AO TOTAL OU PARCIAL DESTE TRABALHO, POR QUALQUER MEIO CONVENCIONAL OU ELETR\^ONICO PARA FINS DE ESTUDO E PESQUISA, DESDE QUE CITADA A FONTE.}
\notabib{Ficha catalogr\'afica elaborada pela Se\c{c}\~ao de Refer\^encia e Atendimento ao Usu\'ario do Servi\c{c}o de Biblioteca e Informa\c{c}\~ao Prof. Johannes R\"udiger Lechat, com os dados fornecidos pelo(a) autor(a)}

\newcommand{\programa}[1]{

% DFQ ==========================================================================
\ifthenelse{\equal{#1}{DFQ}}{
    \area{F\'isico-Qu\'imica}
	\tipotrabalho{Tese (Doutorado em ~\imprimirarea)}
	\tipotrabalhoabs{Tese (Doutorado em ~\imprimirarea)}
	%\opcao{Nome da Op��o}
    % O preambulo deve conter o tipo do trabalho, o objetivo, 
	% o nome da institui��o e a �rea de concentra��o 
	\preambulo{Tese apresentada ao Instituto de Qu\'imica de S\~ao Carlos, da Universidade de S\~ao Paulo, como parte dos requisitos para a obten\c{c}\~ao do t\'itulo de Doutor em Ci\^encias no Programa Qu\'imica.}
	\notaficha{Tese (Doutorado em ~\imprimirarea)}
    }{
% MFQ ===========================================================================
\ifthenelse{\equal{#1}{MFQ}}{
    \area{F\'isico-Qu\'imica}
	\tipotrabalho{Disserta\c{c}\~ao (Mestrado em ~\imprimirarea)}
	\tipotrabalhoabs{Disserta\c{c}\~ao (Mestrado em ~\imprimirarea)}
	%\opcao{Nome da Op��o}
    % O preambulo deve conter o tipo do trabalho, o objetivo, 
	% o nome da institui��o e a �rea de concentra��o 
	\preambulo{Disserta\c{c}\~ao apresentada ao Instituto de Qu\'imica de S\~ao Carlos, da Universidade de S\~ao Paulo, como parte dos requisitos para a obten\c{c}\~ao do t\'itulo de Mestre em Ci\^encias no Programa Qu\'imica.}
	\notaficha{Disserta\c{c}\~ao (Mestrado em ~\imprimirarea)}
    }{
% DQAI ==========================================================================
\ifthenelse{\equal{#1}{DQAI}}{
    \area{Qu\'imica Anal\'itica e Inorg\^anica}
	\tipotrabalho{Tese (Doutorado em ~\imprimirarea)}
	\tipotrabalhoabs{Tese (Doutorado em ~\imprimirarea)}
	%\opcao{Nome da Op��o}
    % O preambulo deve conter o tipo do trabalho, o objetivo, 
	% o nome da institui��o e a �rea de concentra��o 
	\preambulo{Tese apresentada ao Instituto de Qu\'imica de S\~ao Carlos, da Universidade de S\~ao Paulo, como parte dos requisitos para a obten\c{c}\~ao do t\'itulo de Doutor em Ci\^encias no Programa Qu\'imica.}
	\notaficha{Tese (Doutorado em ~\imprimirarea)}
    }{
% MQAI ===========================================================================
\ifthenelse{\equal{#1}{MQAI}}{
	\area{Qu\'imica Anal\'itica e Inorg\^anica}
	\tipotrabalho{Disserta\c{c}\~ao (Mestrado em ~\imprimirarea)}
	\tipotrabalhoabs{Disserta\c{c}\~ao (Mestrado em ~\imprimirarea)}
	%\opcao{Nome da Op��o}
	% O preambulo deve conter o tipo do trabalho, o objetivo, 
	% o nome da institui��o e a �rea de concentra��o 
	\preambulo{Disserta\c{c}\~ao apresentada ao Instituto de Qu\'imica de S\~ao Carlos, da Universidade de S\~ao Paulo, como parte dos requisitos para a obten\c{c}\~ao do t\'itulo de Mestre em Ci\^encias no Programa Qu\'imica.}
	\notaficha{Disserta\c{c}\~ao (Mestrado em ~\imprimirarea)}
    }{
% DQOB ===========================================================================
\ifthenelse{\equal{#1}{DQOB}}{
	\area{Qu\'imica Org\^anica e Biol\'ogica}
	\tipotrabalho{Tese (Doutorado em ~\imprimirarea)}
	\tipotrabalhoabs{Tese (Doutorado em ~\imprimirarea)}
	%\opcao{Nome da Op��o}
    % O preambulo deve conter o tipo do trabalho, o objetivo, 
	% o nome da institui��o e a �rea de concentra��o 
	\preambulo{Tese apresentada ao Instituto de Qu\'imica de S\~ao Carlos, da Universidade de S\~ao Paulo, como parte dos requisitos para a obten\c{c}\~ao do t\'itulo de Doutor em Ci\^encias no Programa Qu\'imica.}
	\notaficha{Tese (Doutorado em ~\imprimirarea)}
	}{
% MQOB ===========================================================================
\ifthenelse{\equal{#1}{MQOB}}{
    \area{Qu\'imica Org\^anica e Biol\'ogica}
	\tipotrabalho{Disserta\c{c}\~ao (Mestrado em ~\imprimirarea)}
	\tipotrabalhoabs{Disserta\c{c}\~ao (Mestrado em ~\imprimirarea)}
	%\opcao{Nome da Op��o}
    % O preambulo deve conter o tipo do trabalho, o objetivo, 
	% o nome da institui��o e a �rea de concentra��o 
	\preambulo{Disserta\c{c}\~ao apresentada ao Instituto de Qu\'imica de S\~ao Carlos, da Universidade de S\~ao Paulo, como parte dos requisitos para a obten\c{c}\~ao do t\'itulo de Mestre em Ci\^encias no Programa Qu\'imica.}
	\notaficha{Disserta\c{c}\~ao (Mestrado em ~\imprimirarea)}
    }{
% Outros 
	\tipotrabalho{Disserta\c{c}\~ao/Tese (Mestrado/Doutorado)}
	\tipotrabalhoabs{Disserta\c{c}\~ao/Tese (Mestrado/Doutorado)}
	\area{Nome da \'Area}
	\opcao{Nome da Op\c{c}\~ao}
	% O preambulo deve conter o tipo do trabalho, o objetivo, 
	% o nome da institui��o, a �rea de concentra��o e op��o quando houver
	\preambulo{Disserta\c{c}\~ao/Tese apresentada ao Instituto de Qu\'imica de S\~ao Carlos, da Universidade de S\~ao Paulo, como parte dos requisitos para a obten\c{c}\~ao do t\'itulo de Mestre/Doutor em Ci\^encias no Programa Qu\'imica.}
	\notaficha{Disserta\c{c}\~ao/Tese (Mestrado/Doutorado em ~\imprimirarea)}	
  
}}}}}}}
				    
        }{
% IQSC-TCC ===========================================================================
	\ifthenelse{\equal{#1}{IQSC-TCC}}{
	%% USPSC-TCC-pre-textual-IQSC.tex
%% Camandos para defini��o do tipo de documento (tese ou disserta��o), �rea de concentra��o, op��o, pre�mbulo, titula��o 
%% referentes ao Programa de P�s-Gradua��o o IQSC
\instituicao{Instituto de Qu\'imica de S\~ao Carlos, Universidade de S\~ao Paulo}
\unidade{INSTITUTO DE QU\'IMICA DE S\~AO CARLOS}
\unidademin{Instituto de Qu\'imica de S\~ao Carlos}
\universidademin{Universidade de S\~ao Paulo}
% O IQSC n�o inclui a nota "Vers�o original", portanto o comando abaixo n�o tem a mensagem entre {}
\notafolharosto{ }
%Para a vers�o revisada tire a % do comando/declara��o abaixo e inclua uma % antes do comando acima

%\notafolharosto{Exemplar revisado \\O exemplar original encontra-se em acervo reservado na Biblioteca do IQSC-USP}

% ---
% dados complementares para CAPA e FOLHA DE ROSTO
% ---
\universidade{UNIVERSIDADE DE S\~AO PAULO}
\titulo{Modelo para TCC em \LaTeX\ utilizando o Pacote USPSC para o IQSC}
%\titulo{Modelo para elabora\c{c}\~ao de trabalhos acad\^emicos em LaTex utilizando o Pacote USPSC para o IQSC}
\titleabstract{Model for TCC in \LaTeX\ using the USPSC Package to the IQSC}
\tituloresumo{Modelo para TCC em \LaTeX\ utilizando o Pacote USPSC para o IQSC}
\autor{Jos\'e da Silva}
\autorficha{Silva, Jos\'e da}
\autorabr{SILVA, J.}

\cutter{S856m}
% Para gerar a ficha catalogr�fica sem o C�digo Cutter, basta 
% incluir uma % na linha acima e tirar a % da linha abaixo
%\cutter{ }

\local{S\~ao Carlos}
\data{2021}
% Quando for Orientador, basta incluir uma % antes do comando abaixo
\renewcommand{\orientadorname}{Orientadora:}
% Quando for Coorientadora, basta tirar a % utilizar o comando abaixo
%\newcommand{\coorientadorname}{Coorientador:}
\orientador{Profa. Dra. Elisa Gon\c{c}alves Rodrigues}
\orientadorcorpoficha{orientadora Elisa Gon\c{c}alves Rodrigues}
\orientadorficha{Rodrigues, Elisa Gon\c{c}alves, orient}
%Se houver co-orientador, inclua % antes das duas linhas (antes dos comandos \orientadorcorpoficha e \orientadorficha) 
%          e tire a % antes dos 3 comandos abaixo
%\coorientador{Prof. Dr. Jo\~ao Alves Serqueira}
%\orientadorcorpoficha{orientadora Elisa Gon\c{c}alves Rodrigues ;  co-orientador Jo\~ao Alves Serqueira}
%\orientadorficha{Rodrigues, Elisa Gon\c{c}alves, orient. II. Serqueira, Jo\~ao Alves, co-orient}

\notaautorizacao{AUTORIZO A REPRODU\c{C}\~AO E DIVULGA\c{C}\~AO TOTAL OU PARCIAL DESTE TRABALHO, POR QUALQUER MEIO CONVENCIONAL OU ELETR\^ONICO PARA FINS DE ESTUDO E PESQUISA, DESDE QUE CITADA A FONTE.}
\notabib{Ficha catalogr\'afica elaborada pela Se\c{c}\~ao de Refer\^encia e Atendimento ao Usu\'ario do Servi\c{c}o de Biblioteca e Informa\c{c}\~ao Prof. Johannes R\"udiger Lechat, com os dados fornecidos pelo(a) autor(a)}

\newcommand{\programa}[1]{

% BQ ==========================================================================
\ifthenelse{\equal{#1}{BQ}}{
    %\area{Qu\'imica}
	\tipotrabalho{Monografia (Trabalho de Conclus\~ao de Curso)}
	\tipotrabalhoabs{Monograph (Conclusion Course Paper)}
	%\opcao{Nome da Op��o}
    % O preambulo deve conter o tipo do trabalho, o objetivo, 
	% o nome da institui��o e a �rea de concentra��o 
	\preambulo{Monografia apresentada ao Instituto de Qu\'imica de S\~ao Carlos, da Universidade de S\~ao Paulo, como parte dos requisitos para a obten\c{c}\~ao do t\'itulo de Bacharel em Qu\'imica.}
	\notaficha{Monografia (Gradua\c{c}\~ao em Qu\'imica)}
    }{
% REQ ===========================================================================
\ifthenelse{\equal{#1}{REQ}}{
    %\area{u\'imica}
	\tipotrabalho{Est\'agio}
	\tipotrabalhoabs{Internship}
	%\opcao{Nome da Op��o}
    % O preambulo deve conter o tipo do trabalho, o objetivo, 
	% o nome da institui��o e a �rea de concentra��o 
	\preambulo{Relat\'orio de est\'agio em  Qu\'imica apresentado ao Instituto de Qu\'imica de S\~ao Carlos, da Universidade de S\~ao Paulo, como parte dos requisitos para a obten\c{c}\~ao do t\'itulo de de Bacharel em Qu\'imica.}
	\notaficha{Relat\'orio de Est\'agio (Gradua\c{c}\~ao em Qu\'imica)}
    }{
% Outros 
	\tipotrabalho{Monografia (Trabalho de Conclus\~ao de Curso)}
	\tipotrabalhoabs{Monografia (Trabalho de Conclus\~ao de Curso)}
	%\area{Nome da \'Area}
	%\opcao{Nome da Op\c{c}\~ao}
	% O preambulo deve conter o tipo do trabalho, o objetivo, 
	% o nome da institui��o, a �rea de concentra��o e op��o quando houver
	\preambulo{Monografia/Relat\'orio de est\'agio apresentada(o) ...}
	\notaficha{Monografia/Relat\'orio de Est\'agio (Gradua\c{c}\~ao em Qu\'imica)}	
  
}}}
				    
       }{
% USPSC ===========================================================================
\ifthenelse{\equal{#1}{USPSC}}{
	%% USPSC-pre-textual-Tutorial.tex
%% Camandos para defini��o do tipo de documento (tese ou disserta��o), �rea de concentra��o, op��o, pre�mbulo, titula��o 
%% referentes ao Programa de P�s-Gradua��o o IQSC
\instituicao{Universidade de S\~ao Paulo}
\unidade{PREFEITURA DO CAMPUS USP DE S\~AO CARLOS \hspace{60 em}INSTITUTO DE ARQUITETURA E URBANISMO \hspace{60 em}ESCOLA DE ENGENHARIA DE S\~AO CARLOS \hspace{60 em}INSTITUTO DE QU\'IMICA DE S\~AO CARLOS \hspace{60 em}INSTITUTO DE F\'ISICA DE S\~AO CARLOS \hspace{60 em}INSTITUTO DE CI\^ENCIAS MATEM\'ATICAS E DE COMPUTA\c{C}\~AO}
\unidademin{Prefeitura do Campus USP de S\~ao Carlos; Instituto de Arquitetura e Urbanismo; Escola de Engenharia de S\~ao Carlos; Instituto de Qu\'imica de S\~ao Carlos; Instituto de F\'isica de S\~ao Carlos; Instituto de Ci\^encias Matem\'aticas e de Computa\c{c}\~ao}
\universidademin{Universidade de S\~ao Paulo}

\notafolharosto{Vers\~ao original}
%Para vers�o original em ingl�s, comente do comando/declara��o 
%     acima(inclua % antes do comando acima) e tire a % do 
%     comando/declara��o abaixo no idioma do texto
%\notafolharosto{Original version} 
%Para vers�o corrigida, comente do comando/declara��o da 
%     vers�o original acima (inclua % antes do comando acima) 
%     e tire a % do comando/declara��o de um dos comandos 
%     abaixo em conformidade com o idioma do texto
%\notafolharosto{Vers\~ao corrigida \\(Vers\~ao original dispon\'ivel na Unidade que aloja o Programa)}
%\notafolharosto{Corrected version \\(Original version available on the Program Unit)}

% ---
% dados complementares para CAPA e FOLHA DE ROSTO
% ---
\universidade{UNIVERSIDADE DE S\~AO PAULO}
\titulo{Tutorial do Pacote USPSC para modelos de trabalhos acad\^emicos em LaTeX - vers\~ao 3.1}
\tituloresumo{Tutorial do Pacote USPSC para modelos de trabalhos acad\^emicos em LaTeX - vers\~ao 3.1}
\titleabstract{USPSC Package tutorial for LaTeX academic work templates - version 3.1}
\autor{Marilza Aparecida Rodrigues Tognetti}
\autorficha{Tognetti, Marilza Aparecida Rodrigues}
%\autorabr{{TOGNETTI, M. A. R.; Calabrez, A. P. A (coord., program.); SIGOLO, B. O. O. (normaliz., padroniz.)} \textit{et al.} }
\autorabr{{TOGNETTI, M. A. R.; CALABREZ, A. P. A. (coord., program.)}}

\cutter{T645t}
% Para gerar a ficha catalogr�fica sem o C�digo Cutter, basta 
% incluir uma % na linha acima e tirar a % da linha abaixo
%\cutter{ }

\local{S\~ao Carlos}
\data{2021}
% Quando for Orientador, basta incluir uma % antes do comando abaixo
\renewcommand{\orientadorname}{Orientadora:}
% Quando for Coorientadora, basta tirar a % utilizar o comando abaixo
%\newcommand{\coorientadorname}{Coorientador:}
%\orientador{Profa. Dra. Elisa Gon\c{c}alves Rodrigues}
\orientador{Normaliza\c{c}\~ao de Brianda de Oliveira Ordonho Sigolo \textit{ et al.}}
%\orientador{Normaliza\c{c}\~ao de Brianda de Oliveira Ordonho Sigolo}
\orientadorcorpoficha{Marilza Aparecida Rodrigues Tognetti, coordenadora, progamadora, normalizadora, padronizadora; Ana Paula Aparecida Calabrez, coordenadora, progamadora, normalizadora, padronizadora; Brianda de Oliveira Ordonho Sigolo, normalizadora e padronizadora ...[et a.]}
\orientadorficha{Rodrigues, Elisa Gon\c{c}alves, orient}
%Se houver co-orientador, inclua % antes das duas linhas (antes dos comandos \orientadorcorpoficha e \orientadorficha) 
%          e tire a % antes dos 3 comandos abaixo
%\coorientador{Prof. Dr. Jo\~ao Alves Serqueira}
%\orientadorcorpoficha{orientadora Elisa Gon\c{c}alves Rodrigues ;  co-orientador Jo\~ao Alves Serqueira}
%\orientadorficha{Rodrigues, Elisa Gon\c{c}alves, orient. II. Serqueira, Jo\~ao Alves, co-orient}

\notaautorizacao{AUTORIZO A REPRODU\c{C}\~AO E DIVULGA\c{C}\~AO TOTAL OU PARCIAL DESTE TRABALHO, POR QUALQUER MEIO CONVENCIONAL OU ELETR\^ONICO PARA FINS DE ESTUDO E PESQUISA, DESDE QUE CITADA A FONTE.}
\notabib{Ficha catalogr\'afica elaborada pela Biblioteca da Prefeitura do Campus USP de S\~ao Carlos - PUSP-SC/USP}

\newcommand{\programa}[1]{

% USPSC ==========================================================================
   \ifthenelse{\equal{#1}{USPSC}}{
				\tipotrabalho{Tutorial}
				%\opcao{Nome da Op��o}
                % \tipotrabalho{Disserta\c{c}\~ao (Mestrado em Ci\^encias)}
				%\area{F\'isica B\'asica}
				%\opcao{Nome da Op��o}
				% O preambulo deve conter o tipo do trabalho, o objetivo, 
				% o nome da institui��o, a �rea de concentra��o e op��o quando houver
				% O preambulo ir� conter dados de autoria 				
				\preambulo{\textbf{Coordena\c{c}\~ao e Programa\c{c}\~ao} \newline Marilza Aparecida Rodrigues Tognetti (PUSP-SC) \newline Ana Paula Aparecida Calabrez (PUSP-SC)\newline \newline \textbf{Normaliza\c{c}\~ao} \newline Ana Paula Aparecida Calabrez (PUSP-SC) \newline Brianda de Oliveira Ordonho Sigolo (IAU) \newline Eduardo Graziosi Silva (EESC) \newline Eliana de C\'assia Aquareli Cordeiro (IQSC) \newline Fl\'avia Helena Cassin (EESC) \newline Maria Cristina Cavarette Dziabas (IFSC) \newline Marilza Aparecida Rodrigues Tognetti (PUSP-SC) \newline Regina C\'elia Vidal Medeiros (ICMC)}	
				%\notaficha{Tutorial}

}{
% Outros
	   \tipotrabalho{Disserta\c{c}\~ao/Tese (Mestrado/Doutorado)}
	   \area{Nome da \'Area}
	   \opcao{Nome da Op\c{c}\~ao}
	   % O preambulo deve conter o tipo do trabalho, o objetivo, 
	   % o nome da institui��o, a �rea de concentra��o e op��o quando houver
	   \preambulo{Disserta\c{c}\~ao/Tese apresentada ao Programa de P\�{\o}s-Gradua\c{c}\~ao em F\�{\i}sica do Instituto de F\�{\i}sica de S\~ao Carlos da Universidade de S\~ao Paulo, para obten�\c{c}\~ao do t\�{\i}tulo de Mestre/Doutor em Ci\^encias.}
	   \notaficha{Disserta\c{c}\~ao/Tese (Mestrado/Doutorado - Programa de P\'os-Gradua\c{c}\~ao em Nome da \'Area)}	
	
}}    
}{
% OUTROS-TCC ===========================================================================
\ifthenelse{\equal{#1}{OUTROS-TCC}}{
	%% USPSC-TCC-pre-textual-OUTROS.tex
%% Camandos para defini��o do tipo de documento (tese ou disserta��o), �rea de concentra��o, op��o, pre�mbulo, titula��o 
%% referentes aos Programas de P�s-Gradua��o
\instituicao{Nome da Unidade USP, Universidade de S\~ao Paulo}
\unidade{NOME DA UNIDADE USP}
\unidademin{Nome da Unidade USP}
\universidademin{Universidade de S\~ao Paulo}

% A EESC n�o inclui a nota "Vers�o original", portanto o comando abaixo n�o tem a mensagem entre {}
\notafolharosto{ }
%Para a vers�o corrigida tire a % do comando/declara��o abaixo e inclua uma % antes do comando acima
%\notafolharosto{VERS\~AO CORRIGIDA}
% ---
% dados complementares para CAPA e FOLHA DE ROSTO
% ---
\universidade{UNIVERSIDADE DE S\~AO PAULO}
\titulo{Modelo para TCC em \LaTeX\ utilizando o Pacote USPSC}
\titleabstract{Model for TCC in \LaTeX\ using the USPSC Package}
\tituloresumo{Modelo para TCC em \LaTeX\ utilizando o Pacote USPSC}
\autor{Jos\'e da Silva}
\autorficha{Silva, Jos\'e da}
\autorabr{SILVA, J.}

\cutter{S856m}
% Para gerar a ficha catalogr�fica sem o C�digo Cutter, basta 
% incluir uma % na linha acima e tirar a % da linha abaixo
%\cutter{ }

\local{S\~ao Carlos}
\data{2021}
% Quando for Orientador, basta incluir uma % antes do comando abaixo
\renewcommand{\orientadorname}{Orientadora:}
% Quando for Coorientadora, basta tirar a % do comando abaixo
%\newcommand{\coorientadorname}{Coorientador:}
\orientador{Profa. Dra. Elisa Gon\c{c}alves Rodrigues}
%\orientadorcorpoficha{orientadora Elisa Gon\c{c}alves Rodrigues}
%\orientadorficha{Rodrigues, Elisa Gon\c{c}alves, orient}
%Se houver co-orientador, inclua % antes das duas linhas (antes dos comandos \orientadorcorpoficha e \orientadorficha) 
%          e tire a % antes dos 3 comandos abaixo
\coorientador{Prof. Dr. Jo\~ao Alves Serqueira}
\orientadorcorpoficha{orientadora Elisa Gon\c{c}alves Rodrigues ;  co-orientador Jo\~ao Alves Serqueira}
\orientadorficha{Rodrigues, Elisa Gon\c{c}alves, orient. II. Serqueira, Jo\~ao Alves, co-orient}

\notaautorizacao{AUTORIZO A REPRODU\c{C}\~AO E DIVULGA\c{C}\~AO TOTAL OU PARCIAL DESTE TRABALHO, POR QUALQUER MEIO CONVENCIONAL OU ELETR\^ONICO PARA FINS DE ESTUDO E PESQUISA, DESDE QUE CITADA A FONTE.}
\notabib{~  ~}

\newcommand{\programa}[1]{     	
% Outros
	\tipotrabalho{Monografia (Trabalho de Conclus\~ao de Curso)}
	\tipotrabalhoabs{Monograph (Conclusion Course Paper)}
	%\area{Nome da �rea}
	%\opcao{Nome da Op��o}
	% O preambulo deve conter o tipo do trabalho, o objetivo, 
	% o nome da institui��o, a �rea de concentra��o e op��o quando houver
	\preambulo{Monografia apresentada ao Curso de XXXXXXX, da Unidade NNNNN da Universidade de S\~ao Paulo, como parte dos requisitos para obten\c{c}\~ao do t\'itulo de XXXXXXX.}
	\notaficha{Monografia (Gradua\c{c}\~ao em XXXXXXX)}	
	}
    
}{
% Outros ========================================================================
                       % ------------------------------------------------------------------------
        %% USPSC-pre-textual-OUTRO.tex
%% Camandos para defini��o do tipo de documento (tese ou disserta��o), �rea de concentra��o, op��o, pre�mbulo, titula��o 
%% referentes ao Programa de P�s-Gradua��o o IQSC
\instituicao{Nome da Unidade USP, Universidade de S\~ao Paulo}
\unidade{NOME DA UNIDADE USP}
\unidademin{Nome da Unidade USP}
\universidademin{Universidade de S\~ao Paulo}

\notafolharosto{Vers\~ao original}
%Para vers�o original em ingl�s, comente do comando/declara��o 
%     acima(inclua % antes do comando acima) e tire a % do 
%     comando/declara��o abaixo no idioma do texto
%\notafolharosto{Original version} 
%Para vers�o corrigida, comente do comando/declara��o da 
%     vers�o original acima (inclua % antes do comando acima) 
%     e tire a % do comando/declara��o de um dos comandos 
%     abaixo em conformidade com o idioma do texto
%\notafolharosto{Vers\~ao corrigida \\(Vers\~ao original dispon\'ivel na Unidade que aloja o Programa)}
%\notafolharosto{Corrected version \\(Original version available on the Program Unit)}

% ---
% dados complementares para CAPA e FOLHA DE ROSTO
% ---
\universidade{UNIVERSIDADE DE S\~AO PAULO}
\titulo{Modelo para teses e disserta\c{c}\~oes em \LaTeX\ utilizando o Pacote USPSC}
\titleabstract{Model for thesis and dissertations in \LaTeX\ using the USPSC Package}
\tituloresumo{Modelo para teses e disserta\c{c}\~oes em \LaTeX\ utilizando o Pacote USPSC}
\autor{Jos\'e da Silva}
\autorficha{Silva, Jos\'e da}
\autorabr{SILVA, J.}

\cutter{S856m}
% Para gerar a ficha catalogr�fica sem o C�digo Cutter, basta 
% incluir uma % na linha acima e tirar a % da linha abaixo
%\cutter{ }

\local{S\~ao Carlos}
\data{2021}
% Quando for Orientador, basta incluir uma % antes do comando abaixo
\renewcommand{\orientadorname}{Orientadora:}
% Quando for Coorientadora, basta tirar a % utilizar o comando abaixo
%\newcommand{\coorientadorname}{Coorientador:}
\orientador{Profa. Dra. Elisa Gon\c{c}alves Rodrigues}
\orientadorcorpoficha{orientadora Elisa Gon\c{c}alves Rodrigues}
\orientadorficha{Rodrigues, Elisa Gon\c{c}alves, orient}
%Se houver co-orientador, inclua % antes das duas linhas (antes dos comandos \orientadorcorpoficha e \orientadorficha) 
%          e tire a % antes dos 3 comandos abaixo
%\coorientador{Prof. Dr. Jo\~ao Alves Serqueira}
%\orientadorcorpoficha{orientadora Elisa Gon\c{c}alves Rodrigues ;  co-orientador Jo\~ao Alves Serqueira}
%\orientadorficha{Rodrigues, Elisa Gon\c{c}alves, orient. II. Serqueira, Jo\~ao Alves, co-orient}

\notaautorizacao{AUTORIZO A REPRODU\c{C}\~AO E DIVULGA\c{C}\~AO TOTAL OU PARCIAL DESTE TRABALHO, POR QUALQUER MEIO CONVENCIONAL OU ELETR\^ONICO PARA FINS DE ESTUDO E PESQUISA, DESDE QUE CITADA A FONTE.}
\notabib{Ficha catalogr\'afica elaborada pela Biblioteca da Unidade USP, com os dados fornecidos pelo(a) autor(a)}

\newcommand{\programa}[1]{

% TCCOUTRO ==========================================================================
\ifthenelse{\equal{#1}{DOUTRO}}{
    \area{Nome da \'Area}
	\tipotrabalho{Tese (Doutorado)}
	\tipotrabalhoabs{Thesis (Doctor)}
	%\opcao{Nome da Op��o}
    % O preambulo deve conter o tipo do trabalho, o objetivo, 
	% o nome da institui��o e a �rea de concentra��o 
	\preambulo{Tese apresentada ao Programa de P\'os-Gradua\c{c}\~ao em XXXXXXX da Unidade USP, Universidade de S\~ao Paulo, como parte dos requisitos para a obten\c{c}\~ao do t\'itulo de Doutor em YYYYYYYYYYY.}
	\notaficha{Tese (Doutorado - Programa de P\'os-Gradua\c{c}\~ao em XXXXXXX e \'Area de Concentra\c{c}\~ao em ~\imprimirarea)}
    }{
% MOUTRO ===========================================================================
\ifthenelse{\equal{#1}{MOUTRO}}{
    \area{Nome da \'Area}
	\tipotrabalho{Disserta\c{c}\~ao (Mestrado)}
	\tipotrabalhoabs{Dissertation (Master)}
	%\opcao{Nome da Op��o}
    % O preambulo deve conter o tipo do trabalho, o objetivo, 
	% o nome da institui��o e a �rea de concentra��o 
	\preambulo{Disserta\c{c}\~ao apresentada ao Programa de P\'os-Gradua\c{c}\~ao em XXXXXXX da Unidade USP, Universidade de S\~ao Paulo, como parte dos requisitos para a obten\c{c}\~ao do t\'itulo de Mestre em YYYYYYYYYYY.}
	\notaficha{Disserta\c{c}\~ao (Mestrado - Programa de P\'os-Gradua\c{c}\~ao em XXXXXXX e \'Area de Concentra\c{c}\~ao em ~\imprimirarea)}
    }{
% Outros
    \tipotrabalho{Disserta\c{c}\~ao/Tese (Mestrado/Doutorado)}
	\tipotrabalhoabs{Dissertation/Thesis (Master/Doctor)}
	\area{Nome da \'Area}
	\opcao{Nome da Op\c{c}\~ao}
	% O preambulo deve conter o tipo do trabalho, o objetivo, 
	% o nome da institui��o e a �rea de concentra��o 
	\preambulo{Disserta\c{c}\~ao/Tese apresentada ao Programa de P\'os-Gradua\c{c}\~ao em XXXXXXX da Unidade USP, Universidade de S\~ao Paulo, como parte dos requisitos para a obten\c{c}\~ao do t\'itulo de Mestre/Doutor em YYYYYYYYYYY.}
	\notaficha{Disserta\c{c}\~ao/Tese (Mestrado/Doutorado - Programa de P\'os-Gradua\c{c}\~ao em XXXXXXX e \'Area de Concentra\c{c}\~ao em ~\imprimirarea)}
    }}}
				 
								  }
                }
							}
						}
				}
			}}}}}}
         	
 + presente em USPSC.cls e USPSC1.cls,  o arquivo USPSC-unidades.tex efetua as chamadas dos arquivos pré-textuais, portanto quando for feita uma customização incluindo novos arquivos pré-textuais e/ou outra Unidade USP e/ou outra instituição de ensino e pesquisa, será necessário fazer as devidas indicações em tais arquivos. 
	 
\subsection{Alternativas de formatação}
O modelo foi concebido de forma a atender as especificidades de cada Unidade e atualmente disponibiliza as seguintes alternativas de formatação:
\subsubsection{Opções de fonte} 
No arquivo USPSC-modelo.tex ou no USPSC-TCC-modelo.tex é possível optar pela fonte desejada, conforme a programação reproduzida abaixo:
				\begin{verbatim}
				\usepackage{lmodern}			% Usa a fonte Latin Modern
					% Para utilizar a fonte Times New Roman, inclua 
					% uma % no início do comando acima  "\usepackage{lmodern}"
					% Abaixo, tire a % antes do comando  \usepackage{times}
					%\usepackage{times}			% Usa a fonte Times New Roman
					% Lembre-se de alterar a fonte no comando que imprime 
					% o preâmbulo no arquivo da Classe USPSC.cls					
				\end{verbatim}
\subsubsection{Opções de cores dos links}\label{coreslinks} 
No arquivo USPSC-modelo.tex e no USPSC-TCC-modelo.tex é possível alterar as cores dos links, de preto para as alternativas pré-estabelecidas, ou optar por outras cores, conforme a programação reproduzida abaixo:
\begin{verbatim}
% informações do PDF
\makeatletter
\hypersetup{
%pagebackref=true,
pdftitle={\@title}, 
pdfauthor={\@author},
pdfsubject={\imprimirpreambulo},
pdfcreator={LaTeX with abnTeX2},
pdfkeywords={abnt}{latex}{abntex}{USPSC}{trabalho acadêmico}, 
colorlinks=true,       		% false: boxed links; true: colored links
linkcolor=black,          	% color of internal links
citecolor=black,        		% color of links to bibliography
filecolor=black,      		% color of file links
urlcolor=black,
%Para {\tiny habili}tar as cores dos links, retire a % antes dos comandos 
abaixo e inclua a % antes das 4 linhas de comando acima 
%linkcolor=blue,            	% color of internal links
%citecolor=blue,        		% color of links to bibliography
%filecolor=magenta,      		% color of file links
%urlcolor=blue,
bookmarksdepth=4	
}
\makeatother				
\end{verbatim}				
\subsubsection{Impressão anverso e verso ou somente anverso}
No arquivo USPSC-modelo.tex ou no USPSC-TCC-modelo.tex é possível optar por impressão em páginas ou em folhas, conforme a seguinte programação:
			  \begin{verbatim}
			  twoside,  % para impressão em anverso (frente) e verso. Oposto 
			            a oneside - Nota: utilizar \imprimirfolhaderosto*
			  %oneside, % para impressão em páginas separadas (somente 
			            anverso) -  Nota: utilizar \imprimirfolhaderosto
			            % inclua uma % antes do comando twoside e exclua a % 
			            antes do oneside 
			  \end{verbatim}			  
\subsubsection{Opção de p. ou f. na referência da Errata, do Resumo e do Abstract} 
 No arquivo USPSC-modelo.tex ou no USPSC-TCC-modelo.tex, indicar p. ou f. em conformidade com a opção de impressão anverso e verso ou somente anverso, conforme a seguinte programação:
			  \begin{verbatim}
			  \pageref{LastPage}p. 
			  %Substitua p. por f. quando utilizar oneside em \documentclass
			  %\pageref{LastPage}f.
			  \end{verbatim}			  
\subsubsection{Tipos de cabeçalhos de páginas} 
Tanto no arquivo USPSC-modelo.tex como no USPSC-TCC-modelo.tex é possível optar por dois tipos de cabeçalhos em conformidade com o definido abaixo:
			  \begin{verbatim}
			  \documentclass[
			  ...
			  % {USPSC-classe/USPSC} configura o cabeçalho contendo apenas o número
			    da página
			  ]{USPSC-classe/USPSC}
			  %]{USPSC-classe/USPSC1}
			  % Inclua % antes de ]{USPSC-classe/USPSC} e retire a % antes 
			    de %]{USPSC-classe/USPSC1} para utilizar o cabeçalho diferenciado
			  % para as páginas pares e ímpares: 
			  %- páginas ímpares: com seções ou subseções e o número da página
			  %- páginas pares: com o número da página e o título do capítulo 
			  \end{verbatim}
\subsubsection{Opções de idiomas do texto}\label{idioma} 
No arquivo USPSC-modelo.tex ou no USPSC-TCC-modelo.tex há duas opções de idiomas do texto: português ou inglês, conforme programação abaixo:			  
			  \begin{verbatim}
			  % Seleciona o idioma do documento (conforme pacotes do babel)
			  \selectlanguage{brazil}
			  % Se o idioma do texto for inglês, inclua uma % antes do 
			  %      comando \selectlanguage{brazil} e 
			  %      retire a % antes do comando abaixo
			  %\selectlanguage{english}			  
			  \end{verbatim}
É importante salientar que quando o idioma do texto for inglês é necessário que o arquivo pré-textual tenha a programação correspondente neste idioma, incluindo preâmbulo, título do documento, área de concentração, dentre outros dados.
 			  
\subsubsection{Utilização de pacotes para a indicação de número de autores nas referências e para citações alfabéticas ou numéricas}
É possível indicar todos os autores nas referências ou utilizar \textbf{\textit{et al}} quando houver mais de três autores. Como somente o IQSC indica todos os autores, adotamos o \textbf{\textit{et al}} e incluímos a orientação de como proceder para alterar a programação para indicar todos, tanto no arquivo USPSC-modelo.tex como no USPSC-TCC-modelo.tex.

Outra possibilidade é de optar por citações alfabéticas ou numéricas, conforme a seguinte orientação contida no arquivo USPSC-modelo.tex e no USPSC-TCC-modelo.tex:	
		  
			 \begin{verbatim}
			 % ---
			 % Pacotes de citações
			 % Citações padrão ABNT
			 % ---
			 % Sistemas de chamada: autor-data ou numérico.
			 % Sistema autor-data
			 \usepackage[alf, abnt-emphasize=bf, abnt-thesis-year=both,
			 abnt-repeated-author-omit=no, abnt-last-names=abnt,
			 abnt-etal-cite, abnt-etal-list=3, abnt-etal-text=it,
			 abnt-and-type=e, abnt-doi=doi, abnt-url-package=none,
			 abnt-verbatim-entry=no]{abntex2cite}
			 \bibliographystyle{USPSC-classe/abntex2-alf-USPSC}
			 % Se o idioma for o inglês, inclua % no comando acima e 
			 exclua o % do comando abaixo
			 %\bibliographystyle{USPSC-classe/abntex2-alfeng-USPSC}
			 
			 % Para o IQSC, que indica todos os autores nas referências, incluir % 
			 no início dos comandos acima e retirar a % dos comandos abaixo 
			 %\usepackage[alf, abnt-emphasize=bf, abnt-thesis-year=both,
			 abnt-repeated-author-omit=no, abnt-last-names=abnt,
			 abnt-etal-cite, abnt-etal-list=0, abnt-etal-text=it,
			 abnt-and-type=e, abnt-doi=doi, abnt-url-package=none,
			 abnt-verbatim-entry=no]{abntex2cite} 
			 %\bibliographystyle{USPSC-classe/abntex2-alf-USPSC}
			 % Se o idioma for o inglês, exclua % no comando acima ou do
			 comando abaixo
			 %\bibliographystyle{USPSC-classe/abntex2-alfeng-USPSC}
			 
			 % Sistema Numérico
			 % Para citações numéricas, sistema adotado pelo IFSC, incluir % no 
			 início dos comandos acima e retirar a % dos comandos abaixo
			 %\usepackage[num, abnt-emphasize=bf, abnt-thesis-year=both,
			 abnt-repeated-author-omit=no, abnt-last-names=abnt,
			 abnt-etal-cite, abnt-etal-list=3, abnt-etal-text=it,
			 abnt-and-type=e, abnt-doi=doi, abnt-url-package=none,
			 abnt-verbatim-entry=no]{abntex2cite} 
			 %\bibliographystyle{USPSC-classe/abntex2-num-USPSC}
			 % Se o idioma for o inglês, exclua % no comando acima ou do
			 comando abaixo
			 %\bibliographystyle{USPSC-classe/abntex2-numeng-USPSC}
			 
			 % Complementarmente, verifique as instruções abaixo sobre os
			 Pacotes de Nota de rodapé
			 % ---
			 % Pacotes de Nota de rodapé
			 % Configurações de nota de rodapé
			  
			 %O presente modelo adota o formato numérico para as notas de rodapés 
			 quando utiliza o sistema de chamada autor-data para citações e 
			 referências. Para utilizar o sistema de chamada numérico para 
			 citações e referências, habilitar um dos comandos abaixo.
			 % São diversas as opções para nota de rodapé no Sistema Numérico.  
			 Para o IFSC, habilitar o comando abaixo.
			  
			 %\renewcommand{\thefootnote}{\fnsymbol{footnote}} %Comando para 
			 inserção de símbolos em nota de rodapé
			  
			 % Outras opções para nota de rodapé no Sistema Numérico:
			 %\renewcommand{\thefootnote}{\alph{footnote}}     %Comando para 
			 inserção de letras minúsculas em nota de rodapé
			 %\renewcommand{\thefootnote}{\Alph{footnote}}     %Comando para 
			 inserção de letras maiúsculas em nota de rodapé
			 %\renewcommand{\thefootnote}{\roman{footnote}}    %Comando para 
			 inserção de números romanos minúsculos  em nota de rodapé
			 %\renewcommand{\thefootnote}{\Roman{footnote}}    %Comando para 
			 inserção de números romanos minúsculos  em nota de rodapé
			  
			 \renewcommand{\footnotesize}{\small} %Comando para diminuir a fonte 
			 das notas de rodapé	
			  
			 % ---
			 % Pacote para agrupar a citação numérica consecutiva
			 % Quando for adotado o Sistema Numérico, a exemplo do IFSC, habilite 
			 % o pacote cite abaixo retirando a porcentagem antes do comando abaixo
			 %\usepackage[superscript]{cite}
			  			  	
			 \end{verbatim}

Quando o idioma do texto for inglês, para o sistema autor-data utilize o comando \verb+\bibliographystyle{USPSC-classe/abntex2-alfeng-USPSC}+ e para o sistema numérico deverá ser utilizado \verb+\bibliographystyle{USPSC-classe/abntex2-numeng-USPSC}+.

Sugerimos que quando for alterada a programação do Sistema autor-data para o numérico e/ou vice-versa, o arquivo original USPSC-modelo.tex ou USPSC-TCC-modelo.tex seja renomeado, pois durante a compilação são gerados arquivos temporários que podem interferir nas mudanças desejadas.			 

\subsubsection{Possibilidades de preâmbulos}

Atualmente disponibiliza 97 possibilidades de preâmbulos codificados nos arquivos pré-textuais, em conformidade com as siglas estabelecidas para os programas de pós-graduação das Unidades do Campus USP de São Carlos \textbf{(APÊNDICES B-J)}, sendo:
	
				  
			   \begin{alineas}
			   	\item EESC – 43;
				\item IAU – 4;
				\item  ICMC – 16;
				\item  IFSC – 28;
				\item  IQSC - 6;
			  \end{alineas}	
			  
O modelo para TCC está disponível inicialmente para a EESC, ICMC e IQSC, sendo que foram codificados 28 preâmbulos referentes aos cursos de graduação destas Unidades nos arquivos USPSC-TCC-pre-textual-UUUU.tex. As siglas estabelecidas estão relacionadas nos \textbf{APÊNDICES H-J}.
				\begin{alineas}
					\item EESC – 10;
					\item  ICMC – 16;
					\item  IQSC - 2;
				\end{alineas}	
	  						
É importante alertar que tais siglas foram estabelecidas com o objetivo de transferência de parâmetros no Pacote USPSC e não são as utilizadas oficialmente pelas Unidades para referenciar os seus programas de pós-graduação e cursos de graduação.

\subsubsection{Versão original ou final/corrigida}
Nos arquivos com os elementos pré-textuais das Unidades é possível especificar a versão do trabalho acadêmico produzido, a exemplo do contido em USPSC-pre-textual-IFSC.tex:	  
			  \begin{verbatim}
			  \notafolharosto{Vers\~ao original}
			  %Para versão original em inglês, comente do comando/declaração 
			  %     acima(inclua % antes do comando acima) e tire a % do 
			  %     comando/declaração abaixo no idioma do texto
			  %\notafolharosto{Original version} 
			  %Para versão corrigida, comente do comando/declaração da 
			  %     versão original acima (inclua % antes do comando acima) 
			  %     e tire a % do comando/declaração de um dos comandos 
			  %     abaixo em conformidade com o idioma do texto
			  %\notafolharosto{Vers\~ao corrigida \\(Vers\~ao original dispon\'ivel na
			  Unidade que aloja o Programa)}
			  %\notafolharosto{Corrected version \\(Original version available on the
			  Program Unit)}
			  \end{verbatim}
			  
\subsubsection{Ficha catalográfica}
É possível elaborar a ficha catalográfica em \LaTeX\ ou incluir a fornecida pela Biblioteca. Para tanto observe a programação contida nos arquivos USPSC-modelo.tex ou USPSC-TCC-modelo.tex  e USPSC-fichacatalografica.tex e/ou gere o arquivo fichacatalografica.pdf.
	  
No arquivo USPSC-modelo.tex ou no USPSC-TCC-modelo.tex faça a sua opção conforme orientações reproduzidas abaixo:

			 \begin{verbatim}
			 
			 % ---
			 % Inserir a ficha catalográfica em pdf
			 % ---
			 % A biblioteca da sua Unidade lhe fornecerá um PDF com a ficha
			 % catalográfica definitiva. 
			 % Quando estiver com o documento, salve-o como PDF no diretório
			 % do seu projeto como fichacatalografica.pdf e inclua o arquivo
			 % utilizando o comando abaixo:
			 %\begin{fichacatalografica}
			 %   \includepdf{fichacatalografica.pdf}
			 %\end{fichacatalografica}
			 % Se você optar por elaborar a ficha catalográfica, deverá 
			 % incluir uma % antes das 3 linhas acima e tirar a % antes
			 % do comando %% USPSC-fichacatalografica.tex
% ---
% Inserir a ficha bibliografica
% ---
% Isto é um exemplo de Ficha Catalográfica, ou ``Dados internacionais de
% catalogação-na-publicação''. Você pode utilizar este modelo como referência. 
% Porém, provavelmente a biblioteca da sua universidade lhe fornecerá um PDF
% com a ficha catalográfica definitiva após a defesa do trabalho. Quando estiver
% com o documento, salve-o como PDF no diretório do seu projeto e substitua todo
% o conteúdo de implementação deste arquivo pelo comando abaixo:
%
\begin{fichacatalografica}
	\hspace{-1.4cm}
	\imprimirnotaautorizacao \\ \\
	%\sffamily
	\vspace*{\fill}					% Posição vertical
\begin{center}					% Minipage Centralizado
  \imprimirnotabib \\
  \begin{table}[htb]
	\scriptsize
	\centering	
	\begin{tabular}{|p{0.9cm} p{8.7cm}|}
		\hline
	      & \\
		  &	  \imprimirautorficha     \\
		
		 \imprimircutter & 
							\hspace{0.4cm}\imprimirtitulo~  / ~\imprimirautor~ ;  ~\imprimirorientadorcorpoficha. -- 	\imprimirlocal, \imprimirdata.   \\
		
		  &  % Para incluir nota referente à versão corrigida no corpo da ficha,
			  % incluir % no início da linha acima e tirar a % do início da linha abaixo
			  %	\hspace{0.4cm} \imprimirtitulo~  / ~\imprimirautor~ ; ~\imprimirorientadorcorpoficha~- ~\imprimirnotafolharosto. -- \imprimirlocal, \imprimirdata.  \\
		
			\hspace{0.4cm}\pageref{LastPage} p. : il. (algumas color.) ; 30 cm.\\ 
		  & \\
		  & 
		    \hspace{0.4cm}\imprimirnotaficha ~--~ 
						  \imprimirunidademin, 
						  \imprimiruniversidademin, 
		                  \imprimirdata. \\ 
		  & \\                 
		   % Para incluir nota referente à versão corrigida em notas,
		    % incluir uma % no início da linha acima e	
		    % tirar a % do início da linha abaixo
		    % & \hspace{0.4cm}\imprimirnotafolharosto \\ 
		  & \\ 
		  & \hspace{0.4cm}1. LaTeX. 2. abnTeX. 3. Classe USPSC. 4. Editoração de texto. 5. Normalização da documentação. 6. Tese. 7. Dissertação. 8. Documentos (elaboração). 9. Documentos eletrônicos. I. \imprimirorientadorficha. 
		   II. Título. \\
	
		     %Se houver co-orientador, inclua % antes da linha (antes de II. Título.) 
		     %          e tire a % antes do comando abaixo 
		     %III. Título. \\   
		  \hline
	\end{tabular}
  \end{table}
\end{center}
\end{fichacatalografica}
% ---


			 %% USPSC-fichacatalografica.tex
% ---
% Inserir a ficha bibliografica
% ---
% Isto é um exemplo de Ficha Catalográfica, ou ``Dados internacionais de
% catalogação-na-publicação''. Você pode utilizar este modelo como referência. 
% Porém, provavelmente a biblioteca da sua universidade lhe fornecerá um PDF
% com a ficha catalográfica definitiva após a defesa do trabalho. Quando estiver
% com o documento, salve-o como PDF no diretório do seu projeto e substitua todo
% o conteúdo de implementação deste arquivo pelo comando abaixo:
%
\begin{fichacatalografica}
	\hspace{-1.4cm}
	\imprimirnotaautorizacao \\ \\
	%\sffamily
	\vspace*{\fill}					% Posição vertical
\begin{center}					% Minipage Centralizado
  \imprimirnotabib \\
  \begin{table}[htb]
	\scriptsize
	\centering	
	\begin{tabular}{|p{0.9cm} p{8.7cm}|}
		\hline
	      & \\
		  &	  \imprimirautorficha     \\
		
		 \imprimircutter & 
							\hspace{0.4cm}\imprimirtitulo~  / ~\imprimirautor~ ;  ~\imprimirorientadorcorpoficha. -- 	\imprimirlocal, \imprimirdata.   \\
		
		  &  % Para incluir nota referente à versão corrigida no corpo da ficha,
			  % incluir % no início da linha acima e tirar a % do início da linha abaixo
			  %	\hspace{0.4cm} \imprimirtitulo~  / ~\imprimirautor~ ; ~\imprimirorientadorcorpoficha~- ~\imprimirnotafolharosto. -- \imprimirlocal, \imprimirdata.  \\
		
			\hspace{0.4cm}\pageref{LastPage} p. : il. (algumas color.) ; 30 cm.\\ 
		  & \\
		  & 
		    \hspace{0.4cm}\imprimirnotaficha ~--~ 
						  \imprimirunidademin, 
						  \imprimiruniversidademin, 
		                  \imprimirdata. \\ 
		  & \\                 
		   % Para incluir nota referente à versão corrigida em notas,
		    % incluir uma % no início da linha acima e	
		    % tirar a % do início da linha abaixo
		    % & \hspace{0.4cm}\imprimirnotafolharosto \\ 
		  & \\ 
		  & \hspace{0.4cm}1. LaTeX. 2. abnTeX. 3. Classe USPSC. 4. Editoração de texto. 5. Normalização da documentação. 6. Tese. 7. Dissertação. 8. Documentos (elaboração). 9. Documentos eletrônicos. I. \imprimirorientadorficha. 
		   II. Título. \\
	
		     %Se houver co-orientador, inclua % antes da linha (antes de II. Título.) 
		     %          e tire a % antes do comando abaixo 
		     %III. Título. \\   
		  \hline
	\end{tabular}
  \end{table}
\end{center}
\end{fichacatalografica}
% ---


			 % As informações que compõem a ficha catalográfica estão 
			 % definidas no arquivo USPSC-pre-textual-UUUU.tex
			 % ---
			 \end{verbatim} 
			 				
É possível incluir ou não o Código Cutter na ficha catalográfica, conforme a seguinte orientação nos respectivos arquivos pré-textuais:

\begin{verbatim}
\cutter{S856m}
% Para gerar a ficha catalográfica sem o Código Cutter, basta 
% incluir uma % na linha acima e tirar a % da linha abaixo
%\cutter{ } 
\end{verbatim} 

Através do arquivo fichacatalografica.tex é possível elaborar a ficha catalográfica em \LaTeX\ . Caso o trabalho possua co-orientador será necessário seguir as orientações contidas também no arquivo com os elementos pré-textuais.	 


\section{Resultados de comandos}\label{sec-divisoes}

O conteúdo desta seção foi baseado no item \textbf{1 Resultados de comandos} do \textbf{Modelo canônico de trabalho acadêmico com abnTEX2} \cite{equipeabntex2}.

% ---
\subsection{Codificação dos arquivos: UTF8}
% ---

A codificação \texttt{UTF8} deve ser utilizada para todos os arquivos do \abnTeX\ . Utilize a mesma codificação nos documentos que escrever, incluindo nos arquivos de bases bibliográficas |.bib|. Para tanto, tanto o arquivo USPSC-modelo.tex  quanto o USPSC-TCC-modelo.tex devem conter o seguinte pacote:

\begin{verbatim}
\usepackage[utf8]{inputenc}	 % Codificacao do documento (conversão
                               automática dos acentos)
\end{verbatim}

% ---
\subsection{Diferentes idiomas e hifenizações}
\label{sec-hifenizacao}
% ---

Para usar hifenizações de diferentes idiomas, inclua nas opções do documento o
nome dos idiomas que o seu texto contém. Os usuários da Classe USPSC devem utilizar:

\begin{verbatim}
\documentclass[
% -- opções da classe memoir --
12pt,		% tamanho da fonte
openright,	% capítulos começam em pág ímpar (insere página vazia caso 
preciso)
twoside,  % para impressão em anverso (frente) e verso. Oposto a oneside - 
Nota: utilizar \imprimirfolhaderosto*
%oneside, % para impressão em páginas separadas (somente anverso) -  
Nota: utilizar \imprimirfolhaderosto
% inclua uma % antes do comando twoside e exclua a % antes do oneside 
a4paper,			% tamanho do papel. 
% -- opções da classe abntex2 --
chapter=TITLE,		% títulos de capítulos convertidos em letras 
maiúsculas
% -- opções do pacote babel --
english,			% idioma adicional para hifenização
french,				% idioma adicional para hifenização
spanish,			% idioma adicional para hifenização
brazil				% o último idioma é o principal do documento
% {USPSC-classe/USPSC} configura o cabeçalho contendo apenas o número 
da página
]{USPSC-classe/USPSC}
%]{USPSC-classe/USPSC1}
% Inclua % antes de ]{USPSC-classe/USPSC} e retire a % antes 
de %]{USPSC-classe/USPSC1} para utilizar o 
% cabeçalho diferenciado para as páginas pares e ímpares:
%- páginas ímpares: cabeçalho com seções ou subseções e o número da página
%- páginas pares: cabeçalho com o número da página e o título do capítulo 
% ---
\end{verbatim}

Desta forma o texto deverá ser escrito idioma português-brasileiro (\texttt{brazil}), podendo ter citações em inglês, francês e espanhol.

O idioma português-brasileiro (\texttt{brazil}) é incluído automaticamente pela
classe \textsf{abntex2}. Porém, mesmo assim a opção \texttt{brazil} deve ser
informada como a última opção da classe para que todos os pacotes reconheçam o
idioma. Vale ressaltar que a última opção de idioma é a utilizada por padrão no
documento. 

Portanto, para Classe USPSC, caso deseje escrever um texto em inglês que tenha
citações em espanhol, português e francês, você deverá usar:

\begin{verbatim}
\documentclass[
% -- opções da classe memoir --
12pt,		% tamanho da fonte
openright,	% capítulos começam em pág ímpar (insere página vazia caso 
preciso)
twoside,  % para impressão em anverso (frente) e verso. Oposto a oneside - 
Nota: utilizar \imprimirfolhaderosto*
%oneside, % para impressão em páginas separadas (somente anverso) -  
Nota: utilizar \imprimirfolhaderosto
% inclua uma % antes do comando twoside e exclua a % antes do oneside 
a4paper,			% tamanho do papel. 
% -- opções da classe abntex2 --
chapter=TITLE,		% títulos de capítulos convertidos em letras 
maiúsculas
% -- opções do pacote babel --
spanish,			% idioma adicional para hifenização
french,				% idioma adicional para hifenização
brazil,				% o último idioma é o principal do documento
english 			% idioma adicional para hifenização
% {USPSC-classe/USPSC} configura o cabeçalho contendo apenas o número 
da página
]{USPSC-classe/USPSC}
%]{USPSC-classe/USPSC1}
% Inclua % antes de ]{USPSC-classe/USPSC} e retire a % antes 
de %]{USPSC-classe/USPSC1} para utilizar o 
% cabeçalho diferenciado para as páginas pares e ímpares:
%- páginas ímpares: cabeçalho com seções ou subseções e o número da página
%- páginas pares: cabeçalho com o número da página e o título do capítulo 
% ---
\end{verbatim}

A lista completa de idiomas suportados, bem como outras opções de hifenização,
estão disponíveis em \citeonline[p.~7-8]{babel2011}. \\

Exemplo de hifenização em inglês\footnote{Extraído de:
	\url{http://en.wikibooks.org/wiki/LaTeX/Internationalization}}:

\begin{otherlanguage*}{english}
	\textit{Text in English language. This environment switches all language-related
		definitions, like the language specific names for figures, tables etc. to the other
		language. The starred version of this environment typesets the main text
		according to the rules of the other language, but keeps the language specific
		string for ancillary things like figures, in the main language of the document.
		The environment hyphenrules switches only the hyphenation patterns used; it can
		also be used to disallow hyphenation by using the language name
		`nohyphenation'.}
\end{otherlanguage*}

Exemplo de hifenização em francês\footnote{Extraído de:
	\url{http://bigbrowser.blog.lemonde.fr/2013/02/17/tu-ne-tweeteras-point-le-vatican-interdit-aux-cardinaux-de-tweeter-pendant-le-conclave/}}:

\begin{otherlanguage*}{french}
	\textit{Texte en français. Pas question que Twitter ne vienne faire une
		concurrence déloyale à la traditionnelle fumée blanche qui marque l'élection
		d'un nouveau pape. Pour éviter toute fuite précoce, le Vatican a donc pris un
		peu d'avance, et a déjà interdit aux cardinaux qui prendront part au vote
		d'utiliser le réseau social, selon Catholic News Service. Une mesure valable
		surtout pour les neuf cardinaux – sur les 117 du conclave – pratiquants très
		actifs de Twitter, qui auront interdiction pendant toute la période de se
		connecter à leur compte.}
\end{otherlanguage*}

Exemplo de hifenização em espanhol\footnote{Extraído de:
	\url{http://internacional.elpais.com/internacional/2013/02/17/actualidad/1361102009_913423.html}}:

\foreignlanguage{spanish}{\textit{Decenas de miles de personas ovacionan al pontífice en su
		penúltimo ángelus dominical, el primero desde que anunciase su renuncia. El Papa se
		centra en la crítica al materialismo}}.

O idioma geral do texto pode ser alterado como no exemplo seguinte:

\begin{verbatim}
\selectlanguage{english}

\end{verbatim}

Isso altera automaticamente a hifenização e todos os nomes constantes de
referências do documento para o idioma inglês. Consulte o manual da classe para obter orientações adicionais sobre internacionalização de documentos produzidos com \textsf{\abnTeX} \cite{abnetxclasse}.

Se a opção de idioma do texto não for o português, é necessário observar o descrito em \ref{idioma}.

% ---
\subsection{Enumerações}
% ---

\index{alíneas}\index{subalíneas}\index{incisos}Quando for necessário enumerar
os diversos assuntos de uma seção que não possua título, esta deve ser
subdividida em alíneas \cite[4.2]{nbr6024}:

\begin{alineas}

  \item os diversos assuntos que não possuam título próprio, dentro de uma mesma
  seção, devem ser subdivididos em alíneas; 
  
  \item o texto que antecede as alíneas termina em dois pontos;
  \item as alíneas devem ser indicadas alfabeticamente, em letra minúscula
  seguida de parêntese. Utilizam-se letras dobradas, quando esgotadas as
  letras do alfabeto;

  \item as letras indicativas das alíneas devem apresentar recuo em relação à
  margem esquerda;

  \item o texto da alínea deve começar por letra minúscula e terminar em
  ponto-e-vírgula, exceto a última alínea que termina em ponto final;

  \item o texto da alínea deve terminar em dois pontos, se houver subalínea;

  \item a segunda e as seguintes linhas do texto da alínea começa sob a
  primeira letra do texto da própria alínea;
  
  \item subalíneas \cite{nbr6024} devem ser conforme as alíneas a
  seguir:

  \begin{alineas}
     \item as subalíneas devem começar por travessão seguido de espaço;

     \item as subalíneas devem apresentar recuo em relação à alínea;

     \item o texto da subalínea deve começar por letra minúscula e terminar em
     ponto-e-vírgula. A última subalínea deve terminar em ponto final, se não
     houver alínea subsequente;

     \item a segunda e as seguintes linhas do texto da subalínea começam sob a
     primeira letra do texto da própria subalínea.
  \end{alineas}
  
  \item no \abnTeX\ estão disponíveis os ambientes \texttt{incisos} e
  \texttt{subalineas}, que em suma é o mesmo que se criar outro nível de
  \texttt{alineas}, como nos exemplos à seguir:
  
  \begin{incisos}
    \item \textit{Um novo inciso em itálico};
  \end{incisos}
  
  \item Alínea em \textbf{negrito}:
  
  \begin{subalineas}
    \item \textit{Uma subalínea em itálico};
    \item \underline{\textit{Uma subalínea em itálico e sublinhado}}; 
  \end{subalineas}
  
  \item Última alínea com \emph{ênfase}.
  
\end{alineas}

% ---
\subsection{Espaçamento entre parágrafos e linhas}\label{sec_espacamento}
% ---

\index{espaçamento!dos parágrafos}O tamanho do parágrafo, espaço entre a margem
e o início da frase do parágrafo, é definido por:

\begin{verbatim}
   \setlength{\parindent}{1.3cm}
\end{verbatim}

\index{espaçamento!do primeiro parágrafo}Por padrão, não há espaçamento no
primeiro parágrafo de cada início de divisão do documento
(\autoref{sec-divisoes-b}). Porém, você pode definir que o primeiro parágrafo
também seja indentado, como é o caso deste documento. Para isso, apenas inclua o
pacote \textsf{indentfirst} no preâmbulo do documento:

\begin{verbatim}
   \usepackage{indentfirst} % Indenta o primeiro parágrafo de cada seção.
\end{verbatim}

\index{espaçamento!entre os parágrafos}O espaçamento entre um parágrafo e outro
pode ser controlado por meio do comando:

\begin{verbatim}
  \setlength{\parskip}{0.2cm}  % tente também \onelineskip
\end{verbatim}

\index{espaçamento!entre as linhas}O controle do espaçamento entre linhas é
definido por:
\begin{verbatim}
  \OnehalfSpacing       % espaçamento um e meio (padrão); 
  \DoubleSpacing        % espaçamento duplo
  \SingleSpacing        % espaçamento simples	
\end{verbatim}

Para isso, também estão disponíveis os ambientes:
\begin{verbatim}
  \begin{SingleSpace} ...\end{SingleSpace}
  \begin{Spacing}{hfactori} ... \end{Spacing}
  \begin{OnehalfSpace} ... \end{OnehalfSpace}
  \begin{OnehalfSpace*} ... \end{OnehalfSpace*}
  \begin{DoubleSpace} ... \end{DoubleSpace}
  \begin{DoubleSpace*} ... \end{DoubleSpace*} 
\end{verbatim}

% ---
\subsection{Tabelas e quadros}

As tabelas e os quadros apresentam os dados de modo resumido, oferecendo uma visão geral do conteúdo em questão, visando facilitar a compreensão do fenômeno em estudo. A diferença básica entre ambas está relacionada ao conteúdo e a formatação. 

Tabela é o conjunto de dados estatísticos, dispostos em determinada ordem de classificação, que expressam as variações qualitativas de um fenômeno. Sua finalidade básica é resumir ou sintetizar dados \cite{sibi2016}.

A construção de tabelas deve obedecer os critérios estabelecidos pelo Instituto Brasileiro de Geografia e Estatística (IBGE) e requeridos pelas normas da ABNT para documentos técnicos e acadêmicos.

A \autoref{tab-ibge} é um exemplo de tabela alinhada que pode ser longa ou curta, conforme padrão do IBGE.

\begin{table}[htb]
	%\begin{table}[H]
	\IBGEtab{%
		\caption{Frequência anual por categoria de usuários}%
		\label{tab-ibge}
	}{%
	\begin{tabular}{ccc}
		\toprule
		Categoria de Usuários & Frequência de Usuários \\
		\midrule \midrule
		Graduação & 72\% \\
		\midrule 
		Pós-Graduação & 15\% \\
		\midrule 
		Docente & 10\% \\
		\midrule 
		Outras & 3\% \\
		\bottomrule
	\end{tabular}%
}{%
\fonte{Elaborada pelos autores.}%
\nota{Exemplo de uma nota.}%
\nota[Anotações]{Uma anotação adicional, que pode ser seguida de várias
	outras.}%

}
\end{table}


\begin{table}[H]
	\IBGEtab{%
		\caption{Níveis descritivos dos testes de comparação de médias entre grupos para profundidade da lesão junto à restauração}%
		\label{tabela-ibge}
	}{%
	\begin{tabular}{p{5.5cm}|p{5.5cm}}
		\hline
		\textbf{Resultado} & \textbf{Nível Descritivo} \\ 
		\hline 
		CIC < Ariston & < 0,0001  \\
		Ariston < Am & 0,0118  \\
		Am = Helio & 0,4576  \\
		-100 = Helio & 0,3360  \\
		\hline
	\end{tabular}%
}{%
\fonte{\citeonline{sibi2009}}%
}
\end{table} 

Os \textbf{APÊNDICES J-K} exemplificam outras formatações de tabelas.

A formatação do quadro é similar à tabela, mas deve ter suas laterais fechadas e conter as linhas horizontais.

% o comando \newpage foi utilizado para forçar a quebra de página

\begin{quadro}[htb]
	\caption{\label{quadro_modelo}Níveis de investigação}
	\begin{tabular}{|p{2.6cm}|p{6.0cm}|p{2.25cm}|p{3.40cm}|}
		\hline
		\textbf{Nível de Investigação} & \textbf{Insumos}  & \textbf{Sistemas de Investigação}  & \textbf{Produtos}  \\
		\hline
		Meta-nível & Filosofia\index{filosofia} da Ciência  & Epistemologia &
		Paradigma  \\
		\hline
		Nível do objeto & Paradigmas do metanível e evidências do nível inferior &
		Ciência  & Teorias e modelos \\
		\hline
		Nível inferior & Modelos e métodos do nível do objeto e problemas do nível inferior & Prática & Solução de problemas  \\
		\hline
	\end{tabular}
	\begin{flushleft}
		%\fonte{\citeonline{van1986}}
		Fonte: \citeonline{van1986}
	\end{flushleft}
\end{quadro} 


Os \textbf{APÊNDICES B-I} apresentam exemplos de quadros.

% ---
\subsection{Figuras}\label{sec_figuras}
% ---
\index{figuras}Figuras podem ser criadas diretamente em \LaTeX,
como o exemplo da \autoref{fig_circulo}. \\ 

\begin{figure}[htb]
	\caption{\label{fig_circulo}A delimitação do espaço}
	\begin{center}
		\setlength{\unitlength}{9cm}
		\begin{picture}(1,1)
		\put(0,0){\line(0,1){1}}
		\put(0,0){\line(1,0){1}}
		\put(0,0){\line(1,1){1}}
		\put(0,0){\line(1,2){.5}}
		\put(0,0){\line(1,3){.3333}}
		\put(0,0){\line(1,4){.25}}
		\put(0,0){\line(1,5){.2}}
		\put(0,0){\line(1,6){.1667}}
		\put(0,0){\line(2,1){1}}
		\put(0,0){\line(2,3){.6667}}
		\put(0,0){\line(2,5){.4}}
		\put(0,0){\line(3,1){1}}
		\put(0,0){\line(3,2){1}}
		\put(0,0){\line(3,4){.75}}
		\put(0,0){\line(3,5){.6}}
		\put(0,0){\line(4,1){1}}
		\put(0,0){\line(4,3){1}}
		\put(0,0){\line(4,5){.8}}
		\put(0,0){\line(5,1){1}}
		\put(0,0){\line(5,2){1}}
		\put(0,0){\line(5,3){1}}
		\put(0,0){\line(5,4){1}}
		\put(0,0){\line(5,6){.8333}}
		\put(0,0){\line(6,1){1}}
		\put(0,0){\line(6,5){1}}
		\end{picture}
	\end{center}
	\legend{Fonte: \citeonline{equipeabntex2}}
\end{figure}

Outra opção é incorporar a figura utilizando um arquivo externo, como é o caso da \autoref{fig_grafico}. Se a figura que for incluída se tratar de um diagrama, um gráfico ou uma ilustração, que você mesmo produza, priorize o uso de imagens vetoriais no formato PDF. Com isso, o tamanho do arquivo final do trabalho será menor e as imagens terão uma apresentação melhor, principalmente quando impressas, uma vez que imagens vetoriais são perfeitamente escaláveis para qualquer dimensão. Nesse caso, se for utilizar o Microsoft Excel para produzir gráficos, ou o Microsoft Word para ilustrações, exporte-os como PDF e os incorpore ao documento conforme o exemplo abaixo. No entanto, para manter a
coerência no uso de software livre (já que você está usando \LaTeX\  e \abnTeX),
teste a ferramenta \textsf{InkScape}\index{InkScape}
(\url{http://inkscape.org/}). Ela é uma excelente opção de código-livre para
produzir ilustrações vetoriais, similar ao CorelDraw\index{CorelDraw} ou ao Adobe
Illustrator\index{Adobe Illustrator}. De todo modo, caso não seja possível
utilizar arquivos de imagens como PDF, utilize qualquer outro formato, como
JPEG, GIF, BMP, etc. Nesse caso, você pode tentar aprimorar as imagens
incorporadas com o software livre \textsf{Gimp}\index{Gimp}
(\url{http://www.gimp.org/}). Ele é uma alternativa livre ao Adobe
Photoshop\index{Adobe Photoshop}. \\

\begin{figure}[H]
	\caption{\label{fig_grafico}Gráfico produzido em Excel e salvo como PDF}
	\includegraphics[scale=0.5]{USPSC-img/USPSC-modelo-img-grafico.pdf}
	\begin{flushleft}
		Fonte: \citeonline[p. 24]{araujo2012}
	\end{flushleft}	
\end{figure}

% ---
\subsubsection{Figuras em minipages}
% ---

As ilustrações devem sempre ter numeração contínua e única em todo o documento:

% O comando \newpage força a quebra de página

\begin{citacao}
	Qualquer que seja o tipo de ilustração, sua identificação aparece na parte
	superior, precedida da palavra designativa (desenho, esquema, fluxograma,
	fotografia, gráfico, mapa, organograma, planta, quadro, retrato, figura,
	imagem, entre outros), seguida de seu número de ordem de ocorrência no texto,
	em algarismos arábicos, travessão e do respectivo título. Após a ilustração, na
	parte inferior, indicar a fonte consultada (elemento obrigatório, mesmo que
	seja produção do próprio autor), legenda, notas e outras informações
	necessárias à sua compreensão (se houver). A ilustração deve ser citada no
	texto e inserida o mais próximo possível do trecho a que se
	refere \cite{nbr14724}.
\end{citacao}

\emph{Minipages} são usadas para inserir textos ou outros elementos em quadros
com tamanhos e posições controladas. Veja o exemplo da
\autoref{fig_minipage_imagem1} e da \autoref{fig_minipage_grafico2}.

\begin{figure}[H]
	\label{teste}
	\centering
	\begin{minipage}{0.4\textwidth}
		\centering
		\caption{Imagem 1 da minipage} \label{fig_minipage_imagem1}
		\includegraphics[scale=0.9]{USPSC-img/USPSC-modelo-img-marca.pdf}
		\legend{Fonte: \citeonline{equipeabntex2}}
	\end{minipage}
	\hfill
	\begin{minipage}{0.4\textwidth}
		\centering
		\caption{Grafico 2 da minipage} \label{fig_minipage_grafico2}
		\includegraphics[scale=0.2]{USPSC-img/USPSC-modelo-img-grafico.pdf}
		\legend{Fonte: \citeonline[p. 24]{araujo2012}}
	\end{minipage}
\end{figure}

% ---
\subsection{Expressões matemáticas}
% ---

\index{expressões matemáticas}Use o ambiente \texttt{equation} para escrever
expressões matemáticas numeradas:

\begin{equation}
\forall x \in X, \quad \exists \: y \leq \epsilon
\end{equation}

Escreva expressões matemáticas entre \$ e \$, como em $ \lim_{x \to \infty}
\exp(-x) = 0 $, para que fiquem na mesma linha.

Também é possível usar colchetes para indicar o início de uma expressão
matemática que não é numerada.

\[
\left|\sum_{i=1}^n a_ib_i\right|
\le
\left(\sum_{i=1}^n a_i^2\right)^{1/2}
\left(\sum_{i=1}^n b_i^2\right)^{1/2}
\]

Consulte mais informações sobre expressões matemáticas em
\url{https://github.com/abntex/abntex2/wiki/Referencias}.

\subsection{Estruturas, reações e mecanismos de reações químicas}\label{Reaquimica}
Para a versão 3.0 do Pacote USPSC, o Grupo Desenvolvedor optou por utilizar os pacotes \textbf{mychemistry},  \textbf{ChemFig} e o \textbf{TikZ}, que fornecem comandos que permitem compor esquemas complexos de reação química com \LaTeX\ . 

Aqui são apresentados alguns exemplos, sendo que a maioria foram retirados do manual do pacote \textbf{ChemFig}\cite{ChemFigPac}. 


A fórmula estrutural do metano é:


\chemfig{C(-[5]H)(-[2]H)(<[:-70]H)(<:[:-20]H)} \\

\begin{verbatim}
\end{verbatim} 

Molecula da Adrenalina é:

\chemfig{*6((-HO)-=-(-(<[::60]OH)-[::-60]-[::-60,,,2]
	HN-[::+60]CH_3)=-(-HO)=)} \\

\begin{verbatim}
\end{verbatim}

Com o comando abaixo, o \textbf{ChemFig} possibilita escrever o nome de uma molécula abaixo dela. 

\begin{verbatim}
\ chemname [hdimi] {\ chemfig {código da molécula}} {hnamei}
\end{verbatim}

Para exemplificar apresentamos uma reação com os nomes das respectivas moléculas:

\schemestart
\chemname{\chemfig{R-C(-[:-30]OH)=[:30]O}}{Ácido carboxílico}
\+
\chemname{\chemfig{R’OH}}{Álcool}
\arrow(.mid east--.mid west)
\chemname{\chemfig{R-C(-[:-30]OR’)=[:30]O}}{Éster}
\+
\chemname{\chemfig{H_2O}}{Água}
\schemestop
\chemnameinit{}
 \\

\begin{verbatim}
\end{verbatim}

Mediante a utilização dos pacotes \textbf{TikZ} e \textbf{ChemFig}, o pacote \textbf{xcolor} é carregado possibilitando o uso de cores nos códigos de comandos, conforme exemplos abaixo:

\chemfig{C|{\color{blue}H_3}-C(=[1]O)-[7]O|{\color{red}H}} 

\begin{verbatim}
\end{verbatim}

Para destacar substâncias individualmente ou parte de um esquema, é possível utilizar recursos tais como os exemplificados a seguir.

\begin{verbatim}
setchemfig{compound style={draw,line width=0.8pt,
semitransparent,text opacity=1,inner sep=8pt,
rounded corners=1mm}}
\schemestart
A\arrow([fill=red]--[fill=blue])[90]
B\arrow(--[fill=gray])
C\arrow(--[fill=green])[-90]
D\arrow(--[draw=none])[-180]
\schemestop
\end{verbatim} 

\begin{verbatim}
\end{verbatim}



\setchemfig{compound style={draw,line width=0.8pt,
		semitransparent,text opacity=1,inner sep=8pt,
		rounded corners=1mm}}
\schemestart
A\arrow([fill=red]--[fill=blue])[90]
B\arrow(--[fill=gray])
C\arrow(--[fill=green])[-90]
D\arrow(--[draw=none])[-180]
\schemestop

\begin{verbatim}
\end{verbatim} 

Mais um exemplo de utilização dos recursos do pacote TikZ é o desenho de duas linhas e um ponto de interseção. O comando \verb+ \draw[blue, thick] + define um elemento gráfico cuja cor é azul e com um traço grosso. A linha é definida por seus dois pontos finais, (-1,2) e (2, -4), unidos por -. O comando \verb+ \filldraw[red] (0,0) circle (2pt) + \\ \verb+node[anchor=west] {Intersection point} + irá desenhar um círculo preenchido com a cor vermelha, sendo que (0,0) define o ponto central, (2pt) determina o raio e, próximo ao ponto, um nó e uma caixa contendo o texto "ponto de interseção", ancorado a oeste do ponto.


\begin{verbatim}
\begin{tikzpicture}
\draw[blue, thick] (-1,2) -- (2,-4);
\draw[green, thick] (-1,-1) -- (2,2);
\filldraw[red] (0,0) circle (2pt) node[anchor=west] {Intersection point};
\end{tikzpicture}
\end{verbatim}

Tais comandos geram os elementos gráficos abaixo:

\begin{tikzpicture}
\draw[blue, thick] (-1,2) -- (2,-4);
\draw[blue, thick] (-1,-1) -- (2,2);
\filldraw[red] (0,0) circle (2pt) node[anchor=west] {Intersection point};
\end{tikzpicture}


Para gerar este documento, no preâmbulo do arquivo principal (USPSC-modelo.tex e USPSC-TCC-modelo.tex ) foram incluídos os seguintes comandos:
\begin{verbatim}
\usepackage{float} 				% Fixa tabelas e figuras no local exato
\usepackage{chemfig,chemmacros} % Para escrever reações químicas
\usepackage{mychemistry}        % Para escrever reações químicas
\usepackage{tikz}               % Para escrever reações químicas e outros
\usetikzlibrary{arrows, babel}	% Para escrever reações químicas e outros
\end{verbatim}

Para informações complementares e recursos adicionais, consulte os manuais dos pacotes utilizados:  \textbf{mychemistry}\cite{mychemistryPac}, \textbf{ChemFig}\cite{ChemFigPac}, \textbf{etoolbox}\cite{etoolboxPack}, \textbf{float}\cite{floatPac}, \textbf{xkeyval}\cite{xkeyvalPac}, \textbf{chemmacros}\cite{chemmacrosPac}, \textbf{TikZ e PGF}\cite{TikZPac} ou de outros necessários para o seu documento.


% ---
\subsection{Inclusão de outros arquivos}\label{sec-include}
% ---

É uma boa prática dividir o seu documento em diversos arquivos, e não
apenas escrever tudo em um único. Esse recurso foi utilizado neste
documento. Para incluir diferentes arquivos em um arquivo principal,
de modo que cada arquivo incluído fique em uma página diferente, utilize o
comando:

\begin{verbatim}
   \include{documento-a-ser-incluido}      % sem a extensão .tex
\end{verbatim}

Para incluir documentos sem quebra de páginas, utilize:

\begin{verbatim}
   \input{documento-a-ser-incluido}      % sem a extensão .tex
\end{verbatim}
% ---
\subsection{Índice(s)}
% ---
Elemento  opcional,  que  consiste  em  lista  de  palavras  ou  frases  ordenadas alfabeticamente (autor, título ou assunto) ou sistematicamente (ordenação por classes, numérica ou cronológica); localiza e remete para as informações contidas no texto. A paginação deve ser contínua, dando seguimento ao texto principal \cite{aguia2020}.

Para criar índice remissivo no \LaTeX\  utilize o pacote makeidx, que deve estar declarado com os demais pacotes. No presente modelo está declarado no arquivo USPSC-modelo.tex, conforme indicado abaixo:

\begin{verbatim}
% ---
% Pacotes básicos - Fundamentais 
% ---
\usepackage[T1]{fontenc}		% Seleção de códigos de fonte.
\usepackage[utf8]{inputenc}		% Codificação do documento (conversão 
automática dos acentos)
\usepackage{lmodern}			% Usa a fonte Latin Modern
% Para utilizar a fonte Times New Roman, inclua uma % no início do comando 
acima  "\usepackage{lmodern}"
% Abaixo, tire a % antes do comando  \usepackage{times}
%\usepackage{times}		    	% Usa a fonte Times New Roman	
% Lembre-se de alterar a fonte no comando que imprime o preâmbulo no 
arquivo da Classe USPSC.cls				
\usepackage{lastpage}			% Usado pela Ficha catalográfica
\usepackage{indentfirst}		% Indenta o primeiro parágrafo de cada seção.
\usepackage{color}				% Controle das cores
\usepackage{graphicx}			% Inclusão de gráficos
\usepackage{float} 				% Fixa tabelas e figuras no local exato
\usepackage{chemfig,chemmacros} % Para escrever reações químicas
\usepackage{mychemistry}        % Para escrever reações químicas
\usepackage{tikz}				% Para escrever reações químicas e outros
\usetikzlibrary{arrows, babel}	% Para escrever reações químicas e outros
\usepackage{microtype} 			% para melhorias de justificação
\usepackage{pdfpages}
\usepackage{makeidx}            % para gerar índice remissivo
% ---
\end{verbatim}

A habilitação dos comandos de indexação foi incluída no arquivo USPSC-modelo.tex da seguinte forma:


\begin{verbatim}
% compila o sumário e índice
\makeindex
% ---
\end{verbatim}

O presente modelo inclui um exemplo de índice, gerado a partir da utilização de comandos similares aos reproduzidos abaixo:

\begin{verbatim}
\index{InkScape}
\index{CorelDraw}
\index{Adobe Illustrator}
\index{Gimp}
\index{Adobe Photoshop}
\index{espaçamento!do primeiro parágrafo}
\index{espaçamento!dos parágrafos}
\index{espaçamento!entre as linhas}
\index{espaçamento!entre os parágrafos}
\end{verbatim}

Os comandos acima estão no arquivo USPSC-Cap2-Desenvolvimento.tex, em textos na  \autoref{sec_espacamento}  e na  \autoref{sec_figuras}.

Para imprimir o índice, no final do arquivo USPSC-modelo.tex foi incluído:

\begin{verbatim}

%---------------------------------------------------------------------
% INDICE REMISSIVO
%--------------------------------------------------------------------
\phantompart
\printindex
%---------------------------------------------------------------------
\end{verbatim}

Para que o índice seja gerado e incluído corretamente no texto é necessário compilá-lo separadamente. No \textbf{TeXstudio 2.9.4}, na barra de menu, selecione \textbf{Tools} e execute \textbf{Index}.


% ---
\subsection{Compilar o documento \LaTeX}
% ---

Geralmente os editores \LaTeX, como o
TeXlipse\footnote{\url{http://texlipse.sourceforge.net/}}, o
Texmaker\footnote{\url{http://www.xm1math.net/texmaker/}}, entre outros,
compilam os documentos automaticamente, de modo que você não precisa se
preocupar com isso.

No entanto, você pode compilar os documentos \LaTeX\ usando os seguintes
comandos, que devem ser digitados no \emph{Prompt de Comandos} do Windows ou no
\emph{Terminal} do Mac ou do Linux:

\begin{verbatim}
   pdflatex ARQUIVO_PRINCIPAL.tex
   bibtex ARQUIVO_PRINCIPAL.aux
   makeindex ARQUIVO_PRINCIPAL.idx 
   makeindex ARQUIVO_PRINCIPAL.nlo -s nomencl.ist -o ARQUIVO_PRINCIPAL.nls
   pdflatex ARQUIVO_PRINCIPAL.tex
   pdflatex ARQUIVO_PRINCIPAL.tex
\end{verbatim}

% ---
\subsection{Remissões internas}
% ---

Ao nomear a \autoref{tab-ibge} e a \autoref{fig_circulo}, apresentamos um exemplo de remissão interna, que também pode ser feita quando indicamos o
\autoref{cap_exemplos}, que tem o nome \emph{\nameref{cap_exemplos}}. O número
do capítulo indicado é \ref{cap_exemplos}, que se inicia à
\autopageref{cap_exemplos}\footnote{O número da página de uma remissão pode ser
	obtida também assim:
	\pageref{cap_exemplos}.}.
Veja a \autoref{sec-divisoes-b} para outros exemplos de remissões internas entre
seções, subseções e subsubseções.

O código usado para produzir o texto desta seção é:

\begin{verbatim}
Ao nomear a \autoref{tab-nivinv} e a \autoref{fig_circulo}, apresentamos 
um exemplo de remissão interna, que também pode ser feita quando indicamos 
o \autoref{cap_exemplos}, que tem o nome \emph{\nameref{cap_exemplos}}. O
número do capítulo indicado é \ref{cap_exemplos}, que se inicia à 
\autopageref{cap_exemplos}\footnote{O número da página de uma remissão 
pode ser obtida também assim: \pageref{cap_exemplos}.}. Veja a 
\autoref{sec-divisoes-b} para outros exemplos de remissões internas entre 
seções, subseções e subsubseções.
\end{verbatim}

% ---
\section{Divisões do documento}\label{sec-divisoes-b}
Esta seção exemplifica o uso de divisões de documentos em conformidade com a ABNT NBR 6024  \cite{nbr6024}.
% ---
% ---
\subsection{Divisões do documento: subseção}\label{sec-divisoes-subsection}
% ---

Um exemplo de seção é a \autoref{sec-divisoes-b}. Esta é a \autoref{sec-divisoes-subsection}.

\subsubsection{Divisões do documento: subsubseção}\label{sec-divisoes-subsubsection}

Isto é uma \texttt{subsubsection} do \LaTeX, mas é denominada de ``subseção'' porque no português não temos a palavra ``subsubseção''.

\subsubsection{Divisões do documento: subsubseção}

Isto é outra subsubseção.

\subsection{Divisões do documento: subseção}\label{sec-exemplo-subsec}

Isto é uma subseção.

\subsubsection{Divisões do documento: subsubseção}

Isto é mais uma subsubseção da \autoref{sec-exemplo-subsec}.


\subsubsubsection{Esta é uma subseção de quinto
nível}\label{sec-exemplo-subsubsubsection}

Esta é uma seção de quinto nível. Ela é produzida com o seguinte comando:

\begin{verbatim}
\subsubsubsection{Esta é uma subseção de quinto
nível}\label{sec-exemplo-subsubsubsection}
\end{verbatim}

\subsubsubsection{Esta é outra subseção de quinto nível}\label{sec-exemplo-subsubsubsection-outro}

Esta é outra seção de quinto nível.


\paragraph{Este é um parágrafo numerado}\label{sec-exemplo-paragrafo}

Este é um exemplo de parágrafo nomeado. Ele é produzido com o comando de
parágrafo:

\begin{verbatim}
\paragraph{Este é um parágrafo numerado}\label{sec-exemplo-paragrafo}
\end{verbatim}

A numeração entre parágrafos numerados e subsubsubseções são contínuas.

\paragraph{Este é outro parágrafo numerado}\label{sec-exemplo-paragrafo-outro}

Este é outro parágrafo numerado.

% ---
\subsection{Este é um exemplo de nome de subseção longa que se aplica a seções e demais divisões do documento. Ele deve estar alinhado à esquerda e a segunda e demais linhas devem iniciar logo abaixo da primeira palavra da primeira linha} 

Observe que o alinhamento do título obedece esta regra também no sumário.
	

% ---
\section{Manual da classe \textsf{\abnTeX}}
% ---

O manual da classe \textsf{\abnTeX} possui uma referência completa das macros e ambientes disponíveis \cite{abnetxclasse}.

Contém informações adicionais sobre as normas ABNT
observadas pelo \textsf{\abnTeX} e considerações sobre eventuais requisitos específicos
não atendidos, como o caso da ABNT NBR 14724 \cite{nbr14724}, que
especifica o espaçamento entre os capítulos e o início do texto, regra
propositalmente não atendida pelo presente modelo.

% ---
\section{Precisa de ajuda sobre \textsf{\abnTeX}?}
% ---

Consulte a FAQ com perguntas frequentes e comuns no portal do \textsf{\abnTeX}:
\url{https://github.com/abntex/abntex2/wiki/FAQ}.

Inscreva-se no grupo de usuários \LaTeX:
\url{http://groups.google.com/group/latex-br}, tire suas dúvidas e ajude
outros usuários.

Participe também do grupo de desenvolvedores do \textsf{\abnTeX}:
\url{http://groups.google.com/group/abntex2} e faça sua contribuição à
ferramenta.

% ---
\section{Você pode ajudar?}
% ---

Sua contribuição é muito importante! Você pode ajudar na divulgação, no
desenvolvimento e de várias outras formas. Veja como contribuir com o \abnTeX\
em \url{https://github.com/abntex/abntex2/wiki/Como-Contribuir}.

% ---
\section{Quer customizar os modelos do \abnTeX\ para sua instituição ou
universidade?}
% ---

Veja como customizar o \abnTeX\ em:
\url{https://github.com/abntex/abntex2/wiki/ComoCustomizar}.

% ---
\section{Precisa de ajuda sobre o Pacote USPSC?}
% ---
Consulte a Seção de Referência da Biblioteca de sua instituição para obter ajuda sobre o Pacote USPSC.

No Campus USP de São Carlos, consulte uma das seguintes equipes de referência:
\begin{verbatim}
EESC - Serviço de Biblioteca Prof. Dr. Sérgio Rodrigues Fontes 
Atendimento ao Usuário
biblioteca.atendimento@eesc.usp.br
(16) 3373-8860

IAU - Biblioteca
Atendimento ao Usuário
bibiau@sc.usp.br
(16) 3373-9282

ICMC - Biblioteca Prof. Achille Bassi
Seção de Atendimento ao Usuário
biblio@icmc.usp.br
(16) 3373-8619

IFSC - Serviço de Biblioteca e Informação Prof. Bernhard Gross
Seção de Atendimento ao Usuário
comut@ifsc.usp.br
(16) 3373-9778

IQSC - Serviço de Biblioteca e Informação Prof. Johannes Rüdiger Lechat
Seção de Atendimento ao Usuário
bibiqsc@iqsc.usp.br
(16) 3373-9936
\end{verbatim}


O Grupo desenvolvedor do Pacote USPSC esclarece que seu objetivo é oferecer um facilitador para os graduandos e pós-graduandos, mas não se compromete a ensinar a Linguagem de Programação \LaTeX .  

% ---
\section{Customize o Pacote USPSC para sua instituição}
% ---

Para customizar o \textbf{Modelo para TCC em \LaTeX\ utilizando o Pacote USPSC} e/ou o \textbf{Modelo para teses e dissertações em \LaTeX\ utilizando o Pacote USPSC} para outras Unidades da USP e demais instituições interessadas em adotar essas normas e padrões, basta criar um arquivo pré-textual contemplando os cursos de graduação e/ou os programas de pós-graduação vigentes e incluir a chamada do mesmo em USPSC-unidades.tex.

Para solicitar orientações como proceder, contactar as responsáveis pela programação:

\begin{verbatim}
Biblioteca da Prefeitura do Campus USP de São Carlos - PUSP-SC/USP
Marilza Aparecida Rodrigues Tognetti
Ana Paula Aparecida Calabrez
biblioteca.prefeitura@sc.usp.br
(16) 3373-8316
\end{verbatim}





