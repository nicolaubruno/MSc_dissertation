% ---
%% USPSC-Cap4-ReferenciasTutorial.tex
% --
% Este capítulo traz os exemplos de referências das "Diretrizes para apresentação de dissertações e teses da USP: documento eletrônico e impresso - Parte I (ABNT)" disponílvel em: http://biblioteca.puspsc.usp.br/pdfFiles_Caderno_Estudos_9_PT_1.pdf


% --- 
%\chapter{Modelos de re$  $ferências}
\chapter{Modelos de referências}
\label{Referências}
% --- 
Elemento obrigatório, que consiste na relação das obras consultadas e citadas no texto, de maneira que permita a identificação individual de cada uma delas. As referências devem ser organizadas em ordem alfabética, caso as citações no texto obedeçam ao sistema autor-data, ou conforme aparecem no texto, quando utilizado o sistema numérico de chamada. \cite{sibi2016}.

Este capítulo foi elaborado com base nas \textbf{Diretrizes para apresentação de dissertações e teses da USP}: documento eletrônico e impresso - Parte I (ABNT), 3ª edição de 2016, e todos os exemplos aqui apresentados constam dessas Diretrizes, porém em conformidade com ABNT NBR 6023:2018. 

Para organização, gerenciamento e editoração das referências em BibTeX foi utilizado o software JabRef versão 2.10.

A ABNT NBR 6023 especifica os elementos a serem incluídos, fixa sua ordem, orienta a preparação e compilação das referências de materiais utilizados para a produção de documentos e para a inclusão em bibliografias, resumos etc. \cite{nbr6023a}.

Normalmente não há problemas em usar caracteres acentuados em arquivos bibliográficos {(*.bib)}. Porém, como as regras da ABNT NBR 6023 exigem a conversão do autor ou organização para letras maiúsculas, é preciso observar o modo como se escrevem os nomes dos autores. No~\autoref{quadro-acentos} você encontra alguns
exemplos das conversões mais importantes. Preste atenção especial para `ç' e `í'
que devem estar envoltos em chaves. A regra geral é sempre usar a acentuação neste modo quando houver conversão para letras maiúsculas. \cite{abnetxcite} \\

\begin{quadro}[H]
	\caption{\label{quadro-acentos}Conversão de acentuação}
		\begin{tabular}{|p{7.5cm}|p{7.5cm}|}
			\hline
			\textbf{Acentos} & \textbf{BibTeX}\\
			\hline
			à á ã & \verb+\`a+ \verb+\'a+ \verb+\~a+\\
			\hline
			í & \verb+{\'\i}+\\
			\hline
			ç & \verb+{\c c}+\\
			\hline
		\end{tabular}
		\begin{flushleft}
			Fonte: \citeonline{abnetxcite}
		\end{flushleft}	
\end{quadro}


\section{Monografias}

Nesta categoria são incluídos livros, folhetos, guias, catálogos, folderes, dicionários e trabalhos acadêmicos.

Elementos essenciais: autoria, título, edição, local de publicação, editora e ano de publicação.

Elementos complementares: responsabilidade (tradutor, revisor, ilustrador, entre outros), paginação, série, notas e ISBN.

O prenome pode estar abreviado ou por extenso, porém deve estar padronizado em toda a listagem. \\

\subsection{Monografia no todo}

\begin{tabular}{|l|c|} \hline
SOBRENOME, Prenome(s) do(s) autor(es). \textbf{Título da obra}: subtítulo \\ (se houver). Edição (se houver). Local de publicação (cidade):	Editora, data \\
de publicação.  Paginação. Série. Notas. ISBN.\\\hline
\end{tabular}\\

\subsubsection{Um autor}

\begin{tabular}{|l|c|} \hline
 ESPÍRITO SANTO, A. \textbf{Essências de metodologia científica}: aplicada \\
 à educação. Londrina: Universidade Estadual, 1987. \\\hline
\end{tabular}\\

\textbf{Campos em LATEX:}

\begin{verbatim}
\@Book{EspiritoSanto1987,
Title                    = {Essências de metodologia científica},
Address                  = {Londrina},
Author                   = {Esp{\'\i}rito, Santo, A.},
Publisher                = {Universidade Estadual},
Subtitle                 = {aplicada à educação},
Year                     = {1987},
Owner                    = {apcalabrez},
Timestamp                = {2015.09.21}
}
\end{verbatim}

\begin{tabular}{|l|c|} \hline
DE ROSE JUNIOR, D. \textbf{Minibasquetebol na escola}. São Paulo: Ícone, \\ 2015. 128 p. \\\hline
\end{tabular}\\

\newpage

\textbf{Campos em LATEX:}

\begin{verbatim}
@Book{DeRose2015,
Title                    = {Minibasquetebol na escola},
Address                  = {São Paulo},
Author                   = {De, Rose Júnior, D.},
Pages                    = {128},
Publisher                = {Ícone},
Year                     = {2015},
Owner                    = {apcalabrez},
Timestamp                = {2015.09.21}
}
\end{verbatim}

\begin{tabular}{|l|c|} \hline
	SMITH, E. B. \textbf{Basic chemical thermodynamics}. 6th ed. London:\\ Imperial College Press, 2014.  \\\hline
\end{tabular}\\

\textbf{Campos em LATEX:}

\begin{verbatim}
@Book{DeRose2015,
Title                    = {Basic chemical thermodynamics},
Address                  = {London},
Edition                  = {6th ed},
Author                   = {SMITH, E. B.},
Publisher                = {Imperial College Press},
Year                     = {2014},
Owner                    = {apcalabrez},
Timestamp                = {2015.09.21}
}
\end{verbatim}

\subsubsection{Dois autores}

\begin{tabular}{|l|c|} \hline
GOMES, C.B.; KEIL, K. \textbf{Brazilian stone meteorites} Albuquerque: \\ Univerity of New Mexico, 1980. \\\hline
\end{tabular}\\

\textbf{Campos em LATEX:}
\begin{verbatim}
@Book{Novak1967,
Title                    = {Brazilian stone meteorites},
Address                  = {Albuquerque},
Author                   = {Gomes, C. B. and Keil, K.},
Publisher                = {University of New Mexico},
Year                     = {1980},
Owner                    = {apcalabrez},
Timestamp                = {2015.09.21}
}
\end{verbatim}

\begin{tabular}{|l|c|} \hline
DIAS, R. B.; COTO, N. P. \textbf{Odontologia do esporte}: história e \\ evolução. Rio de Janeiro: MedBook, 2014. \\\hline
\end{tabular}\\

\textbf{Campos em LATEX:}

\begin{verbatim}
@Book{DIAS2014,
Title                    = {Odontologia do esporte},
Subtitle                 = {história e evolução},
Address                  = {Rio de Janeiro},
Author                   = {DIAS, R. B.and COTO, N. P.},
Publisher                = {MedBook},
Year                     = {2014},
Owner                    = {apcalabrez},
Timestamp                = {2015.09.21}
}
\end{verbatim}

\subsubsection{Três autores}

\begin{tabular}{|l|c|} \hline
GIANNINI, S. D.; FORTI, N.; DIAMENT, J. \textbf{Cardiologia preventiva}:\\ prevenção primária e secundária. São Paulo: Atheneu, 2000. \\\hline
\end{tabular}\\
\\
\textbf{Campos em LATEX:}

\begin{verbatim}
@Book{Giannini2000,
Title                    = {Cardiologia preventiva},
Address                  = {São Paulo},
Author                   = {Giannini, S. D. and Forti, N. and Diament, 
J.},
Publisher                = {Atheneu},
Subtitle                 = {prevenção primária e secundária},
Year                     = {2000},
Owner                    = {apcalabrez},
Timestamp                = {2015.09.21}
}
\end{verbatim}

\begin{tabular}{|l|c|} \hline
PAMMI, M.; VALLEJO, J. G.; ABRAMS, S. A. \textbf{Nutrition-infection} \\ \textbf{interactions and impacts on human health}. Hoboken: Taylor and Francis, \\ 2014. \\\hline
\end{tabular}\\

\textbf{Campos em LATEX:}

\begin{verbatim}
@Book{Pammi2000,
Title                    = { Nutrition-infection interactions
and impacts on human health},
Address                  = {Hoboken},
Author                   = {PAMMI, M. and VALLEJO, J. G. and ABRAMS, 
S. A.},
Publisher                = {Taylor and Francis},
Year                     = {2014},
Owner                    = {apcalabrez},
Timestamp                = {2015.09.21}
}
\end{verbatim}

\subsubsection{Quatro autores}

\begin{tabular}{|l|c|} \hline
BAST JUNIOR, C. \textit{et al.} (ed.). \textbf{Cancer medicine}. 
5th ed.	New York: \\ American Cancer Society, 2000. 
\\\hline
\end{tabular}\\

\begin{verbatim}
@Book{bast2000,
Title                    = {Cancer medicine},
Address                  = {New York},
Address                  = {Hamilton},
Edition                  = {5th ed},
Author                   = {Bast Junior, C. et al.},
Publisher                = {American Cancer Society},
Publisher                = {BC Decker},
Year                     = {2000},
Owner                    = {apcalabrez},
Timestamp                = {2015.09.21}
}
\end{verbatim}

\begin{tabular}{|l|c|} \hline
PASQUARELLI, M. L. R. \textit{et al.} \textbf{Avaliação do uso de periódicos}. 
São \\ Paulo: SIBi-USP, 1987. 14 p.\\\hline
\end{tabular}\\

\textbf{Campos em LATEX:}

\begin{verbatim}
@Book{Pasquarelli1987,
Title                    = {Avaliação do uso de periódicos},
Address                  = {São Paulo},
Author                   = {Pasquarelli, M. L. R. and Krzyzanowski,
R. F.; Imperatriz, I. M. M.; Noronha, D. P.; Andrade, E.; Zapparoli,
M. C. M.; Bonesio, M. C. M.; Lobo, M. P.; Almeida, M. S.; Arruda, 
R. M. A.; Plaza, R. T. T.},
Pages                    = {14},
Publisher                = {SIBi-USP},
Year                     = {1987},
Owner                    = {apcalabrez},
Timestamp                = {2015.09.21}
}
\end{verbatim}

\textbf{Nota:} quando houver quatro ou mais autores, convém indicar todos. Permite-se a indicação do primeiro autor, seguido da expressão \textit{et al.} 

Para desativar a substituição dos autores por ‘et al.’, nas referências você deve incluir o pacote com a seguinte opção: \verb+\usepackage[alf,abnt-etal-cite=0]{abntex2cite}+

No ~\autoref{quadro-opcoes-etal} estão descritos os comandos dos pacotes de alteração da composição dos estilos bibliográficos para alterar o estilo ‘et al.’.

A obrigatoriedade do \textit{et al.} ser em itálico, segundo a ABNT NBR 6023:2018, é atendida mediante o parâmetro \textbf{abnt-etal-text=it} na chamada do pacote \textbf{abntex2cite} no arquivo principal do projeto USPSC-modelo.tex ou USPSC-TCC-modelo.tex. 



\begin{quadro}[H]
	\caption{\label{quadro-opcoes-etal}Opções de alteração da composição dos estilos bibliográficos para utilização da sigla ‘et al.’}
		\begin{tabular}{|p{4.0cm}|p{2.0cm}|p{8.5cm}|}
			\hline
			\textbf{Campo} & \textbf{Opções} & \textbf{Descrição} \\ 
			\hline
			\emph{abnt-etal-cite} &  & controla como e quando os co-autores são
			substituídos por \emph{et al.}  Note que a substituição
			por \emph{et al.} continua ocorrendo \emph{sempre} se os co-autores tiverem sido indicados
			como \texttt{others}.\\
			\hline
			\texttt{abnt-etal-cite=0}&\texttt{0}& não abrevia a lista de autores.\\
			\hline
			\texttt{abnt-etal-cite=2}& \texttt{2} & abrevia com mais de 2 autores.\\
			\hline
			\texttt{abnt-etal-cite=3}& \texttt{3} & abrevia com mais de 2 autores.\\
			\hline
			$\vdots$ & $\vdots$ & \\
			\hline
			\texttt{abnt-etal-cite=5}& \texttt{5} & abrevia com mais de 5 autores.\\
			\hline
		\end{tabular}
	\begin{flushleft}
		Fonte: \citeonline{abnetxcite}
	\end{flushleft}	
\end{quadro}

Para ver as demais opções e o modo de uso dos pacotes de especificidades para formatação de referências veja o documento \textbf{O pacote abntex2cite}. \cite{abnetxcite}.

Sendo assim, para que todos os nomes dos autores constem da referência basta acrescentar o pacote: 

\verb+\usepackage[alf,abnt-etal-cite=0]{abntex2cite}+

E a referência será escrita da seguinte forma: \\

\begin{tabular}{|l|c|} \hline
PASQUARELLI, M. L. R.; KRZYZANOWSKI, R. F.; IMPERATRIZ, I. M.\\
M.; NORONHA, D. P.; ANDRADE, E.; ZAPPAROLI, M. C. M.; BONESIO, \\
M. C. M.; LOBO, M. P.; ALMEIDA, M. S.; ARRUDA, R. M. A.; PLAZA, R. \\ \textbf{Avaliação do uso de periódicos}. São Paulo: SIBi-USP, 1987. 14 p.\\\hline
\end{tabular}\\

\textbf{Campos em LATEX:} permanecerão transcritos da mesma forma.\\

\begin{verbatim}
@Book{Pasquarelli1987,
Title                    = {Avaliação do uso de periódicos},
Address                  = {São Paulo},
Author                   = {Pasquarelli, M. L. R. and Krzyzanowski,
R. F.; Imperatriz, I. M. M.; Noronha, D. P.; Andrade, E.; Zapparoli,
M. C. M.; Bonesio, M. C. M.; Lobo, M. P.; Almeida, M. S.; Arruda, 
R. M. A.; Plaza, R. T. T.},
Pages                    = {14},
Publisher                = {SIBi-USP},
Year                     = {1987},
Owner                    = {apcalabrez},
Timestamp                = {2015.09.21}
}
\end{verbatim}


\subsubsection{Responsabilidade pelo conjunto da obra (editor, organizador,coordenador, compilador entre outros)}

\begin{tabular}{|l|c|} \hline
	DEL VECCHIO, M. (comp.). \textbf{A vista de antejo longa mira}: los \\antejos
	del  Luxottica, as lunetas do Museo Luxottica. Tradução: G. Lizabe \\M. Maglione,  Monique Di Prima. Milão: Arti Grafiche Salea Luxottica, 1995.  \\\hline
\end{tabular}\\

\textbf{Campos em LATEX:}

\begin{verbatim}
@Book{delvecchio1995,
Title                    = {A Vista de antejo longa mira},
Address                  = {Milão},
Editor                   = {Del, Vecchio, M},
Editortype               = {comp.},
Furtherresp              = {Tradução: G. Lizabe M. Maglione, Monique 
Di Prima},
Publisher                = {Arti Grafiche Salea Luxottica},
Subtitle                 = {los antejos del Luxottica, as lunetas do 
Museo Luxottica.},
Year                     = {1995},
Owner                    = {apcalabrez},
Timestamp                = {2015.09.21}
}
\end{verbatim}

\begin{tabular}{|l|c|} \hline
	PLOTKIN, S. A.; ORENSTEIN, W. A. (ed.). \textbf{Vaccines}. 3rd ed. Philadelphia: \\ W.B. Saunders, 1999. 1230 p.  \\\hline
\end{tabular}\\

\textbf{Campos em LATEX:}

\begin{verbatim}
@Book{Plotkin1999,
Title                    = {Vaccines.},
Address                  = {Philadelphia},
Editor                   = {Plotkin, S. A. and Orenstein W. A.},
Editortype               = {ed.},
Pages                    = {1230},
Publisher                = {W.B. Saunders},
Year                     = {1999},
Edition                  = {3rd ed},
Owner                    = {apcalabrez},
Timestamp                = {2016.03.31}
}
\end{verbatim}

\begin{tabular}{|l|c|} \hline
	CAVALCANTI, M. G. P. \textit{et al.} (org.). \textbf{Tomografia computadorizada por feixe} \\ \textbf{cônico}: interpretação e diagnóstico para o cirurgião-dentista. São Paulo: Santos, \\ 2010.\\\hline
\end{tabular}\\

\textbf{Campos em LATEX:}

\begin{verbatim}
@Book{Cavalcanti2010,
Title                    = { Tomografia computadorizada 
por feixe cônico},
Address                  = {São Paulo},
Editor                   = {Cavalcanti, M. G. P. and Santos, C. P. and 
Silva, A. M.
and Souza, T. B.},
Editortype               = {org.},
Publisher                = {Santos},
Subtitle                 = {interpretação e diagnóstico para o 
cirurgião-dentista},
Year                     = {2010},
Owner                    = {apcalabrez},
Timestamp                = {2015.09.22}
}
\end{verbatim}

\subsubsection{Outros tipos de responsabilidade (tradutor, prefaciador, ilustrador entre outros)} 


\begin{tabular}{|l|c|} \hline
	BERGSTEIN, R. \textbf{Do tornozelo para baixo:}  a história dos
	sapatos e \\ como eles definem as mulheres. Tradução: Débora Guimarães Isidoro.\\ Rio de
	Janeiro: Casa da Palavra, 2013. \\\hline
\end{tabular}\\

\textbf{Campos em LATEX:}

\begin{verbatim}
@Book{Bergstein2013,
Title                    = {Do tornozelo para baixo},
Subtitle                 = {a história dos sapatos 
e como eles definem as mulheres}
Author                   = {Bergstein, R.}
Address                  = {Philadelphia},
Editor                   = {Fonseca, R. J.},
Furtherresp              = {Tradução: Débora Guimarães
Isidoro},
Publisher                = {Casa da Palavra},
Year                     = {2013},
Owner                    = {apcalabrez},
Timestamp                = {2015.09.17}
}
\end{verbatim}

\begin{tabular}{|l|c|} \hline
	FONSECA, R. J. (ed.). \textbf{Oral and maxillofacial surgery}. Illustrated by\\
	William M. Winn. Philadelphia: Saunders, 2000. \\\hline
\end{tabular}\\

\textbf{Campos em LATEX:}

\begin{verbatim}
@Book{Fonseca2000,
Title                    = {Oral and maxillofacial surgery},
Address                  = {Philadelphia},
Editor                   = {Fonseca, R. J.},
Editortype               = {ed.},
Furtherresp              = {llustrated by William M. Winn},
Publisher                = {Saunders},
Year                     = {2000},
Owner                    = {apcalabrez},
Timestamp                = {2015.09.17}
}
\end{verbatim}


\subsubsection{Autor entidade (entidades coletivas, governamentais, públicas, particulares etc.)} 

As obras de responsabilidade de autor entidade (órgãos governamentais, empresas, associações, comissões, congressos, seminários etc.) têm entrada pelo próprio nome da entidade, por extenso. Seu nome é precedido pelo nome do órgão superior, ou pelo nome da jurisdição geográfica à qual pertence. 

No capítulo \ref{Citações} foram exemplificados algumas citações com  as referências para entidades coletivas. Conforme exposto anteriormente os arquivos.bib de referências para entidade coletiva deve conter o comando Org-short que equivale a forma como a referência será citada no texto. \\

\begin{tabular}{|l|c|} \hline
	INSTITUTO DE PESQUISAS TECNOLÓGICAS DO ESTADO DE SÃO \\
	PAULO.  \textbf{Mapeamento de riscos em encostas e margens de rios}. \\ 
	Brasília: Ministério das Cidades: IPT, 2007.   \\\hline
\end{tabular}\\

\textbf{Campos em LATEX:}

\begin{verbatim}
@Book{Instituto2007,
Title                    = {Mapeamento de riscos em encostas
e margens de rios},
Address                  = {Brasília: Ministério das Cidades},
Organization             = {Instituto de Pesquisa
Tecnológicas do Estado de São Paulo},
Publisher                = {IPT},
Year                     = {2007},
Owner                    = {apcalabrez},
Timestamp                = {2015.09.23}
}
\end{verbatim}

\begin{tabular}{|l|c|} \hline
	SÃO PAULO (Estado). Secretaria da Agricultura. \textbf{O café}: estatística de \\produção e commercio 1935-1936. São Paulo: Typ. Brasil de Rothschild, \\1937. 261 p.  \\\hline
\end{tabular}\\

\textbf{Campos em LATEX:}

\begin{verbatim}
@Book{saopaulo1937,
Title                    = {O café},
Address                  = {São Paulo},
Org-short                = {S\~ao Paulo},
Organization             = {S\~ao Paulo {(Estado). Secretaria da 
Agricultura}},
Pages                    = {261},
Publisher                = {Typ. Brasil de Rothschild},
Subtitle                 = {estatística de produção e commercio 1935-
1936.},
Year                     = {1937},
Owner                    = {apcalabrez},
Timestamp                = {2015.09.23}
}
\end{verbatim}
T

\begin{tabular}{|l|c|} \hline
	U.S. NATIONAL INSTITUTE OF PUBLIC HEALTH. \textbf{Siphonaptera}: \\  a study
	of species infesting wild hares and rabbits of North America,\\ North of Mexico. Washington: GPO, 1988. Não paginado.  \\\hline
\end{tabular}\\

\textbf{Campos em LATEX:}

\begin{verbatim}
@Book{Health1988,
Title                    = {Siphonaptera},
Address                  = {Washington},
Organization             = {U. S. National Institute of Public Health}},
Note                     = {Não paginado},
Publisher                = {GPO},
Subtitle                 = {a study of species infesting 
wild hares and rabbits of North America,},
Year                     = {1988},
Owner                    = {apcalabrez},
Timestamp                = {2015.09.23}
}
\end{verbatim}

Em caso de duplicidade de nomes, deve-se acrescentar entre parêntese a unidade geográfica que identifica a jurisdição a que pertence. \\


\begin{tabular}{|l|c|} \hline
	BIBLIOTECA NACIONAL (Brasil). \textbf{Movimento de vanguarda na Euro-} \\ \textbf{pa e modernismo brasileiro (1909-1924)}. Rio de Janeiro, 1976.	83 p.   \\\hline
\end{tabular}\\

\textbf{Campos em LATEX:}

\begin{verbatim}
@Book{bibliotecanacional1976,
Title                    = {Movimento de vangarda na Europa e modernismo
brasileiro (1909-1924)},
Address                  = {Rio de Janeiro},
Org-short                = {Biblioteca Nacional {(Brasil)}},
Organization             = {Biblioteca Nacional {(Brasil)}},
Pages                    = {83},
Year                     = {1976},
Owner                    = {apcalabrez},
Timestamp                = {2015.09.23}
}
\end{verbatim}

\begin{tabular}{|l|c|} \hline
	BIBLIOTECA NACIONAL (Portugal). \textbf{O 24 de Julho de 1833 e a} \\ \textbf{guerra civil de 1829-1834}. Lisboa, 1983. 95 p.   \\\hline
\end{tabular}\\

\textbf{Campos em LATEX:}

\begin{verbatim}
@Book{bibliotecanacional1983,
Title                    = {O 24 de Julho de 1833  e a guerra civil de 
1829-1834},
Address                  = {Lisboa},
Org-short                = {Biblioteca Nacional {(Portugal)}},
Organization             = {Biblioteca Nacional {(Portugal)}},
Pages                    = {95},
Year                     = {1983},
Owner                    = {apcalabrez},
Timestamp                = {2015.09.23}
}
\end{verbatim}

\subsubsection{Autoria desconhecida} 

Quando a autoria não puder ser identificada no documento inicia-se a referência pelo título.

\begin{tabular}{|l|c|} \hline
	A BETTER investiment climate for everyone. Washington: Oxford University \\ Press, 2004.\\\hline
\end{tabular}\\

\textbf{Campos em LATEX:}

\begin{verbatim}
@Book{abetter2004,
Title                    = {A BETTER investiment climate for everyone},
Address                  = {Washington},
Org-short                = {A Better},
Publisher                = {Oxford University Press},
Year                     = {2004},
Owner                    = {apcalabrez},
Timestamp                = {2015.09.21}
}
\end{verbatim}

\begin{tabular}{|l|c|} \hline
	EDUCAÇÃO para todos: o imperativo da qualidade. Brasília, DF: Unesco,\\ 2005.\\\hline
\end{tabular}\\

\textbf{Campos em LATEX:}

\begin{verbatim}
@Book{educacao2005,
Title                    = {Educa{\c c}\~ao para todos},
Address                  = {Brasília, DF},
Org-short                = {Educa{\c c}\~ao},
Publisher                = {Unesco},
Subtitle                 = {o imperativo da qualidade},
Year                     = {2005},
Owner                    = {apcalabrez},
Timestamp                = {2015.09.21}
}
\end{verbatim}

\subsubsection{Autor(es) com mais de uma obra referenciada } 

Quando se referenciam várias obras do mesmo autor, até a ABNT NBR 6023:2002, era permitido substituir as seguintes por um traço sublinear (equivalente a seis espaços) e ponto.

A ABNT NBR 6023:2018 define que o uso do traço sublinear não é mais indicado para representar a mesma autoria do documento anterior na lista de referência, devendo-se repeti-la quantas vezes forem necessário.

Sendo assim, no Pacote USPSC o parâmetro \textbf{abnt-repeated-author-omit} está configurado com a opção \textbf{no} para exibir os autores em todas as referências, conforme exemplificado abaixo: 

\verb+\usepackage[alf, abnt-emphasize=bf, abnt-thesis-year=both, + \\ \verb+abnt-repeated-author-omit=no, abnt-last-names=abnt, abnt-etal-cite,+ \\
\verb+abnt-etal-list=3, abnt-etal-text=it, abnt-and-type=e, abnt-doi=doi,+ \\ \verb+abnt-url-package=none, abnt-verbatim-entry=no]{abntex2cite}+ \\

As obras de mesmos autores serão listadas conforme abaixo: \\

\begin{tabular}{|l|c|} \hline
	PICCINI, A. \textbf{Casa de Babylonia}: estudo da habitação rural no interior de \\São Paulo. São Paulo: Annablume, 1996. 165 p. \\
	
	PICCINI, A. \textbf{Cortiços na cidade}: conceito e preconceito na reestruturação do \\centro urbano de  São Paulo. São Paulo: Annablume, 1999. 166 p.   \\\hline
\end{tabular}\\

\textbf{Campos em LATEX:}

\begin{verbatim}
@Book{Piccini1996,
Title                    = {Casa de Babylonia},
Address                  = {São Paulo},
Author                   = {Piccini, A.},
Pages                    = {165},
Publisher                = {Annablume},
Subtitle                 = {estudo da habitação rural no interior de 
São Paulo},
Year                     = {1996},
Owner                    = {apcalabrez},
Timestamp                = {2015.09.23}
}
\end{verbatim}

\begin{verbatim}
Book{Piccini1999,
Title                    = {Cortiços na cidade},
Address                  = {São Paulo},
Author                   = {Piccini, A.},
Pages                    = {166},
Publisher                = {Annablume},
Subtitle                 = {conceito e preconceito na reestruturação do 
centro urbano de São Paulo},
Year                     = {1999},
Owner                    = {apcalabrez},
Timestamp                = {2015.09.21}
}
\end{verbatim}

\subsubsection{Mais de um volume}

\begin{tabular}{|l|c|} \hline
	KUHN, H. A.; LASCH, H. G. \textbf{Avaliação clínica e funcional do doente}. \\São Paulo:  E.P.U., 1977. 4 v.    \\\hline
\end{tabular}\\

\textbf{Campos em LATEX:}

\begin{verbatim}
@Book{Kuhn1977,
Title                    = {Avaliação clínica e funcional do doente},
Address                  = {São Paulo},
Author                   = {Kuhn, H. A. and Lasch, H. G.},
Publisher                = {E. P. U.},
Year                     = {1977},
Volume                   = {4},
Owner                    = {apcalabrez},
Timestamp                = {2016.04.11}
}
\end{verbatim}

\begin{tabular}{|l|c|} \hline
	MATSUO, T. \textit{et al.}  \textbf{ Science of the rice plant}. Tokyo: Food and \\ Agriculture Policy Research Center, 1977. 4 v: Genetics. \\\hline
\end{tabular}\\

\textbf{Campos em LATEX:}

\begin{verbatim}
@Book{Matsuo1977,
Title                    = {Science of the rice plant},
Address                  = {Tokyo},
Author                   = {Matsuo, T and Sakomoto, H. A. and Tieko, H. 
G. and Suzuki, A.},
Publisher                = { Food and Agriculture
Policy Research Center,},
Year                     = {1977},
Volume                   = {4: Genetics},
Owner                    = {apcalabrez},
Timestamp                = {2016.04.11}
}
\end{verbatim}

\subsubsection{Série}

\begin{tabular}{|l|c|} \hline
	PHILLIPI JÚNIOR, A. \textit{et al.} \textbf{Interdisciplinaridade em ciências ambien-}\\ 
	\textbf{tais}. São Paulo: Signus, 2000. 318 p. (Série textos básicos para a formação \\ambiental, 5). \\\hline
\end{tabular}\\

\textbf{Campos em LATEX:}

\begin{verbatim}
@Book{PhillipiJunior2000,
Title                 = {Interdisciplinaridade em ciências ambientais},
Address               = {São Paulo},
Author                = {Phillipi, Junior, A. and Medeiros, C. B. and 
Silva, A. M. and Piccini, A.},
Pages                 = {318},
Publisher             = {Signus},
Series                = {Série textos básicos para a formação ambiental, 
5},
Year                  = {2000},
Owner                 = {apcalabrez},
Timestamp             = {2015.09.21}
}
\end{verbatim}

\begin{tabular}{|l|c|} \hline
STEPHENSON, J. B.; KING, M. D. \textbf{ Handbook of neurological}\\ \textbf{investigations in children}. London: Wright, 1989. (Handbooks of \\
investigations in children). \\\hline
\end{tabular}\\

\textbf{Campos em LATEX:}

\begin{verbatim}
@Book{Stephenson1989,
Title                 = {Handbook of neurological
investigations in children},
Address               = {London},
Author                = {Stephenson, J. B and King, M. D.},
Publisher             = {Wright},
Series                = {Handbooks of
investigations in children},
Year                  = {1989},
Owner                 = {apcalabrez},
Timestamp             = {2015.09.21}
}
\end{verbatim}


\subsubsection{Catálogo}

\begin{tabular}{|l|c|} \hline
	BIBLIOTECA NACIONAL (Brasil). \textbf{500 anos de Brasil na Biblioteca }\\ \textbf{Nacional}: catálogo. Rio de Janeiro, 2000. 143 p. Catálogo da exposição em \\comemoração aos 500  anos do Brasil e aos 190 anos da Biblioteca Nacional, \\13 de dezembro de 2000 a 20 de abril de 2001.    \\\hline
\end{tabular}\\

\textbf{Campos em LATEX:}

\begin{verbatim}
@Book{bibliotecanacional2000,
Title                    = {500 anos de Brasil na Biblioteca Nacional},
Address                  = {Rio de Janeiro},
Note                     = {Catálogo da exposição em comemoração aos 500
anos do Brasil e aos 190 anos da Biblioteca Nacional, 13 de dezembro de
2000 a 20 de abril de 2001},
Org-short                = {Biblioteca Nacional {(Brasil)}},
Organization             = {Biblioteca Nacional {(Brasil)}},
Pages                    = {143},
Subtitle                 = {catálogo},
Year                     = {2000},
Owner                    = {apcalabrez},
Timestamp                = {2015.09.18}
}
\end{verbatim}

\begin{tabular}{|l|c|} \hline
	DEMAKOPOULOU, K. \textit{et al.} \textbf{Gods and heroes of the european}\\ \textbf{bronze age}. London:  Thames and Hudson, 2000. 303 p. Catalog.    \\\hline
\end{tabular}\\

\textbf{Campos em LATEX:}

\begin{verbatim}
@Book{Demakopoulou2000,
Title                    = {Gods and heroes of the european bronze age},
Address                  = {London},
Author                   = {Demakopoulou, K. and Arruda, M. L. and Souza,
L. S. and Saadi, S.},
Note                     = {Catalog},
Pages                    = {303},
Publisher                = {Thames and Hudson},
Year                     = {2000},
Owner                    = {apcalabrez},
Timestamp                = {2015.09.18}
}
\end{verbatim}

\begin{tabular}{|l|c|} \hline
	FARIAS, A. A. C \textbf{Amor = love}: catálogo. São Paulo: Thomas \\ Cohn, 2001. Catálogo de exposição artística Beth Moysés.    \\\hline
\end{tabular}\\

\textbf{Campos em LATEX:}

\begin{verbatim}
@Book{Farias2001,
Title                    = {Amor = love},
Subtitle                 = {catálogo},
Address                  = {São Paulo},
Author                   = {Farias, A. A. C.},
Note                     = {Catálogo de exposição
artística Beth Moysés},
Publisher                = {Thomas Cohn},
Year                     = {2001},
Owner                    = {apcalabrez},
Timestamp                = {2015.09.18}
}
\end{verbatim}

\begin{tabular}{|l|c|} \hline
	UNIVERSIDADE DE SÃO PAULO. Museu de Arqueologia e Etnologia.  \\ \textbf{Brasil 50 mil anos}: uma viagem ao passado pré-colonial, guia temático \\ para professores: catálogo. [São Paulo]: Universidade de São Paulo, Museu \\ de
	Arqueologia e Etnologia, [2001]. 28 p. il. 19 pranchas. Catálogo de \\ exposição.   \\\hline
\end{tabular}\\

\textbf{Campos em LATEX:}

\begin{verbatim}
@Book{USPmuseu2001,
Title                    = {Brasil 50 mil anos},
Address                  = {[São Paulo]},
Org-short                = {Universidade de S\~ao Paulo},
Organization             = {Universidade de S\~ao Paulo. {Museu de 
Arqueologia e Etnologia}},
Pages                    = {28},
Publisher                = {Universidade de São Paulo, Museu de 
Arqueologia e Etnologia},
Subtitle                 = { uma viagem ao passado
pré-colonial, guia temático para professores: catálogo},
Year                     = {[2001]},
Note                     = { il. 19 pranchas. Catálogo de exposição}
Owner                    = {apcalabrez},
Timestamp                = {2015.09.23}
}
\end{verbatim}

\subsubsection{Relatório e parecer técnico}

\begin{tabular}{|l|c|} \hline
	CASTRO, M. C. \textit{et al.} \textbf{Cooperação técnica na implementação do
		Pro-}\\\textbf{grama Integrado de Desenvolvimento - Polonordeste}. Brasília, DF: \\ PNUD: FAO, 1990. 47 p. Relatório da Missão de Avaliação do Projeto \\ BRA/87/037.     \\\hline
\end{tabular}\\

\textbf{Campos em LATEX:}

\begin{verbatim}
@Book{Castro,
Title                    = {Cooperação técnica na implementação do 
Programa Integrado 
de Desenvolvimento - Polonordeste},
Address                  = {Brasília},
Author                   = {Castro, M. C. and Souza, L. S. and Cardoso, 
R. F and Arruda, M. L.},
Note                     = {Relatório da Missão de Avaliação do 
Projeto BRA/87/037},
Pages                    = {47},
Publisher                = {PNUD: FAO},
Year                     = {1990},

Owner                    = {apcalabrez},
Timestamp                = {2015.09.17}
}
\end{verbatim}

\begin{tabular}{|l|c|} \hline
	COMPANHIA ESTADUAL DE TECNOLOGIA DE SANEAMENTO AMBI-\\ENTAL. \textbf{Bacia hidrográfica do Ribeirão Pinheiros}: relatório técnico. São \\Paulo: CETESB, 1994. 39 p.   \\\hline
\end{tabular}\\

\textbf{Campos em LATEX:}

\begin{verbatim}
@Book{Castro,
@Book{CETESB1994,
Title                    = {Bacia hidrográfica do Ribeirão Pinheiros},
Address                  = {São Paulo},
Organization             = {Companhia Estadual de Tecnologia de 
Saneamento Ambiental},
Pages                    = {39},
Publisher                = {CETESB},
Subtitle                 = {relatório técnico},
Year                     = {1994},
Owner                    = {apcalabrez},
Timestamp                = {2015.09.17}
}
\end{verbatim}

\begin{tabular}{|l|c|} \hline
	GUBITOSO, M. D. \textbf{Máquina worm}: simulador de máquinas paralelas. \\São Paulo: IME- USP, 1989. 29 p. Relatório técnico, Rt-Mac-8908.   \\\hline
\end{tabular}\\

\textbf{Campos em LATEX:}

\begin{verbatim}
@Book{Gubitoso1989,
Title                    = {Máquina worm},
Address                  = {São Paulo},
Author                   = {Gubitoso, M. D.},
Note                     = {Relatório técnico, Rt-Mac-8908},
Pages                    = {29},
Publisher                = {IME-USP},
Subtitle                 = {simulador de máquinas paralelas},
Year                     = {1989},
Owner                    = {apcalabrez},
Timestamp                = {2015.09.17}
\end{verbatim}


\begin{tabular}{|l|c|} \hline
	POGGIANI, F. \textit{et al.} \textbf{ Parecer sobre o Projeto de Revegetação nas} \\ \textbf{Áreas do
		Gasoduto de Merluza}. Piracicaba: IPEF: ESALQ, Depto. \\ Ciências Florestais,
	1992. 5 p. Parecer técnico apresentado à Petrobrás,\\Cubatão.   \\\hline
\end{tabular}\\




\textbf{Campos em LATEX:}

\begin{verbatim}
@Book{Poggiani1992,
Title                    = {Parecer sobre o Projeto
de Revegetação nas Áreas do Gasoduto de Merluza},
Address                  = {Piracicaba},
Author                   = {Poggiani, A. B. and Gusmão, M. D. and 
Silva, B. C. and Machado, D. B.},
Note                     = {Parecer técnico
apresentado à Petrobrás, Cubatão},
Pages                    = {5},
Publisher                = {IPEF: ESALQ, Depto. Ciências Florestais},
Year                     = {1992},
Owner                    = {apcalabrez},
Timestamp                = {2015.09.17}
}
\end{verbatim}

\begin{tabular}{|l|c|} \hline
WORLD HEALTH ORGANIZATION.  Study Group on Integration on \\ Health Care Delivery.  \textbf{Report}. Geneva, 1996. (WHO technical report \\ series, 861)  \\\hline
\end{tabular}\\

\textbf{Campos em LATEX:}

\begin{verbatim}
@Book{World1996,
Title                    = {Report},
Address                  = {Geneva},
Organization             = {World Health Organization. {Study Group on 
Integration on Health Care Delivery}},
Org-short                = {World Health Organization}
Series                   = {WHO technical report series, 861}
Year                     = {1996},
Owner                    = {apcalabrez},
Timestamp                = {2015.09.17}
}
\end{verbatim}


\subsubsection{Dicionário}

\begin{tabular}{|l|c|} \hline
	DORLAND'S illustrated medical dictionary. 29th. ed. Philadelphia: W.\\B. Saunders, 2000.   \\\hline
\end{tabular}\\


\textbf{Campos em LATEX:}

\begin{verbatim}
@Book{Dorlands2000,
Title                    = {Dorland's illustrated medical dictionary},
Address                  = {Philadelphia},
Org-short                = {DORLAND'S},
Publisher                = {W.B. Saunders},
Year                     = {2000},
Edition                  = {29th.},
Owner                    = {apcalabrez},
Timestamp                = {2015.09.24}
}
\end{verbatim}


\begin{tabular}{|l|c|} \hline
	SCHEARZ, R. G. (org.). \textbf{Dicionário de direito do trabalho, de} \\ \textbf{direito processual do trabalho e de direito previdenciário} \\ \textbf{aplicado ao direito do trabalho}. São Paulo: LTr, 2012.  \\\hline
\end{tabular}\\


\textbf{Campos em LATEX:}

\begin{verbatim}
@Book{Schearz2012,
Title                    = {Dicionário de direito do trabalho, 
de direito processual do trabalho e de direito previdenciário 
aplicado ao direito do trabalho},
Address                  = {São Paulo},
Editor                   = {Schearz, R. G.},
Editortype               = {org.},
Publisher                = {LTr},
Year                     = {2012},
Owner                    = {apcalabrez},
Timestamp                = {2015.09.24}
}
\end{verbatim}

\subsubsection{Trabalhos acadêmicos}

\textbf{Elementos essenciais}\\

\begin{tabular}{|l|c|} \hline
	SOBRENOME, Prenome do autor. \textbf{Título}: subtítulo (se houver). Ano. \\ Tipo de trabalho. Grau (Mestrado / Doutorado / Especialização entre \\ outros e curso entre parênteses) - Vinculação Acadêmica, Unidade de \\ defesa, local, data de defesa.\\\hline
\end{tabular}\\

\textbf{Exemplos}\\

\begin{tabular}{|l|c|} \hline
	ALVES, J. M. \textbf{Competividade e tendência da produção de manga} \\ \textbf{para exportação do nordeste do Brasil}. 2002. Tese (Doutorado em \\ Economia Aplicada) - Escola Superior de Agricultura "Luiz de Queiroz", \\ Universidade de São Paulo, Piracicaba, 2002.    \\\hline
\end{tabular} \\

\textbf{Campos em LATEX:}

\begin{verbatim}
@Phdthesis{Alves2002,
Title                    = {Competividade e tendência da produção de 
manga para exportação do nordeste do Brasil},
Address                  = {Piracicaba},
Author                   = {Alves, J. M.},
School                   = {Escola Superior de Agricultura "Luiz de 
Queiroz", Universidade de São Paulo},
Type                     = {Doutorado em Economia Aplicada},
Year                     = {2002},
Owner                    = {apcalabrez},
Timestamp                = {2015.09.23}
}
\end{verbatim} 

\begin{tabular}{|l|c|} \hline
	DIAS, F. L. F. \textbf{Efeito da aplicação de calcário, lodo de esgoto e vinhaça} \\ \textbf{em solo cultivado em sorgo granífero (\textit{Sorghum bicolor} L. Moench)}. 1994. \\ Trabalho de Conclusão do Curso (Engenharia Agronômica) - Faculdade de Ciências\\  Agrárias e Veterinárias, Universidade Estadual Paulista "Júlio de Mesquita Filho", \\ Jaboticabal, 1994.     \\\hline
\end{tabular} \\

\textbf{Campos em LATEX:} 

\begin{verbatim}
@Thesis{Dias1994,
Title                    = {Efeito da aplicação de calcário, lodo de 
esgoto e vinhaça em solo cultivado em sorgo granífero (Sorghum bicolor 
L. Moench)},
Address                  = {Jaboticabal},
Author                   = {Dias, F. L. F.},
School                   = {Faculdade de Ciências Agrárias e Veterinárias, 
Universidade Estadual Paulista "Júlio de Mesquita Filho"},
Type                     = {Trabalho de Conclusão do Curso (Engenharia 
Agronômica)},
Year                     = {1994},
Owner                    = {apcalabrez},
Timestamp                = {2015.09.23}
}
\end{verbatim}


\begin{tabular}{|l|c|} \hline
	DOOD, M. J. \textbf{Silicon photonic crystals and spontaneous emission.} \\ 2002.
	 Ph. D. Thesis (Physics) - FOM Institute for Atomic and Molecular
	 \\ Physics, University of Utrecht, Utrecht, 2002.    \\\hline
\end{tabular} \\

\textbf{Campos em LATEX:} \\

\begin{verbatim}
@PhdThesis{Dood2002,
Title                    = {Silicon photonic
crystals and spontaneous emission},
Address                  = {Utrecht},
Author                   = {Dood, M. J.},
School                   = {FOM Institute for
Atomic and Molecular Physics, University of
Utrecht},
Type                     = {Physics},
Year                     = {2002},
Owner                    = {apcalabrez},
Timestamp                = {2015.09.23}
}
\end{verbatim}

\subsection{Parte de monografia}	

\begin{tabular}{|l|c|} \hline
	SOBRENOME, Prenome(s) do(s) autor(es) do capítulo. Título do capítulo. \textit{In}: \\ SOBRENOME, Prenome(s) do(s) autor(es) do documento. \textbf{Título da obra}: \\ subtítulo (se houver). Edição (se houver). Local de publicação (cidade): Editora, data da\\ publicação. Páginas ou indicação do capítulo. Série. Notas. ISBN.     \\\hline
\end{tabular} \\

\subsubsection{Autor distinto da obra no todo} 

\begin{tabular}{|l|c|} \hline
	CATANI, A. M. O que é capitalismo. \textit{In}: SPINDEL, A. \textbf{Que é socialismo e o}\\ \textbf{que é comunismo}. São Paulo: Círculo do Livro, 1989. p. 7-87. (Primeiros \\passos, 1).    \\\hline
\end{tabular} \\

\textbf{Campos em LATEX:} 

\begin{verbatim}
@Incollection{Catani1989,
Title                    = {O que é capitalismo},
Author                   = {Catani, A. M.},
Booktitle                = {O que é socialismo e o que é comunismo},
Organization             = {Spindel, A.},
Publisher                = {Círculo do Livro},
Year                     = {1989},
Address                  = {São Paulo},
Note                     = {(Primeiros Passos, 1)},
Pages                    = {7-87},
Owner                    = {apcalabrez},
Timestamp                = {2015.09.25}
}
\end{verbatim}


\begin{tabular}{|l|c|} \hline
	MOSS, D. W.; HENDERSON, A. R. Clinical enzymology. \textit{In}: BURTIS, C. \\A.; ASHWOOD, E. R. (ed.). \textbf{Tietz textbook of clinical chemistry}. 3rd\\ ed. Philadelphia: W. B. Saunders, 1999. cap. 22, p. 617-721.  \\\hline
\end{tabular} \\

\textbf{Campos em LATEX:} 

\begin{verbatim}
@Incollection{Moss1999,
Title                    = {Clinical enzymology},
Author                   = {Moss, D. W. and Henderson, A. R.},
Booktitle                = {Tietz textbook of clinical chemistry},
Publisher                = {W. B. Saunders},
Year                     = {1999},
Address                  = {Philadelphia},
Chapter                  = {22},
Edition                  = {3rd},
Editor                   = {Burtis, C. A. and Ashwood, E. R.},
editortype               = {ed.},
Pages                    = {617-721},
Owner                    = {apcalabrez},
Timestamp                = {2015.09.25}
}
\end{verbatim}

\subsubsection{Mesmo autor da obra no todo}

Repete-se a autoria. \\

\begin{tabular}{|l|c|} \hline
	MONTGOMERY, R.; CONWAY, T. W.; SPECTOR, A. A. Estructuras de \\las proteínas.  \textit{In}: MONTGOMERY, R.; CONWAY, T. W.; SPECTOR, A. A. \\
	\textbf{Bioquímica}: casos y texto. 5th ed. St. Louis:
	Mosby, 1992. cap. 2, p. 41-90.  \\\hline
\end{tabular} \\ 

\textbf{Campos em LATEX:} 

\begin{verbatim}
@InBook{Montgomery1992,
author       = {Montgomery, R. and Conway, T. W. and Spector, A. A.},
title        = {Estructuras de las proteínas},
booktitle    = {Bioquímica},
year         = {1992},
booksubtitle = {casos y texto. 5th ed.},
organization = {Montgomery, R.; Conway, T. W.; Spector, A. A.},
publisher    = {Mosby},
chapter      = {2},
pages        = {41-90},
address      = {St. Louis},
owner        = {apcalabrez},
timestamp    = {2015.09.25},
}
\end{verbatim}

\begin{tabular}{|l|c|} \hline
	RAMOS, M. E. M. Serviços administrativos na Bicen da UEPG. \textit{In}:
	RAMOS, \\
	M. E. M. \textbf{Tecnologia e novas formas de gestão em bibliotecas} \\
		\textbf{universitárias}. Ponta Grossa: UEPG, 1999. p. 157-182.   \\\hline
\end{tabular} \\ 

\textbf{Campos em LATEX:} 

\begin{verbatim}
@Inbook{Ramos1999,
Title                    = {Serviços administrativos na {Bicen da UEPG}},
Author                   = {Ramos, M. E. M.},
Pages                    = {157-182},
Publisher                = {UEPG},
Year                     = {1999},
Address                  = {Ponta Grossa},
Booktitle                = {Tecnologia e novas formas de gestão em 
bibliotecas universitárias},
Organization             = {Ramos, M. E. M.}, 
Owner                    = {apcalabrez},
Timestamp                = {2015.09.25}
}
\end{verbatim}

\subsection{Monografia em suporte eletrônico}	 

\begin{tabular}{|l|c|} \hline
	SOBRENOME, Prenome(s) do(s) autor(es). \textbf{Título da obra}: 
	subtítulo (se \\ houver). Edição (se houver). Local de publicação (cidade): Editora, data da publicação. \\ Disponível em: endereço eletrônico. Acesso em: dia mês abreviado e ano.     \\\hline
\end{tabular} \\

Exemplos: \\ 

\begin{tabular}{|l|c|} \hline
	DUDEK, S. G. (ed.). \textbf{Nutrition essentials for nursing practice}. 5th ed.\\  Philadelphia: Lippincott \& Williams  Wilkins, 2006. Disponível em: http:// \\gateway.ut.ovid.com/gw1/ovidweb.cgi. Acesso em: 24 out. 2006.  \\\hline
\end{tabular} \\ 

\textbf{Campos em LATEX:} 

\begin{verbatim}
@Book{Dudek2006,
Title                    = {Nutrition essentials for nursing practice},
Address                  = {Philadelphia},
Editor                   = {Dudek, S. G.},
Editortype               = {ed.},
Publisher                = {Lippincott Williams \& Wilkins},
Year                     = {2006},
Edition                  = {5th},
Url                      = {http://gateway.ut.ovid.com/gw1/ovidweb.cgi},
Urlaccessdate            = {24 out. 2011},
Owner                    = {apcalabrez},
Timestamp                = {2015.09.28}
}
\end{verbatim}


\begin{tabular}{|l|c|} \hline
	FOREST PHARMACEUTICALS. \textbf{Frequently asked questions}. New York, \\ 2005. Disponível em:  http://www.celexa.com/Celexa/faq.aspx. Acesso em: \\ 17 out. 2005.   \\\hline
\end{tabular} \\ 

\textbf{Campos em LATEX:} 

\begin{verbatim}
@Book{forest2005,
Title                    = {Frequently asked questions.},
Address                  = {New York},
Year                     = {2005},
Edition                  = {7th},
Url                      = {http://www.celexa.com/Celexa/faq.aspx},
Urlaccessdate            = {17 out. 2005},
Owner                    = {apcalabrez},
Timestamp                = {2015.09.28}
}
\end{verbatim}


\begin{tabular}{|l|c|} \hline
	THOMÉ, V. M. R. \textit{et al.} \textbf{Zoneamento agroecológico e socioeconômico} \\ \textbf{do Estado de Santa Catarina}:  versão preliminar. Florianópolis: EPAGRI, \\
	1999. 1 CD-ROM.  \\\hline
\end{tabular} \\ 

\textbf{Campos em LATEX:} 

\begin{verbatim}
@Book{Thome1999,
Title                    = {Zoneamento agroecológico e socioeconômico do 
Estado de Santa Catarina},
Address                  = {Florianópolis},
Author                   = {Thom\'e, V. M. R. and Souza, L. S. and 
Oliveira, A. P. and Silva, A. M.},
Note                     = {1 CD-ROM},
Publisher                = {EPAGRI},
Subtitle                 = {versão preliminar},
Year                     = {1999},
Owner                    = {apcalabrez},
Timestamp                = {2015.09.28}
}
\end{verbatim}


 \subsubsection{Parte de monografia em suporte eletrônico}
	 
	 \begin{tabular}{|l|c|} \hline
	 	SOBRENOME, Prenome(s) do(s) autor(es) do capítulo. Título do capítulo. \\ \textit{In}:
	 	SOBRENOME, Prenome(s)  do(s) autor(es) do documento.  \textbf{Título da} \\ \textbf{obra}: subtítulo (se houver).Edição (se houver). Local de publicação (cidade): Editora, \\ data da publicação. Páginas ou indicação do capítulo. Disponível em: \\ endereço eletrônico. Acesso em: dia mês abreviado e ano.  \\\hline
	 \end{tabular} \\ 
	 
	 Exemplos: \\ 
	 
	 \begin{tabular}{|l|c|} \hline
		ZELEN, M. Theory and practice of clinical trials. \textit{In}: BAST JUNIOR, R. C. \\\textit{et al.} (ed.). \textbf{Cancer medicine e.5.} Hamilton: BC Decker; New York: \\American Cancer Society, 2000. CD-ROM  \\\hline
	\end{tabular} \\ 
	
	\textbf{Campos em LATEX:} 
	
	
	\begin{verbatim}
	@InCollection{Zelen2000,
	author     = {Zelen, M.},
	title      = {Theory and practice of clinical trials},
	booktitle  = {Cancer medicine e.5},
	year       = {2000},
	editor     = {Bast, Junior, R. C. and Arruda, A. C. and Marques, A. P. 
	and Oliveira, A. C.},
	editortype = {ed.},
	note       = {CD-ROM},
	publisher  = {BC Decker},
	address    = {Hamilton},
	owner      = {apcalabrez},
	timestamp  = {2015.09.28},
	}
	\end{verbatim}
		
	 Como a versão atual do pacote \textbf{abntex2cite} tem restrições para elaboração de alguns tipos de referências, houve necessidade de adaptações para atender as especificidades da ABNT NBR 6023:2018. As  referências abaixo são exemplos desta questão, que mesmo sendo uma parte de monografia foi necessário utilizar o tipo  \textbf{Book} ao invés do \textbf{Inbook} ou \textbf{Incollection} do \textbf{BibTeX}. \\
	 
	 \begin{tabular}{|l|c|} \hline
	 	FOOD AND DRUG ADMINISTRATION. Code of federal regulations, \\ 21CFR202. \textit{In}: FOOD AND DRUG ADMINISTRATION. \textbf{Food and drugs}. \\Rockville, 2005. cap. 1. Disponível em: http://www.accessdata.fda.gov/\\ scripts,cdrh/cfdocs/cfcfr/CFRPart=202s\&howFR=1. Acesso em: 14 out. 2005. \\\hline
	 \end{tabular} \\ 
	 
	 \textbf{Campos em LATEX:} 
	 
	 \begin{verbatim}
	 @Book{Food2005,
	 Title                    = {Food and drugs},
	 Organization             = {Food and Drug Administration. {Code of 
	 federal regulations, 21CFR202. \textit{In}: FOOD AND DRUG 
	 ADMINISTRATION}},
	 org-short                = {Food and Drug Administration},
	 Url                      = {http://www.accessdata.fda.gov/scripts,cdrh/
	 cfdocs/cfcfr/CFRPart=202&showFR=1},
	 Urlaccessdate            = {14 out. 2005},
	 Year                     = {2005},
	 Address                  = {Rockville},
	 note                     = {{cap. 1}},
	 Owner                    = {marilza},
	 Timestamp                = {2019.10.08}
	 }
	 \end{verbatim}
	 
		 
	 \begin{tabular}{|l|c|} \hline
	 	SÃO PAULO (Estado). Secretaria do Meio Ambiente. Tratados e organizações\\ ambientais em matéria de meio ambiente. \textit{In}: SÃO PAULO (Estado). Secretaria \\ do Meio Ambiente. \textbf{Entendendo o meio ambiente}. São Paulo, 1999. v. 1. \\ Disponível em: http://www/bdf.org.br/sma/entendendo/atual.htm. Acesso \\ em: 9 mar. 1999.  \\\hline
	 \end{tabular} \\ 
	 
	  \textbf{Campos em LATEX:} 

	  \begin{verbatim}
	   @Book{tratados1999,
	   title         = {Entendendo o meio ambiente},
	   year          = {1999},
	   volume        = {1},
	   url           = {http://www/bdf.org.br/sma/entendendo/atual.htm},
	   urlaccessdate = {9 mar. 1999},
	   address       = {São Paulo},
	   org-short     = {S\~ao Paulo (Estado). Secretaria do Meio Ambiente},
	   organization  = {S\~ao Paulo {(Estado). Secretaria do Meio Ambiente. 
	   Tratados e organizações ambientais em matéria de meio ambiente. 
	   \textit{In}: SÃO PAULO (Estado). Secretaria do Meio Ambiente}},
	   owner         = {marilza},
	   timestamp     = {2019.10.08},
	   }
	 \end{verbatim}
	 
	

\subsection{Evento}
%\textbf{4.1.4 Evento} \\

Conjunto dos documentos reunidos em um produto final com denominação
de: atas, anais, proceedings, resumos entre outros. \\
%\newpage
%\textbf{Elementos essenciais:} 

\begin{tabular}{|l|c|} \hline
	
	NOME DO EVENTO, numeração do evento em arábico (se
	houver), ano, lo-\\cal (cidade) de realização do evento. \textbf{Título do documento [...]} (Anais, Atas, \\Resumos etc.). Local: Editora, data de publicação. \\\hline
\end{tabular} \\ 

\subsubsection{Evento completo} 

\begin{tabular}{|l|c|} \hline
	ANNUAL MEETING OF THE AMERICAN SOCIETY OF INTERNATIONAL \\ LAW, 65., 1967,  Washington. \textbf{Proceedings [...]}. Washington: ASIL, 1967. \\\hline
\end{tabular} \\

\textbf{Campos em LATEX:} 

\begin{verbatim}
@Proceedings{law1967,
Title                    = {Proceedings [...]},
Address                  = {Washington},
Conference-location      = {Washington},
Conference-number        = {65},
Conference-year          = {1997},
Organization             = {Annual Meeting of the American Society of 
International Law},
Publisher                = {ASIL},
Year                     = {1967},
Owner                    = {apcalabrez},
Timestamp                = {2015.09.28}
}
\end{verbatim}

\begin{tabular}{|l|c|} \hline
	CONGRESSO DE INICIAÇÃO CIENTÍFICA DA UFPe, 4., 1996, Recife. \\ \textbf{Anais eletrônicos [...]} Recife: UFPe, 1996. Disponível em: http://www.\\ propesq.ufpe.br/anais/anais/educ/ce04.htm. Acesso em: 21 jan. 1997 \\\hline
\end{tabular} \\

\textbf{Campos em LATEX:} 

\begin{verbatim}
@Proceedings{law1967,
Title                    = {Anais eletrônicos [...]},
Address                  = {Recife},
Conference-location      = {recife},
Conference-number        = {4},
Conference-year          = {1996},
Organization             = {Congresso De Inicia{\c c}\~ao Científica Da 
{UFPe}},
Publisher                = {UFPe},
Year                     = {1996},
Url                      = {http://www.\\ propesq.ufpe.br/anais/anais/
educ/ce04.htm},
Urlaccessdate            = {21 jan. 1997},
Owner                    = {apcalabrez},
Timestamp                = {2015.09.28}
}
\end{verbatim}

\begin{tabular}{|l|c|} \hline
	REUNIÃO ANUAL DA SOCIEDADE BRASILEIRA DE QUÍMICA, 20., \\1997, Poços de Caldas. \textbf{Química}: academia, indústria, sociedade: livro de \\resumos. São Paulo: Sociedade Brasileira de Química, 1997.  \\\hline
\end{tabular} \\

\textbf{Campos em LATEX:} 

\begin{verbatim}
@Proceedings{quimica1997,
Title                    = {Química},
Address                  = {São Paulo},
Conference-location      = {Poços de Caldas},
Conference-number        = {20},
Conference-year          = {1997},
Organization             = {Reuni\~ao Anual da Sociedade Brasileira de 
Qu{\'\í}mica},
Publisher                = {Sociedade Brasileira de Química},
Subtitle                 = {academia, indústria, sociedade: livro de 
resumos},
Year                     = {1997},
Owner                    = {apcalabrez},
Timestamp                = {2015.09.28}
}
\end{verbatim}

\subsubsection{Trabalho apresentado em evento}

\begin{tabular}{|l|c|} \hline
	BRAYNER, A. R. A.; MEDEIROS, C. B. Incorporação do tempo em SGBD \\orientado a objetos. \textit{In}: SIMPÓSIO BRASILEIRO DE BANCO DE DADOS, \\9., 1994, São Paulo. \textbf{Anais [...]} São Paulo: USP, 1994. p. 16-29.  \\\hline
\end{tabular} \\

\textbf{Campos em LATEX:} 

\begin{verbatim}
@InProceedings{BRAYNER1994,
Title                    = {Incorporação do tempo em {SGBD} orientado a 
objetos},
Author                   = {Brayner, A. R. A. and Medeiros, C. B.},
Booktitle                = {Anais [...]},
Conference-location      = {São Paulo},
Conference-number        = {9},
Conference-year          = {1994},
Year                     = {1994},
Address                  = {São Paulo},
Organization             = {Simp\'osio Brasileiro de Banco de Dados},
Pages                    = {16-29},
Publisher                = {USP},
Owner                    = {Ana Paula},
Timestamp                = {2015.09.10}
}
\end{verbatim}

\begin{tabular}{|l|c|} \hline
	VALARINI, M. J.; VIEIRA, M. L. C. Avaliação da fixação de nitrogênio \\ em \textit{Stylosantes guyanensis}0 derivado de cultura de tecidos. \textit{In}: SIMPÓSIO \\ BRASILEIRO SOBRE MICROBIOLOGIA DO SOLO, 3.; REUNIÃO DE \\ LABORATÓRIOS PARA RECOMENDAÇÃO DE ESTIRPES DE \\ \textit{RHIZOBIUM} E \textit{BRADYRHIZOBIUM}, 6., 1994, Londrina. \textbf{Resumos [...]} \\ Londrina: IAPAR, 1994. p. 34. \\\hline
\end{tabular} \\

\textbf{Campos em LATEX:} 


\begin{verbatim}
@InProceedings{Nitrogenio1994,
author              = {Valarini, M. J. and Vieira, M. L. C.},
title               = {Avaliação da fixação de nitrogênio em 
\textit{Stylosantes guyanensis} derivado de cultura de tecidos.},
booktitle           = {Resumos...},
year                = {1994},
organization        = {Simp\'osio Brasileiro sobre Microbiologia do Solo, 
3.; Reuni\~ao de Laboratórios para Recomendação de Estirpes de Rhizobium 
e Bradyrhizobium de Banco de Dados},
publisher           = {IAPAR},
pages               = {34},
address             = {Londrina},
conference-location = {Londrina},
conference-number   = {6},
conference-year     = {1994},
owner               = {Ana Paula},
timestamp           = {2015.09.10}
}
\end{verbatim}

\begin{tabular}{|l|c|} \hline
	KRONSTRAND, R. \textit{et al.} Relationship between melanin and codeine
	concen-\\trations in hair after oral administration. \textit{In}: ANNUAL MEETINGS OF THE \\AMERICAN  ACADEMY OF FORENSIC SCIENCE, 1999, Orlando. \\\textbf{Proceedings [...]} Orlando:  Academic Press, 1999. p. 12.   \\\hline
\end{tabular} \\

\textbf{Campos em LATEX:} 

\begin{verbatim}
@Inproceedings{kronstrand1994,
Title                    = {Relationship between melanin and codeine
concentrations in hair after oral administration},
Author                   = {Kronstrand, R. and Arruda, M. L. and Kuhn, 
H. A. and Braams, J.},
Booktitle                = {Proceedings [...]},
Conference-location      = {Orlando},
Conference-year          = {1999},
Year                     = {1994},
Address                  = {Orlando},
Organization             = {Annual Meetings of the American Academy of 
Forensic Science},
Pages                    = {12},
Publisher                = {Academic Press},
Owner                    = {Ana Paula},
Timestamp                = {2015.09.10}
}
\end{verbatim}

\subsubsection{Trabalho de evento publicado em periódico} 

\begin{tabular}{|l|c|} \hline
	MINGRONI-NETTO, R. C. Origin of fmr-1 mutation: study of closely linked \\microsatellite loci in fragile x syndrome. \textbf{Brazilian Journal of Genetics}, \\Ribeirão Preto, v. 19, n.3, p. 144, 1996. Supplement. Program and abstract \\42nd. National Congress of Genetics, 1996. 
	\\\hline
\end{tabular} \\

\textbf{Campos em LATEX:} 

\begin{verbatim}
@Article{Mingroni-Netto1996,
Title                    = {Origin of fmr-1 mutation: study of closely 
linked microsatellite loci in fragile x syndrome},
Author                   = {Mingroni-Netto, R. C},
Journal                  = {Brazilian Journal of Genetics},
Year                     = {1996},
Address                  = {Ribeirão Preto},
Note                     = {Supplement. Program and abstract 42nd. 
National Congress of Genetics, 1996},
Number                   = {3},
Pages                    = {144},
Volume                   = {19},
Owner                    = {AnaPaula},
Timestamp                = {2015.10.02}
\end{verbatim} \\

\subsubsection{Evento em suporte eletrônico} 

\begin{tabular}{|l|c|} \hline
	NOME DO EVENTO, numeração do evento em arábico (se
	houver), ano, \\local de realização do evento. \textbf{Título do
		documento [...]} (Anais, Atas, \\Resumos etc.)  Local: Editora, data de publicação. Paginação. Disponível \\ em: endereço eletrônico. Acesso em: dia mês abreviado. ano. Mídia.
	\\\hline
\end{tabular} \\

\textbf{Exemplo:} \\

\begin{tabular}{|l|c|} \hline
	SIMPÓSIO INTERNACIONAL DE INICIAÇÃO CIENTÍFICA DA
	UNI-\\VERSIDADE DE SÃO PAULO, 8., 2000, São Paulo. \textbf{Resumos [...]}
	São \\ Paulo: USP, 2000. 1 CD-ROM.  \\\hline
\end{tabular} \\

\textbf{Campos em LATEX:} 

\begin{verbatim}
@Proceedings{Simposio2000,
Title                    = {Resumos [...]},
Address                  = {São Paulo},
Conference-location      = {São Paulo},
Conference-number        = {8},
Conference-year          = {2000},
Organization             = {Simp\'osio Internacional de Iniciação 
Cient{\'\i}fica da Universidade de São Paulo},
Publisher                = {USP},
Year                     = {2000},
Note                     = {1 CD-ROM},
Owner                    = {apcalabrez},
Timestamp                = {2015.09.28}
}
\end{verbatim}

\subsubsection{Trabalho de evento em suporte eletrônico}

\begin{tabular}{|l|c|} \hline
	SABROZA, P. C. Globalização e saúde: impacto nos perfis
	epidemiológicos \\das populações. \textit{In}: CONGRESSO BRASILEIRO DE
	EPIDEMIOLOGIA,\\4., 1998, Rio de Janeiro. \textbf{Anais eletrônicos [...]} Rio de
	Janeiro: ABRASCO,\\1998. Mesa-redonda. Disponível em:
	http://www.abrasco.com.br/epino98/.\\ Acesso em: 17 jan. 1999.\\\hline 
\end{tabular} \\

\textbf{Campos em LATEX:} 

\begin{verbatim}
@Inproceedings{Sabroza1998,
Title                    = {Globalização e saúde},
Author                   = {Sabroza, P. C.},
Booktitle                = {Anais eletrônicos [...]},
Conference-location      = {Rio de Janeiro},
Conference-number        = {4},
Conference-year          = {1998},
Subtitle                 = {impacto nos perfis epidemiológicos das 
populações},
Year                     = {1998},
Address                  = {Rio Janeiro},
Note                     = {Mesa-redonda},
Organization             = {Congresso Brasileiro de Epidemiologia},
Publisher                = {ABRASCO},
Url                      = {http://www.abrasco.com.br/epino98/},
Urlaccessdate            = {17 jan. 1999},
Owner                    = {apcalabrez},
Timestamp                = {2015.10.01}
}
\end{verbatim}

\section{Publicações Periódicas}

Revistas, jornais, publicações anuais e séries monográficas, quando
tratadas como publicação periódica. \\

Quando for explicitada a periodicidade da publicação, para que a referência fique em conformidade com a ABNT NBR 6023:2018, o ISSN deverá ser indicado no campo \textbf{note} e o campo \textbf{ISSN} deverá ser suprimido, conforme exemplos abaixo. Na ausência da indicação de periodicidade, mantem-se o campo \textbf{ISSN}. \\

\subsection{Coleção como um todo}

\textbf{Exemplo:} \\

\begin{tabular}{|l|c|} \hline
	NATURE. London, GB: Macmillan Magazines, 1869- . ISSN
	0028-0836.\\Semanal.\\\hline
\end{tabular} \\

\textbf{Campos em LATEX:} 

\begin{verbatim}
@Journalpart{Nature1869,
Title                    = {Nature},
Address                  = {London, GB},
Note                     = {ISSN 0028-0836. Semanal},
Publisher                = {Macmillan Magazines},
Year                     = {1869-},
Owner                    = {apcalabrez},
Timestamp                = {2015.10.01}
}
\end{verbatim}

\begin{tabular}{|l|c|} \hline
	SÃO PAULO MEDICAL JOURNAL = REVISTA PAULISTA DE\\MEDICINA.  São Paulo: Associação Paulista de Medicina, 1941- . \\ISSN 0035-0362.\\\hline
\end{tabular} \\

\textbf{Campos em LATEX:} 

\begin{verbatim}
@Journalpart{Medicaljournal1941,
address   = {São Paulo},
ISSN      = {0035-0362},
publisher = {Associação Paulista de Medicina},
title     = {São Paulo Medical Journal = Revista Paulista de Medicina},
year      = {1941-},
owner     = {apcalabrez},
timestamp = {2015.10.01}
}
\end{verbatim}


\subsection{Artigo de revista}


\begin{tabular}{|l|c|} \hline
	BOYD, A. L.; SAMID, D. Molecular biology of transgenic animals. \textbf{Journal } \\ \textbf{of  Animal Science}, Albany, v. 71, n. 3, p. 1-9, 1993.
	\\\hline
\end{tabular} \\

\textbf{Campos em LATEX:} 

\begin{verbatim}
@Article{Boyd1993,
Title                    = {Molecular biology of transgenic animals},
Author                   = {Boyd, A. L and Samid, D.},
Journal                  = {Journal of Animal Science},
Year                     = {1993},
Address                  = {Albany},
Number                   = {3},
Pages                    = {1-9},
Volume                   = {71},
Owner                    = {apcalabrez},
Timestamp                = {2015.10.02}
}
\end{verbatim}

\begin{tabular}{|l|c|} \hline
	KRAUSS, J. K. \textit{et al.} Flow void of cerebrospinal fluid in idiopathic normal\\
	pressure hydrocephalus of the elderly: can it predict outcome after
	shunting? \\\textbf{Neurosurgery}, Baltimore, v. 40, n. 1, p. 67-73, 1997.
	Discussion 73-74. 
	\\\hline
\end{tabular} \\

\textbf{Campos em LATEX:} 

\begin{verbatim}
@Article{Krauss1997,
Title                    = {Flow void of cerebrospinal fluid in idiopathic 
normal pressure hydrocephalus of the elderly:},
Author                   = {Krauss, J. K. and Souza, L. S. and Silva, A. M. 
and Arruda, M. L. and Mansilla, H. C. F.},
Journal                  = {Neurosurgery},
Subtitle                 = {can it predict outcome after shunting?},
Year                     = {1997},
Address                  = {Baltimore},
Note                     = {Discussion 73-74},
Number                   = {1},
Pages                    = {67-73},
Volume                   = {40},
Owner                    = {apcalabrez},
Timestamp                = {2015.10.02}
}
\end{verbatim}

\begin{tabular}{|l|c|} \hline
	RIVITTI, E. A. Departamento de Dermatologia: histórico, seus professores
	e \\ suas contribuições científicas. \textbf{Revista de Medicina}, São Paulo, v. 81, p. 7-13, \\ nov. 2002. Número especial.
	\\\hline
\end{tabular} \\

\textbf{Campos em LATEX:} 

\begin{verbatim}
@Article{Riviti2002,
Title                    = {Departamento de Dermatologia},
Author                   = {Rivitti, E. A.},
Journal                  = {Revista de Medicina},
Subtitle                 = {histórico, seus professores
e suas contribuições científicas},
Month                    = {nov.},

Year                     = {1997},
Address                  = {São Paulo},
Note                     = {Número especial},
Pages                    = {7-13},
Volume                   = {81},
Owner                    = {apcalabrez},
Timestamp                = {2015.10.02}
}
\end{verbatim}

\subsection{Editorial} 

\begin{tabular}{|l|c|} \hline
	BRENNAN, R. J.; SONDORP, E. Humanitarian aid: some political realities. \\ \textbf{British Medical Journal}, London, v. 333, n. 7573, p. 817-818, Oct. 2006. \\Editorial. Disponível em: http://bmj.bmjjournals.com/cgi/reprint/333/75\\73/817. Acesso em: 24 out. 2006. \\\hline
\end{tabular} \\

\textbf{Campos em LATEX:} 

\begin{verbatim}
@Article{Brennan2006,
Title                    = {Humanitarian aid},
Author                   = {Brennan, R. J. and Sondorp, E.},
Journal                  = {British Medical Journal},
Subtitle                 = {some political realities},
Year                     = {2006},
Address                  = {London},
Month                    = {Oct.},
Note                     = {Editorial},
Number                   = {7573},
Pages                    = {817-818},
Url                      = {http://bmj.bmjjournals.com/cgi/reprint/333/
7573/817},
Urlaccessdate            = {24 out. 2006},
Volume                   = {333},
Owner                    = {apcalabrez},
Timestamp                = {2015.10.02}
}
\end{verbatim}

\begin{tabular}{|l|c|} \hline
	COSTA, S. Os sertões: cem anos. \textbf{Revista USP}, São Paulo, v. 54, p. 5, jul./\\ago. 2002. Editorial.\\\hline
\end{tabular} \\

\textbf{Campos em LATEX:} 

\begin{verbatim}
@Article{Costa2002,
Title                    = {Os sertões},
Author                   = {Costa, S.},
Journal                  = {Revista USP},
Subtitle                 = {cem anos},
Year                     = {2002},
Address                  = {São Paulo},
Month                    = {jul./ago.},
Note                     = {Editorial},
Owner                    = {apcalabrez},
Timestamp                = {2015.10.02}
}
\end{verbatim}

\subsection{Entidade coletiva}

\begin{tabular}{|l|c|} \hline
	COCHRANE INJURIES GROUP ALBUMIN REVIEWERS. Human \\albumin administration in critically ill patients: systematic review of \\randomized controlled trials. \textbf{British Medical} \textbf{Journal}, London, v. 317, \\n. 7153, p. 235-240, 1998. 
	\\\hline
\end{tabular} \\

\textbf{Campos em LATEX:} 

\begin{verbatim}
@Article{Cochrane1998,
Title                    = {Human albumin administration in critically 
ill patients:systematic review of randomized controlled trials.},
Journal                  = {British Medical Journal},
Org-short                = {Cochrane Injuries Group Albumin Reviewers},
Organization             = {Cochrane Injuries Group Albumin Reviewers},
Year                     = {1998},
Address                  = {London},
Number                   = {7153},
Pages                    = {235-240},
Volume                   = {317},
Owner                    = {apcalabrez},
Timestamp                = {2015.10.02}
}
\end{verbatim}

\subsection{Artigos em suplementos ou em números especiais}

\begin{tabular}{|l|c|} \hline
	BOYD, A. L.; SAMID, D. Molecular biology of transgenic animals. \textbf{Journal } \\ \textbf{of Animal Science}, Albany, v. 71, p. 1-9, 1993. Supplement 3. 
	\\\hline
\end{tabular} \\

\textbf{Campos em LATEX:} 

\begin{verbatim}
@Article{Boyd1993,
Title                    = {Molecular biology of transgenic animals},
Author                   = {Boyd, A. L and Samid, D.},
Journal                  = {Journal of Animal Science},
Year                     = {1993},
Address                  = {Albany},
Note                     = {Supplement 3},
Pages                    = {1-9},
Volume                   = {71},
Owner                    = {apcalabrez},
Timestamp                = {2015.10.02}
}
\end{verbatim}

\begin{tabular}{|l|c|} \hline
	HOOD, D. W. The utility of complete genome sequences in the study of \\pathogenic bacteria. \textbf{Parasitology}, Cambridge, v. 118, p. S3-S9, 1999. \\Supplement. \\\hline
\end{tabular} \\

\textbf{Campos em LATEX:} 

\begin{verbatim}
@Article{Hood1999,
Title                    = {The utility of complete genome sequences in 
the study of pathogenic bacteria},
Author                   = {Hood, D. W.},
Journal                  = {Parasitology},
Year                     = {1999},
Address                  = {Cambridge},
Note                     = {Supplement},
Pages                    = {S3-S9},
Volume                   = {118},
Owner                    = {apcalabrez},
Timestamp                = {2015.10.02}
}
\end{verbatim}

\begin{tabular}{|l|c|} \hline
	PAYNE, D. K.; SULLIVAN, M. D.; MASSIE, M. J. Women's psychological \\ reactions to breast cancer. \textbf{Seminars in Oncology},  New York, v. 23, \\n. 1, p. 89-97, 1996. Supplement 2.
	\\\hline
\end{tabular} \\

\textbf{Campos em LATEX:} 

\begin{verbatim}
@Article{Payne1996
Title                    = {Women's psychological
reactions to breast cancer},
Author                   = {Payne, D. K. and Sullivan, M. D. and Massie, 
M. J.},
Journal                  = {Seminars in Oncology},
Year                     = {1996},
Address                  = {New York},
Note                     = {Supplement 2},
Pages                    = {89-97},
Volume                   = {23},
Number                   = {1},
Owner                    = {apcalabrez},
Timestamp                = {2015.10.02}
}
\end{verbatim}

\begin{tabular}{|l|c|} \hline
	TOLLIVET, M. Agricultura e meio ambiente: reflexões sociológicas. \\\textbf{Estudos Econômicos},  São Paulo, v. 24, p. 138-198, 1994. Número \\especial. 
	\\\hline
\end{tabular} \\

\textbf{Campos em LATEX:} 

\begin{verbatim}
@Article{Tollivet1994,
Title                    = {Agricultura e meio ambiente: reflexões 
sociológicas},
Author                   = {Tollivet, M},
Journal                  = {Estudos Econômicos},
Year                     = {1994},
Address                  = {São Paulo},
Note                     = {Número especial},
Pages                    = {138-198},
Volume                   = {24},
Owner                    = {apcalabrez},
Timestamp                = {2015.10.02}
}
\end{verbatim}

\subsection{Artigo publicado em partes}

\begin{tabular}{|l|c|} \hline
	ABEND, S. M.; KULISH, N. The psychoanalytic method from an\\
	epistemological viewpoint. \textbf{International Journal of Psycho-Analysis}, \\London, v. 83, pt. 2, p. 491-495, 2002. \\\hline
\end{tabular} \\

\textbf{Campos em LATEX:} 

\begin{verbatim}
@Article{Abend2002,
Title                    = {The psychoanalytic method from an 
epistemological viewpoint},
Author                   = {Abend, S. M. and Kulish},
Journal                  = {International Journal of Psycho-Analysis},
Year                     = {2002},
Address                  = {London},
Pages                    = {491-495},
Volume                   = {83, pt. 2},
Owner                    = {apcalabrez},
Timestamp                = {2015.10.02}
}
\end{verbatim}

\subsection{Artigo com errata publicada}

\begin{tabular}{|l|c|} \hline
	MALINOWSKI, J. M.; BOLESTA, S. Rosiglitazone in the treatment of
	type \\2 diabetes mellitus: a critical review. Clinical Therapetucis,
	Princeton, v. 22, \\n. 10, p. 1151-1168, 2000. Errata em: \textbf{Clinical
		Therapeutics}, Princeton, \\v. 23, n. 2, p. 309, 2001.
	\\\hline
\end{tabular} \\

\textbf{Campos em LATEX:} 

\begin{verbatim}
@Article{Malinowski2000,
Title                    = {Rosiglitazone in the treatment of type 
2 diabetes mellitus},
Author                   = {Malinowski, J. M and Bolesta, S.},
Journal                  = {Clinical Therapetucis},
Subtitle                 = {a critical review},
Year                     = {2000},
Address                  = {Princeton},
Note                     = {Errata em: \textbf{Clinical Therapeutics}, 
Princeton, v. 23, n. 2, p. 309, 2001},
Number                   = {10},
Pages                    = {1151-1168},
Volume                   = {22},
Owner                    = {apcalabrez},
Timestamp                = {2015.10.02}
}
\end{verbatim}

\subsection{Artigo publicado com indicação do mês}

\begin{tabular}{|l|c|} \hline
	HARRISON, P. Update on pain management for advanced genitourinary	\\cancer. \textbf{Journal of Urology}, Baltimore, v. 165, n. 6, p. 1849-1858, June \\2001. 
	\\\hline
\end{tabular} \\

\textbf{Campos em LATEX:} 

\begin{verbatim}
@Article{Harrison2001,
Title                    = {Update on pain management for advanced 
genitourinary 
cancer},
Author                   = {Harrison, P.},
Journal                  = {Journal of Urology},
Year                     = {2001},
Address                  = {Baltimore},
Month                    = {June},
Number                   = {6},
Pages                    = {1849-1858},
Volume                   = {165},
Owner                    = {AnaPaula},
Timestamp                = {2015.10.02}
}
\end{verbatim}

\begin{tabular}{|l|c|} \hline
	OLIVEIRA, R. \textit{et al.} Preparações radiofarmacêuticas e suas aplicações.\\
	\textbf{Revista Brasileira de Ciências Farmacêuticas}, São Paulo, v. 42, n. 2,\\
	p. 151-165, abr./jun. 2006. \\\hline
\end{tabular} \\

\textbf{Campos em LATEX:} 

\begin{verbatim}
@Article{Oliveira2006,
Title                    = {Preparações radiofarmacêuticas e suas 
aplicações},
Author                   = {Oliveira, R. and Silva, A. M. and Arruda, 
M. L. and 
Malinowski, J. M},
Journal                  = {Revista Brasileira de Ciências Farmacêuticas},
Year                     = {2006},
Address                  = {São Paulo},
Month                    = {abr./jun.},
Number                   = {2},
Pages                    = {151-165},
Volume                   = {42},
Owner                    = {AnaPaula},
Timestamp                = {2015.10.02}
}
\end{verbatim}

\subsection{Artigo no prelo}

É considerado no prelo o artigo já aceito para publicação pelo Conselho
Editorial do periódico.

\begin{tabular}{|l|c|} \hline
	ELEWA, H. H. Water resources and geomorphological characteristics of
	\\Tushka and west of Lake Nasser, Agypt. \textbf{Hydrogeology Journal}, Berlin,
	\\v. 16, n. 1, 2006. \textit{Ahead of print}. \\\hline
\end{tabular} \\

\textbf{Campos em LATEX:} 

\begin{verbatim}
@Article{Elewa2006,
Title                    = {Water resources and geomorphological 
characteristics of Tushka and west of Lake Nasser, Agypt},
Author                   = {Elewa, H. H.},
Journal                  = {Hydrogeology Journal},
Year                     = {2006},
Address                  = {Berlin},
Note                     = {\textit{Ahead of print}},
Number                   = {1},
Volume                   = {16},
Owner                    = {AnaPaula},
Timestamp                = {2015.10.02}
}
\end{verbatim}

\begin{tabular}{|l|c|} \hline
	PAULA, F. C. E. \textit{et al.} Incinerador de resíduos líquidos e pastosos.
	\textbf{Revista } \\ \textbf{de Engenharia e Ciências Aplicadas}, São Paulo, v. 5, n. 2,
	2001. No \\prelo. \\\hline
\end{tabular} \\

\textbf{Campos em LATEX:} 

\begin{verbatim}
@Article{Paula2001,
Title                    = {Incinerador de resíduos líquidos e pastosos},
Author                   = {Paula, F. C. E and Cardoso, R. F and Oliveira, 
A. P. and Silva, A. M. and Guimarães, P. C.},
Journal                  = {Revista de Engenharia e Ciências Aplicadas},
Year                     = {2001},
Address                  = {São Paulo},
Note                     = {No prelo},
Volume                   = {5},
Owner                    = {apcalabrez},
Timestamp                = {2015.09.16}
}
\end{verbatim}

\subsection{Publicações periódicas em suporte eletrônico}

\begin{tabular}{|l|c|} \hline
	PALAGACHEV, D. K.; RECKE, L.; SOFTOVA, L. G. Applications of\\ the
	differential calculus to nonlinear elliptic operators with discontinuous\\
	coefficients.  \textbf{Mathematische Annalen}, Berlin, v. 336, n. 3, p. 617-637,
	\\Nov. 2006. Disponível em:
	http://www.springerlink.com.w10077.dotlib.\\com.br/content/y767134777
	841722/fulltext.pdf. Acesso em: 17 nov. \\2006. 
	\\\hline
\end{tabular} \\

\textbf{Campos em LATEX:} 

\begin{verbatim}
Title                    = {Applications of the differential calculus 
to nonlinear
elliptic operators with discontinuous coefficients.},
Author                   = {Palagachev, D. K. and Recke, L and 
Softova, L. G.},
Journal                  = {Mathematische Annalen},
Year                     = {2006},
Address                  = {Berlin},
Month                    = {nov.},
Number                   = {3},
Pages                    = {617-637},
Url                      = {http://www.springerlink.com.w10077.dotlib.
com.br/content/y767134777841722/fulltext.pdf},
Urlaccessdate            = {17 nov. 2006},
Volume                   = {336},
Owner                    = {AnaPaula},
Timestamp                = {2015.10.02}
}
\end{verbatim}

\begin{tabular}{|l|c|} \hline
	PUECH-LEÃO, P. \textit{et al.} Prevalence of abdominal aortic aneurysms: a\\
	screening program in São Paulo, Brazil. \textbf{São Paulo Medical Journal},\\
	São Paulo, v. 122, n. 4, p. 158-160, 2004.  Disponível em: http://www.scielo.\\
	br/scielo.php?script= sciarttextpid=S1516-93322006000200007lngennrm=iso. \\
	Acesso em: 18 out. 2006. \\\hline
\end{tabular} \\

\textbf{Campos em LATEX:} 

\begin{verbatim}
Title                    = {Prevalence of abdominal aortic aneurysms},
Subtitle                 = {a screening program in São Paulo, Brazil},
Author                   = {Puech-Leão, P. and Celzi, P. A. and Facchi, A. 
B. and Louis, D. F.},
Journal                  = {São Paulo Medical Journal},
Year                     = {2004},
Address                  = {São Paulo},
Number                   = {4},
Pages                    = {158-160},
Url                      = {http://www.scielo.br/scielo.php?script=sci_
arttext&pid=S1516-31802004000400005&lng=en&nrm=iso},
Urlaccessdate            = {18 out. 2004},
Volume                   = {122},
Owner                    = {AnaPaula},
Timestamp                = {2015.10.02}
}
\end{verbatim}

\begin{tabular}{|l|c|} \hline
	SILVA, R. C. da; GIOIELLI, L. A. Propriedades físicas de lipídeos \\estruturados obtidos a partir de banha e óleo de soja. \textbf{Revista Brasileira} \\\textbf{de Ciências Farmacêuticas}, São Paulo, v. 42, n. 2, p. 223-235, 2006.\\
	Disponível em: http://www.scielo.br/scielo.php?script=sci-arttextpid=\\
	S1516-93322006000200007lng=ennrm=iso. Acesso em: 17 out. 2006. \\\hline
\end{tabular} \\

\textbf{Campos em LATEX:} 

\begin{verbatim}
Title                    = {Propriedades físicas de lipídeos estruturados
obtidos a partir de banha e óleo de soja},
Author                   = {SILVA, R. C. da and  GIOIELLI, L. A},
Journal                  = {Revista Brasileira de Ciências
Farmacêuticas},
Year                     = {2006},
Address                  = {São Paulo},
Number                   = {2},
Pages                    = {223-235},
Url                      = {http://www.scielo.br/
scielo.php?script=sci_arttext&pid=S1516-31802004000400005&lng=en&nrm=i
so},
Urlaccessdate            = {17 out. 2004},
Volume                   = {42},
Owner                    = {AnaPaula},
Timestamp                = {2015.10.02}
}
\end{verbatim}

\begin{tabular}{|l|c|} \hline
	WU, H. \textit{et al.} Parametric sensitivity in fixed-bed catalytic reactors with \\
	reverse flow operation. \textbf{Chemical Engineering Science}, London, v. 54,\\
	n. 20, 1999. Disponível em: http://www.probe.br/sciencedirect.html. \\Acesso em: 8 nov. 1999. \\\hline
\end{tabular} \\

\textbf{Campos em LATEX:} 

\begin{verbatim}
@Article{Wu1999,
Title                    = {Parametric sensitivity in fixed-bed 
catalytic reactors with reverse flow operation},
Author                   = {Wu, H. and Silva, A. M. and Montgomery, 
R. and Arruda, M. L.},
Journal                  = {Chemical Engineering Science},
Year                     = {1999},
Address                  = {London},
Number                   = {20},
Url                      = {http://www.probe.br/sciencedirect.html},
Urlaccessdate            = {8 nov. 1999},
Volume                   = {54},
Owner                    = {AnaPaula},
Timestamp                = {2015.10.02}
}
\end{verbatim}


\subsection{Artigo e/ou matéria de jornal}

\begin{tabular}{|l|c|} \hline
	HOFLING, E. Livro descreve os 134 tipos de aves do campus da USP. \textbf{O} \\ \textbf{Estado de S. Paulo}, São Paulo, 15 out. 1993. Cidades, Caderno 7, p. 15. \\Depoimento a Luiz Roberto de Souza Queiroz.	\\\hline
\end{tabular} \\

\begin{verbatim}
@Article{Hofling1993,
Title                    = {Livro descreve os 134 tipos de aves do campus
da USP},
Author                   = {Hofling, E.},
Journal                  = {O Estado de S. Paulo},
Year                     = {1993},
Address                  = {São Paulo},
Month                    = {15 out.},
Note                     = {Cidades, Caderno 7, p. 15. Depoimento a Luiz 
Roberto de Souza Queiroz},
Owner                    = {AnaPaula},
Timestamp                = {2015.10.02}
}
\end{verbatim}

\textbf{-- Em suporte eletrônico} \\

\begin{tabular}{|l|c|} \hline
	PORTER, E. This time, it's not the economy. \textbf{The New York Times}, \\New 
	York, 24 Oct. 2006. Disponível em: http://www.nytimes.com/2006\\/10/24/
	business/usinessoref=slogin. Acesso em: 24 out. 2006. \\\hline
\end{tabular} \\

\textbf{Campos em LATEX:} 

\begin{verbatim}
@Article{Porter2006,
Title                    = {This time, it's not the economy},
Author                   = {Porter, E.},
Journal                  = {The New York Times},
Year                     = {2006},
Address                  = {New York},
Month                    = {24 Oct.},
Url                      = {http://www.nytimes.com/2006/10/24/
business/usinessoref=slogin},
Urlaccessdate            = {24 out. 2006},
Owner                    = {AnaPaula},
Timestamp                = {2015.10.02}
}
\end{verbatim}

\subsection{Artigo publicado com correção}

\textbf{-- Correção de} \\

\begin{tabular}{|l|c|} \hline
	MEYAARD, L. \textit{et al.} The epithelial celular adhesion molecule (Ep-CAM)\\
	is a ligand for the leukocyte-associated immunoglobulin-like receptor
	\\(LAIR). \textbf{Journal of Experimental Medicine}, New York, v. 198, n. 7,\\ 
	p.	1129, Oct. 2003. Correção de: MEYAARD, L. \textit{et al.} \textbf{Journal of Experi-}\\ \textbf{mental Medicine}, New York, v. 194, n. 1, p. 107-112, July 2001.\\\hline
\end{tabular} \\

\textbf{Campos em LATEX:} 

\begin{verbatim}
@Article{Meyaard2003,
Title                    = {The epithelial celular adhesion molecule 
(Ep-CAM) is a ligand for the leukocyte-associated immunoglobulin-like 
receptor (LAIR).}, 
Author                   = {Meyaard, L and Arruda, M. L. and Silva, 
A. M. and Montgomery, R. and Malinowski, J. M},
Journal                  = {Journal of Experimental Medicine},
Year                     = {2003},
Address                  = {New York},
Month                    = {Oct.},
Note                     = {Correção de: MEYAARD, L. \textit{et al.} 
Journal of Experimental Medicine, New York, v. 194, n. 1, p. 107-112, 
July 2001},
Number                   = {7},
Pages                    = {1129},
Volume                   = {198},
Owner                    = {AnaPaula},
Timestamp                = {2015.10.02}
}
\end{verbatim}

\textbf{-- Correção em} \\

\begin{tabular}{|l|c|} \hline
	MEYAARD, L. \textit{et al.} The epithelial celular adhesion molecule (Ep-CAM)
	\\is a ligand for the leukocyte-associated immunoglobulin-like receptor
	(LAIR). \\Journal of Experimental Medicine, New York, v. 194, n. 1, p. 107-112, July \\2001. Correção em: MEYAARD, L. \textit{et al.} \textbf{Journal of Experimental}\\ \textbf{Medicine}, New York, v. 198, n. 7, p. 1129, Oct. 2003. 
	\\\hline
\end{tabular} \\

\textbf{Campos em LATEX:} 

\begin{verbatim}
@Article{Meyaard2003,
Title                    = {The epithelial celular adhesion molecule 
(Ep-CAM) is a ligand for the leukocyte-associated immunoglobulin-like 
receptor (LAIR).},
Author                   = {Meyaard, L and Arruda, M. L. and Silva, 
A. M. and Montgomery, R. and Malinowski, J. M},
Journal                  = {Journal of Experimental Medicine},
Year                     = {2001},
Address                  = {New York},
Month                    = {July},
Note                     = {Correção em: MEYAARD, L. \textit{et al.} 
\textbf{Journal of Experimental Medicine}, New York, v. 198, n. 7, 
p. 1129, Oct. 2003.},
Number                   = {1},
Pages                    = {107-112},
Volume                   = {194},
Owner                    = {AnaPaula},
Timestamp                = {2015.10.02}
}
\end{verbatim}

\section{Patentes}

Os exemplos abaixo são diferentes dos apresentados nas \textbf{Diretrizes para apresentação de dissertações e teses da USP}: documento eletrônico e impresso - Parte I (ABNT), 3ª edição de 2016, para melhor exemplificar as especificidades da ABNT NBR 6023:2018 para patentes. \\

\begin{tabular}{|l|c|} \hline
	BAGNATO, Vanderlei Salvador. \textbf{Processo de fotoalvejamento de tecidos}. \\ 
	Int. CI. D06L 3/12; D06L 3/16 BR 102016014269-5 A2. Depósito: 2 jan. 2018.
	\\\hline
\end{tabular} \\

\textbf{Campos em LATEX:} 

\begin{verbatim}
@Patent{Bagnato2018,
Title                    = {Processo de fotoalvejamento de tecidos},
Furtherresp              = {Int. CI. D06L 3/12; D06L 3/16 BR 102016014269-
5 A2. Depósito: 2 jan},
Author                   = {{BAGNATO, Vanderlei Salvador}},
year                     = {2018},
Owner                    = {marilza},
Timestamp                = {2019.10.09}
}
\end{verbatim}



\begin{tabular}{|l|c|} \hline
	VICENTE, Marcos Fernandes. \textbf{Reservatório para sabão em pó com} \\ 
	\textbf{suporte para escova}. Depositante: Marcos Fernandes Vicente: \\
	MU8802281-1U2, 15 out. 2008, 29 jun, 2010. Depósito: 15 out. 2018. \\
	Concessão: 29 jun. 2010.
	\\\hline
\end{tabular} \\


\textbf{Campos em LATEX:} 

\begin{verbatim}
@Patent{Vicente2010,
Title                    = {Reservatório para sabão em pó com suporte para 
escova},
Author                   = {{VICENTE, Marcos Fernandes}},
Furtherresp              = {Depositante: Marcos Fernandes Vicente: 
MU8802281-1U2, 15 out. 2008, 29 jun, 2010. Depósito: 15 out. 2018. 
Concessão: 29 jun},
year                     = {2010},
Owner                    = {marilza},
Timestamp                = {2019.10.21}
}
\end{verbatim}

\textbf{-- Em suporte eletrônico } \\


\begin{tabular}{|l|c|} \hline
	ROCHA, Flavio Alves da. \textbf{Composição veterinária à base de disofenol} \\ 
	\textbf{e suas variadas apresentações, para o combate ao carrapato em} \\ 
	\textbf{caninos}. Depositante: Flavio Alves da Rocha. Procurador: São \\
	Paulo Marcas e Patentes Ltda. BR 10 2017 003276 0 A2. Depósito: 17 fev. \\
	2017. Disponível em: \verb+ https://gru.inpi.gov.br/pePI/servlet/Patente+ \\
	\verb+Action=detail&CodPedido=1409935&SearchParameter=COMPOSI%C7%C3O%+ \\
    \verb+ServletController?20VETERIN%C1RIA%20%C0%20BASE%20DE%20DISOFENOL%+ \\
    \verb+20%20%20%20%20%20&Resumo=&Titulo=+. Acesso em: 1 abr. 2019. \\\hline
\end{tabular} \\

\textbf{Campos em LATEX:} 

\begin{verbatim}
@Patent{Rocha2017,
Title                    = {Composição veterinária à base de disofenol e 
suas variadas apresentações, para o combate ao carrapato em caninos},
Author                   = {{ROCHA, Flavio Alves da}},
Furtherresp              = {Depositante: Flavio Alves da Rocha. Procurador: 
São Paulo Marcas e Patentes Ltda. 	BR 10 2017 003276 0 A2. Depósito: 17
 fev},
Year                     = {2017},
URL                      =
{https://gru.inpi.gov.br/pePI/servlet/PatenteServletController?Action=
detail&CodPedido=1409935&SearchParameter=COMPOSI%C7%C3O%20VETERIN%C1RIA%
20%C0%20BASE%20DE%20DISOFENOL%20%20%20%20%20%20&Resumo=&Titulo=}, 
urlaccessdate            = {1 abr. 2019},
Owner                    = {marilza},
Timestamp                = {2019.10.21}
}
\end{verbatim}


\begin{tabular}{|l|c|} \hline
	OLIVEIRA, Luiz Antonio de \textit{et al.} \textbf{Ponta removível de fibra óptica para} \\ \textbf{uso de
	laser em odontologia e seu processo de fabricação}. Depositante: \\ MM Optics
	Ltda (BR/SP). Procurador: Marcio Loreti. PI 0504038-8 A2, \\ Depósito: 9 set.
	2005. Disponível em: https://gru.inpi.gov.br/pePI/servlet/\\PatenteServletController?Action=detailCodPedido=687788SearchParameter\\=LASER20
	EM20ODONTOLOGIA. Acesso em: 04 nov. 2015. 
	\\\hline
\end{tabular} \\

\textbf{Campos em LATEX:} 

\begin{verbatim}
@Patent{Oliveira2005,
Title                    = {Ponta removível de fibra óptica para uso de
laser em odontologia e seu processo de fabricação},
Author                   = {Oliveira, Luiz Antonio and Sousa, M. C. 
and Silva, E. D. and Juarez, R. S.},
HowPublished             = {9 set.
2005},
Number                   = {Depositante: MM Optics
Ltda (BR/SP). Procurador: Marcio Loreti. PI 0504038-8 A2},
Url                      = {https://gru.inpi.gov.br/pePI/servlet/
PatenteServletController?Action=detail
&CodPedido=687788&SearchParameter=LASER%
20EM%20ODONTOLOGIA},
Urlaccessdate            = {04 nov. 2002},
Owner                    = {apcalabrez},
Timestamp                = {2015.09.15}
}
\end{verbatim}

\section{Documentos Jurídicos}

Documentos referentes à legislação, jurisprudência (decisões judiciais) e
atos administrativos.

\subsection{Legislação}

 Inclui Constituição, Decreto, Decreto-Lei, Emenda Constitucional, Emenda à Lei Orgânica, Lei Complementar, Lei Delegada, Lei Ordinária, Lei Orgânica e Medida Provisória, entre outros.\\

\textbf{Elementos essenciais}

Jurisdição, ou cabeçalho da entidade, m letras maiúsculas; epigrafe e ementa transcrita conforme publicada; dados da publicação.

 
\textbf{Exemplos:} \\

\begin{tabular}{|l|c|} \hline
	BRASIL. \textbf{Código civil}. Organização dos textos, notas remissivas e índices: \\Juarez de Oliveira. 46. ed. São Paulo: Saraiva, 1995. 
	\\\hline
\end{tabular} \\

\textbf{Campos em LATEX:} 
\begin{verbatim}
@Book{codigo1985,
Title                    = {Código civil},
Address                  = {São Paulo},
Furtherresp              = {Organização dos textos, notas remissivas e 
índices: Juarez de Oliveira},
Org-short                = {Brasil},
Organization             = {Brasil},
Publisher                = {Saraiva},
Year                     = {1985},
Edition                  = {46},
Owner                    = {AnaPaula},
Timestamp                = {2015.10.02}
}
\end{verbatim}

\begin{tabular}{|l|c|} \hline
	BRASIL. Congresso. Senado. Resolução nº 17, de 1991. Autoriza o desbloqueio \\ de Letras Financeiras do Tesouro do Estado do Rio Grande do Sul, através de \\ revogação do parágrafo 2º, do artigo 1º da resolução nº 72, de
	1990. \textbf{Coleção} \\ \textbf{de leis da República Federativa do Brasil}, Brasília, DF, v.
	183, p. 1156-\\1157, maio/jun. 1991.
	\\\hline
\end{tabular} \\

\textbf{Campos em LATEX:} 


\begin{verbatim}
@Article{brasil1991,
Title                    = {Resolução nº 17, de
1991. Autoriza o desbloqueio de Letras Financeiras do Tesouro do 
Estado do Rio Grande do Sul, através de revogação do parágrafo 2º, 
do artigo 1º da resolução nº 72, de 1990},
Journal                  = {Coleção de leis da República Federativa do 
Brasil},
Organization             = {Brasil.  Congresso. Senado},
Org-short                = {Brasil},
Year                     = {1991},
Month                    = {maio/jun},
Address                  = {Brasília, DF},
Volume                   = {183},
Pages                    = {1156-1157},
Owner                    = {Ana Paula},
Timestamp                = {2015.09.10}
}
\end{verbatim}

\begin{tabular}{|l|c|} \hline
	BRASIL. \textbf{Constituição (1988)}. Constituição da República Federativa do \\Brasil. Brasília, DF: Senado, 1988. 
	\\\hline
\end{tabular} \\

\textbf{Campos em LATEX:} 

\begin{verbatim}
@Book{constituicao1988,
Title                    = {Constituição (1988)},
Address                  = {Brasília, DF},
Furtherresp              = {Constituição da República Federativa 
do Brasil.},
Org-short                = {Brasil},
Organization             = {Brasil},
Publisher                = {Senado},
Year                     = {1988},
Owner                    = {AnaPaula},
Timestamp                = {2015.10.02}
}
\end{verbatim}

\begin{tabular}{|l|c|} \hline
	BRASIL. Constituição (1988). Emenda Constitucional 
	nº 9, de 9 de \\novembro de 1995. Dá nova redação ao art. 177 da Constituição
	Federal,\\ alterando e inserindo parágrafos. \textbf{Lex}: legislação federal marginalia, São \\ Paulo, v. 59, p. 1966, out./dez. 1995.  
	\\\hline
\end{tabular} \\

\textbf{Campos em LATEX:} \\

\begin{verbatim}
@Article{Emenda1995,
Title                    = {Emenda Constitucional nº 9, de 9 de novembro 
de 1995. Dá nova redação ao art. 177 da Constituição Federal, alterando 
e inserindo parágrafos.},
Journal                  = { },
Org-short                = {Brasil},
Organization             = {Brasil. {Constituição (1988)}},
Year                     = {1995},
Address                  = {{\textbf{Lex}: legislação federal e marginalia, 
São Paulo}},
Month                    = {out./dez.},
Pages                    = {1966},
Volume                   = {59},
Owner                    = {Marilza},
Timestamp                = {2019.10.16}
}
\end{verbatim}

\begin{tabular}{|l|c|} \hline
	BRASIL. Decreto-lei nº 5452, de 1 de maio de 1943. Aprova a consolidação \\ das leis do trabalho. \textbf{Lex}: coletânea de legislação: edição federal, São Paulo, \\ v. 7, 1943. Suplemento.
	\\\hline
\end{tabular} \\

\textbf{Campos em LATEX:} 

\begin{verbatim}
@Article{brasil1943,
Title                    = {Decreto-lei nº 5452, de 1 de maio de 1943. 
Aprova a consolidação das leis do trabalho.},
Journal                  = { },
Organization             = {Brasil},
Year                     = {1943},
Address                  = {{\textbf{Lex}: legislação federal e marginalia, 
São Paulo}},
Volume                   = {7},
Note                     = {Suplemento},
Owner                    = {Marilza},
Timestamp                = {2019.10.16}
}
\end{verbatim}

\begin{tabular}{|l|c|} \hline
	BRASIL. Lei nº 7.000, de 20 de dezembro de 1990. Dispõe sobre a proibição
	da \\ pesca. \textbf{Diário Oficial da União}, Brasília, DF, 21 jan. 1991. Seção 1, p. 51.
	\\\hline
\end{tabular} \\

\textbf{Campos em LATEX:} 

\begin{verbatim}
@Article{brasil1943,
Title                    = {Lei nº 7.000, de 20 de dezembro de 1990. 
Dispõe sobre a proibição
da pesca},
Journal                  = {Diário Oficial da
União},
Organization             = {Brasil},
Year                     = {1991},
Address                  = {São Paulo},
Volume                   = {7},
Note                     = {Suplemento},
Owner                    = {Ana Paula},
Timestamp                = {2015.09.10}
}
\end{verbatim}

\begin{tabular}{|l|c|} \hline
	BRASIL. Medida provisória nº 1.569-9, de 11 de dezembro de 1997.
	Estabelece \\ multa em operações de importação, e dá outras providências.
	\textbf{Diário Oficial} \\ \textbf{[da] República Federativa do Brasil}, Poder Executivo,
	Brasília, DF, 14 dez.\\
	1997. Seção 1, p. 29514. \\\hline
\end{tabular} \\

\textbf{Campos em LATEX:} 

\begin{verbatim}
@Article{brasil1943,
Title                    = {Medida provisória nº 1.569-9, de 11 de 
dezembro de 1997. Estabelece multa em operações de importação, e 
dá outras providências},
Journal                  = {Diário Oficial [da] República Federativa do 
Brasil},
Organization             = {Brasil},
Month                    = {14 dez.},
Year                     = {1997},
Address                  = {Poder Executivo,
Brasília, DF},
Volume                   = {7},
Note                     = {Seção 1, p. 29514},
Owner                    = {Ana Paula},
Timestamp                = {2015.09.10}
}
\end{verbatim}


\begin{tabular}{|l|c|} \hline
	BRASIL. Secretaria da Receita Federal. Desliga a Empresa de Correios e \\ Telégrafos - ECT do sistema de arrecadação. Portaria nº 12, 21 de \\ março de 1996. \textbf{Lex}: coletânea de legislação e jurisprudência, São Paulo,\\ p. 742-743,
	mar./abr., 2. trim. 1996. 
	\\\hline
\end{tabular} \\

\textbf{Campos em LATEX:} 


\begin{verbatim}
@Article{brasil1996ECT,
title         = {Desliga a {Empresa de Correios e Telégrafos - 
ECT} do sistema de arrecadação. Portaria nº 12, 21 de
março de 1996.},
Organization  = {BRASIL. {Secretaria da Receita Federal}},
Org-short     = {Brasil},
year          = {1996},
pages         = {742-743},
month         = {mar./abr., 2. trim.},
Address       = {{\textbf{Lex}: coletânea de legislação e jurisprudência, 
São Paulo}},
Owner         = {Marilza},
Timestamp     = {2019.10.16}, 
}
\end{verbatim}


\begin{tabular}{|l|c|} \hline
	SÃO PAULO (Estado). Decreto nº 42.822, de 20 de janeiro de 1998. Dispõe \\ sobre a desativação de unidades administrativas de órgãos da administração \\
	direta e das autarquias do Estado e dá providências correlatas. \textbf{Lex}: coletânea \\ de legislação e jurisprudência, São Paulo, v. 62, n. 3, p. 217-220,
	1998.
	\\\hline
\end{tabular} \\

\textbf{Campos em LATEX:} 


\begin{verbatim}
@Article{brasil1991,
Title                    = {Decreto nº 42.822, de 20 de janeiro de 1998. 
Dispõe sobre a desativação de unidades administrativas de órgãos da 
administração direta e das autarquias do Estado e dá providências 
correlatas.},
Journal                  = { },
Organization             = {SÃO PAULO (Estado)},
Org-short                = {São Paulo},
Year                     = {1998},
Address                  = {\textbf{Lex}: legislação federal e marginalia, 
São Paulo},
Volume                   = {62},
Number                   = {3},
Pages                    = {217-220},
Owner                    = {Marilza},
Timestamp                = {2019.10.16}
}
\end{verbatim}
\subsection{Jurisprudência}

Inclui acórdão, decisão interlocutória, despacho, sentença, súmula, entre outros. \\

\textbf{Elementos essenciais}

Jurisdição (em letra maiúsculas), nome da corte ou trbunal; turma e/ou região (entre parênteses, se houver); tipo de documento (agravo, despacho, entre outros); número do processo (se houver); ementa (se houver); vara, ofício, cartório, câmara ou outra unidade do tribunal; mone do relator (precedido da palavra Relator, se houver), data do julgamento (se houver); dados da publicação. \\

\textbf{Exemplos:} \\


\begin{tabular}{|l|c|} \hline
	BRASIL. Tribunal Regional Federal. (5. Região). Administrativo. Escola\\
	Técnica Federal. Pagamento de diferenças referente a enquadramento de \\
	servidor decorrente da implantação de Plano Único de Classificação e \\
	Distribuição de Cargos e Empregos, instituído pela Lei nº 8.270/91. \\
	Predominância da lei sobre a portaria. Apelação cível nº 42.441-PE \\
	(94.05.01629-6). Apelante: Edilemos Mamede dos Santos e outros. Ape-\\
	lada: Escola Técnica Federal de Pernambuco. Relator: Juiz Nereu San-\\
	tos. Recife, 4 de março de 1997. \textbf{Lex}: jurisprudência do STJ e Tribu-\\
	nais Regionais Federais, São Paulo. v. 10, n.103, p. 558-562, mar. 1998. \\\hline
\end{tabular} \\

\textbf{Campos em LATEX:} 

\begin{verbatim}
@Article{brasillex1998,
Title                    = {Tribunal Regional Federal. Regi\~ao, 5. 
Administrativo. Escola T\’ecnica Federal. Pagamento de diferen{\c c}as 
referente a enquadramento de servidor decorrente de implanta{\c c}\~ao 
de Plano {{\’U}}nico de Classifica{\cc}\~ao e Distribui{\c c}\~ao de 
Cargos e Empregos, institu{\’\i}do pela Lei n{$^o$}~8.270/91. 
Predomin\^ancia da lei sobre a portaria. Apela{\cc}\~ao c{\’\i}vel
n{$^o$}~42.441-{PE} (94.05.01629-6). Apelante: Edilemos Mamede dos Santos
e outros. Apelada: Escola T\’ecnica Federal de Pernambuco. Relator: Juiz
Nereu Santos. Recife, 4 de mar{\c c}o de 1997},
Journal                  = { },
Organization             = {Brasil},
Year                     = {1998},
Address                  = {\textbf{Lex}: jurisprud\^encia do STJ e 
Tribunais Regionais Federais, S\~ao Paulo},
Month                    = {mar.},
Number                   = {103},
Pages                    = {558-562},
Volume                   = {10},
Owner                    = {Marilza},
Timestamp                = {2019.10.16}
}
\end{verbatim}

\subsection{Doutrina}

Qualquer discussão técnica sobre questões legais (monografias, artigos
de periódicos, papers etc.), referenciada conforme o tipo de publicação. 

\textbf{Exemplos:} \\

\begin{tabular}{|l|c|} \hline
	BARROS, Raimundo Gomes de. Ministério Público: sua legitimação	frente ao\\
	Código do Consumidor. \textbf{Revista Trimestral de Jurisprudência dos}\\
	\textbf{Estados}, São Paulo, v. 19, n. 139, p. 53-72, ago. 1995. \\\hline
\end{tabular} \\

\textbf{Campos em LATEX:} 

\begin{verbatim}
@Article{barros1995,
Title                    = {Ministério Público},
Author                   = {Barros, Raimundo Gomes de},
Journal                  = {Revista Trimestral de Jurisprudência dos 
Estados},
Subtitle                 = {sua legitimação
frente ao Código do Consumidor},
Year                     = {1995},
Address                  = {São Paulo,},
Month                    = {ago},
Number                   = {139},
Pages                    = {53-72},
Volume                   = {19},
Owner                    = {apcalabrez},
Timestamp                = {2016.04.26}
}
\end{verbatim}

\subsection{Documentos Jurídicos em suporte eletrônico}
%\textbf{4.5.4 Documentos Jurídicos em suporte eletrônico} \\

\begin{tabular}{|l|c|} \hline
	BRASIL. Lei nº 9.887, de 7 de dezembro de 1999. Altera a legislação tributária\\
	federal. \textbf{Diário Oficial [da] República Federativa do Brasil}, Brasília,\\
	DF, 8 dez. 1999. Disponível em: http://www.in.gov.br/mpleis/leistexto.asp?\\
	ld=LEI209887. Acesso em: 22 dez. 1999. 
	\\\hline
\end{tabular} \\

\textbf{Campos em LATEX:} 

\begin{verbatim}
@Article{1999,
Title                    = {Lei nº 9.887, de 7 de dezembro de 1999. Altera 
a legislação tributária federal},
Journal                  = {Diário Oficial da República Federativa do 
Brasil},
Organization             = {Brasil},
Year                     = {1999},
Address                  = {Brasília, DF},
Month                    = {8 dez.},
Url                      = {http://www.in.gov.br/mp_leis/leis_texto.aps?
Id=Lei209887},
Urlaccessdate            = {22 dez. 1999},
Owner                    = {Ana Paula},
Timestamp                = {2015.09.10}
}
\end{verbatim}


\section{Normas}

Norma é o documento estabelecido por consenso e aprovado por um organismo reconhecido, que fornece regras, diretrizes ou características mínimas para atividades ou para seus resultados, visando à obtenção de um grau ótimo de ordenação em um dado contexto.

\textbf{Exemplos:} \\

\begin{tabular}{|l|c|} \hline
	ASSOCIAÇÃO BRASILEIRA DE NORMAS TÉCNICAS. \textbf{NBR 10520}: \\informação e documentação: citações em documentos: apresentação. Rio \\de Janeiro, 2002a. 7 p. 
	\\\hline
\end{tabular} \\

\textbf{Campos em LATEX:} 

\begin{verbatim}
@Book{nbr10520,
Title                    = {NBR 10520},
Address                  = {Rio de Janeiro},
Org-short                = {Associa{\c c}\~ao Brasileira de Normas 
T\'ecnicas},
Organization             = {Associa{\c c}\~ao Brasileira de Normas 
T\'ecnicas},
Pages                    = {7},
Subtitle                 = {informação e documentação: citações 
em documentos: 
apresentação},
Year                     = {2002a},
Owner                    = {apcalabrez},
Timestamp                = {2015.10.16}
}
\end{verbatim}


\section{Materiais especiais}

Filmes cinematográficos ou científicos, gravações de vídeo e som,
esculturas, maquetes, objetos de museu, animais empalhados, jogos,
modelos, protótipos etc. \\

\begin{tabular}{|l|c|} \hline
	TÍTULO. Diretor, produtor. Local: Produtora, data. Especificação do	suporte\\
	em unidades físicas. Notas complementares. \\
	
	ou\\	
	
	SOBRENOME, Prenome(s) do(s) autor(es). \textbf{Título} (quando não 	existir,\\
	deve-se atribuir uma denominação ou a indicação sem 	título, entre col-\\
	chetes). Ano. Especificação do objeto. 
	\\\hline
\end{tabular} \\

\textbf{Exemplos:} \\

\begin{tabular}{|l|c|} \hline
	BULE de porcelana: família Rosa, decorado com buquês e guirlandas de flores\\ 
	sobre fundo branco, pegador de tampa em formato de fruto. [China: Compa-\\
	nhia das Índias, 18--]. 1 bule.  
	\\\hline
\end{tabular} \\

\textbf{Campos em LATEX:} 

\begin{verbatim}
@Book{bule18,
Title                    = {Bule de porcela},
Note                     = {[China: Companhia das Índias, 18--]. 1 
bule.}, 
Org-short                = {Bule, 18--},
Subtitle                 = {família Rosa, decorado com buquês e 
guirlandas de flores sobre fundo branco, pegador de tampa em formato de 
fruto},
Owner                    = {apcalabrez},
Timestamp                = {2015.10.08}
}
\end{verbatim}

\begin{tabular}{|l|c|} \hline
	CENTRAL do Brasil. Direção: Walter Salles Júnior. Produção: Martire de\\
	Clermont-Tonnerre e Arthur Cohn. Intérpretes: Fernanda Montenegro; Ma-\\
	rília Pera; Vinicius de Oliveira; Sônia Lira; Othon Bastos; Matheus\\ 
	Nachtergaele e outros. Roteiro: Marcos Bernstein, João Emanuel Carnei-\\
	ro e Walter Salles Júnior. [\textit{S.l.}]: Le Studio Canal; Riofilme; MACT \\
	Productions, 1998. 1 bobina cinematográfica (106 min), son., color., 
	\\35 mm. 
	\\\hline
\end{tabular} \\

\textbf{Campos em LATEX:} 

\begin{verbatim}
@Book{central1998,
Title                    = {Central do Brasil},
Furtherresp              = {Direção: Walter Salles Júnior. Produção: 
Martire de Clermont-Tonnerre e Arthur Cohn. Intérpretes: Fernanda 
Montenegro; Marília Pera; Vinicius de Oliveira; Sônia Lira; Othon 
Bastos; Matheus Nachtergaele e outros. Roteiro: Marcos Bernstein, 
João Emanuel Carneiro e Walter Salles Júnior},
Note                     = {1 bobina cinematográfica (106 min), 
son., color., 35 mm},
Org-short                = {Central},
Publisher                = {Le Studio Canal; Riofilme; MACT 
Productions},
Year                     = {1998},
Owner                    = {apcalabrez},
Timestamp                = {2015.10.08}
}
\end{verbatim}

\begin{tabular}{|l|c|} \hline
	KOBAYASHI, K. \textbf{Doença dos xavantes}. 1980. 1 fotografia, color., 16 cm x \\
	56 cm. 
	\\\hline
\end{tabular} \\

\textbf{Campos em LATEX:} 

\begin{verbatim}
@Book{Kobayashi1980,
Title                    = {Doença dos xavantes},
Author                   = {Kobayashi, K.},
Note                     = {1 fotografia, color., 16 cm x 56 cm},
Year                     = {1980},
Owner                    = {apcalabrez},
Timestamp                = {2015.10.08}
}

Ou

@Misc{KOBAYASHI1980,
Title                    = {Doenças dos xavantes},
Author                   = {Kobayashi, K.},
Note                     = {1 fot., color. 16 cm X 56 cm.},
Year                     = {1980},
Owner                    = {Ana Paula},
Timestamp                = {2015.09.10}
}
\end{verbatim}


\subsection{Documentos Cartográficos}

Mapa, atlas, globo, fotografia aérea, imagem de satélite etc. 
\subsubsection{No todo}

\begin{tabular}{|l|c|} \hline
	SOBRENOME, Prenome(s) do(s) autor(es). \textbf{Título}: subtítulo. Local: \\
	Editora, ano. Designação específica. Escala
	\\\hline
\end{tabular} \\

\textbf{Exemplos:} \\

\begin{tabular}{|l|c|} \hline
	ATLAS Mirador Internacional. Rio de Janeiro: Enciclopédia Britânica do\\
	Brasil, 1981. 1 atlas. Escalas variam. 
	\\\hline
\end{tabular} \\

\textbf{Campos em LATEX:} 

\begin{verbatim}
@Book{atlas1981,
Title                    = {Atlas Mirador Internacional},
Address                  = {Rio de Janeiro},
Note                     = {1 atlas. Escalas variam},
Org-short                = {Atlas},
Publisher                = {Enciclopédia Britânica do Brasil},
Year                     = {1981},
Owner                    = {apcalabrez},
Timestamp                = {2015.10.08}
}
\end{verbatim}

\begin{tabular}{|l|c|} \hline
	BRASIL e parte da América do Sul: mapa político, escolar, rodoviário,
	turís-\\ 
	tico e regional. São Paulo: Michalany, 1981. 1 mapa, color., 79 cm x\\
	95 cm. Escala 1:600. 
	\\\hline
\end{tabular} \\

\textbf{Campos em LATEX:} 

\begin{verbatim}
@Book{brasil1981,
Title                    = {Brasil e parte da América do Sul},
Address                  = {São Paulo},
Note                     = {1 mapa, color., 79 cm x 95 cm. Escala 1:600},
Org-short                = {Brasil},
Publisher                = {Michalany},
Subtitle                 = {mapa político, escolar, rodoviário, turístico 
e regional},
Year                     = {1981},
Owner                    = {apcalabrez},
Timestamp                = {2015.10.08}
}
\end{verbatim}

\subsubsection{Em suporte eletrônico}

\begin{tabular}{|l|c|} \hline
	SOBRENOME, Prenome(s) do(s) autor(es). \textbf{Título}: subtítulo. Local: Editora,\\
	ano. Designação específica. Escala. Disponível em: endereço eletrônico. \\
	Acesso em: dia mês abreviado. Ano. 
	\\\hline
\end{tabular} \\

\textbf{Exemplos:} \\

\begin{tabular}{|l|c|} \hline
	ATLAS ambiental da Bacia do Rio Corumbataí. Rio Claro: CEAPLA, IGCE,\\
	UNESP, 2001. Disponível em: http://www.rc.unesp.br/igce/ceapla/atlas.\\
	Acesso em: 8 abr. 2002. 
	\\\hline
\end{tabular} \\

\textbf{Campos em LATEX:} 

\begin{verbatim}
@Book{atlas2001,
Title                    = {Atlas ambiental da Bacia do Rio Corumbataí},
Address                  = {Rio Claro},
Org-short                = {Atlas},
Publisher                = {CEAPLA, IGCE, UNESP},
Year                     = {2001},
Url                      = {http://www.rc.unesp.br/igce/ceapla/atlas},
Urlaccessdate            = {8 abr. 2002},
Owner                    = {apcalabrez},
Timestamp                = {2015.10.08}
}
\end{verbatim}

\subsection{Documentos sonoros}

Discos, CD, fita cassete, fita magnética etc. \\
\subsubsection{No todo}

\begin{tabular}{|l|c|} \hline
	COMPOSITOR(ES) OU INTÉRPRETE(S). \textbf{Título}. Local: Gravadora, ano. \\
	Especificação do suporte. 
	\\\hline
\end{tabular} \\

\textbf{Exemplos:} \\

\begin{tabular}{|l|c|} \hline
	FAGNER, R. \textbf{Revelação}. Rio de Janeiro: CBS, 1988. 1 cassete sonoro (60 \\
	min), 3 3/4 pps, estéreo.  
	\\\hline
\end{tabular} \\

\textbf{Campos em LATEX:} 

\begin{verbatim}
@Book{Fagner1988,
Title                    = {Revelação},
Address                  = {Rio de Janeiro},
Author                   = {Fagner, R.},
Note                     = {1 cassete sonoro (60 min), 3 3/4 pps, 
estéreo},
Publisher                = {CBS},
Year                     = {1988},
Owner                    = {AnaPaula},
Timestamp                = {2015.10.08}
}
\end{verbatim}

\begin{tabular}{|l|c|} \hline
	DENVER, John. \textbf{Poems, prayers \& promises}. São Paulo: RCA Records,\\
	1974. 1 disco (38 min): 33 1/3 rpm, microssulco, estéreo. 104.4049. \\\hline
\end{tabular} \\

\textbf{Campos em LATEX:} 

\begin{verbatim}
@Book{Denver1974,
Title                    = {Poems, prayers \& promises},
Address                  = {São Paulo},
Author                   = {Denver, John},
Note                     = {1 disco (38 min): 33 1/3 rpm, microssulco, 
estéreo. 104.4049},
Publisher                = {RCA records},
Year                     = {1974},
Owner                    = {apcalabrez},
Timestamp                = {2015.10.08}
}
\end{verbatim}

\subsubsection{Em parte}
%\textbf{4.6.2.2 Em parte} \\

\begin{tabular}{|l|c|} \hline
	COSTA. S.; SILVA, A. Jura secreta. Intérprete: Simone. \textit{In}: SIMONE. \textbf{Face}\\ \textbf{a face}. [\textit{S.l.}]: Emi-Odeon Brasil, p1977. 1 CD. Faixa 7. 
	\\\hline
\end{tabular} \\

\textbf{Campos em LATEX:} 

\begin{verbatim}
@InCollection{simone1977,
author       = {Costa, S and Silva, A.},
title        = {Jura secreta. {Intérprete: Simone}},
booktitle    = {Face a face},
year         = {1977},
note         = {1 CD. Faixa 7},
publisher    = {Emi-Odeon Brasil},
org-short    = {Simone},
organization = {Simone},
owner        = {apcalabrez},
timestamp    = {2015.10.08},
}
\end{verbatim}

\subsection{Partituras}

\subsubsection{Impressa}


\begin{tabular}{|l|c|} \hline
	SOBRENOME, Prenome do autor. \textbf{Título}: subtítulo. Local: Editora, ano.\\
	Designação do material (unidades físicas: número de partituras ou de partes,\\
	páginas e/ou folhas). Instrumento a que se destina. 
	\\\hline
\end{tabular} \\

\textbf{Exemplos:} \\

\begin{tabular}{|l|c|} \hline
	VILLA-LOBOS, H. \textbf{Coleções de quartetos modernos}: cordas. Rio de \\Janeiro: [\textit{s.n}], 1916. 1 partitura [23 p.]. Violoncelo. 
	\\\hline
\end{tabular} \\

\textbf{Campos em LATEX:} 

\begin{verbatim}
@Book{Villa-Lobos1916,
author    = {Villa-Lobos, H.},
title     = {Coleções de quartetos modernos},
year      = {1916},
address   = {Rio de Janeiro},
subtitle  = {cordas},
publisher = {},
note      = {1 partitura [23 p.]. Violoncelo},
owner     = {apcalabrez},
timestamp = {2015.10.08},
}
\end{verbatim}

\subsubsection{Em suporte eletrônico}

\begin{tabular}{|l|c|} \hline
	SOBRENOME, Prenome do autor. \textbf{Título}: subtítulo. Local: Editora,
	ano. \\Designação do material (unidades físicas: número de
	partituras ou de partes).\\Instrumento a que se destina. Disponível
	em: endereço eletrônico. Acesso \\em: dia mês abreviado. Ano. 
	\\\hline
\end{tabular} \\

\textbf{Exemplos:} \\

\begin{tabular}{|l|c|} \hline
	OLIVA, Marcos; MOCOTÓ, Tiago. \textbf{Fervilhar}: frevo. [\textit{s.n}], [19--]. 1 partitura.
	\\Piano. Disponível em: http://openlink.inter.net/picolino/partitur.htm.
	\\Acesso: 5 jan. 2002. 
	\\\hline
\end{tabular} \\

\textbf{Campos em LATEX:} 

\begin{verbatim}
@Book{Oliva1900,
Title                    = {Fervilhar},
Author                   = {Oliva, M. and Mocot\'o, T.},
Note                     = {1 partitura. Piano},
Subtitle                 = {frevo},
Year                     = {{[19--]}},
Url                      = {http://openlink.inter.net/picolino/partitur.
htm},
Urlaccessdate            = {5 jan. 2002},
Owner                    = {apcalabrez},
Timestamp                = {2015.10.08}
}
\end{verbatim}

\subsection{Bula de medicamento}

\begin{tabular}{|l|c|} \hline
	TÍTULO da medicação. Responsável técnico (se houver). Local: Laboratório, \\ano de fabricação. Bula de remédio. 
	\\\hline
\end{tabular} \\

\textbf{Exemplos:} \\

\begin{tabular}{|l|c|} \hline
	RESPRIN: comprimidos. Responsável técnico Delosmar R. Bastos. São José \\dos Campos: Johnson \& Johnson, 1997. Bula de remédio. 
	\\\hline
\end{tabular} \\

\textbf{Campos em LATEX:} 

\begin{verbatim}
@Book{resprin1997,
Title                    = {Resprin},
Address                  = {São José dos Campos},
Furtherresp              = {Responsável técnico Delosmar R. Bastos},
Note                     = {Bula de remédio},
Publisher                = {Johnson \& Johnson},
Subtitle                 = {comprimidos},
Year                     = {1997},
Owner                    = {apcalabrez},
Timestamp                = {2015.09.14}
}
\end{verbatim}

\textbf{-- Em suporte eletrônico} \\

\begin{tabular}{|l|c|} \hline
	BUSCOPAN: composto. Responsável Técnico Dímitra Apostolopoulou.\\ Itacerica da Serra: Boehringer Ingelheim Brasil, 2013. Bula de remédio. \\Disponível em: http://www.buscopan.com.br/content/dam/internet/\\chc/buscopan/pt-BR/documents/bula-buscopan-composto-comprimidos-\\revestidos-paciente.pdf. Acesso em: 14 set. 2015.
	\\\hline
\end{tabular} \\

\textbf{Campos em LATEX:} 

\begin{verbatim}
@Book{buscopan2013,
Title                    = {Buscopan},
Address                  = {Itacerica da Serra},
Furtherresp              = {Responsável Técnico Dímitra 
Apostolopoulou},
Note                     = {Bula de remédio},
Publisher                = {Boehringer Ingelheim Brasil},
Subtitle                 = {composto},
Year                     = {2013},
Url                      = {http://www.buscopan.com.br/
content/dam/internet/chc/buscopan/pt_BR/documents/bula_
buscopan_composto_comprimidos_revestidos_paciente.pdf},
Urlaccessdate            = {14 set. 2015},
Owner                    = {apcalabrez},
Timestamp                = {2015.09.14}
}
\end{verbatim}

\subsection{Website}

\textbf{Exemplo:} \\

\begin{tabular}{|l|c|} \hline
	UNIVERSIDADE DE SÃO PAULO. Disponível em: http://www.usp.br.\\ Acesso em: 16 out. 2014
	\\\hline
\end{tabular} \\

\textbf{Campos em LATEX:} 

\begin{verbatim}
@Misc{usp2014,
Title                    = {UNIVERSIDADE DE SÃO PAULO},
Org-short                = {Universidade de São Paulo},
Url                      = {http://www.usp.br},
Urlaccessdate            = {16 out. 2014},
Owner                    = {marilza},
Timestamp                = {2019.10.16}
}
\end{verbatim}

\subsection{Artigo ahead of print}

Artigo aceito para publicação e disponível on-line, antes da impressão,
sem ter um número de fascículo associado. \\

\textbf{Exemplo:} \\


\begin{tabular}{|l|c|} \hline
	TEIXEIRA JÚNIOR, A. L.; CARAMELLI, P. Apatia na doença de \\Alzheimer. \textbf{Revista Brasileira de Psiquiatria}, São Paulo, 2006. No\\ prelo. Disponível em:
	http://www.scielo.br/pdf/rbp/nahead/ahead1b.\\pdf. Acesso em: 8 ago.
	2006. 
	\\\hline
\end{tabular} \\

\textbf{Campos em LATEX:} 

\begin{verbatim}
@Article{Teixeira2006,
Title                    = {Apatia na doença de Alzheimer},
Author                   = {Teixeira, Junior, A. L. and Caramelli, 
P.},
Journal                  = {Revista Brasileira de Psiquiatria},
Year                     = {2006},
Address                  = {São Paulo},
Note                     = {No prelo},
Url                      = {http://www.scielo.br/pdf/rbp/nahead
/ahead1b.pdf},
Urlaccessdate            = {8 ago. 2006},
Owner                    = {apcalabrez},
Timestamp                = {2016.04.26}
}
\end{verbatim}

\subsection{Digital Object Identifier (DOI)}

Representa um sistema de identificação numérico para localizar e
acessar materiais na web (publicações em periódicos, livros etc.), muitas
das quais localizadas em bibliotecas virtuais. Foi desenvolvido pela
Associação de Publicadores Americanos (AAP) com a finalidade de autenticar a base administrativa de conteúdo digital. Este número de
identificação da obra é composto por duas sequências: um prefixo (ou
raiz) que identifica o publicador do documento e um sufixo determinado
pelo responsável pela publicação do documento. \cite{Doic2016}.

Por exemplo: 34.7111.9 / ISBN (ou ISSN).

O prefixo DOI é nomeado pela International DOI Foundation (IDF),
garantindo identidade única a cada documento. \\


\begin{tabular}{|l|c|} \hline
	SUKIKARA, M. H. \textit{et al.} Opiate regulation of behavioral selection during \\lactation. \textbf{Pharmacology, Biochemistry and Behavior}, Phoenix, v. 87,\\ p. 315-320, 2007. DOI: 10.1016/j.pbb.2007.05.005. 
	\\\hline
\end{tabular} \\

\textbf{Campos em LATEX:} 

\begin{verbatim}
@Article{Sukikara2007,
Title                    = {Opiate regulation of behavioral selection 
during lactation},
Author                   = {Sukikara, M. H. and Arruda, M. L. and 
Softova,
L. G. and Malinowski, J. M},
Journal                  = {Pharmacology, Biochemistry and Behavior},
Year                     = {2007},
Address                  = {Phoenix},
Note                     = {DOI: 10.1016/j.pbb.2007.05.005},
Pages                    = {315-320},
Volume                   = {87},
Owner                    = {apcalabrez},
Timestamp                = {2015.10.08}
}
\end{verbatim}

\textbf{Documento em suporte eletrônico} 

\begin{tabular}{|l|c|} \hline
	DANTAS, J. A. \textit{et al.} Regulação da auditoria em sistemas bancários: análise do \\ cenário internacional e fatores determinantes. Revista Contabilidade \& Finanças,\\ São Paulo, v. 25, n. 64, p. 07–18, jan./abr. 2014. DOI: http://dx.doi.org/10.1590/\\S1519-707720140001000002. Disponível em: https://www.scielo.br/scielo.php?pid\\=S1519-70772014000100002\&script=sciarttext. Acesso em: 21 maio 2014.	
	\\\hline
\end{tabular} \\

\textbf{Campos em LATEX:} 

\begin{verbatim}
	@article{dantas2014,
		Title={Regula{\c{c}}{\~a}o da auditoria em sistemas banc{\'a}rios},
		Subtitle={an{\'a}lise do cen{\'a}rio internacional e fatores determinantes},
		Author={Dantas, J. A. and Costa, F. M. and Niyama, J. K. and Medeiros, O. R.},
		Journal={Revista Contabilidade \& Finan{\c{c}}as},
		Address={S{\~a}o Paulo},
		Volume={25},
		Number={64},
		Pages={07--18},
		Month={jan./abr.},
		year={2014},
		Note={DOI: http://dx.doi.org/10.1590/S1519-707720140001000002},
		Url={https://www.scielo.br/scielo.php?pid=S1519-70772014000100002&script=sci_arttext},
		Urlaccessdate={21 maio 2014},	
	}
\end{verbatim}


\subsection{CD-ROM e disquete}

\textbf{Exemplo} \\

\begin{tabular}{|l|c|} \hline
	MICROSOFT Project for Windows 95: project planning software. Version \\4.1. [\textit{S.l.}]: Microsoft Corporation, 1995. 1 CD-ROM. 
	\\\hline
\end{tabular} \\

\textbf{Campos em LATEX:} 

\begin{verbatim}
@Book{microsoft1995,
Title                    = {Microsoft Project for Windows 95},
Note                     = {1 CD-ROM},
Org-short                = {Microsoft},
Publisher                = {Microsoft Corporation},
Subtitle                 = {project planning software. Version 4.1.},
Year                     = {1995},
Owner                    = {apcalabrez},
Timestamp                = {2015.10.08}
}
\end{verbatim}

\subsection{Mensagens eletrônicas}

\textbf{Exemplo} \\

\begin{tabular}{|l|c|} \hline
	SCIENCEDIRECT MESSAGE CENTER. \textbf{ScienceDirect Search Alert}: \\34 New articles Available on ScienceDirect [mensagem pessoal]. Mensagem \\recebida por mjkarval@usp.br em 17 nov. 2006. 
	\\\hline
\end{tabular} \\

\textbf{Campos em LATEX:} 

\begin{verbatim}
@Book{science2006,
Title                    = {ScienceDirect Search Alert},
Note                     = {Mensagem recebida por mjkarval@usp.br 
em 17 nov. 2006},
Org-short                = {Sciencedirect Message Center},
Organization             = {Sciencedirect Message Center},
Subtitle                 = {34 New articles Available on 
ScienceDirect [mensagem pessoal]},
Owner                    = {apcalabrez},
Timestamp                = {2015.10.08}
}
\end{verbatim}

As referências das citações presentes em \textbf{REFERÊNCIAS} também servem de exemplos para elaboração de bibliografia em BibTeX e constam do arquivo.bib.

