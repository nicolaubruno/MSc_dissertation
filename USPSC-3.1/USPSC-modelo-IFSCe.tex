%% USPSC-modelo-IFSCe.tex
% ---------------------------------------------------------------
% USPSC: Modelo de Trabalho Academico (tese de doutorado, dissertacao de
% mestrado e trabalhos monograficos em geral) em conformidade com 
% ABNT NBR 14724:2011: Informacao e documentacao - Trabalhos academicos -
% Apresentacao
%----------------------------------------------------------------
%% Esta é uma customização do abntex2-modelo-trabalho-academico.tex de v-1.9.5 laurocesar 
%% para as Unidades do Campus USP de São Carlos:
%% EESC - Escola de Engenharia de São Carlos
%% IAU - Instituto de Arquitetura e Urbanismo
%% ICMC - Instituto de Ciências Matemáticas e de Computação
%% IFSC - Instituto de Física de São Carlos
%% IQSC - Instituto de Química de São Carlos
%%
%% Este trabalho utiliza a classe USPSC.cls que é mantida pela seguinte equipe:
%% 
%% Coordenação e Programação:
%%   - Marilza Aparecida Rodrigues Tognetti - marilza@sc.usp.br (PUSP-SC)
%%   - Ana Paula Aparecida Calabrez - aninha@sc.usp.br (PUSP-SC)
%% Normalização:
%%   - Brianda de Oliveira Ordonho Sigolo - brianda@usp.br (IAU)
%%   - Eduardo Graziosi Silva - edu.gs@sc.usp.br (EESC)
%%   - Eliana de Cássia Aquareli Cordeiro - eliana@iqsc.usp.br (IQSC)
%%   - Flávia Helena Cassin - cassinp@sc.usp.br (EESC)
%%   - Maria Cristina Cavarette Dziabas - mcdziaba@ifsc.usp.br (IFSC)
%%   - Regina Célia Vidal Medeiros - rcvmat@icmc.usp.br (ICMC)
%%
%% O USPSC-modelo.tex e USPSC-TCC-modelo.tex utilizam diversos arquivos relacionado em 
%% 2.1 Pacote USPSC: Classe USPSC e modelos de trabalhos acadêmicos	do Tutorial do Pascote 
%%  USPSC para modelos de trabalhos de acadêmicos em LaTeX - versão 3.1


%----------------------------------------------------------------
%% Sobre a classe abntex2.cls:
%% abntex2.cls, v-1.9.5 laurocesar
%% Copyright 2012-2015 by abnTeX2 group at https://www.abntex.net.br/ 
%%
%----------------------------------------------------------------

\documentclass[
% -- opções da classe memoir --
12pt,		% tamanho da fonte
openright,	% capítulos começam em pág ímpar (insere página vazia caso preciso)
twoside,  % para impressão em anverso (frente) e verso. Oposto a oneside - Nota: utilizar \imprimirfolhaderosto*
%oneside, % para impressão em páginas separadas (somente anverso) -  Nota: utilizar \imprimirfolhaderosto
% inclua uma % antes do comando twoside e exclua a % antes do oneside 
a4paper,			% tamanho do papel. 
% -- opções da classe abntex2 --
chapter=TITLE,		% títulos de capítulos convertidos em letras maiúsculas
% -- opções do pacote babel --
english,			% idioma adicional para hifenização
french,				% idioma adicional para hifenização
spanish,			% idioma adicional para hifenização
brazil				% o último idioma é o principal do documento
% {USPSC-classe/USPSC} configura o cabeçalho contendo apenas o número da página
]{USPSC-classe/USPSC}
%]{USPSC-classe/USPSC1}
% Inclua % antes de ]{USPSC-classe/USPSC} e retire a % antes de %]{USPSC-classe/USPSC1} para utilizar o 
% cabeçalho diferenciado para as páginas pares e ímpares:
%- páginas ímpares: com seções ou subseções e o número da página
%- páginas pares: com o número da página e o título do capítulo 
% ---
% ---
% Pacotes básicos - Fundamentais 
% ---
\usepackage[T1]{fontenc}		% Seleção de códigos de fonte.
\usepackage[utf8]{inputenc}		% Codificação do documento (conversão automática dos acentos)
\usepackage{lmodern}			% Usa a fonte Latin Modern
% Para utilizar a fonte Times New Roman, inclua uma % no início do comando acima  "\usepackage{lmodern}"
% Abaixo, tire a % antes do comando  \usepackage{times}
%\usepackage{times}		    	% Usa a fonte Times New Roman	
% Para usar a fonte , lembre-se de tirar a % do comando %\renewcommand{\ABNTEXchapterfont}{\rmfamily}, localizado mais abaixo, logo após "Outras opções para nota de rodapé no Sistema Numérico" 						
\usepackage{lastpage}			% Usado pela Ficha catalográfica
\usepackage{indentfirst}		% Indenta o primeiro parágrafo de cada seção.
\usepackage{color}				% Controle das cores
\usepackage{graphicx}			% Inclusão de gráficos
\usepackage{float} 				% Fixa tabelas e figuras no local exato
\usepackage{chemfig,chemmacros} % Para escrever reações químicas
\usepackage{tikz}				% Para escrever reações químicas e outros
\usetikzlibrary{positioning}
\usepackage{pdfpages}
\usepackage{makeidx}            % para gerar índice remissivo
\usepackage{hyphenat}          % Pacote para retirar a hifenizacao DO TEXTO
\usepackage[absolute]{textpos} % Pacote permite o posicionamento do texto
\usepackage{eso-pic}           % Pacote para incluir imagem de fundo
\usepackage{makebox}           % Pacote para criar caixa de texto
% ---

% ---
% Pacotes de citações
% Citações padrão ABNT
% ---
% Sistemas de chamada: autor-data ou numérico.
% Sistema autor-data
%\usepackage[alf, abnt-emphasize=bf, abnt-thesis-year=both, abnt-repeated-author-omit=no, abnt-last-names=abnt, abnt-etal-cite, abnt-etal-list=3, abnt-etal-text=it, abnt-and-type=e, abnt-doi=doi, abnt-url-package=none, abnt-verbatim-entry=no]{abntex2cite}
%\bibliographystyle{USPSC-classe/abntex2-alf-USPSC}
% Se o idioma for o inglês, exclua % no comando acima ou do comando abaixo
%\bibliographystyle{USPSC-classe/abntex2-alfeng-USPSC}

% Para o IQSC, que indica todos os autores nas referências, incluir % no início dos comandos acima e retirar a % dos comandos abaixo 
%\usepackage[alf, abnt-emphasize=bf, abnt-thesis-year=both, abnt-repeated-author-omit=no, abnt-last-names=abnt, abnt-etal-cite, abnt-etal-list=0, abnt-etal-text=it, abnt-and-type=e, abnt-doi=doi, abnt-url-package=none, abnt-verbatim-entry=no]{abntex2cite} 
%\bibliographystyle{USPSC-classe/abntex2-alf-USPSC}
% Se o idioma for o inglês, exclua % no comando acima ou do comando abaixo
%\bibliographystyle{USPSC-classe/abntex2-alfeng-USPSC}

% Sistema Numérico
% Para citações numéricas, sistema adotado pelo IFSC, incluir % no início dos comandos acima e retirar a % dos comandos abaixo 
\usepackage{cite}              % agrupa citações numéricas consecutivas
\usepackage[num, abnt-emphasize=bf, abnt-thesis-year=both, abnt-repeated-author-omit=no, abnt-last-names=abnt, abnt-etal-cite, abnt-etal-list=3, abnt-etal-text=it, abnt-and-type=e, abnt-doi=doi, abnt-url-package=none, abnt-verbatim-entry=no]{abntex2cite} 
%\bibliographystyle{USPSC-classe/abntex2-num-USPSC}
% Se o idioma for o inglês, inclua % no comando acima e exclua o % do comando abaixo
\bibliographystyle{USPSC-classe/abntex2-numeng-USPSC}

% Complementarmente, verifique as instruções abaixo sobre os Pacotes de Nota de rodapé
% ---
% Pacotes de Nota de rodapé
% Configurações de nota de rodapé

% O presente modelo adota o formato numérico para as notas de rodapés quando utiliza o sistema de chamada autor-data para citações e referências. Para utilizar o sistema de chamada numérico para citações e referências, habilitar um dos comandos abaixo.
% Há diversa opções para nota de rodapé no Sistema Numérico.  Para o IFSC, habilitade o comando abaixo.

\renewcommand{\thefootnote}{\fnsymbol{footnote}}  %Comando para inserção de símbolos em nota de rodapé

% Outras opções para nota de rodapé no Sistema Numérico:
%\renewcommand{\thefootnote}{\alph{footnote}}      %Comando para inserção de letras minúscula em nota de rodapé
%\renewcommand{\thefootnote}{\Alph{footnote}}      %Comando para inserção de letras maiúscula em nota de rodapé
%\renewcommand{\thefootnote}{\roman{footnote}}     %Comando para inserção de números romanos minúsculos  em nota de rodapé
%\renewcommand{\thefootnote}{\Roman{footnote}}     %Comando para inserção de números romanos minúsculos  em nota de rodapé

\renewcommand{\footnotesize}{\small} %Comando para diminuir a fonte das notas de rodapé
%Para utilizar a fonte Times New Roman, inclua retire % do início do comando abaixo 
%\renewcommand{\ABNTEXchapterfont}{\rmfamily}

% ---
% Pacote para agrupar a citação numérica consecutiva
% Quando for adotado o Sistema Numérico, a exemplo do IFSC, habilite 
% o pacote cite abaixo retirando a porcentagem antes do comando abaixo
%\usepackage[superscript]{cite}	

% ---
% Pacotes adicionais, usados apenas no âmbito do Modelo Canônico do abnteX2
% ---
\usepackage{lipsum}				% para geração de dummy text
% ---

% pacotes de tabelas
\usepackage{multicol}	% Suporte a mesclagens em colunas
\usepackage{multirow}	% Suporte a mesclagens em linhas
\usepackage{longtable}	% Tabelas com várias páginas
\usepackage{threeparttablex}    % notas no longtable
\usepackage{array}

% ----
% Compatibilização com a ABNT NBR 6023:2018
% Para tirar <> da URL
%\DeclareFieldFormat{url}{\bibstring{urlfrom}\addcolon\addspace\url{#1}}
\usepackage{USPSC-classe/ABNT6023-2018}
% As demais compatibilizações estão nos arquivos abntex2-alf-USPSC.bst e abntex2-num-USPSC.bst, chamados através do comando \bibliographystyle{USPSC-classe/abntex2-alf-USPSC} ou %\bibliographystyle{USPSC-classe/abntex2-num-USPSC}, dependendo se o Sistemas de chamada for autor-data ou numérico.
% ----

% ---
% DADOS INICIAIS - Define sigla com título, área de concentração e opção do Programa 
% Consulte a tabela referente aos Programas, áreas e opções de sua unidade contante do
% arquivo USPSC-Siglas estabelecidas para os Programas de Pós-Graduação nos APÊNDICES B-J
\siglaunidade{IFSC}
\programa{DFAe}
% Os demais dados deverão ser fornecidos no arquivo USPSC-pre-textual-UUUU ou USPSC-TCC-pre-textual-UUUU, onde UUUU é a sigla da Unidade. 
% Exemplo:USPSC-pre-textual-IFSC.tex
% ---
% Configurações de aparência do PDF final
% alterando o aspecto da cor azul
\definecolor{blue}{RGB}{41,5,195}

% informações do PDF
\makeatletter
\hypersetup{
	%pagebackref=true,
	pdftitle={\@title}, 
	pdfauthor={\@author},
	pdfsubject={\imprimirpreambulo},
	pdfcreator={LaTeX with abnTeX2},
	pdfkeywords={abnt}{latex}{abntex}{USPSC}{trabalho acadêmico}, 
	colorlinks=true,       		% false: boxed links; true: colored links
	linkcolor=black,          	% color of internal links
	citecolor=black,        		% color of links to bibliography
	filecolor=black,      		% color of file links
	urlcolor=black,
	%Para habilitar as cores dos links, retire a % antes dos comandos abaixo e inclua a % antes das 4 linhas de comando acima 
	%linkcolor=blue,            	% color of internal links
	%citecolor=blue,        		% color of links to bibliography
	%filecolor=magenta,      		% color of file links
	%urlcolor=blue,
	bookmarksdepth=4	
}
\makeatother
% --- 
% --- 
% Espaçamentos entre linhas e parágrafos 
% --- 

% O tamanho do parágrafo é dado por:
\setlength{\parindent}{1.3cm}

% Controle do espaçamento entre um parágrafo e outro:
\setlength{\parskip}{0.2cm}  % tente também \onelineskip

% ---
% compila o sumário e índice
\makeindex
% ---

% ----
% Início do documento
% ----
\begin{document}

% Seleciona o idioma do documento (conforme pacotes do babel)
%\selectlanguage{brazil}
% Se o idioma do texto for inglês, inclua uma % antes do 
%      comando \selectlanguage{brazil} e 
%      retire a % antes do comando abaixo
\selectlanguage{english}

% Retira espaço extra obsoleto entre as frases.
\frenchspacing 

% --- Formatação dos Títulos
\renewcommand{\ABNTEXchapterfontsize}{\fontsize{12}{12}\bfseries}
\renewcommand{\ABNTEXsectionfontsize}{\fontsize{12}{12}\bfseries}
\renewcommand{\ABNTEXsubsectionfontsize}{\fontsize{12}{12}\normalfont}
\renewcommand{\ABNTEXsubsubsectionfontsize}{\fontsize{12}{12}\normalfont}
\renewcommand{\ABNTEXsubsubsubsectionfontsize}{\fontsize{12}{12}\normalfont}


% ----------------------------------------------------------
% ELEMENTOS PRÉ-TEXTUAIS
% ----------------------------------------------------------
% ---
% Capa
% ---
\imprimircapa
% ---
% Folha de rosto
% (o * indica impressão em anverso (frente) e verso )
% ---
\imprimirfolhaderosto*
%\imprimirfolhaderosto
% ---
% ---
% Inserir a ficha catalográfica em pdf
% ---
% A biblioteca da sua Unidade lhe fornecerá um PDF com a ficha
% catalográfica definitiva. 
% Quando estiver com o documento, salve-o como PDF no diretório
% do seu projeto como fichacatalografica.pdf e inclua o arquivo
% utilizando o comando abaixo:

\includepdf{USPSC-TA-PreTextual/USPSC-fichacatalografica.pdf}

% Se você optar por elaborar a ficha catalográfica, deverá 
% incluir uma % antes da linha % antes
% do comando %% USPSC-fichacatalografica.tex
% ---
% Inserir a ficha bibliografica
% ---
% Isto é um exemplo de Ficha Catalográfica, ou ``Dados internacionais de
% catalogação-na-publicação''. Você pode utilizar este modelo como referência. 
% Porém, provavelmente a biblioteca da sua universidade lhe fornecerá um PDF
% com a ficha catalográfica definitiva após a defesa do trabalho. Quando estiver
% com o documento, salve-o como PDF no diretório do seu projeto e substitua todo
% o conteúdo de implementação deste arquivo pelo comando abaixo:
%
\begin{fichacatalografica}
	\hspace{-1.4cm}
	\imprimirnotaautorizacao \\ \\
	%\sffamily
	\vspace*{\fill}					% Posição vertical
\begin{center}					% Minipage Centralizado
  \imprimirnotabib \\
  \begin{table}[htb]
	\scriptsize
	\centering	
	\begin{tabular}{|p{0.9cm} p{8.7cm}|}
		\hline
	      & \\
		  &	  \imprimirautorficha     \\
		
		 \imprimircutter & 
							\hspace{0.4cm}\imprimirtitulo~  / ~\imprimirautor~ ;  ~\imprimirorientadorcorpoficha. -- 	\imprimirlocal, \imprimirdata.   \\
		
		  &  % Para incluir nota referente à versão corrigida no corpo da ficha,
			  % incluir % no início da linha acima e tirar a % do início da linha abaixo
			  %	\hspace{0.4cm} \imprimirtitulo~  / ~\imprimirautor~ ; ~\imprimirorientadorcorpoficha~- ~\imprimirnotafolharosto. -- \imprimirlocal, \imprimirdata.  \\
		
			\hspace{0.4cm}\pageref{LastPage} p. : il. (algumas color.) ; 30 cm.\\ 
		  & \\
		  & 
		    \hspace{0.4cm}\imprimirnotaficha ~--~ 
						  \imprimirunidademin, 
						  \imprimiruniversidademin, 
		                  \imprimirdata. \\ 
		  & \\                 
		   % Para incluir nota referente à versão corrigida em notas,
		    % incluir uma % no início da linha acima e	
		    % tirar a % do início da linha abaixo
		    % & \hspace{0.4cm}\imprimirnotafolharosto \\ 
		  & \\ 
		  & \hspace{0.4cm}1. LaTeX. 2. abnTeX. 3. Classe USPSC. 4. Editoração de texto. 5. Normalização da documentação. 6. Tese. 7. Dissertação. 8. Documentos (elaboração). 9. Documentos eletrônicos. I. \imprimirorientadorficha. 
		   II. Título. \\
	
		     %Se houver co-orientador, inclua % antes da linha (antes de II. Título.) 
		     %          e tire a % antes do comando abaixo 
		     %III. Título. \\   
		  \hline
	\end{tabular}
  \end{table}
\end{center}
\end{fichacatalografica}
% ---

 
% e retirar o % do comando abaixo
%%% USPSC-fichacatalografica.tex
% ---
% Inserir a ficha bibliografica
% ---
% Isto é um exemplo de Ficha Catalográfica, ou ``Dados internacionais de
% catalogação-na-publicação''. Você pode utilizar este modelo como referência. 
% Porém, provavelmente a biblioteca da sua universidade lhe fornecerá um PDF
% com a ficha catalográfica definitiva após a defesa do trabalho. Quando estiver
% com o documento, salve-o como PDF no diretório do seu projeto e substitua todo
% o conteúdo de implementação deste arquivo pelo comando abaixo:
%
\begin{fichacatalografica}
	\hspace{-1.4cm}
	\imprimirnotaautorizacao \\ \\
	%\sffamily
	\vspace*{\fill}					% Posição vertical
\begin{center}					% Minipage Centralizado
  \imprimirnotabib \\
  \begin{table}[htb]
	\scriptsize
	\centering	
	\begin{tabular}{|p{0.9cm} p{8.7cm}|}
		\hline
	      & \\
		  &	  \imprimirautorficha     \\
		
		 \imprimircutter & 
							\hspace{0.4cm}\imprimirtitulo~  / ~\imprimirautor~ ;  ~\imprimirorientadorcorpoficha. -- 	\imprimirlocal, \imprimirdata.   \\
		
		  &  % Para incluir nota referente à versão corrigida no corpo da ficha,
			  % incluir % no início da linha acima e tirar a % do início da linha abaixo
			  %	\hspace{0.4cm} \imprimirtitulo~  / ~\imprimirautor~ ; ~\imprimirorientadorcorpoficha~- ~\imprimirnotafolharosto. -- \imprimirlocal, \imprimirdata.  \\
		
			\hspace{0.4cm}\pageref{LastPage} p. : il. (algumas color.) ; 30 cm.\\ 
		  & \\
		  & 
		    \hspace{0.4cm}\imprimirnotaficha ~--~ 
						  \imprimirunidademin, 
						  \imprimiruniversidademin, 
		                  \imprimirdata. \\ 
		  & \\                 
		   % Para incluir nota referente à versão corrigida em notas,
		    % incluir uma % no início da linha acima e	
		    % tirar a % do início da linha abaixo
		    % & \hspace{0.4cm}\imprimirnotafolharosto \\ 
		  & \\ 
		  & \hspace{0.4cm}1. LaTeX. 2. abnTeX. 3. Classe USPSC. 4. Editoração de texto. 5. Normalização da documentação. 6. Tese. 7. Dissertação. 8. Documentos (elaboração). 9. Documentos eletrônicos. I. \imprimirorientadorficha. 
		   II. Título. \\
	
		     %Se houver co-orientador, inclua % antes da linha (antes de II. Título.) 
		     %          e tire a % antes do comando abaixo 
		     %III. Título. \\   
		  \hline
	\end{tabular}
  \end{table}
\end{center}
\end{fichacatalografica}
% ---


% As informações que compõem a ficha catalográfica estão 
% definidas no arquivo USPSC-pre-textual-UUUU.tex
% ---

% ---
% Folha de rosto adicional
% Para imprimir a folha de rosto adicional, exigida por algumas Unidades, a exemplo do ICMC,
% retire a % antes do comando abaixo

%\imprimirfolhaderostoadic

% ---
% ---
% Inserir errata
% ---

%% USPSC-Errata.tex
\begin{errata}
	%\OnehalfSpacing 			
	A errata é um elemento opcional, que consiste de uma lista de erros da obra, precedidos pelas folhas e linhas onde eles ocorrem e seguidos pelas correções correspondentes. Deve ser inserida logo após a folha de rosto e conter a referência do trabalho para facilitar sua identificação, conforme a ABNT NBR 14724 \cite{nbr14724}.
	
	Modelo de Errata:
		
	\begin{flushleft} 
			\setlength{\absparsep}{0pt} % ajusta o espaçamento da referência	
			\SingleSpacing 
			\imprimirautorabr~ ~\textbf{\imprimirtituloresumo}.	\imprimirdata. \pageref{LastPage}p. 
			%Substitua p. por f. quando utilizar oneside em \documentclass
			%\pageref{LastPage}f.
			\imprimirtipotrabalho~-~\imprimirinstituicao, \imprimirlocal, \imprimirdata. 
 	\end{flushleft}
\vspace{\onelineskip}
\OnehalfSpacing 
\center
\textbf{ERRATA}
\vspace{\onelineskip}
\OnehalfSpacing 
\begin{table}[htb]
	\center
	\footnotesize
	\begin{tabular}{p{2cm} p{2cm} p{4cm} p{4cm} }
		\hline
		\textbf{Folha} & \textbf{Linha}  & \textbf{Onde se lê}  & \textbf{Leia-se}  \\
			\hline
			1 & 10 & auto-conclavo & autoconclavo\\
		\hline
	\end{tabular}
\end{table}
\end{errata}
% ---

% ---

% ---
% Inserir folha de aprovação
% ---

% A Folha de aprovação é um elemento obrigatório da NBR 4724/2011 (seção 4.2.1.3). 
% Após a defesa/aprovação do trabalho, gere o arquivo folhadeaprovacao.pdf da página assinada pela banca 
% e iclua o arquivo utilizando o comando abaixo:
\includepdf{USPSC-TA-PreTextual/USPSC-folhadeaprovacao.pdf}
% Alternativa para a Folha de Aprovação:
% Se for a sua opção elaborar uma folha de aprovação, insira uma % antes do comando acima que inclui o arquivo folhadeaprovacao.pdf,
% tire o % do comando abaixo e altere o arquivo folhadeaprovacao.tex conforme suas necessidades
%\include{folhadeaprovacao}
\includepdf{USPSC-TA-PreTextual/USPSC-PaginaEmBranco.pdf}

% ---
% Dedicatória
% ---
%% USPSC-Dedicatoria.tex
\begin{dedicatoria}
   \vspace*{\fill}
   \centering
   \noindent
   \textit{For my family and friends who support me during all my academic life.} \vspace*{\fill}
\end{dedicatoria}
% ---
% ---

% ---
% Agradecimentos
% ---
%% USPSC-Agradecimentos.tex
\begin{agradecimentos}
	I would like to acknowledge the support and patience of my advisor, Professor Emanuel A. L. Henn, who helped me proceed with my work even through the difficulties.

	Many thanks to my father, Rogério C. Santos, my mother, Adeli C. C. Santos, and my sister Andressa C. N. Santos, who supported me throughout the process by keeping me motivated and always being present when I most needed them. I could not have finished this work without them. I would also like to thank my girlfriend and also my partner, Denise S. Ribeiro, who follows my dedication and keeps supporting me during the process.

	Lastly, but not less important, I would like to thank my closest friends Adonai H. Silva, Cainã de Oliveira, Henrique Malavazzi, Raian Westin, and Vinicius Martinez (in alphabetical order since I value them equally) for all their support, insightful conversions, and unforgeable moments.

	This study was financed in part by the Coordenação de Aperfeiçoamento de Pessoal de Nível Superior – Brasil (CAPES) – Finance Code 001



	
\end{agradecimentos}
% ---

% ---

% ---
% Epígrafe
% ---
\include{USPSC-TA-PreTextual/USPSC-Epigrafe}
% ---

% A T E N Ç Ã O
% Se o idioma do texto for em inglês, o abstract deve preceder o resumo
% Abstract
% ---
%% USPSC-Abstract.tex
%\autor{Silva, M. J.}
\begin{resumo}[ABSTRACT]
 \begin{otherlanguage*}{english}
	\begin{flushleft} 
		\setlength{\absparsep}{0pt} % ajusta o espaçamento dos parágrafos do resumo		
 		\SingleSpacing  		\imprimirautorabr~~\textbf{\imprimirtitleabstract}.	\imprimirdata.  \pageref{LastPage}p. 
		%Substitua p. por f. quando utilizar oneside em \documentclass
		%\pageref{LastPage}f.
		\imprimirtipotrabalhoabs~-~\imprimirinstituicao, \imprimirlocal, 	\imprimirdata. 
 	\end{flushleft}
	\OnehalfSpacing 
   In this work, we proposed and implemented a Monte Carlo simulation to estimate experimental quantities of narrow-line magneto-optical traps (nMOTs). Our model relies on sampling the atoms' movement under laser light and a quadrupole magnetic field as a Markov chain. We assume that the involved electronic transition can be modelled as a four-level system and split into three independent two-level systems, which is only valid for nMOTs in the power-broadened regime. We were able to estimate quantities for three nMOT arrangements at different laboratories. The first nMOT traps dysprosium atoms on an electronic transition with narrowness $ \eta = 43.8 $, which is not an ideal value since it is slightly larger. Nevertheless, we could obtain estimated quantities that were more accurate than the theoretical predictions. The other two nMOTs trap strontium atoms on an electronic transition with narrowness $ \eta = 1.6 $. We could also obtain estimated quantities close to the experimental measures in this case.

   \vspace{\onelineskip}
 
   \noindent 
   \textbf{Keywords}: Magneto-optical trap. Monte Carlo simulation. Laser cooling.
 \end{otherlanguage*}
\end{resumo}

% ---
% resumo em português
%
% Resumo
% ---
%% USPSC-Resumo.tex
\setlength{\absparsep}{18pt} % ajusta o espaçamento dos parágrafos do resumo		
\begin{resumo}
	\begin{flushleft} 
			\setlength{\absparsep}{0pt} % ajusta o espaçamento da referência	
			\SingleSpacing 
			\imprimirautorabr~~\textbf{\imprimirtituloresumo}.	\imprimirdata. \pageref{LastPage}p. 
			%Substitua p. por f. quando utilizar oneside em \documentclass
			%\pageref{LastPage}f.
			\imprimirtipotrabalho~-~\imprimirinstituicao, \imprimirlocal, \imprimirdata. 
 	\end{flushleft}
\OnehalfSpacing 			
Nesse trabalho, propomos e implementamos uma simulação de Monte Carlo para estimar quantidades experimentais de armadilhas magneto-ópticas de linhas estreita (nMOTs). Nosso modelo se baseia em amostrar o movimento dos átomos expostos a luz laser e um campo magnético quadrupolar como uma cadeia de Markov. Nós assumimos que a transição eletrônica envolvida pode ser modelada como um sistema de quatro níveis e, então, a dividimos em três sistemas de dois níveis independentes sobre a suposição em que nMOT esteja no regime de alargamento por potência. Nós fomos capazes de estimar três nMOTs de diferente laboratórios. O primeiro aprisiona átomos de disprósio em uma transição cuja estreiteza é 43,8, o que não é um valor ideal por ser ligeiramente elevado. Apesar disso, nós obtivemos quantidades estimadas mais precisas que as previstas teoricamente. Os outros dois nMOTs aprisionam estrôncio em uma transição com estreiteza 1,6. Nesse caso, nós também obtivemos quantidades estimadas próximas as medidas experimentais.
 

 \textbf{Palavras-chave}: Armadilha magneto-óptica. Simulação de Monte Carlo. Resfriamento a laser.
\end{resumo}

% ---

% ---
% inserir lista de figurass
% ---
\pdfbookmark[0]{\listfigurename}{lof}
\listoffigures*
\cleardoublepage
% ---

% ---
% inserir lista de tabelas
% ---
\pdfbookmark[0]{\listtablename}{lot}
\listoftables*
\cleardoublepage
% ---

% ---
% inserir lista de quadros
% ---
\pdfbookmark[0]{\listofquadroname}{loq}
\listofquadro*
\cleardoublepage
% ---

% ---
% inserir lista de abreviaturas e siglas
% ---
% USPSC-AbreviaturasSiglas.tex
\begin{siglas}
    \item[ABNT] Associação Brasileira de Normas Técnicas
    \item[abnTeX] ABsurdas Normas para TeX
	\item[IBGE] Instituto Brasileiro de Geografia e Estatística
	\item[LaTeX] Lamport TeX
	\item[USP] Universidade de São Paulo
	\item[USPSC] Campus USP de São Carlos
\end{siglas}

% ---

% ---
% inserir lista de símbolos
% ---
\include{USPSC-TA-PreTextual/USPSC-Simbolos}
% ---
% ---
% inserir o sumario
% ---
\pdfbookmark[0]{\contentsname}{toc}
\tableofcontents*
\cleardoublepage
% ---
% ----------------------------------------------------------
% ELEMENTOS TEXTUAIS
% ----------------------------------------------------------
\textual
% Os capítulos são inseridos como arquivos externos 

% Capítulo 1 - Introdução
% ---
%% USPSC-Introducao.tex

% ----------------------------------------------------------
% Introdução (exemplo de capítulo sem numeração, mas presente no Sumário)
% ----------------------------------------------------------
\chapter[Introdução]{Introdução}
\label{Introdução}


Parte inicial do texto, que contém a delimitação do assunto tratado, objetivos da pesquisa e outros elementos necessários para apresentar o tema do trabalho \cite{aguia2020}.



	

% ---

% ---
% Capítulo 2
% ---
%% USPSC-Cap2-Desenvolvimento.tex 

% ---
% Este capítulo, utilizado por diferentes exemplos do abnTeX2, ilustra o uso de
% comandos do abnTeX2 e de LaTeX.
% ---

\chapter{Desenvolvimento}\label{cap_exemplos}
Este capítulo é parte principal do trabalho acadêmico e deve conter a exposição ordenada e detalhada do assunto. Divide-se em seções e subseções, em conformidade com a abordagem do tema e do método, abrangendo: revisão bibliográfica, materiais e métodos, técnicas utilizadas, resultados obtidos e discussão.

Abaixo são apresentados minimamente exemplos tabelas, quadros, divisões de documentos e outros itens. Consulte o \textbf{Tutorial do Pacote USPSC para modelos de trabalhos de acad\^emicos em LaTeX - vers\~ao 3.1} para demais informações. 

\section{Resultados de comandos}\label{sec-divisoes}

% ---
\subsection{Tabelas e quadros}

O \textbf{Tutorial do Pacote USPSC para modelos de trabalhos de acad\^emicos em LaTeX - vers\~ao 3.1} apresenta orientações completas e diversas formatações de tabelas, dentre elas a \autoref{tab-ibge}, que é um exemplo de tabela alinhada que pode ser longa ou curta, conforme padrão do Instituto Brasileiro de Geografia e Estatística (IBGE).

%\begin{table}[H]
\begin{table}[htb]
	\IBGEtab{%
		\caption{Frequência anual por categoria de usuários}%
		\label{tab-ibge}
	}{%
		\begin{tabular}{ccc}
			\toprule
			Categoria de Usuários & Frequência de Usuários \\
			\midrule \midrule
			Graduação & 72\% \\
			\midrule 
			Pós-Graduação & 15\% \\
			\midrule 
			Docente & 10\% \\
			\midrule 
			Outras & 3\% \\
			\bottomrule
		\end{tabular}%
	}{%
		\fonte{Elaborada pelos autores.}%
		\nota{Exemplo de uma nota.}%
		\nota[Anotações]{Uma anotação adicional, que pode ser seguida de várias
			outras.}%
		
	}
\end{table}


A formatação do quadro é similar à tabela, mas deve ter suas laterais fechadas e conter as linhas horizontais.
\newpage

% o comando \newpage foi utilizado para forçar a quebra de página

\begin{quadro}[htb]
	\caption{\label{quadro_modelo}Níveis de investigação}
	\begin{tabular}{|p{2.6cm}|p{6.0cm}|p{2.25cm}|p{3.40cm}|}
		\hline
		\textbf{Nível de Investigação} & \textbf{Insumos}  & \textbf{Sistemas de Investigação}  & \textbf{Produtos}  \\
		\hline
		Meta-nível & Filosofia\index{filosofia} da Ciência  & Epistemologia &
		Paradigma  \\
		\hline
		Nível do objeto & Paradigmas do metanível e evidências do nível inferior &
		Ciência  & Teorias e modelos \\
		\hline
		Nível inferior & Modelos e métodos do nível do objeto e problemas do nível inferior & Prática & Solução de problemas  \\
		\hline
	\end{tabular}
	\begin{flushleft}
		%\fonte{\citeonline{van1986}}
		Fonte: \citeonline{van1986}
	\end{flushleft}
\end{quadro} 


No \textbf{Tutorial do Pacote USPSC para modelos de trabalhos de acad\^emicos em LaTeX - vers\~ao 3.1} são apresentados mais exemplos de quadros.

% ---
\subsection{Figuras}\label{sec_figuras}
% ---
\index{figuras}Figuras podem ser criadas diretamente em \LaTeX,
como o exemplo da \autoref{fig_circulo}. \\ 

\begin{figure}[htb]
	\caption{\label{fig_circulo}A delimitação do espaço}
	\begin{center}
		\setlength{\unitlength}{9cm}
		\begin{picture}(1,1)
		\put(0,0){\line(0,1){1}}
		\put(0,0){\line(1,0){1}}
		\put(0,0){\line(1,1){1}}
		\put(0,0){\line(1,2){.5}}
		\put(0,0){\line(1,3){.3333}}
		\put(0,0){\line(1,4){.25}}
		\put(0,0){\line(1,5){.2}}
		\put(0,0){\line(1,6){.1667}}
		\put(0,0){\line(2,1){1}}
		\put(0,0){\line(2,3){.6667}}
		\put(0,0){\line(2,5){.4}}
		\put(0,0){\line(3,1){1}}
		\put(0,0){\line(3,2){1}}
		\put(0,0){\line(3,4){.75}}
		\put(0,0){\line(3,5){.6}}
		\put(0,0){\line(4,1){1}}
		\put(0,0){\line(4,3){1}}
		\put(0,0){\line(4,5){.8}}
		\put(0,0){\line(5,1){1}}
		\put(0,0){\line(5,2){1}}
		\put(0,0){\line(5,3){1}}
		\put(0,0){\line(5,4){1}}
		\put(0,0){\line(5,6){.8333}}
		\put(0,0){\line(6,1){1}}
		\put(0,0){\line(6,5){1}}
		\end{picture}
	\end{center}
	\legend{Fonte: \citeonline{equipeabntex2}}
\end{figure}

Consulte o \textbf{Tutorial do Pacote USPSC para modelos de trabalhos de acad\^emicos em LaTeX - vers\~ao 3.1} para conhecer mais recursos referentes à figuras. 

% ---
\section{Divisões do documento}\label{sec-divisoes-b}
Esta seção exemplifica o uso de divisões de documentos em conformidade com a ABNT NBR 6024  \cite{nbr6024}.
% ---
% ---
\subsection{Divisões do documento: subseção}\label{sec-divisoes-subsection}
% ---

Um exemplo de seção é a \autoref{sec-divisoes-b}. Esta é a \autoref{sec-divisoes-subsection}.

\subsubsection{Divisões do documento: subsubseção}\label{sec-divisoes-subsubsection}

Isto é uma \texttt{subsubsection} do \LaTeX, mas é denominada de ``subseção'' porque no português não temos a palavra ``subsubseção''.

\subsubsection{Divisões do documento: subsubseção}

Isto é outra subsubseção.

\subsection{Divisões do documento: subseção}\label{sec-exemplo-subsec}

Isto é uma subseção.

\subsubsection{Divisões do documento: subsubseção}

Isto é mais uma subsubseção da \autoref{sec-exemplo-subsec}.


\subsubsubsection{Esta é uma subseção de quinto
nível}\label{sec-exemplo-subsubsubsection}

Esta é uma seção de quinto nível. Ela é produzida com o seguinte comando:

\begin{verbatim}
\subsubsubsection{Esta é uma subseção de quinto
nível}\label{sec-exemplo-subsubsubsection}
\end{verbatim}

\subsubsubsection{Esta é outra subseção de quinto nível}\label{sec-exemplo-subsubsubsection-outro}

Esta é outra seção de quinto nível.


\paragraph{Este é um parágrafo numerado}\label{sec-exemplo-paragrafo}

Este é um exemplo de parágrafo nomeado. Ele é produzido com o comando de
parágrafo:

\begin{verbatim}
\paragraph{Este é um parágrafo nomeado}\label{sec-exemplo-paragrafo}
\end{verbatim}

A numeração entre parágrafos numerados e subsubsubseções são contínuas.

\paragraph{Esta é outro parágrafo numerado}\label{sec-exemplo-paragrafo-outro}

Este é outro parágrafo nomeado.

% ---
\subsection{Este é um exemplo de nome de subseção longa que se aplica a seções e demais divisões do documento. Ele deve estar alinhado à esquerda e a segunda e demais linhas devem iniciar logo abaixo da primeira palavra da primeira linha} 

Observe que o alinhamento do título obedece esta regra também no sumário.
	








% Capítulo 3 - Conclusão
% ---
%% USPSC-Cap3-Conclusao.tex
% ---
% Conclusão
% ---
\chapter{Conclusão}
% ---
% O comando abaixo insere parágrafos aleatórios só para exemplificar
Apresentar as conclusões correspondentes aos objetivos ou hipóteses propostos para o desenvolvimento do trabalho, podendo incluir  sugestões para novas pesquisas.


% ---

% ----------------------------------------------------------
% ELEMENTOS PÓS-TEXTUAIS
% ----------------------------------------------------------
\postextual
% ----------------------------------------------------------

% -----------------------------------------------------------
% Referências bibliográficas
% ----------------------------------------------------------
\bibliography{USPSC-bib/USPSC-modelo-references}


% ----------------------------------------------------------
% Glossário
% ----------------------------------------------------------
%
% Consulte o manual da classe abntex2 para orientações sobre o glossário.
%
%\glossary

% ----------------------------------------------------------
% Apêndices
% ----------------------------------------------------------
%% USPSC-Apendice.tex
% ---
% Inicia os apêndices
% ---

\begin{apendicesenv}
% Imprime uma página indicando o início dos apêndices
\partapendices
\chapter{Apêndice(s)}
Elemento opcional, que consiste em texto ou documento elaborado pelo autor, a fim de complementar sua argumentação, conforme a ABNT NBR 14724 \cite{nbr14724}.

Os apêndices devem ser identificados por letras maiúsculas consecutivas, seguidas de hífen e pelos respectivos títulos. Excepcionalmente, utilizam-se letras maiúsculas dobradas na identificação dos apêndices, quando esgotadas as 26 letras do alfabeto. A paginação deve ser contínua, dando seguimento ao texto principal. \cite{aguia2020}
% ----------------------------------------------------------
\chapter{Exemplo de tabela centralizada verticalmente e horizontalmente}
\index{tabelas}A \autoref{tab-centralizada} exemplifica como proceder para obter uma tabela centralizada verticalmente e horizontalmente.
% utilize \usepackage{array} no PREAMBULO (ver em USPSC-modelo.tex) obter uma tabela centralizada verticalmente e horizontalmente
\begin{table}[htb]
\ABNTEXfontereduzida
\caption[Exemplo de tabela centralizada verticalmente e horizontalmente]{Exemplo de tabela centralizada verticalmente e horizontalmente}
\label{tab-centralizada}

\begin{tabular}{ >{\centering\arraybackslash}m{6cm}  >{\centering\arraybackslash}m{6cm} }
\hline
 \centering \textbf{Coluna A} & \textbf{Coluna B}\\
\hline
  Coluna A, Linha 1 & Este é um texto bem maior para exemplificar como é centralizado verticalmente e horizontalmente na tabela. Segundo parágrafo para verificar como fica na tabela\\
  Quando o texto da coluna A, linha 2 é bem maior do que o das demais colunas  & Coluna B, linha 2\\
\hline
\end{tabular}
\begin{flushleft}
		Fonte: Elaborada pelos autores.\
\end{flushleft}
\end{table}

% ----------------------------------------------------------
\chapter{Exemplo de tabela com grade}
\index{tabelas}A \autoref{tab-grade} exemplifica a inclusão de traços estruturadores de conteúdo para melhor compreensão do conteúdo da tabela, em conformidade com as normas de apresentação tabular do IBGE.
% utilize \usepackage{array} no PREAMBULO (ver em USPSC-modelo.tex) obter uma tabela centralizada verticalmente e horizontalmente
\begin{table}[htb]
\ABNTEXfontereduzida
\caption[Exemplo de tabelas com grade]{Exemplo de tabelas com grade}
\label{tab-grade}
\begin{tabular}{ >{\centering\arraybackslash}m{8cm} | >{\centering\arraybackslash}m{6cm} }
\hline
 \centering \textbf{Coluna A} & \textbf{Coluna B}\\
\hline
  A1 & B1\\
\hline
  A2 & B2\\
\hline
  A3 & B3\\
\hline
  A4 & B4\\
\hline
\end{tabular}
\begin{flushleft}
		Fonte: Elaborada pelos autores.\
\end{flushleft}
\end{table}


\end{apendicesenv}
% ---

% ----------------------------------------------------------
% Anexos
% ----------------------------------------------------------
%% USPSC-Anexos.tex
% ---
% Inicia os anexos
% ---
\begin{anexosenv}

% Imprime uma página indicando o início dos anexos
\partanexos

% ---
\chapter{Exemplo de anexo}
% ---
Elemento opcional, que consiste em um texto ou documento não elaborado pelo autor, que serve de fundamentação, comprovação e ilustração, conforme a ABNT NBR 14724. \cite{nbr14724}.

O \textbf{ANEXO B} exemplifica como incluir um anexo em pdf.

\chapter{Acentuação (modo texto - \LaTeX)}
\begin{figure}[H]
	\begin{center}
	\caption{\label{fig_anexob}Acentuação (modo texto - \LaTeX)}
	\includegraphics[scale=1.0]{USPSC-img/USPSC-AcentuacaoLaTeX.png} \\
	Fonte: \citeonline{comandos}
	\end{center}	
\end{figure}

\end{anexosenv}


%---------------------------------------------------------------------
% INDICE REMISSIVO
%--------------------------------------------------------------------
%% USPSC-IndicexRemissivos.tex
% ---
% Inicia os Índices Remissivos
% ---
%---------------------------------------------------------------------
% INDICE REMISSIVO
%--------------------------------------------------------------------
\phantompart
\printindex
%---------------------------------------------------------------------


%---------------------------------------------------------------------

\end{document}
