%% USPSC-Resumo.tex
\setlength{\absparsep}{18pt} % ajusta o espaçamento dos parágrafos do resumo		
\begin{resumo}
	\begin{flushleft} 
			\setlength{\absparsep}{0pt} % ajusta o espaçamento da referência	
			\SingleSpacing 
			\imprimirautorabr~~\textbf{\imprimirtituloresumo}.	\imprimirdata. \pageref{LastPage}p. 
			%Substitua p. por f. quando utilizar oneside em \documentclass
			%\pageref{LastPage}f.
			\imprimirtipotrabalho~-~\imprimirinstituicao, \imprimirlocal, \imprimirdata. 
 	\end{flushleft}
\OnehalfSpacing 			
Nesse trabalho, propomos e implementamos uma simulação de Monte Carlo para estimar quantidades experimentais de armadilhas magneto-ópticas de linhas estreita (nMOTs). Nosso modelo se baseia em amostrar o movimento dos átomos expostos a luz laser e um campo magnético quadrupolar como uma cadeia de Markov. Nós assumimos que a transição eletrônica envolvida pode ser modelada como um sistema de quatro níveis e, então, a dividimos em três sistemas de dois níveis independentes sobre a suposição em que nMOT esteja no regime de alargamento por potência. Nós fomos capazes de estimar três nMOTs de diferente laboratórios. O primeiro aprisiona átomos de disprósio em uma transição cuja estreiteza é 43,8, o que não é um valor ideal por ser ligeiramente elevado. Apesar disso, nós obtivemos quantidades estimadas mais precisas que as previstas teoricamente. Os outros dois nMOTs aprisionam estrôncio em uma transição com estreiteza 1,6. Nesse caso, nós também obtivemos quantidades estimadas próximas as medidas experimentais.
 

 \textbf{Palavras-chave}: Armadilha magneto-óptica. Simulação de Monte Carlo. Resfriamento a laser.
\end{resumo}
