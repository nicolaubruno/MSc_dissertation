%% USPSC-Abstract.tex
%\autor{Silva, M. J.}
\begin{resumo}[ABSTRACT]
 \begin{otherlanguage*}{english}
	\begin{flushleft} 
		\setlength{\absparsep}{0pt} % ajusta o espaçamento dos parágrafos do resumo		
 		\SingleSpacing  		\imprimirautorabr~~\textbf{\imprimirtitleabstract}.	\imprimirdata.  \pageref{LastPage}p. 
		%Substitua p. por f. quando utilizar oneside em \documentclass
		%\pageref{LastPage}f.
		\imprimirtipotrabalhoabs~-~\imprimirinstituicao, \imprimirlocal, 	\imprimirdata. 
 	\end{flushleft}
	\OnehalfSpacing 
   In this work, we proposed and implemented a Monte Carlo simulation to estimate experimental quantities of narrow-line magneto-optical traps (nMOTs). Our model relies on sampling the atoms' movement under laser light and a quadrupole magnetic field as a Markov chain. We assume that the involved electronic transition can be modelled as a four-level system and split into three independent two-level systems, which is only valid for nMOTs in the power-broadened regime. We were able to estimate quantities for three nMOT arrangements at different laboratories. The first nMOT traps dysprosium atoms on an electronic transition with narrowness $ \eta = 43.8 $, which is not an ideal value since it is slightly larger. Nevertheless, we could obtain estimated quantities that were more accurate than the theoretical predictions. The other two nMOTs trap strontium atoms on an electronic transition with narrowness $ \eta = 1.6 $. We could also obtain estimated quantities close to the experimental measures in this case.

   \vspace{\onelineskip}
 
   \noindent 
   \textbf{Keywords}: Magneto-optical trap. Monte Carlo simulation. Laser cooling.
 \end{otherlanguage*}
\end{resumo}
