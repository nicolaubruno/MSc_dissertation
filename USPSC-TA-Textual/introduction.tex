%
%
% Introduction Section
%-----------------------------------
\chapter{Introduction}
\label{ch:introduction}
%-----------------------------------

The deep understanding of light-matter interaction brought several scientific possibilities such as improvements in the atom interferometry \cite{peters2001high}, accurate spectroscopic methods \cite{mukamel2020roadmap}, and control of ultracold atoms. 
The Nobel Prize in Physics of 1997 was awarded jointly to Steven Chu \cite{chu1998nobel}, Claude Cohen-Tannoudji \cite{cohen1998nobel}, and William D. Phillips \cite{phillips1998nobel} for developing methods to cool and trap atoms with laser light, also known as laser cooling \cite{metcalf2007laser}. This achievement has enabled modern technologies, including accurate atomic clocks \cite{ludlow2015optical}, qubits for quantum computing \cite{schneider2012quantum}, and quantum sensors \cite{zhang2016precision}. Laser cooling also allowed the experimental confirmation of the Bose-Einstein condensation (BEC), motivating the Nobel Prize of Physics in 2001 \cite{cornell2002nobel, ketterle2002nobel}.

The workhorse of laser cooling is the magneto-optical trap (MOT) \cite{krzysztof2010magneto}, a technique to trap and cool a dilute atomic gas until temperatures in a range of $\mu K$. A standard MOT consists of three pairs of counter-propagating laser beams mutually orthogonal and a magnetic quadrupole field. Briefly, the atoms scatter photons from the laser light through atomic transitions, which cause momentum exchanges. The average momentum exchange yields a trapping and drag force on the atoms (MOT force). The frequency of these random scatterings increases with the atomic linewidth and defines the magnitude of both MOT force and its temperature. Overall, MOTs operating with narrow transitions reach lower temperatures at the cost of trapping efficiency. When the linewidth is comparable to the photonic recoil, we have the narrow line magneto-optical trap (nMOT) \cite{frisch2012narrow, maier2014narrow, miyazawa2021narrow}.

The quantitative MOT theories based upon Doppler cooling are limited to predict quantities such as temperature \cite{lett1988observation} and atomic cloud size \cite{gattobigio2010scaling} from analytical expressions. In many experiments, there is either the absence of theoretical predictions or the necessity of adjustable scaling factors \cite{loo2003investigations}. Furthermore, most of these theories are restricted to an unfeasible one-dimensional MOT \cite{metcalf2007laser, balykin2000electromagnetic}. The difficulty arises from the three-dimensional laser beams arrangement in the presence of a magnetic quadrupole field. A more accurate quantitative analysis of MOTs often involves a computational approach \cite{chaudhuri2006realization, atutov2001sodium,hanley2018quantitative}. The case of nMOTs is even more delicate since gravity can be comparable with the optical forces and then must be included.

In this thesis, we proposed a \textbf{Monte Carlo simulation} to predict experimental quantities of dysprosium and strontium nMOTs. Our model relies on sampling ensembles of atomic trajectories as a discrete stochastic process. Then, we estimate quantities such as temperature, centre of mass, and cloud size by averaging distributions of position and velocity. We sampled trajectories for standard six-beam nMOTs of dysprosium and strontium on which we acquired results in agreement with experimental measurements.

%-----------------------------------
\section{The thesis}
\label{sec:introduction-thesis}
%-----------------------------------

In the framework of this thesis, we perform a deep study of atom-light interactions aiming to understand the atom dynamics from a semiclassical perspectives. Firstly, we investigate the Einstein rate equations. Although this approach is proper to attend an elementary understanding, it does not contemplate coherent effects such as Rabi oscillations and the nature of line broadening mechanisms. To take these phenomena into account, we introduce the density operator formalism, treating the atom-light interaction semiclassically. Afterwards, we verify the mechanical effects of atom-light interaction introducing the optical forces.

We analyse the current theory of MOTs exploring a simplified unidimensional model and discussing the three-dimensional case. Moreover, we also analyse narrow-line MOTs and their three operating regimes. We are interested in scenarios in which gravity plays a relevant role.

Finally, we study Monte Carlo simulation applied to a stochastic process and optimized computational algorithms to develop methods that can predict experimental quantities accurately.

In the section 2, we present the whole study about atom-light interaction. Then, we introduce the theory of MOT and nMOTs in the section 3. In section 4, we discuss our Monte Carlo model and the data analysis. Finally, we present two set of simulated data for dysprosium and strontium nMOTs in the section 5, comparing them with experimental data.
