%
%
% Introduction Section
%-----------------------------------
\chapter{INTRODUCTION}
\label{ch:introduction}
%-----------------------------------

The deep understanding of light-matter interaction brought several scientific possibilities such as improvements in the atom interferometry \cite{peters2001high}, accurate spectroscopic methods \cite{mukamel2020roadmap}, and control of ultracold atoms. 
The Nobel Prize in Physics of 1997 was awarded jointly to Steven Chu \cite{chu1998nobel}, Claude Cohen-Tannoudji \cite{cohen1998nobel}, and William D. Phillips \cite{phillips1998nobel} for developing methods to cool and trap atoms with laser light, also known as laser cooling \cite{metcalf2007laser}. This achievement has enabled modern technologies, including accurate atomic clocks \cite{ludlow2015optical}, qubits for quantum computing \cite{schneider2012quantum}, and quantum sensors \cite{zhang2016precision}. Laser cooling also allowed the experimental confirmation of the Bose-Einstein condensation (BEC), motivating the Nobel Prize of Physics in 2001 \cite{cornell2002nobel, ketterle2002nobel}.

The workhorse of laser cooling is the magneto-optical trap (MOT) \cite{krzysztof2010magneto}, a technique to trap and cool a dilute atomic gas until temperatures in a range of $\mu K$. A standard MOT consists of three pairs of counter-propagating laser beams mutually orthogonal and a quadrupole magnetic field. Briefly, the atoms scatter photons from the laser light through atomic transitions, which cause momentum exchanges. The average momentum exchange yields a trapping and drag force on the atoms (MOT force). The frequency of these random scatterings increases with the atomic linewidth and defines the magnitude of both MOT force and its temperature. Overall, MOTs operating with narrow transitions reach lower temperatures at the cost of trapping efficiency. When the linewidth is comparable to the photonic recoil, we have the narrow line magneto-optical trap (nMOT) \cite{frisch2012narrow, maier2014narrow, miyazawa2021narrow}.

The currently theories of MOT based on Doppler cooling are limited to accurate predict experimental quantities such as temperature \cite{lett1988observation} and atomic cloud profile \cite{gattobigio2010scaling}. In many experiments, there is either the absence of theoretical predictions or the necessity of adjustable scaling factors \cite{loo2003investigations}. Furthermore, most of these theories are restricted to an unfeasible one-dimensional MOT \cite{metcalf2007laser, balykin2000electromagnetic}. The difficulty arises from the three-dimensional laser beams arrangement in the presence of a magnetic quadrupole field. The case of nMOTs is even more delicate since the gravity effect can be comparable to the optical forces and then must be included. This problem is often approach by computational solutions \cite{chaudhuri2006realization, atutov2001sodium,hanley2018quantitative} that are capable of analysing MOTs qualitatively and quantitatively.

In this thesis, we propose a Monte Carlo simulation in order to predict experimental quantities of nMOTs. Our model relies on sampling the atoms' movement by assuming it is a discrete stochastic process. Then, we estimate the probability distributions of the atoms' position and velocity, which define the atomic cloud profile and the temperature. We could obtain estimated quantities that are more accurate than the theoretical predictions for three nMOT arrangements at different laboratories.

%-----------------------------------
\section{The thesis}
\label{sec:introduction-thesis}
%-----------------------------------

In the framework of this thesis, we perform a deep study of atom-light interactions in Chapter \ref{ch:atom-light-interaction}. Firstly, in Section \ref{sec:rate-equations-model}, we investigate the Einstein rate equations to get a start on the basic concepts. Although this approach is proper for an elementary understanding, it does not contemplate coherent effects such as Rabi oscillations or the nature of line broadening mechanisms. To take these phenomena into account, we introduce the optical Bloch equations in Section \ref{sec:optical-Bloch-equations}, which tackle the electronic transitions within quantum mechanics. Afterwards, we analyse mechanical effects by introducing optical forces in Section \ref{sec:optical-forces}.

In Chapter \ref{ch:MOT}, we introduce the theory of MOTs for electronic transitions $ \ket{J = 0} \longrightarrow \ket{J = 1} $ by analysing a simplified unidimensional model in Section \ref{sec:one-dimensional-model}. Then, we discuss the three-dimensional case in Section \ref{sec:three-dimensional-MOT}. We also analyse narrow-line MOTs and their three operating regimes in Section \ref{sec:narrow-line-MOT}. In Chapter \ref{ch:Monte-Carlo-simulation}, we detailed our model to simulate the atoms' movement as a stochastic process and then estimate their position and velocity. We get through the parameters of the simulation and its output in Section \ref{sec:input-outputs}. Finally, in Chapter \ref{ch:results}, we present estimated quantities for three nMOT arrangements at different laboratories and then compare them with experimental measures and theoretical predictions.
