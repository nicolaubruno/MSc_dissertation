%
%--- 
%-----------------------------------
\chapter{Conclusion}
\label{ch:conclusion}
%-----------------------------------
%--- 
%

In this Master's degree work, we propose and implement a Monte Carlo simulation to estimate experimental quantities of the narrow-line magneto-optical traps.
We were able to simulate three nMOT arrangements reproduced by different laboratories and obtain estimated quantities that are more accurate than the theoretical ones. It is possible to use our simulation to optimize parameters of nMOTs without the necessity of experimental data. It is also a tool to analyse the feasibility of unusual nMOT arrangements such as nMOTs with fewer laser beams.

Our simulation is capable of estimating the atomic cloud profile and the temperature of nMOTs. We analysed the limits of our model regarding the three nMOT regimes and verified that it works exclusively in the power-broadened regime. Therefore, to obtain accurate quantities, it is necessary to guarantee a set of parameters that keep the nMOT in this regime.

We validated our model for nMOTs small and slightly large narrowness.

We estimated the atomic cloud profile and the temperature of the nMOT arrangement reproduced by Davide Dreon and his research group.

The magneto-optical traps are the workhorse of laser cooling. They were essential for at least two Nobel Prize laureates works since they allowed a myriad of remarkable achievements such as the Bose-Einstein condensation, accurate atomic clocks, quantum computers, and quantum sensors. Although MOTs work well experimentally, the theory of MOTs does not provide accurate predictions for some experimental quantities. The case of narrow-line MOTs is even more complicated due to the gravity effect.

The first part of this work was focused on to understand the interaction between atoms and light, which is essential to study MOTs properly. from two approaches. The first one is the rate equations model proposed by Albert Einstein. We explored such approach to get start to basic concepts. The second approach is the optical Bloch equations, which contemplates coherent effects and fundamental understanding of the line broadening mechanisms. We then deduced and analysed the optical forces, which is a key concept to understand the theory of the magneto-optical trap. We introduced the MOT theory by an simply one-dimension model. Afterwards, we presented the three-dimensional case as well as the narrow-line magneto-optical trap.

