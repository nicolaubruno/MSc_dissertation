%
%--- 
%-----------------------------------
\chapter{Monte Carlo simulation}
\label{ch:Monte-Carlo-simulation}
%-----------------------------------
%--- 
%

In order to predict experimental quantities in nMOTs, we propose a Monte Carlo simulation to estimate the probability distributions of both the atoms' position and velocity. The goal is to acquire simulated quantities in agreement with experimental measures for different nMOTs by considering as many parameters as possible. Our simulation can be used to optimize the parameters of nMOTs without the necessity of experimental data. It is also a tool to analyse the feasibility of unusual nMOT arrangements such as nMOTs with fewer laser beams. To ensure optimization, we developed a module for Python using the C programming language\footnote{We choose the C language since it is one of the fastest languages available.}, which deploys our model. We also developed a Python program that applies the model-view-controller pattern and parallelism. Our code is available in \cite{nMOT_sim}. In this section, we shall introduce our model as well as its deployment.

% Stochastic evolution
%-----------------------------------
%
%-----------------------------------
\section{Stochastic evolution}
\label{sec:stochastic-evolution}
%-----------------------------------
%

Let us consider an atom at $ \mathbf{r_0} $ and with the velocity $ \mathbf{v_0} $ in the presence of laser beams and a quadrupole magnetic field. After a period $ \delta t $, this atom can absorb a photon with momentum $ \hbar \mathbf{k_j} $ and then emit a photon with momentum $ \hbar |\mathbf{k}_j| \mathbf{\hat{u}} $, where $ \mathbf{\hat{u}} $ is an uniform unit random vector and $ \mathbf{k}_j $ is the wave vector of the $j$-th laser beam. Also, since there is a magnetic field gradient, a magnetic force acts on the atom as discussed in section \ref{sec:magnetic-force} that yields a position-dependent acceleration $ \mathbf{a}_{B}(\mathbf{r_0}) $. Therefore, the atom's velocity $ \mathbf{v}_i $ and position $ \mathbf{r}_i $ at instant $ i \delta t $, where $ i = 0, 1, \cdots $, are
\begin{equation}
    \mathbf{v}_{i} = \left\{ \begin{array}{lr}
        \mathbf{v}_{i - 1} + \hbar \frac{|\mathbf{k}_j|}{m}(\mathbf{\hat{k}}_j + \mathbf{\hat{u}}) + (\mathbf{a}_B(\mathbf{r}_{i - 1})- g \mathbf{\hat{z}}) \delta t, & \textrm{with probability}\ P_{i,j}
        \\
        \mathbf{v}_{i - 1} + (\mathbf{a}_B(\mathbf{r}_{i - 1})- g \mathbf{\hat{z}}) \delta t, & \textrm{with probability}\ 1 - \sum_{j} P_{i,j}
        \end{array} \right.
        \label{eq:atom-velocity-iteration}
\end{equation}
and
\begin{equation}
    \mathbf{r}_i = \mathbf{r}_{i - 1} + \mathbf{v}_{i}\delta t + (\mathbf{a}_B(\mathbf{r}_{i - 1}) - g \mathbf{\hat{z}}) \delta t^2 / 2,
    \label{eq:atom-position-iteration}
\end{equation}
where $ P_{i,j} $ is the probability of happing a scattering event due to the $ j $-th laser beam also known as \textbf{transition probability}. Since $ P_{i,j} $ is a conditional probability that depends on the previous atom state, the dynamics is then a \textbf{memoryless stochastic process}, also known as \textbf{Markov chain}. The goal is to sample trajectories $ \mathbf{r}(t) $ and velocities $ \mathbf{v}(t) $ in order to estimate the probability distributions of the atom's position and velocity. To perform this, we iterate the equations (\ref{eq:atom-velocity-iteration}) and (\ref{eq:atom-position-iteration}) and then fill up histograms.

We shall assume that simulating a single atom's motion for a long period of time is equivalent to simulating the motion of many atoms over smaller periods of time, which is known as \textbf{the ergodic hypothesis}. Thus, we shall work with ensembles of atoms' motions. We also shall neglect the interaction between atoms so that the atoms' motions will be independent of each other. These assumptions allow us to efficiently apply parallelism and then decrease the computational time.


%-----------------------------------
\subsection{Equilibrium}
\label{sec:equilibrium}
%-----------------------------------

Let us consider the vector $ \rho(t) = \rho(i\delta) = \rho_i $ in which each component is the probability of an atom being at specific position $ \mathbf{r} = (x, y, z) $ with a specific velocity $ \mathbf{v} = (v_x, v_y, v_z) $ at a instant $ t $.
We shall call this vector as \textbf{atom state}. Although position and velocity are continuous variables, we must discrete both to perform a feasible computation. If we consider $ N $ possible values for $ x $, $ y $, $ z $, $ v_x $, $ v_y $, and $ v_z $, the dimension of the vector $ \rho_i $ is $ N^6 $, which means that the computational spatial complexity is very high and can leads to memory issues. However, we can reduce such complexity by getting only the the marginal probabilities $ \rho_{i,\alpha} $, where $ \alpha \in \{x, y, z, v_x, v_y, v_z\} $, since they are independent, $ \rho_{i} = \sum_{\alpha} \rho_{i, \alpha} $. Hence, the total dimension will be reduced to $ 6N $.

Let us represent the transition probabilities $ P_i $ in the same vector space as a matrix $ \mathbf{P} $ of which each element is $ P_{i,j} $\footnote{We are assuming that $ \mathbf{r} $ and $ \mathbf{v} $ are discrete variables}. Therefore, after $ n $ iterations, the atom state will be \cite[Section~23.2]{wasserman2004all}
\begin{equation}
    \rho_n =  \rho_0 \underbrace{(\mathbf{P} \times \mathbf{P} \times \cdots \times \mathbf{P})}_{\textrm{mutiply the matrix $n$ times}} = \rho_0 \mathbf{P}^n,
\end{equation}
where $ \rho_0 $ is the initial atom state given by a deterministic vector since we set the initial atom's position and velocity. We expect that $ \rho $ be constant after a sufficient number of iterations so that $ \rho = \rho\mathbf{P} $ for large enough $ n $. In this case, we say that $ \rho $ is \textbf{stationary} and then the Markov chain settle into an \textbf{equilibrium}. We can ensure this assumption based on the \textbf{thermodynamics equilibrium}. To make sure that we are only getting samples in the equilibrium, we only store $ \mathbf{r} $ and $ \mathbf{v} $ after a specified number of iterations.

%-----------------------------------
\subsection{Magnetic field frame}
\label{sec:magnetic-field-frame}
%-----------------------------------

We shall consider the assumptions introduced in section \ref{sec:MOT-force}, which simplifies a four-level system in three independent two-level systems associated with the polarizations $ \sigma_{\pm} $ and $ \pi $. In this condition, for each laser beam, there are three possible transitions, each one with its scattering rate $ R_{i,j,l} $, where $ l \in \{\sigma_{+}, \sigma_{-}, \pi \} $. Initially, we define the laser polarizations on the basis $ \mathbf{B} = \{\hat{\sigma}_{+}, \hat{\sigma}_{-}, \hat{\pi} \} $ so that
\begin{equation}
    \hat{\sigma}_+ = \frac{\mathbf{\hat{x}} + i\mathbf{\hat{y}}}{\sqrt{2}},\ \ \hat{\sigma}_- = \frac{\mathbf{\hat{x}} - i\mathbf{\hat{y}}}{\sqrt{2}},\ \ \hat{\pi} = \mathbf{\hat{z}},
\end{equation}
where $ A = \{\mathbf{\hat{x}}, \mathbf{\hat{y}}, \mathbf{\hat{z}}\} $ is the basis of the \textbf{laboratory frame}.
\begin{figure}[!ht]
    \centering
    \caption{Magnetic field frame}
    \includegraphics[width=0.4\textwidth]{USPSC-img/polarization_basis.png}
    \vspace{5px}
    \legend{basis $ A' = \{\mathbf{\hat{x}}', \mathbf{\hat{y}}', \mathbf{\hat{z}}'\} $ of the magnetic field frame in relation to the basis $ A = \{\mathbf{\hat{x}}, \mathbf{\hat{y}}, \mathbf{\hat{z}}\} $ of the laboratory frame. \\ Source: author}
    \label{fig:magnetic-field-frame}
\end{figure}

To account the magnetic field $ \mathbf{B} $ effect in $ R_{i,j,l} $, we must analyse the magnetic field in a frame that defines the quantization axis parallel to $ \mathbf{B} $. Let us consider the basis $ A' = \{\mathbf{\hat{x}}', \mathbf{\hat{y}}', \mathbf{\hat{z}}'\} $ in the \textbf{magnetic field frame} as illustrated in figure \ref{fig:magnetic-field-frame}. The polarization basis $ \mathbf{B}' = \{\hat{\sigma}_{+}', \hat{\sigma}_{-}', \hat{\pi}' \} $ in the magnetic field frame is given by
\begin{equation}
    \hat{\sigma}_+' = \frac{\mathbf{\hat{x}'} + i\mathbf{\hat{y}}'}{\sqrt{2}},\ \ \hat{\sigma}_-' = \frac{\mathbf{\hat{x}'} - i\mathbf{\hat{y}}'}{\sqrt{2}},\ \ \hat{\pi}' = \mathbf{\hat{z}}' \| \mathbf{B}.
\end{equation}
The polarization vector $ \hat{\epsilon}_j $ of the $ j $-th laser beam is then
\begin{equation}
    [\hat{\epsilon}_j]_B = \left[ \begin{matrix} \epsilon_{i,\sigma_+} \\ \epsilon_{i,\sigma_-} \\ \epsilon_{i,\pi} \end{matrix} \right]\ \Rightarrow\ \hat{\epsilon}_j = \epsilon_{i,\sigma_+} \hat{\sigma}_{+} + \epsilon_{i,\sigma_-} \hat{\sigma}_{-} +  \epsilon_{i,\pi} \hat{\pi},
\end{equation}
where $ [\hat{\epsilon}_j]_B $ is a column matrix whose elements are the components of $ \hat{\epsilon}_j $ on the laboratory basis B. Hence, the components of $ \hat{\epsilon}_j $ on the magnetic field basis B are
\begin{equation}
    [\hat{\epsilon}_j]_B' = M [\hat{\epsilon}_j]_B,
\end{equation}
where $ M $ is the change-of-basis matrix. We must consider two other change-of-basis matrices to obtain $ M $. The first one is the matrix $ M' $ that change a polarization basis such as $ B $ to a Cartesian basis such as $ A $. The second one is the rotation matrix $ R(\theta) $. Thus, the change-of-basis $ M $ is given by
\begin{equation}
    M = (M')^{\dagger}R(\theta)M',\ \ (M')^{\dagger} = ((M')^*)^T = (M')^{-1},
\end{equation}
where
\begin{equation}
    M' = \left[ \begin{matrix}
        1/\sqrt{2} & 1/\sqrt{2} & 0 \\
        -i/\sqrt{2} & i/\sqrt{2} & 0 \\
        0 & 0 & 1
    \end{matrix} \right],\ \
    R(\theta) = \left[ \begin{matrix}
        1 & 0 & 0 \\
        0 & \cos(\theta) & -\sin(\theta) \\
        0 & \sin(\theta) & \cos(\theta)
    \end{matrix} \right].
\end{equation}

%-----------------------------------
\subsection{Transition probabilities}
\label{sec:transition-probabilities}
%-----------------------------------

The transition probability $ P_{i,j} $ is the probability of an atom at state $ i $ transiting to the state $ j $ by scattering a photon of the $j$-th laser in a nMOT arrangement. Let us impose the independent two-level systems assumption discussed in section \ref{sec:MOT-force} such that there will be three independent transitions for each laser beam. Thus,
\begin{equation}
    P_{i,j} = \sum_{l = 0}^{2} P_{i,j,l},
\end{equation}
where $ P_{i,j,l} $ is the probability of an atom at state $ i $ scattering a photon of the $j$-th laser due to the $l$-th electronic transition. This assumption is only valid when the Zeeman shift is strong enough to decouple the excited states so that we can neglect coherent transitions between them. Therefore, our model is reliable when the trapped atoms are far away from the magnetic field origin, which happens in the power-broadened and quantum regimes (see section \ref{sec:nMOT-operating-regimes}). Although the quantum regime matches this condition, we are not analysing the atom dynamics in quantum mechanics since we are treating the atoms' position and velocity classically. Therefore, we expect that our model will fail in the Doppler and quantum regime.

The time derivative of $ P_{i,j,l} $ is the scattering rate $ R_{i,j,l} $ given by
\begin{equation}
    R_{i,j,l} = \frac{\Gamma}{2}\frac{s(\mathbf{r})}{1 + s(\mathbf{r}) + (2\Delta_l / \Gamma)^2},\ \ s(\mathbf{r}) = \exp\left[\ -\frac{2(x^2 + y^2)}{w^2} \right],\ \ \Delta_l = \delta + \delta_Z^{(l)} + \delta_D,
    \label{eq:scattering-rate-each-transition}
\end{equation}
where $ w $ and $ \delta $ is, respectively, the waist and the laser detuning of the $ j $-th laser beam, $ \delta_Z^{(l)} $ is the Zeeman shift due to the $ l $ transition given by equation (\ref{eq:Zeeman-shift}), and $ \delta_D = - \mathbf{k} \cdot \mathbf{v}_{i - 1} $ is the Doppler shift. To increase accuracy, we are taking into account the Gaussian profile of the laser beams in $ s(\mathbf{r}) $. Hence, the probability $ P_{i,j,l} $ of happening a scattering event during a small time interval $ \delta t $ can be approximate by
\begin{equation}
    R_{i,j,l} = \frac{\partial P_{i,j,l}}{\partial t} \Rightarrow P_{i,j,l} \simeq R_{i,j,l} \delta t.
\end{equation}
Each transition is associated with one laser polarization. The probability of a photon from the $j$-th laser beam having the polarization associated with the $ l $-transition is $ \left| \braket{\hat{\epsilon}_j|\mathbf{\hat{e}}_l} \right|^2 $, where $ \mathbf{\hat{e}}_l \in B' $. Thus, the transition probability $ P_{i,j} $ is given by
\begin{equation}
    P_{i,j} = \sum_{l} \left| \braket{\hat{\epsilon}_j|\mathbf{\hat{e}}_l} \right|^2 P_{i,j,k} =  \sum_{l} \left| \braket{\hat{\epsilon}_j|\mathbf{\hat{e}}_l} \right|^2 R_{i,j,l} \delta t.
    \label{eq:transition-probabilities}
\end{equation}

%-----------------------------------
%

% Monte Carlo Simulation
%-----------------------------------
%
%-----------------------------------
\section{Input and outputs}
\label{sec:input-outputs}
%-----------------------------------
%

In this section, we define the parameters (inputs) to execute the simulation as well as the experimental quantities (outputs) obtained from the estimated distributions of the atom's position and velocity.

%-----------------------------------
\subsection{Input}
\label{sec:input}
%-----------------------------------

Since we seek to obtain accurate quantities from the simulation, we must take into account detailed information about the nMOT. There are two groups of parameters, the experiment parameters and the performance parameters. The \textbf{experiment parameters}, related to the nMOT, define the laser beam arrangement, the magnetic field profile, the involved electronic transitions, and the atom. Whilst the \textbf{performance parameters} define the computational resolution as well as precision parameters.

The experiment parameters are presented in the tables \ref{tab:transition-parameters}, \ref{tab:magnetic-field-profile-parameters}, and \ref{tab:laser-beams-arrangement-parameters}.
\begin{table}[ht!]
    \centering
    \begin{tabular}{|c|c|c|}
        \hline
        \textbf{Symbol} & \textbf{Description} & \textbf{Unit} \\ \hline
        $ \Gamma $ & Natural linewidth & $ 2\pi \times kHz $ \\
        $ \lambda $ & Resonance wavelength & $ nm $ \\
        $ J_{\textrm{gnd}} $ & Total angular momentum of the ground state & dimensionless \\
        $ J_{\textrm{exc}} $ & Total angular momentum of the excited state & dimensionless \\
        $ g_{\textrm{gnd}} $ & \textit{Landé} factor of the ground state & dimensionless \\
        $ g_{\textrm{exc}} $ & \textit{Landé} factor of the excited state & dimensionless \\
        $ m $ & Atomic mass & $ Da $ (Dalton unit) \\
        \hline
    \end{tabular}
    \caption{Parameters that defines the electronic transition.}
    \label{tab:transition-parameters}
\end{table}

The next set of parameters presented in table \ref{tab:magnetic-field-profile-parameters} defines the magnetic field profile. Besides the quadrupole magnetic field, some experiments have a residual linear gradient and a constant magnetic field to control the magnetic field origin.
\begin{table}[ht!]
    \centering
    \begin{tabular}{|c|c|c|}
        \hline
        \textbf{Symbol} & \textbf{Description} & \textbf{Unit} \\ \hline
        $ B_0 $ & Magnetic field gradient in equation (\ref{eq:magnetic-field-profile}) & $ G/cm $ \\
        $ B_{\textrm{axial}} $ & Axial direction of the magnetic field & 3D vector \\
        $ B_{\textrm{lingrad}} $ & Residual magnetic field gradient & 3D vector \\
        $ B_{\textrm{bias}} $ & Constant magnetic field & 3D vector \\
        \hline
    \end{tabular}
    \caption{Parameters that defines the magnetic field profile.}
    \label{tab:magnetic-field-profile-parameters}
\end{table}

The final set of experimental quantities defines the laser beam arrangement. The parameters shown in table \ref{tab:laser-beams-arrangement-parameters} must be defined for each laser beam.
\begin{table}[ht!]
    \centering
    \begin{tabular}{|c|c|c|}
        \hline
        \textbf{Symbol} & \textbf{Description} & \textbf{Unit} \\ \hline
        $ \delta $ & Laser detuning & $ 2\pi \times kHz $ \\
        $ s_0 $ & Saturation parameter & dimensionless \\
        $ w $ & Waist & $ cm $ \\
        $ \hat{\mathbf{k}} $ & Wavevector direction & 3D vector \\
        $ \hat{\epsilon} $ & Polarization vector in the laboratory frame & 3D vector \\
        \hline
    \end{tabular}
    \caption{Parameters that defines the laser beams arrangement.}
    \label{tab:laser-beams-arrangement-parameters}
\end{table}

%-----------------------------------
\subsection{Output}
\label{sec:output}
%-----------------------------------

As discussed in section \ref{sec:stochastic-evolution}, the simulation estimates distributions for the atoms' position and velocity. In total, we have 6 set distributions $ d_x $, $ d_y $, $ d_z $, $ d_{v_x} $, $ d_{v_y} $, and $ d_{v_z} $. With this distributions, we can estimate the atomic cloud profile and temperature. We are interested in two atomic cloud variables: the centre of mass and the cloud size. The centre of mass is the easiest one to obtain from the distributions,
\begin{equation}
    x_c = \braket{d_x} =
\end{equation}

%-----------------------------------
%
