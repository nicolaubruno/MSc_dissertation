%
%-----------------------------------
\section{Dysprosium nMOT}
\label{sec:dysprosium-nMOT}
%-----------------------------------
%

We choose to simulate the dysprosium nMOT \cite{dreon2017optical} reproduced by Davide Dreon and his research group since it matches the conditions of our model and is thorough detailed in the Dreon PhD thesis \cite{dreon2017designing}, which is necessary to improve accuracy as discussed in section \ref{sec:input-outputs}.

\begin{wrapfigure}{l}{0.5\linewidth}
    \centering
    \caption{Electronic transitions of the dysprosium nMOT}
    \includegraphics[width=0.45\textwidth]{USPSC-img/Dy-Dreon-transitions.png}
    \vspace{5px}
    \legend{. \\ Source: \cite{dreon2017optical}}
    \label{fig:Dy-Dreon-electronic-transitions}
\end{wrapfigure}
The involved electronic transition presented in table \ref{tab:electronic-transition-Dy-Dreon} yields a narrowness of $ \eta =  43.8 $ that is not small enough to reach the quantum regime, but it is enough to reach the power-broadened regime. Furthermore, in the presence of a magnetic field, the involved electronic transition $ J = 8 \longrightarrow J' = 9 $ has 36 possible states, which is much more complicated than the transition $ J = 0 \longrightarrow J = 1 $ presented in chapter \ref{ch:Monte-Carlo-simulation}. However, previous works \cite{lu2011strongly,aikawa2012bose} on nMOTs with Lanthanide atoms experimentally confirmed a spontaneous spin polarization that allows to simplify the transitions to $ \ket{J = 8, m_J = 8} \longrightarrow \ket{J' = 9, m_J' = -7, -8, -9} $ as  illustrated in figure \ref{fig:Dy-Dreon-electronic-transitions}.

\begin{table}[ht!]
    \centering
    \begin{tabular}{|c|c|c|}
        \hline
        \textbf{Symbol} & \textbf{Quantity} & \textbf{Value} \\ \hline
        $ \Gamma $ & Natural Linewidth & $ 2\pi \times 136\ kHz $ \\
        $ \lambda $ & Resonant wavelength & $ 626\ nm $ \\
        $ J_{gnd} $ & Ground state angular momentum & $ 8 $ \\
        $ g_{gnd} $ & Ground state Landè factor & $ 1.24 $ \\
        $ g_{exc} $ & Excited state Landè factor & $ 1.29 $ \\
        $ m $ & Mass & $ 164\ u $ \\
        \hline
    \end{tabular}
    \caption{Eletronic transition parameters of the dysprosium nMOT reproduced by \cite{dreon2017designing}.}
    \label{tab:electronic-transition-Dy-Dreon}
\end{table}

$ T_D = 3.26\ \mu K $

\begin{table}[ht!]
    \centering
    \begin{tabular}{|c|c|c|}
        \hline
        \textbf{Symbol} & \textbf{Quantity} & \textbf{Value} \\ \hline
        $ w $ & Waist & $ 2.0\ cm $ \\
        $ s_0 $ & Saturation parameter & $ 0.65 $ \\
        \hline
    \end{tabular}
    \caption{Laser setup features}
    \label{tab:lasers-setup}
\end{table}

\begin{table}[ht!]
    \centering
    \begin{tabular}{|c|c|c|}
        \hline
        \textbf{Symbol} & \textbf{Quantity} & \textbf{Value} \\ \hline
        $ B_0 $ & Axial gradient & $ 1.71 G $ \\
        $ B $ & Magnetic Field & $ B_0(-\hat{x} + \hat{y}/2 + \hat{z} / 2) $ \\
        $ B_{bias} $ & Bias & $ (-0.094 \hat{z})\ G / cm $ \\
        \hline
    \end{tabular}
    \caption{Magnetic quadrupole field}
    \label{tab:magnetic-field}
\end{table}

%-----------------------------------
\subsection{Atomic cloud profile}
\label{sec:cloud-profile-dysprosium}
%-----------------------------------

\begin{figure}[!ht]
    \centering
    \caption{Simulated ${}^{164}Dy$ cloud profile}
    \includegraphics[width=0.7\textwidth]{USPSC-img/dy_dreon_cloud_profile.png}
    \vspace{5px}
    \legend{Simulated dysprosium cloud profile for three laser detunings: $ \delta = -5.0\Gamma' $ (a), $ \delta = -8.6\Gamma' $ (b), and $ \delta = -11.1\Gamma' $ (c), where $ \Gamma' = \Gamma \sqrt{1 + s_0} $ is the power-broadened linewidth. \\ Source: author}
    \label{fig:dy-atomic-cloud-profile}
\end{figure}

\begin{figure}[!ht]
    \centering
    \caption{Centre of mass of the ${}^{164}Dy$ nMOT}
    \includegraphics[width=0.7\textwidth]{USPSC-img/dy_centre_of_mass.png}
    \vspace{5px}
    \legend{centre of mass.\\ Source: author}
    \label{fig:dy-centre-of-mass}
\end{figure}

\begin{figure}[!ht]
    \centering
    \caption{Cloud size of the ${}^{164}Dy$ nMOT}
    \includegraphics[width=0.7\textwidth]{USPSC-img/dy_cloud_size.png}
    \vspace{5px}
    \legend{cloud size.\\ Source: author}
    \label{fig:dy-cloud-size}
\end{figure}

\begin{figure}[!ht]
    \centering
    \caption{Cloud size ratio of the ${}^{164}Dy$ nMOT}
    \includegraphics[width=0.55\textwidth]{USPSC-img/dy_cloud_size_ratio.png}
    \vspace{5px}
    \legend{cloud size ratio.\\ Source: author}
    \label{fig:dy-cloud-size-ratio}
\end{figure}

%-----------------------------------
\subsection{Temperature}
\label{temperature}
%-----------------------------------

\begin{figure}[!ht]
    \centering
    \caption{Temperature of the ${}^{164}Dy$ nMOT as a function of the laser detuning}
    \includegraphics[width=0.6\textwidth]{USPSC-img/dy_temperature.png}
    \vspace{5px}
    \legend{temperature.\\ Source: author}
    \label{fig:dy-temperature}
\end{figure}
