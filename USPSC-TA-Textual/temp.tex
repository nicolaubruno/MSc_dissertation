This approach matched the available light sources in the first half of the 20th century, providing an insightful comprehension of the basic concepts of light-matter interaction. However, the advent of the laser in 1958 makes an intense, collimated, spectrally narrow and phase-coherent light possible, demanding a new theory to treat the interaction between laser light and atom. 

A first attempt to approach this problem is to use the Schrodinger equation, which is enough as long as we are only interested in coherent stimulated processes. These processes only contemplate the atom interaction with a single-mode light. However, we have to consider the atom interacting with the light modes of the vacuum to take spontaneous emission into account. Therefore, the atom must be described by a distribution of states That can not be described by a single wavefunction, 

we deduce the radiation pressure force through the 
stationary solution of the Optical Bloch Equations. After, we interpret this 
force as a process of absorption and spontaneous emission of photons. Then, we 
deduce a net force on the atoms in a low-speed regime, checking the limit for 
narrow transitions.

\textcolor{red}{The thesis is structured as follows. We initially introduce 
concepts of light-matter interaction in chapter 
\ref{ch:ligh-matter-interaction}. In particular, the interaction between a 
two-level atomic system and a classical electromagnetic field. Then, we present 
the MOT and nMOT theory in chapter \ref{ch:MOT}, introducing a didactic 
unidimensional model and expending it to a three-dimensional configuration. 
Afterwards, we model the simulation as a Markov chain in chapter 
\ref{ch:simulation}, showing an optimized implementation using parallel 
programming. In chapter \ref{ch:results}, we check the simulation comparing 
simulated data with experimental values of nMOTs. We also present the analysis 
of nMOTs with four and three beams using a  picture and simulated results. 
Furthermore, we summarize our findings and give an outlook to further 
possibilities in chapter \ref{ch:conclusion}.}

\textcolor{red}{Our model rely on to sample ensembles of random atomic 
trajectories mediate by scattering events and infer MOT properties through 
statistical calculations, which is a stochastic process. more precisely a Markov 
chain since the probability of happening a scattering events only depends of the 
current state of the atom. Furthermore, we propose a trapping efficiency 
parameter to verify the feasibility of MOTs with a reduced number of beams. We 
obtain simulated data of six-beam nMOTs with dysprosium and strontium in 
agreement with experimental values. Moreover, we propose four and three-beam 
nMOTs, using simulated data and a semiclassical analysis to verify trapping 
efficiency and other properties like temperature and atomic cloud shape.}